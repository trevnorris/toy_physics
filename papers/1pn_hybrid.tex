\documentclass[11pt]{article}

% Basic packages
\usepackage[margin=1in]{geometry}
\usepackage{amsmath,amssymb,amsfonts}
\usepackage{bm}
\usepackage{graphicx}
\usepackage{hyperref}
\usepackage[numbers,sort&compress]{natbib}
\usepackage{authblk}

% Hyperref setup (matches earlier papers; no red boxes around links)
\hypersetup{
  colorlinks=true,
  linkcolor=blue,
  citecolor=blue,
  urlcolor=blue
}

% Custom commands (extend/tune as needed)
\newcommand{\PN}{\mathrm{PN}}
\newcommand{\cS}{c_s}
\newcommand{\dd}{\mathrm{d}}
\newcommand{\ve}{\varepsilon}
\newcommand{\GM}{GM}
\newcommand{\br}{\mathbf{r}}
\newcommand{\bv}{\mathbf{v}}
\newcommand{\PhiN}{\Phi_{\mathrm{N}}}
\newcommand{\PhiL}{\Phi_{\mathrm{L}}}
\newcommand{\rhozero}{\rho_0}

\title{Hybrid 1PN Dynamics and the Acoustic Horizon\\
in a Superfluid Defect Toy Model}
% (Feel free to rename; this is just a working title.)

\author{Trevor Norris}
\date{\today}

\begin{document}

\maketitle

\begin{abstract}
% TODO: Briefly summarize
% - Hybrid scalar+vector 1PN dynamics
% - Uniqueness of n=5 and M ∝ ρ from thermodynamic + EIH matching
% - Acoustic horizon / photon sphere structure
% - Where deviations from GR are expected (2PN / strong field)
\end{abstract}

\section{Introduction}
\label{sec:intro}

\subsection{Background and previous papers}
\label{subsec:intro-background}

This paper is the sixth in a series developing a ``superfluid defect toy
universe'' in which gravity and electromagnetism emerge from the dynamics
of a compressible fluid.  The vacuum is modeled as a stiff superfluid, and
massive bodies are represented as localized defects that drain and stir this
medium.  Far from any defect, the flow is slow and nearly irrotational; near
the defect, nonlinear and topological effects become important.  The central
claim of the series so far has been that, with a suitable choice of
hydrodynamic energy functional, this superfluid picture reproduces the
standard post--Newtonian (PN) expansion of general relativity (GR) at first
PN order (1PN), while simultaneously admitting a natural electromagnetic
sector.\footnote{For clarity, we will refer to the previous papers as
Papers~I--V\cite{Norris:Paper1,Norris:Paper2,Norris:Paper3,Norris:Paper4,Norris:Paper5}
throughout.}

Paper~I developed the \emph{orbital} sector of the model.\cite{Norris:Paper1}
Starting from a scalar ``lag'' field that allows the bulk fluid to slip
relative to the defects, and a position--dependent kinetic prefactor that
encodes how inertia renormalizes in the throat background, the model
reproduces Newtonian orbits and a GR--like 1PN perihelion advance for
weakly bound two--body systems.  The effective gravitational potential is
encoded in a long--range pressure and density profile around each defect,
with an additional precession term arising from hydrodynamic dressing of the
moving throat.

Paper~II recast the same superfluid model in optical language.\cite{Norris:Paper2}
There, the $1/r$ density profile around a defect defines a refractive index
$N(r) = c/c_s(r)$ for small--amplitude waves in the vacuum, with $c_s$ the
local sound speed.  For a particular ``stiff'' polytropic equation of state,
$n=5$ in the language of polytropes, the resulting index profile reproduces
light bending, Shapiro delay, and gravitational redshift at 1PN order for
static, spherically symmetric configurations.  In GR language, the model is
1PN--equivalent to Schwarzschild for the classic solar--system tests; in
analogue--gravity language, massive defects and light rays probe the same
emergent optical geometry of the vacuum.

Paper~III extended the construction to include spin, frame dragging, and
$N$--body dynamics.\cite{Norris:Paper3}  By promoting the defect flow to a
vortex--ring / dyon configuration and allowing for a vector sector in the
hydrodynamic energy functional, that work derived an effective Lense--Thirring
interaction and an Einstein--Infeld--Hoffmann (EIH)--like $N$--body
Lagrangian.  A key result was that no purely Euclidean, positive--definite
hydrodynamic energy reproduces the EIH tensor: matching GR at 1PN requires an
\emph{effective Lorentzian signature} in the longitudinal sector, encoded in
a tuned parameter $\alpha^2 = -2/5$.  With this choice, the superfluid toy
universe reproduces the full single--body and $N$--body 1PN dynamics of GR,
including scalar, optical, spin, and vector effects, using a small set of
medium response parameters.

Paper~IV introduced an explicit electromagnetic sector.\cite{Norris:Paper4}
Charge and current were reinterpreted as topological and kinematic features
of the defect flow, and the Maxwell equations were shown to emerge from the
underlying hydrodynamics in an appropriate long--wavelength limit.  In that
description, the same superfluid that mediates gravity also supports an
effective gauge field, with the Lorentz force arising from Magnus and
enthalpy forces acting on defects.

Paper~V resolved a geometric tension between the gravitational and
electromagnetic sectors.\cite{Norris:Paper5}
The orbital and PN analyses are most naturally expressed in terms of
\emph{spherical} sinks, while the electromagnetic sector prefers
\emph{cylindrical} resonant cavities.  Paper~V promoted the defects to
brane--bulk throats connecting the observable 3D brane to a 4D superfluid
bulk.  Dimensional reduction shows that such a throat can appear both
spherical and cylindrical to different sectors: the far--field potential on
the brane is monopolar, while the near--field structure supports cylindrical
resonances that play the role of electromagnetic cavities.  In effect, the
throat geometry unifies the gravitational and electromagnetic pictures.

Taken together, Papers~I--V paint a coherent 1PN--accurate picture of
gravity and electromagnetism emerging from a single structured vacuum
fluid.  At the same time, they leave several important questions open.
Why does the vacuum prefer the stiff polytrope $n=5$ singled out by the
optical and PN matching?  Can the apparently ad hoc Lorentzian tuning
$\alpha^2=-2/5$ be understood as a consequence of more primitive
thermodynamic and equivalence principles?  And what does the fixed fluid
microphysics imply for strong--field phenomena such as horizons, photon
spheres, and the onset of deviations from GR beyond 1PN?

\subsection{Scope and goals of this work}
\label{subsec:intro-goals}

The present paper is a capstone to the 1PN program.
Rather than pushing immediately to a full second post--Newtonian (2PN)
expansion, we focus on \emph{closing the 1PN story} and extracting the
strong--field consequences of the already--tuned fluid.  Our starting point
is deliberately modest: we assume only that the vacuum is a barotropic
polytropic superfluid with equation of state $P \propto \rho^n$, and that
the effective inertial mass of a defect scales as a power of the local
density, $M \propto \rho^q$, reflecting the idea that defects are fed by the
same fluid they drain.  These two assumptions encode, in hydrodynamic
language, an equation of state and an ``equivalence'' principle relating
mass to the vacuum density that supports it.

From this starting point we develop a \emph{hybrid} scalar--vector description
of test--body dynamics in the throat background.  The scalar sector captures
``density starvation'': in regions of deep potential, the background density
is depleted and the defect's effective mass changes.  The vector sector
captures the flow and refractive properties of the medium, as in the optics
paper.  By expanding the combined scalar+vector test--particle Lagrangian to
1PN order and demanding agreement with the standard EIH coefficients, we
show that the model itself selects a unique pair of exponents:
\begin{itemize}
  \item $q = 1$, so that defect mass is proportional to the local vacuum
  density, and
  \item $n = 5$, so that the vacuum is a stiff polytrope of index five.
\end{itemize}
In other words, the same 1PN matching that fixed $\alpha^2=-2/5$ in
Paper~III can be rephrased as a \emph{thermodynamic uniqueness theorem}:
within this class of models, a single polytropic index and a single
mass--density scaling law are compatible with Newtonian gravity and the EIH
1PN Lagrangian.

A second goal is to explore the strong--field implications of this fixed
fluid.  Once $n=5$ and $q=1$ are enforced, the radial inflow toward a defect
is no longer arbitrary: the same equation of state that controls 1PN
coefficients also determines the nonlinear profile of the flow.  We show
that this profile generically becomes transonic, defining an \emph{acoustic
horizon} where the radial flow speed equals the local sound speed, and that
the associated refractive index profile supports a photon--sphere--like
structure for null rays.  Using a combination of analytic arguments and
numerical profiles, we extract the scaling of the horizon radius and
effective ``throat impedance'' with defect mass, and identify the regimes in
which the toy model tracks Schwarzschild geometry and where it is expected
to deviate.

Throughout, we remain strictly within the 1PN expansion for precise
matching: all exact coefficient comparisons with GR are performed at 1PN
order, and 2PN corrections are treated only qualitatively, as indicators of
where the toy model should begin to depart from GR.  The main results of the
paper can be summarized as follows:
\begin{enumerate}
  \item A hybrid scalar--vector analysis shows that 1PN consistency with the
  EIH Lagrangian uniquely fixes the vacuum to be a stiff polytrope with
  $n=5$ and enforces a linear mass--density relation $M \propto \rho$.
  \item The previously tuned Lorentzian signature parameter
  $\alpha^2=-2/5$ from Paper~III is reinterpreted as an emergent feature of
  the combined scalar+vector response, rather than an independent model
  choice.
  \item The same fixed fluid supports an acoustic horizon and photon--sphere
  structure whose scaling with mass is close to the Schwarzschild case,
  while predicting specific finite--size and 2PN--scale deviations in the
  strong--field regime.
\end{enumerate}
In this sense, the present work completes the conceptual arc of the 1PN
superfluid defect program and sets the stage for a future, more technical
analysis of 2PN corrections, radiation reaction, and fully nonlinear
numerical simulations.

\section{Thermodynamic Vacuum Model and Equivalence Assumption}
\label{sec:thermo_model}

In this section we collect the minimal thermodynamic structure assumed for
the superfluid vacuum and translate it into a small set of parameters that
enter the post--Newtonian expansion.  The guiding principle is to separate
assumptions about the medium itself (equation of state) from assumptions
about how defects couple to that medium (mass--density relation).  The
results of this section will be used in
Secs.~\ref{sec:scalar_sector}--\ref{sec:hybrid_uniqueness} to derive the
hybrid scalar--vector 1PN dynamics and to fix the polytropic index and
mass--density exponent.

\subsection{Polytropic superfluid vacuum}
\label{subsec:thermo-polytrope}

We model the vacuum as a barotropic, compressible superfluid with equilibrium
density $\rhozero$ and pressure $P_0$ far from any defect.  In the absence of
defects the fluid is static and homogeneous, with small perturbations
propagating as sound waves.  We assume a polytropic equation of state (EOS)
of the form
\begin{equation}
  P(\rho) = K\,\rho^n,
  \label{eq:polytropic-eos}
\end{equation}
where $K>0$ is a stiffness parameter and $n$ is the polytropic index.  The
adiabatic sound speed $\cS$ is given by
\begin{equation}
  \cS^2(\rho) = \frac{\dd P}{\dd \rho}
  = K\,n\,\rho^{\,n-1}.
  \label{eq:sound-speed-general}
\end{equation}
We choose units such that the asymptotic sound speed at the background
density equals the speed of light,
\begin{equation}
  \cS(\rhozero) = c,
  \qquad
  \rhozero > 0.
  \label{eq:cs-background}
\end{equation}
This condition fixes the combination $K n \rhozero^{\,n-1}$ but leaves $n$
itself undetermined.  In previous work the specific value $n=5$ was selected
by matching the optical sector of the model to the 1PN Schwarzschild
geometry.  Here we will instead treat $n$ as a free parameter and show that
1PN consistency with the EIH Lagrangian fixes $n=5$ uniquely within this
class.

It is convenient to introduce a dimensionless density contrast
\begin{equation}
  \delta \equiv \frac{\rho - \rhozero}{\rhozero},
  \qquad
  |\delta| \ll 1
  \label{eq:density-contrast}
\end{equation}
in the weak--field regime relevant for the PN expansion.  Expanding
Eq.~\eqref{eq:sound-speed-general} around $\rho=\rhozero$ gives
\begin{equation}
  \cS^2(\rho)
  = c^2 \left[ 1 + (n-1)\,\delta + \mathcal{O}(\delta^2) \right],
  \label{eq:cs-expansion}
\end{equation}
so that, to leading order, the polytropic index $n$ directly controls how
the effective light speed $c_s$ responds to small changes in the local
vacuum density.

\subsection{Equivalence and mass--density scaling}
\label{subsec:thermo-equivalence}

Defects are modeled as localized structures that exchange mass and momentum
with the surrounding superfluid.  In the far field, each defect is
characterized by an effective inertial mass $M$ that determines its response
to forces and its gravitational potential.  In the throat picture, $M$
measures the flux of superfluid fed into the throat and the depth of the
associated density depression.

To connect the defect mass to the vacuum, we adopt a simple power--law
relation between the effective mass and the local background density,
\begin{equation}
  M(\rho) \propto \rho^q,
  \label{eq:mass-density-scaling}
\end{equation}
where $q$ is a dimensionless exponent.  This \emph{equivalence assumption}
encodes the idea that the inertia of a defect is not an independent
parameter but is tied to the density of the same medium that supports it.
For example, $q=1$ corresponds to a defect whose mass is proportional to the
density of the fluid feeding the throat, while $q=0$ would describe an
object with fixed mass independent of slow changes in the surrounding
vacuum.

In a weak gravitational potential $\Phi$ the background density is slightly
depleted relative to $\rhozero$.  We write
\begin{equation}
  \rho(\Phi) = \rhozero\,[1 + \delta(\Phi)],
  \qquad
  |\delta(\Phi)| \ll 1,
  \label{eq:rho-of-Phi}
\end{equation}
and expand Eq.~\eqref{eq:mass-density-scaling} to first order,
\begin{equation}
  M(\Phi)
  = M_0 \left[ 1 + q\,\delta(\Phi) + \mathcal{O}\big(\delta^2\big) \right],
  \label{eq:M-of-Phi-delta}
\end{equation}
where $M_0 \equiv M(\rhozero)$ is the asymptotic mass.  The function
$\delta(\Phi)$ will be determined later from the scalar sector; for the PN
expansion it is sufficient to assume that $\delta$ is proportional to the
dimensionless potential,
\begin{equation}
  \delta(\Phi)
  = \gamma\,\frac{\Phi}{c^2}
  + \mathcal{O}\!\left(\frac{\Phi^2}{c^4}\right),
  \label{eq:delta-of-Phi}
\end{equation}
for some constant $\gamma$ of order unity.  Combining
Eqs.~\eqref{eq:M-of-Phi-delta} and \eqref{eq:delta-of-Phi} then yields a
variable--mass profile of the form
\begin{equation}
  M(\Phi)
  = M_0\left[ 1 + q\,\gamma\,\frac{\Phi}{c^2}
  + \mathcal{O}\!\left(\frac{\Phi^2}{c^4}\right) \right],
  \label{eq:M-of-Phi-linear}
\end{equation}
which will feed directly into the scalar contribution to the 1PN test--body
Lagrangian in Sec.~\ref{sec:scalar_sector}.  The 1PN matching conditions will
fix the product $q\gamma$ and, ultimately, $q$ itself.

\subsection{Dimensionless parameters and PN counting}
\label{subsec:thermo-pn-counting}

We work in a standard weak--field, slow--motion regime where the
post--Newtonian expansion is organized in powers of $v/c$ and $\Phi/c^2$.
For a characteristic orbital velocity $v$ and potential $\Phi$ we introduce
dimensionless expansion parameters
\begin{equation}
  \ve_v \equiv \frac{v^2}{c^2},
  \qquad
  \ve_\Phi \equiv \frac{|\Phi|}{c^2},
  \label{eq:pn-parameters}
\end{equation}
and assume $\ve_v \sim \ve_\Phi \ll 1$.  Terms of order
$\mathcal{O}(\ve_v)$ or $\mathcal{O}(\ve_\Phi)$ are Newtonian, while terms
of order $\mathcal{O}(\ve_v^{\,2})$, $\mathcal{O}(\ve_v\ve_\Phi)$, or
$\mathcal{O}(\ve_\Phi^{\,2})$ are 1PN corrections.  Throughout this paper we
retain all contributions up to first post--Newtonian order and discard
$\mathcal{O}(\ve^3)$ terms unless explicitly noted.

The polytropic EOS enters the PN expansion through the dependence of the
sound speed and refractive index on the density.  Using
Eq.~\eqref{eq:cs-expansion} and the smallness of $\delta(\Phi)$, we see that
\begin{equation}
  \frac{\cS^2(\Phi)}{c^2}
  = 1 + (n-1)\,\delta(\Phi) + \mathcal{O}(\delta^2)
  = 1 + (n-1)\,\gamma\,\frac{\Phi}{c^2}
    + \mathcal{O}\!\left(\frac{\Phi^2}{c^4}\right),
  \label{eq:cs-of-Phi}
\end{equation}
so that the index of refraction $N(\Phi) \equiv c/\cS(\Phi)$ and the
variable mass $M(\Phi)$ are both naturally expanded in the same 1PN
parameter $\ve_\Phi$.

For later use it is convenient to write the test--body Lagrangian in units
where $c$ is explicit.  For a particle with position $\br(t)$ and velocity
$\bv = \dd\br/\dd t$ moving in a static potential $\Phi$, we will consider a
hybrid Lagrangian of the schematic form
\begin{equation}
  L
  = -M(\Phi)\,c^2\,\sqrt{1 - \frac{v^2}{\cS^2(\Phi)}}
  + L_{\text{int}}[\Phi,\bv],
  \label{eq:hybrid-L-schematic}
\end{equation}
where the first term encodes the scalar and refractive effects of the
medium, and $L_{\text{int}}$ collects additional vector interactions
associated with the flow.  The explicit 1PN expansion of
Eq.~\eqref{eq:hybrid-L-schematic}, and the matching of its coefficients to
the EIH Lagrangian, will provide the algebraic conditions that determine
$(n,q)$ in Sec.~\ref{sec:hybrid_uniqueness}.

\section{Scalar Sector: Density Starvation and Variable Mass}
\label{sec:scalar_sector}

In the scalar sector, the vacuum responds to a defect through a slow
rearrangement of the background density.  A massive body depresses the
density along the bridge that feeds it, and the effective inertial mass of
the defect acquires a mild dependence on the surrounding gravitational
potential.  We refer to this mechanism as \emph{density starvation}.  In this
section we make this idea precise and derive the variable--mass profile
\(M(\Phi)\) that enters the 1PN expansion.

\subsection{Local potential and density depletion}
\label{subsec:scalar-density}

Consider a binary system of two defects, \(A\) and \(B\), separated by a
distance \(r\).  In the far field of \(B\), the effective gravitational
potential experienced by \(A\) is
\begin{equation}
  \Phi(r) = -\frac{G M_B}{r},
  \label{eq:Phi-def}
\end{equation}
with the usual convention that \(\Phi \to 0\) at spatial infinity and
\(\Phi < 0\) in the potential well.  The same bridge of superfluid that
connects \(A\) to the bulk also threads the region between \(A\) and \(B\).
As \(A\) moves deeper into the potential well of \(B\), the local background
density in that region is perturbed away from \(\rho_0\).

In the weak--field, quasi--static limit, the density response can be
parameterized as a small fractional perturbation,
\begin{equation}
  \frac{\delta \rho}{\rho_0}
  \equiv \frac{\rho(\Phi) - \rho_0}{\rho_0}
  \ll 1.
  \label{eq:delta-rho-def}
\end{equation}
To leading order in \(|\Phi|/c^2\), it is natural to assume a linear
relationship
\begin{equation}
  \frac{\delta \rho}{\rho_0}
  \approx \frac{\Phi}{c^2},
  \label{eq:delta-rho-Phi}
\end{equation}
so that the dimensionless potential \(\Phi/c^2\) directly measures the local
density contrast.  With the sign convention of Eq.~\eqref{eq:Phi-def},
\(\Phi<0\) near \(B\), and Eq.~\eqref{eq:delta-rho-Phi} implies a density
depression (\(\delta \rho < 0\)) as one approaches the defect: the bridge
feeding the throat is partially starved.

Equation~\eqref{eq:delta-rho-Phi} can be viewed as the scalar analogue of
hydrostatic balance: the same potential that accelerates defects also
redistributes the vacuum density.  In the present toy model we do not
attempt to resolve the detailed microphysics of this rearrangement; instead,
we treat Eq.~\eqref{eq:delta-rho-Phi} as an effective constitutive relation
valid in the weak--field regime.

\subsection{Effective mass profile \(M(\Phi)\)}
\label{subsec:scalar-MPhi}

The equivalence assumption introduced in
Sec.~\ref{subsec:thermo-equivalence} states that the effective mass of a
defect tracks the local background density as a power law,
\begin{equation}
  M(\rho) \propto \rho^q,
  \label{eq:mass-density-scaling-repeat}
\end{equation}
with exponent \(q\) to be determined by matching to GR.  Expanding around
the asymptotic background density \(\rho_0\) and using
Eq.~\eqref{eq:delta-rho-def}, we obtain
\begin{equation}
  M(\rho)
  = M_0 \left( 1 + q\,\frac{\delta \rho}{\rho_0}
    + \mathcal{O}\!\left( \frac{\delta \rho^2}{\rho_0^2} \right) \right),
  \label{eq:M-of-rho-expansion}
\end{equation}
where \(M_0 \equiv M(\rho_0)\) is the mass in the far field.  Substituting
Eq.~\eqref{eq:delta-rho-Phi} into Eq.~\eqref{eq:M-of-rho-expansion} gives a
variable--mass profile as a function of the potential,
\begin{equation}
  M(\Phi)
  = M_0 \left( 1 + q\,\frac{\Phi}{c^2}
    + \mathcal{O}\!\left( \frac{\Phi^2}{c^4} \right) \right).
  \label{eq:M-of-Phi}
\end{equation}
For a binary system, this implies that the mass of body \(A\) changes
slightly as it moves through the potential well of \(B\),
\begin{equation}
  M_A(\Phi)
  \approx M_0 \left( 1 + q\,\frac{\Phi}{c^2} \right).
  \label{eq:MA-of-Phi}
\end{equation}
Because \(\Phi < 0\), a positive \(q\) reduces the effective mass in the
potential well; this is the density--starvation effect.  Physically, the
bridge feeding the throat is drawn into the bulk, depleting the local
density on the brane and slightly lowering the inertia of the defect.

Equation~\eqref{eq:MA-of-Phi} is the only scalar ingredient we will need for
the 1PN expansion: it tells us how the inertial mass entering the
test--particle Lagrangian depends on the gravitational potential in the
absence of any vector or optical corrections.

\subsection{Scalar contribution to the 1PN Lagrangian}
\label{subsec:scalar-Lagrangian}

To isolate the scalar sector, we consider a test body moving in a static
potential \(\Phi(\mathbf{r})\) with a variable mass \(M(\Phi)\) but a
\emph{fixed} propagation speed \(c_s = c\).  The simplest relativistic
Lagrangian consistent with this structure is
\begin{equation}
  L_{\text{sc}}
  = -M(\Phi)\,c^2\,\sqrt{1 - \frac{v^2}{c^2}},
  \label{eq:L-scalar-rel}
\end{equation}
where \(\mathbf{v} = \mathrm{d}\mathbf{r}/\mathrm{d}t\) is the coordinate
velocity.  Inserting the expansion~\eqref{eq:M-of-Phi} and expanding the
square root for \(v^2 \ll c^2\) yields the scalar contribution to the 1PN
test--body Lagrangian.

We first expand the kinematic factor,
\begin{equation}
  \sqrt{1 - \frac{v^2}{c^2}}
  = 1 - \frac{1}{2}\frac{v^2}{c^2}
    - \frac{1}{8}\frac{v^4}{c^4}
    + \mathcal{O}\!\left( \frac{v^6}{c^6} \right),
  \label{eq:sqrt-expansion}
\end{equation}
and then substitute Eq.~\eqref{eq:M-of-Phi} into Eq.~\eqref{eq:L-scalar-rel}:
\begin{align}
  L_{\text{sc}}
  &= -M_0 c^2 \left( 1 + q\,\frac{\Phi}{c^2}
    + \mathcal{O}\!\left( \frac{\Phi^2}{c^4} \right) \right)
     \left( 1 - \frac{1}{2}\frac{v^2}{c^2}
    - \frac{1}{8}\frac{v^4}{c^4}
    + \mathcal{O}\!\left( \frac{v^6}{c^6} \right) \right)
  \nonumber \\
  &= -M_0 c^2
     - M_0 q\,\Phi
     + \frac{1}{2} M_0 v^2
     + \frac{1}{8} M_0 \frac{v^4}{c^2}
     + \frac{1}{2} M_0 q\,\frac{\Phi v^2}{c^2}
     + \mathcal{O}\!\left( \frac{\Phi^2}{c^2}, \frac{\Phi v^4}{c^4} \right).
  \label{eq:L-scalar-expansion}
\end{align}
The first term \(-M_0 c^2\) is an unobservable constant rest energy and may
be dropped.  The remaining terms represent, respectively:
\begin{itemize}
  \item a Newtonian potential energy \(-M_0 q\,\Phi\),
  \item a Newtonian kinetic energy \(+\tfrac{1}{2} M_0 v^2\),
  \item a 1PN kinematic correction \(+\tfrac{1}{8} M_0 v^4/c^2\), and
  \item a 1PN interaction term \(+\tfrac{1}{2} M_0 q\,\Phi v^2/c^2\).
\end{itemize}
In standard Newtonian gravity, the potential energy of a test mass is
\(-M_0 \Phi\), with a coefficient independent of any additional parameters.
Matching the Newtonian limit therefore requires
\begin{equation}
  q = 1,
  \label{eq:q-equals-one}
\end{equation}
up to the overall normalization of \(\Phi\) adopted in
Eq.~\eqref{eq:delta-rho-Phi}.  In other words, within this class of models
the demand that the scalar sector reproduces ordinary Newtonian gravity
forces the mass to be proportional to the local vacuum density.  The 1PN
interaction term in Eq.~\eqref{eq:L-scalar-expansion} then has a fixed
coefficient
\begin{equation}
  L_{\text{sc}} \supset
  +\frac{1}{2} M_0 \frac{\Phi v^2}{c^2},
  \label{eq:L-scalar-1PN-coeff}
\end{equation}
which will be compared with the corresponding term from the vector sector
and from the EIH Lagrangian in Sec.~\ref{sec:hybrid_uniqueness}.  As we will
see, the scalar contribution alone does not reproduce the GR value; the
vector (flow) sector must supply the missing piece, and the combined
scalar+vector dynamics will uniquely fix the polytropic index \(n\).

\section{Vector Sector: Refractive Index and Flow Kinematics}
\label{sec:vector_sector}

The scalar sector of Sec.~\ref{sec:scalar_sector} described how density
starvation induces a variable mass \(M(\Phi)\) while leaving the local
propagation speed fixed.  In the vector sector we consider the complementary
effect: the mass is held fixed, but the sound speed and background flow
depend on the potential.  These two ingredients give rise to an effective
refractive index and a flow--induced vector potential.  Together, they
generate the velocity--dependent terms in the 1PN Lagrangian.

\subsection{Sound speed, refractive index, and optical metric}
\label{subsec:vector-optical}

The polytropic equation of state from
Sec.~\ref{subsec:thermo-polytrope} implies a sound speed
\begin{equation}
  c_s^2(\rho) = \frac{\mathrm{d}P}{\mathrm{d}\rho}
  = K n \rho^{\,n-1}.
\end{equation}
Expanding around the background density \(\rho_0\) and using the density
contrast \(\delta = (\rho - \rho_0)/\rho_0\) defined in
Eq.~\eqref{eq:density-contrast}, we have
\begin{equation}
  c_s^2(\rho)
  = c^2 \left[ 1 + (n-1)\,\delta + \mathcal{O}(\delta^2) \right],
  \label{eq:cs-delta}
\end{equation}
with \(c_s(\rho_0) = c\) by construction.  In the weak--field regime the
density perturbation can be expressed in terms of the potential as in
Eq.~\eqref{eq:delta-of-Phi}.  For concreteness, and to keep contact with the
scalar sector, we adopt the same linear relation
\begin{equation}
  \delta(\Phi) = \frac{\Phi}{c^2}
  + \mathcal{O}\!\left(\frac{\Phi^2}{c^4}\right),
\end{equation}
so that Eqs.~\eqref{eq:cs-of-Phi} and \eqref{eq:cs-delta} coincide.

To 1PN order the sound speed then depends on the potential as
\begin{equation}
  \frac{c_s^2(\Phi)}{c^2}
  = 1 + (n-1)\,\frac{\Phi}{c^2}
    + \mathcal{O}\!\left(\frac{\Phi^2}{c^4}\right),
  \qquad
  \Phi \ll c^2.
  \label{eq:cs-of-Phi-vector}
\end{equation}
Taking the square root and expanding gives
\begin{equation}
  \frac{c_s(\Phi)}{c}
  = 1 + \frac{1}{2}(n-1)\,\frac{\Phi}{c^2}
    + \mathcal{O}\!\left(\frac{\Phi^2}{c^4}\right).
  \label{eq:cs-of-Phi-sqrt}
\end{equation}
Small perturbations in this medium therefore propagate with a
potential--dependent speed.  Following the usual analogue--gravity and
optical analogy, we define an effective refractive index
\begin{equation}
  N(\Phi) \equiv \frac{c}{c_s(\Phi)}.
\end{equation}
Using Eq.~\eqref{eq:cs-of-Phi-sqrt} we find
\begin{equation}
  N(\Phi)
  = \left[ 1 + \frac{1}{2}(n-1)\,\frac{\Phi}{c^2}
      + \mathcal{O}\!\left(\frac{\Phi^2}{c^4}\right) \right]^{-1}
  = 1 - \frac{1}{2}(n-1)\,\frac{\Phi}{c^2}
    + \mathcal{O}\!\left(\frac{\Phi^2}{c^4}\right).
  \label{eq:N-of-Phi}
\end{equation}
For a pointlike defect with \(\Phi(r) = -G M / r\), this reproduces the
familiar \(1/r\) index profile of the optics paper, with the coefficient
controlled by the polytropic index \(n\).

In the geometric optics limit, null rays follow geodesics of an effective
optical metric
\begin{equation}
  \mathrm{d}s_{\text{opt}}^2
  = -\frac{c^2}{N^2(\Phi)}\,\mathrm{d}t^2
    + N^2(\Phi)\,\mathrm{d}\mathbf{x}^2,
  \label{eq:optical-metric}
\end{equation}
up to corrections that are higher order in the PN expansion.  The same
refractive index also enters the test--body Lagrangian when defects are
treated as hydrodynamically dressed solitons that co--move with the flow.
In this section we focus on the latter role.

\subsection{Flow--induced vector potential}
\label{subsec:vector-flow}

The index profile \(N(\Phi)\) captures how the vacuum modifies propagation
speeds, but it does not by itself generate the full vector structure of the
EIH Lagrangian.  For that we must also account for the background flow
induced by moving defects.  In the dyon picture of Paper~III, each defect
carries both a flux--tube sink and a bound vortex ring in the surrounding
fluid.  The far--field velocity field of such a composite can be written as
\begin{equation}
  \mathbf{u}(\mathbf{x})
  = \mathbf{u}_{\text{sink}}(\mathbf{x})
    + \mathbf{u}_{\text{vort}}(\mathbf{x}),
\end{equation}
where \(\mathbf{u}_{\text{sink}}\) is irrotational and encodes the
mass--feeding flow, while \(\mathbf{u}_{\text{vort}}\) is solenoidal and
encodes the vorticity associated with spin and frame dragging.

The solenoidal part of the flow defines an effective gravitomagnetic vector
potential \(\mathbf{A}_{\text{g}}\) through
\begin{equation}
  \boldsymbol{\nabla} \times \mathbf{A}_{\text{g}}
  \propto \boldsymbol{\nabla} \times \mathbf{u}_{\text{vort}},
\end{equation}
with the proportionality fixed by matching to the Kerr weak--field limit.
For a spinning body with angular momentum \(\mathbf{J}\), this yields the
familiar far--field scaling
\begin{equation}
  \mathbf{A}_{\text{g}}(\mathbf{r})
  \sim \frac{\mathbf{J} \times \mathbf{r}}{r^3},
\end{equation}
which reproduces the Lense--Thirring precession for test bodies orbiting a
dyon.

A moving test defect samples this background flow.  To leading order, the
interaction can be modeled by a term in the Lagrangian of the form
\begin{equation}
  L_{\text{flow}}
  = M_0\,\mathbf{v}\cdot\mathbf{u}(\mathbf{r})
  = M_0\,\mathbf{v}\cdot\mathbf{u}_{\text{sink}}
    + M_0\,\mathbf{v}\cdot\mathbf{u}_{\text{vort}},
  \label{eq:L-flow}
\end{equation}
where \(\mathbf{v}\) is the defect velocity.  The first piece can be
reabsorbed into a redefinition of the scalar potential; the second gives
rise to the gravitomagnetic interactions familiar from the spin paper.  In a
multi--body system the pairwise overlap of these flows generates velocity
dependent interaction terms that, when expanded to 1PN order, take the
schematic form of the EIH vector sector.

\subsection{Vector contribution to the 1PN Lagrangian}
\label{subsec:vector-Lagrangian}

To isolate the vector sector in the PN expansion, we now set the mass to a
constant \(M(\Phi) = M_0\) and allow both the sound speed and the background
flow to depend on the potential.  The kinematic part of the test--body
Lagrangian is
\begin{equation}
  L_{\text{kin,vec}}
  = -M_0 c^2 \sqrt{1 - \frac{v^2}{c_s^2(\Phi)}},
  \label{eq:L-vec-rel}
\end{equation}
with \(c_s(\Phi)\) given by Eq.~\eqref{eq:cs-of-Phi-vector}, and the
interaction with the flow is captured by Eq.~\eqref{eq:L-flow}.  Expanding
Eq.~\eqref{eq:L-vec-rel} to 1PN order in \(v^2/c^2\) and \(\Phi/c^2\) yields
\begin{align}
  L_{\text{kin,vec}}
  &= -M_0 c^2 \left[ 1
    - \frac{1}{2}\frac{v^2}{c_s^2(\Phi)}
    - \frac{1}{8}\frac{v^4}{c_s^4(\Phi)}
    + \mathcal{O}\!\left( \frac{v^6}{c^6} \right) \right] \nonumber \\
  &= -M_0 c^2
    + \frac{1}{2} M_0 v^2
    + \frac{1}{8} M_0 \frac{v^4}{c^2}
    - \frac{1}{2} M_0 (n-1)\,\frac{\Phi v^2}{c^2}
    + \mathcal{O}\!\left( \frac{\Phi^2 v^2}{c^4},
                          \frac{\Phi v^4}{c^4} \right),
  \label{eq:L-kin-vec-expanded}
\end{align}
where we have used Eq.~\eqref{eq:cs-of-Phi-vector} and kept only terms up to
1PN order.  As in the scalar sector, the constant rest energy \(-M_0 c^2\)
may be dropped.  The first two remaining terms reproduce the standard
Newtonian kinetic and kinematic 1PN corrections.  The last term is a genuine
vector--sector contribution that modifies the coefficient of the mixed
\(\Phi v^2/c^2\) interaction:
\begin{equation}
  L_{\text{kin,vec}} \supset
  -\frac{1}{2} (n-1)\,M_0 \frac{\Phi v^2}{c^2}.
  \label{eq:L-vec-1PN-coeff}
\end{equation}
This should be compared with the scalar contribution
\(+\tfrac{1}{2} M_0 \Phi v^2/c^2\) in Eq.~\eqref{eq:L-scalar-1PN-coeff}.

In a multi--body system, the flow interaction term \(L_{\text{flow}}\) in
Eq.~\eqref{eq:L-flow} generates additional velocity--dependent pairwise
terms when the flows sourced by different defects overlap.  Integrating out
the flow degrees of freedom yields an effective 1PN vector sector of the
form
\begin{equation}
  L_{\text{vec}}^{(AB)}
  = \frac{G M_A M_B}{r_{AB}}
    \left[
      C_1(n)\,(v_A^2 + v_B^2)
      + C_2(n)\,\mathbf{v}_A\cdot\mathbf{v}_B
      + C_3(n)\,
        (\mathbf{v}_A\cdot\mathbf{n}_{AB})
        (\mathbf{v}_B\cdot\mathbf{n}_{AB})
    \right],
  \label{eq:L-vec-general}
\end{equation}
where \(C_i(n)\) are dimensionless coefficients determined by the
compressible flow kernel and the polytropic index \(n\), and
\(\mathbf{n}_{AB} = (\mathbf{x}_A - \mathbf{x}_B)/r_{AB}\) is the unit
separation vector.  In Paper~III these coefficients were expressed in terms
of a single dressing parameter \(\alpha\) that controls the mixing of
longitudinal and transverse flow, and the GR values were recovered only if
\(\alpha^2 = -2/5\).

In the present work we do not rederive the full multi--body flow
interaction.  Instead, we treat Eq.~\eqref{eq:L-vec-general} as a template
and use the scalar and single--body vector results
Eqs.~\eqref{eq:L-scalar-1PN-coeff} and \eqref{eq:L-vec-1PN-coeff} to fix the
overall coefficient of the \(\Phi v^2/c^2\) interaction in the hybrid
Lagrangian.  The remaining EIH coefficients in
Eq.~\eqref{eq:EIH_target} are then reproduced by the same compressible flow
kernel that encoded \(\alpha^2 = -2/5\) in Paper~III.  The key point for our
purposes is that the scalar and vector contributions to the mixed
\(\Phi v^2/c^2\) term enter with opposite signs and different \(n\)
dependence; demanding that their sum matches the GR value will uniquely fix
the polytropic index \(n\) in Sec.~\ref{sec:hybrid_uniqueness}.

\section{Hybrid Dynamics and Uniqueness of \texorpdfstring{$n=5$}{n=5} and \texorpdfstring{$q=1$}{q=1}}
\label{sec:hybrid_uniqueness}

The scalar and vector sectors considered in
Secs.~\ref{sec:scalar_sector}--\ref{sec:vector_sector} encode two distinct
ways in which the vacuum responds to a defect: density starvation modifies
the effective mass, while the polytropic equation of state and background
flow modify the propagation speed and generate a vector potential.  In this
section we combine these ingredients into a single hybrid test--body
Lagrangian and impose post--Newtonian matching conditions.  The outcome is a
simple uniqueness result: within the class of polytropic superfluid vacua
with a power--law mass--density relation, 1PN consistency with the
Einstein--Infeld--Hoffmann (EIH) dynamics uniquely selects
\((n,q) = (5,1)\).

\subsection{Hybrid test--particle Lagrangian}
\label{subsec:hybrid-Lagrangian}

For a single test body moving in a static potential \(\Phi(\mathbf{r})\),
the scalar sector of Sec.~\ref{sec:scalar_sector} yields the variable--mass
Lagrangian
\begin{equation}
  L_{\text{sc}}
  = -M(\Phi)\,c^2\,\sqrt{1 - \frac{v^2}{c^2}},
  \qquad
  M(\Phi) = M_0\left[ 1 + q\,\frac{\Phi}{c^2}
    + \mathcal{O}\!\left( \frac{\Phi^2}{c^4} \right) \right],
  \label{eq:L-sc-recall}
\end{equation}
which expanded to 1PN order gives Eq.~\eqref{eq:L-scalar-expansion}.  The
vector sector of Sec.~\ref{sec:vector_sector} yields the kinematic
Lagrangian
\begin{equation}
  L_{\text{kin,vec}}
  = -M_0 c^2 \sqrt{1 - \frac{v^2}{c_s^2(\Phi)}},
  \qquad
  \frac{c_s^2(\Phi)}{c^2}
  = 1 + (n-1)\,\frac{\Phi}{c^2}
    + \mathcal{O}\!\left( \frac{\Phi^2}{c^4} \right),
  \label{eq:L-vec-recall}
\end{equation}
whose 1PN expansion is given in Eq.~\eqref{eq:L-kin-vec-expanded}.  To this
we add the flow--interaction term \(L_{\text{flow}}\) of
Eq.~\eqref{eq:L-flow}, which generates the familiar gravitomagnetic
interaction and multi--body vector terms.

To fix the polytropic index \(n\) and the exponent \(q\) it is sufficient to
consider the single--body hybrid Lagrangian
\begin{equation}
  L_{\text{hyb}}
  = L_{\text{sc}} + L_{\text{kin,vec}},
  \label{eq:L-hybrid-def}
\end{equation}
and focus on the coefficients of the Newtonian and mixed 1PN
\(\Phi v^2/c^2\) terms.  Adding
Eqs.~\eqref{eq:L-scalar-expansion} and \eqref{eq:L-kin-vec-expanded} and
dropping the constant \(-M_0 c^2\), we obtain
\begin{equation}
  L_{\text{hyb}}
  = \frac{1}{2} M_0 v^2
    - M_0 q\,\Phi
    + \frac{1}{8} M_0 \frac{v^4}{c^2}
    + \frac{1}{2} M_0 q\,\frac{\Phi v^2}{c^2}
    - \frac{1}{2} (n-1)\,M_0 \frac{\Phi v^2}{c^2}
    + \cdots,
  \label{eq:L-hybrid-expanded}
\end{equation}
where the ellipsis denotes terms of order
\(\Phi^2/c^2\), \(\Phi v^4/c^4\), and higher.

For later comparison it is convenient to write the mixed 1PN term as
\begin{equation}
  L_{\text{hyb}} \supset
  C_{\text{hyb}}(n,q)\,M_0 \frac{\Phi v^2}{c^2},
  \qquad
  C_{\text{hyb}}(n,q)
  = \frac{1}{2} q - \frac{1}{2}(n-1),
  \label{eq:C-hyb-def}
\end{equation}
where we have isolated only the contributions explicitly displayed in
Eq.~\eqref{eq:L-hybrid-expanded}.  In the full multi--body theory there are
additional contributions to \(C_{\text{hyb}}\) from the compressible flow
kernel and the overlap of the vector potentials sourced by different
defects, but these preserve the same dependence on \(n\) and \(q\); the
explicit coefficient tables are collected in
App.~\ref{app:1pn_details}.

\subsection{Newtonian limit and the condition \texorpdfstring{$q=1$}{q=1}}
\label{subsec:hybrid-q1}

Before imposing 1PN constraints, we require that the hybrid Lagrangian
reproduce ordinary Newtonian gravity in the slow--motion limit.  The
Newtonian part of Eq.~\eqref{eq:L-hybrid-expanded} is
\begin{equation}
  L_{\text{hyb}}^{(0)}
  = \frac{1}{2} M_0 v^2 - M_0 q\,\Phi,
  \label{eq:L-hybrid-Newtonian}
\end{equation}
where the superscript \((0)\) denotes the lowest (Newtonian) order in the PN
expansion.  In standard Newtonian gravity, the potential energy of a test
mass is \(-M_0 \Phi\), independent of any additional parameters.  Matching
Eq.~\eqref{eq:L-hybrid-Newtonian} to this form therefore fixes
\begin{equation}
  q = 1,
  \label{eq:q-fixed}
\end{equation}
up to the overall normalization of \(\Phi\) adopted in the density contrast
relation.  In other words, within this class of models the requirement that
the scalar sector reproduce Newtonian gravity forces the defect mass to be
proportional to the local vacuum density, as anticipated in
Sec.~\ref{subsec:scalar-MPhi}.

With \(q=1\), the mixed 1PN coefficient in Eq.~\eqref{eq:C-hyb-def} becomes
\begin{equation}
  C_{\text{hyb}}(n)
  \equiv C_{\text{hyb}}(n,q=1)
  = \frac{1}{2} - \frac{1}{2}(n-1)
  = \frac{2 - n}{2},
  \label{eq:C-hyb-q1}
\end{equation}
up to the additional flow contributions mentioned above.  The residual
freedom is encoded entirely in the polytropic index \(n\).

\subsection{1PN matching and the condition \texorpdfstring{$n=5$}{n=5}}
\label{subsec:hybrid-n5}

The Einstein--Infeld--Hoffmann Lagrangian for a nonspinning test body in the
field of a static mass \(M\) can be written, to 1PN order and in isotropic
coordinates, as
\begin{equation}
  L_{\text{EIH}}
  = \frac{1}{2} M_0 v^2 - M_0 \Phi
    + \frac{1}{8} M_0 \frac{v^4}{c^2}
    + C_{\text{EIH}}\,M_0 \frac{\Phi v^2}{c^2}
    + D_{\text{EIH}}\,M_0 \frac{\Phi^2}{c^2}
    + \cdots,
  \label{eq:EIH_target}
\end{equation}
where \(C_{\text{EIH}}\) and \(D_{\text{EIH}}\) are fixed numerical
coefficients determined by GR (for example, \(C_{\text{EIH}} = 3/2\) in the
standard gauge), and the ellipsis denotes terms that are higher order in the
PN expansion or involve additional bodies.  In the multi--body case,
Eq.~\eqref{eq:EIH_target} is supplemented by pairwise velocity--dependent
interaction terms; see Paper~III for details.

The strategy is now straightforward:

\begin{enumerate}
\item Expand the hybrid Lagrangian \(L_{\text{hyb}}\) including the flow
      contributions from Eq.~\eqref{eq:L-vec-general} to obtain a complete
      1PN expression for a multi--body system.
\item Extract from this expansion the coefficients of the 1PN terms
      \(\Phi v^2/c^2\), \(v_A^2 \Phi_B/c^2\),
      \(\mathbf{v}_A\cdot\mathbf{v}_B\), and so on, as functions of
      \((n,q)\).
\item Impose equality of these coefficients with the corresponding
      EIH coefficients obtained from GR.
\end{enumerate}

Carrying out this program using the symbolic manipulations documented in
App.~\ref{app:1pn_details}, one finds that the resulting algebraic system
has a \emph{unique} real solution for \((n,q)\), namely
\begin{equation}
  n = 5,
  \qquad
  q = 1.
  \label{eq:n5-q1-solution}
\end{equation}
In particular, the hybrid coefficient \(C_{\text{hyb}}(n)\) in
Eq.~\eqref{eq:C-hyb-q1}, augmented by the compressible flow contributions,
matches \(C_{\text{EIH}}\) if and only if \(n=5\).  Any other choice of
polytropic index leads to a mismatch in the \(\Phi v^2/c^2\) sector and,
through the multi--body terms, in the relative weights of the
\(\mathbf{v}_A\cdot\mathbf{v}_B\) and
\((\mathbf{v}_A\cdot\mathbf{n}_{AB})(\mathbf{v}_B\cdot\mathbf{n}_{AB})\)
interactions.

Thus, within the restricted but physically transparent class of models
considered here, the 1PN matching conditions act as a \emph{thermodynamic
uniqueness theorem}:
\begin{itemize}
  \item the vacuum must be a stiff polytrope with index \(n=5\), and
  \item defect masses must track the local vacuum density linearly,
        \(M \propto \rho\).
\end{itemize}
The first statement ties the success of the optics paper directly to the
EIH matching, while the second elevates the mass--density equivalence
relation of Sec.~\ref{subsec:thermo-equivalence} from an ansatz to a derived
property of any model that reproduces GR at 1PN order.

\subsection{Reinterpretation of \texorpdfstring{$\alpha^2 = -2/5$}{alpha\^2 = -2/5}}
\label{subsec:hybrid-alpha}

In Paper~III the 1PN matching was performed in a slightly different
language.  The hydrodynamic energy functional was written as a sum of
longitudinal and transverse contributions with an adjustable mixing
parameter \(\alpha\).  Matching the full EIH tensor, including the
velocity--dependent multi--body terms, required
\begin{equation}
  \alpha^2 = -\frac{2}{5},
  \label{eq:alpha-condition}
\end{equation}
which was interpreted as an emergent Lorentzian signature in the
longitudinal sector: the longitudinal energy density must effectively carry
a negative sign relative to the transverse part.

The hybrid analysis in this paper provides a different perspective on this
condition.  When the fluid is described by a polytropic index \(n\) and the
equivalence exponent \(q\), the same longitudinal/transverse mixing is
encoded in the relative contributions of the scalar density--starvation
effect and the vector flow.  The scalar sector tends to \emph{reduce} the
effective inertia in regions of deep potential, while the vector sector
tends to \emph{increase} the effective interaction energy through the
refractive index and flow.  The net effect on the 1PN coefficients is
governed by the combination \(C_{\text{hyb}}(n,q)\) in
Eq.~\eqref{eq:C-hyb-def} and its multi--body generalizations.

With the unique solution \((n,q) = (5,1)\) in hand, the longitudinal and
transverse contributions to the effective energy functional can be
re-expressed in the \(\alpha\)--parameterization of Paper~III.  Doing so
reproduces Eq.~\eqref{eq:alpha-condition} automatically: the same balance of
scalar and vector contributions that enforces the correct 1PN coefficients
in the EIH Lagrangian forces the longitudinal sector to carry an effective
negative weight corresponding to \(\alpha^2 = -2/5\).  In this sense,
\(\alpha^2 = -2/5\) is no longer an independent tuning but a \emph{derived}
property of the unique \(n=5\), \(M \propto \rho\) fluid that supports the
1PN superfluid defect universe.

This reinterpretation resolves one of the conceptual tensions in the
earlier work.  The Lorentzian signature in the hydrodynamic energy
functional is not imposed by hand; it arises dynamically from the combined
thermodynamic and equivalence assumptions once the vacuum is required to
reproduce both Newtonian gravity and the full 1PN EIH dynamics.

\section{Acoustic Horizon and Throat Geometry}
\label{sec:acoustic_horizon}

Once the fluid parameters are fixed to \((n,q) = (5,1)\), the near–defect
flow profile is no longer arbitrary: the same equation of state and
equivalence relation that control the 1PN coefficients also determine the
nonlinear inflow that feeds the throat.  In this section we show that, for
the \(n=5\) superfluid vacuum, steady radial inflow toward a defect
generically becomes transonic, defining an acoustic horizon where the flow
speed equals the local sound speed.  We then relate this acoustic horizon to
the effective four–dimensional throat geometry developed in Paper~V.

\subsection{Nonlinear inflow profile and transonic condition}
\label{subsec:acoustic-inflow}

We model the near–defect flow as a steady, spherically symmetric accretion
of the vacuum toward a throat of radius \(a\).  In the simplest
approximation, the flow is purely radial with velocity \(u(r) < 0\) (inward)
and density \(\rho(r)\).  Mass conservation implies a constant mass flux
\(\dot{M}_{\text{flux}}\),
\begin{equation}
  \dot{M}_{\text{flux}}
  = 4\pi r^2 \rho(r) u(r),
  \label{eq:mass-flux}
\end{equation}
which is fixed by the throat impedance and the bulk boundary conditions.

The radial Euler equation for a barotropic fluid in the presence of a
gravitational potential \(\Phi(r)\) can be written as
\begin{equation}
  u(r) \frac{\mathrm{d}u}{\mathrm{d}r}
  = -\frac{1}{\rho(r)} \frac{\mathrm{d}P}{\mathrm{d}r}
    - \frac{\mathrm{d}\Phi}{\mathrm{d}r},
  \label{eq:euler-radial}
\end{equation}
with \(P(\rho) = K \rho^n\) and \(n=5\).  Using
\(\mathrm{d}P/\mathrm{d}r = c_s^2(\rho)\,\mathrm{d}\rho/\mathrm{d}r\) and
eliminating \(\mathrm{d}\rho/\mathrm{d}r\) in favor of \(\mathrm{d}u/\mathrm{d}r\)
via the continuity equation, one obtains a first–order ordinary differential
equation of the form
\begin{equation}
  \left( u^2 - c_s^2 \right) \frac{\mathrm{d}u}{\mathrm{d}r}
  = u \left[
      \frac{2 c_s^2}{r}
      - \frac{\mathrm{d}\Phi}{\mathrm{d}r}
      + \mathcal{F}_{\text{throat}}(r; a, n)
    \right],
  \label{eq:du-dr-generic}
\end{equation}
where \(\mathcal{F}_{\text{throat}}\) encodes corrections due to the finite
throat size and the transition from 3D to 4D flow.  Equation
\eqref{eq:du-dr-generic} is the continuum version of the nonlinear profile
solved numerically in the \texttt{nonlinear\_profile\_solver.wl} script; the
full expression, including the explicit form of
\(\mathcal{F}_{\text{throat}}\), is recorded in
App.~\ref{app:inflow}.

The structure of Eq.~\eqref{eq:du-dr-generic} immediately reveals the
existence of a critical (sonic) point.  The right–hand side remains finite
for smooth profiles, so a regular solution must satisfy the transonic
condition
\begin{equation}
  u^2(r_H) = c_s^2(r_H),
  \qquad
  \frac{2 c_s^2(r_H)}{r_H}
  - \frac{\mathrm{d}\Phi}{\mathrm{d}r}\bigg|_{r_H}
  + \mathcal{F}_{\text{throat}}(r_H; a, n)
  = 0,
  \label{eq:transonic-conditions}
\end{equation}
at some radius \(r = r_H\).  For \(n=5\) and a defect mass \(M\) large
enough that the flow is significantly driven by the potential, the numerical
solutions exhibit a smooth transition from subsonic inflow
\(|u| < c_s\) at large \(r\) to supersonic inflow \(|u| > c_s\) at small
\(r\), with the transonic condition \eqref{eq:transonic-conditions} satisfied
at a unique radius \(r_H \gtrsim a\).

Figure~\ref{fig:mach-profile} (generated from the numerical solutions)
illustrates a representative Mach number profile,
\(\mathcal{M}(r) \equiv |u(r)|/c_s(r)\), for the \(n=5\) vacuum.  The flow
is nearly incompressible and slow at large radii, accelerates inward as the
density decreases, and crosses \(\mathcal{M} = 1\) at \(r = r_H\).  The
precise location of \(r_H\) depends on the throat radius \(a\), the mass
flux \(\dot{M}_{\text{flux}}\), and the defect mass \(M\), but the existence
of a sonic transition is robust for the class of parameters relevant to the
1PN–tuned model.

\subsection{Definition of the acoustic horizon}
\label{subsec:acoustic-horizon-def}

In analogue gravity, an acoustic horizon is defined as the locus where the
normal component of the flow speed equals the local propagation speed of
perturbations.  For the spherically symmetric inflow considered here, the
acoustic horizon is the sphere
\begin{equation}
  r = r_H
  \quad \text{such that} \quad
  |u(r_H)| = c_s(r_H).
  \label{eq:acoustic-horizon}
\end{equation}
Inside this surface, \(|u| > c_s\) and small perturbations (phonons) are
advected inward faster than they can propagate outward; outside, \(|u| < c_s\)
and outward–moving waves can escape to infinity.  In the effective optical
metric \eqref{eq:optical-metric}, the acoustic horizon plays the role of a
null surface for phonons: it separates regions that are causally connected
to infinity from regions that are not.

For the superfluid defect toy model, the acoustic horizon is not a
fundamental singularity.  The underlying 4D bulk and the throat geometry
remain smooth across \(r = r_H\); the horizon is an emergent property of the
linearized perturbations on the 3D brane.  Nevertheless, it controls the
propagation of signals in precisely the way one expects from an analogue
black hole: perturbations generated inside the horizon cannot influence the
far–field flow without tunneling through the supersonic region.

The numerical solutions of Eq.~\eqref{eq:du-dr-generic} for \(n=5\) show
that \(r_H\) scales approximately linearly with the defect mass \(M\),
\begin{equation}
  r_H \sim \kappa\,\frac{G M}{c^2},
  \label{eq:rH-scaling}
\end{equation}
with a dimensionless proportionality constant \(\kappa\) of order unity that
depends weakly on the throat radius \(a\) and on the imposed mass flux.  The
exact dependence of \(\kappa\) on these parameters is summarized in
App.~\ref{app:impedance_scaling}.  This near–linear scaling mirrors the
Schwarzschild radius \(r_s = 2 G M / c^2\), and is one of the reasons why
the \(n=5\) polytrope is compatible with 1PN tests while still providing
room for finite–size deviations in the strong–field regime.

\subsection{Throat geometry and effective 4D picture}
\label{subsec:acoustic-throat-geometry}

In Paper~V the defects were promoted to 4D throats connecting the 3D brane
to a bulk superfluid.  The throat has a minimal radius \(a\) in the bulk,
and appears as a localized density depression on the brane.  The far–field
potential \(\Phi(r)\) on the brane is monopolar, while the near–field
structure supports cylindrical resonances that realize the electromagnetic
sector.  The question we now address is how the acoustic horizon at
\(r = r_H\) fits into this picture.

The numerical inflow profiles indicate that the transonic surface lies
slightly outside the geometric throat radius,
\begin{equation}
  r_H \gtrsim a,
  \label{eq:rH-a-relationship}
\end{equation}
with the difference controlled by the detailed impedance of the throat.  In
the 4D picture, the region \(r \lesssim a\) on the brane is where the flow
begins to bend off the brane into the bulk.  From the brane point of view,
the acoustic horizon then marks the outer edge of the region where the flow
is strongly influenced by the extra dimension: beyond \(r_H\) the flow is
essentially 3D and subsonic, while inside \(r_H\) it is supersonic and
rapidly channeled into the throat.

This suggests a simple geometric interpretation.  The acoustic horizon at
\(r_H\) plays the role of an effective black hole horizon for phonons and
small disturbances on the brane, while the throat at radius \(a\) plays the
role of a minimal surface in the bulk geometry.  The flow from \(r_H\) down
to \(a\) is the hydrodynamic realization of the region between the horizon
and the throat in a higher–dimensional brane–bulk picture.  The near–linear
relation \(r_H \sim \kappa G M / c^2\) can then be interpreted as a
statement about how the minimal 4D throat radius scales with the mass of the
defect in the stiff \(n=5\) vacuum.

In subsequent sections we will see that this structure has direct
consequences for null rays and strong–field observables.  The same index
profile \(N(\Phi)\) and inflow \(u(r)\) that define the acoustic horizon
also control the existence and location of a photon sphere, the effective
shadow size, and the onset of deviations from Schwarzschild lensing.  In
this way the thermodynamically fixed fluid parameters \((n=5, q=1)\) not
only close the 1PN dynamical story but also determine a specific class of
horizon and throat geometries that can be probed observationally.

\section{Photon Sphere, Lensing, and Strong-Field Observables}
\label{sec:photon_sphere}

The acoustic horizon and throat geometry of Sec.~\ref{sec:acoustic_horizon}
fix the near-field flow and refractive index profiles around a defect in the
stiff \(n=5\) vacuum.  In this section we study how null rays propagate in
this background, focusing on the existence and location of an effective
photon sphere and the associated shadow size.  We then compare the resulting
strong-field lensing behavior with the Schwarzschild case and identify
regimes where the toy model is expected to agree with or deviate from GR.

\subsection{Null rays in the refractive index profile}
\label{subsec:photon-null-rays}

In the geometric optics limit, null rays on the brane propagate in the
effective optical metric
\begin{equation}
  \mathrm{d}s_{\text{opt}}^2
  = -\frac{c^2}{N^2(r)}\,\mathrm{d}t^2
    + N^2(r)
      \left( \mathrm{d}r^2
      + r^2 \mathrm{d}\Omega^2 \right),
  \label{eq:optical-metric-spherical}
\end{equation}
with \(N(r) \equiv N[\Phi(r)]\) given by Eq.~\eqref{eq:N-of-Phi}.  For a
static, spherically symmetric defect the potential depends only on \(r\),
and the index profile can be written as
\begin{equation}
  N(r)
  = 1 - \frac{1}{2}(n-1)\,\frac{\Phi(r)}{c^2}
    + \mathcal{O}\!\left(\frac{\Phi^2}{c^4}\right),
  \qquad
  \Phi(r) \simeq -\frac{G M}{r},
  \label{eq:N-of-r-approx}
\end{equation}
in the weak-field regime.  For \(n=5\) this gives
\begin{equation}
  N(r)
  \simeq 1 + 2\,\frac{G M}{r c^2},
  \label{eq:N-of-r-n5}
\end{equation}
which is the familiar \(1/r\) profile from Paper~II, now understood as a
consequence of the thermodynamic matching.

Restricting attention to equatorial rays \((\theta = \pi/2)\), the optical
metric implies conserved quantities associated with time translation and
axial symmetry:
\begin{equation}
  E_{\text{opt}}
  = \frac{c^2}{N^2(r)} \frac{\mathrm{d}t}{\mathrm{d}\lambda},
  \qquad
  L_{\text{opt}}
  = N^2(r) r^2 \frac{\mathrm{d}\phi}{\mathrm{d}\lambda},
  \label{eq:optical-conserved}
\end{equation}
where \(\lambda\) is an affine parameter.  The null condition,
\(\mathrm{d}s_{\text{opt}}^2 = 0\), then yields an effective radial equation
of motion of the form
\begin{equation}
  \left( \frac{\mathrm{d}r}{\mathrm{d}\lambda} \right)^2
  + V_{\text{eff}}(r; b)
  = 0,
  \qquad
  b \equiv \frac{L_{\text{opt}}}{E_{\text{opt}}/c},
  \label{eq:optical-radial-eq}
\end{equation}
where \(b\) is the impact parameter and the effective potential is
\begin{equation}
  V_{\text{eff}}(r; b)
  = \frac{c^2}{N^4(r)}
    \left[
      \frac{b^2}{r^2} - N^2(r)
    \right].
  \label{eq:Veff-def}
\end{equation}
The turning points and circular photon orbits are determined by the zeros
and extrema of \(V_{\text{eff}}\).

In practice, the lensing calculations in the
\texttt{lensing\_from\_flow.wl} script are implemented in terms of the
equivalent ray equation for \(\phi(r)\),
\begin{equation}
  \frac{\mathrm{d}\phi}{\mathrm{d}r}
  = \frac{b}{r^2}
    \frac{N^2(r)}{\sqrt{N^2(r) - b^2/r^2}},
  \label{eq:dphi-dr}
\end{equation}
which is integrated numerically using the full index profile obtained from
the inflow solution.  The analytic discussion here is intended to highlight
the key structures in the effective potential that control the photon sphere
and shadow.

\subsection{Photon sphere and shadow size}
\label{subsec:photon-sphere-shadow}

A photon sphere is defined as a marginally stable circular null orbit.
In the effective potential language, this corresponds to a radius \(r_{\text{ph}}\)
satisfying
\begin{equation}
  V_{\text{eff}}(r_{\text{ph}}; b_{\text{ph}}) = 0,
  \qquad
  \frac{\partial V_{\text{eff}}}{\partial r}
    (r_{\text{ph}}; b_{\text{ph}}) = 0,
  \label{eq:photon-sphere-conditions}
\end{equation}
for some critical impact parameter \(b_{\text{ph}}\).  Using
Eq.~\eqref{eq:Veff-def}, these conditions can be expressed more directly in
terms of the index profile as
\begin{equation}
  \frac{b_{\text{ph}}^2}{r_{\text{ph}}^2}
  = N^2(r_{\text{ph}}),
  \qquad
  \frac{\mathrm{d}}{\mathrm{d}r}
  \left[ \frac{N^2(r)}{r^2} \right]_{r = r_{\text{ph}}}
  = 0.
  \label{eq:photon-sphere-N}
\end{equation}
The second condition implies
\begin{equation}
  \left. \frac{2 N(r) N'(r)}{r^2}
    - \frac{2 N^2(r)}{r^3}
  \right|_{r = r_{\text{ph}}}
  = 0,
\end{equation}
or equivalently
\begin{equation}
  \frac{N'(r_{\text{ph}})}{N(r_{\text{ph}})}
  = \frac{1}{r_{\text{ph}}}.
  \label{eq:photon-sphere-slope}
\end{equation}
For a simple \(1/r\) index profile, \(N(r) = 1 + \beta G M/(r c^2)\), this
condition yields
\begin{equation}
  r_{\text{ph}}
  \simeq \frac{3 \beta G M}{c^2},
  \qquad
  b_{\text{ph}}
  \simeq \sqrt{3}\,r_{\text{ph}},
  \label{eq:rph-beta}
\end{equation}
up to corrections of order \((G M / r c^2)^2\).  For the Schwarzschild
metric in isotropic coordinates, the effective index has \(\beta = 2\), and
one recovers the standard relation \(r_{\text{ph}} = 3 G M / c^2\) in
Schwarzschild radius units.

In the superfluid defect model, the actual index profile deviates from a
pure \(1/r\) law inside the acoustic horizon due to the nonlinear inflow and
finite throat size.  The numerical solutions in
\texttt{lensing\_from\_flow.wl} use the full \(N(r)\) obtained from the
\(n=5\) inflow to solve Eqs.~\eqref{eq:dphi-dr} and
\eqref{eq:photon-sphere-N} directly.  The results can be summarized as
follows:
\begin{itemize}
  \item A photon-sphere-like orbit exists at a radius
        \begin{equation}
          r_{\text{ph}}
          \simeq \kappa_{\text{ph}}\,\frac{G M}{c^2},
          \label{eq:rph-scaling}
        \end{equation}
        with \(\kappa_{\text{ph}}\) a dimensionless constant of order
        unity that is slightly larger than the Schwarzschild value \(3\).
  \item The corresponding critical impact parameter scales as
        \begin{equation}
          b_{\text{ph}}
          \simeq \eta_{\text{ph}}\,\frac{G M}{c^2},
          \label{eq:bph-scaling}
        \end{equation}
        with \(\eta_{\text{ph}}\) close to the GR value \(\sqrt{27}\).
  \item The shadow radius seen by a distant observer, defined by
        \(b_{\text{sh}} \approx b_{\text{ph}}\), is therefore nearly
        proportional to \(M\) with a proportionality constant that deviates
        from the Schwarzschild value at the few to few-tens of percent
        level, depending on the throat radius \(a\) and the details of the
        inflow.
\end{itemize}
In other words, the \(n=5\) superfluid vacuum predicts a photon-sphere and
shadow structure that is qualitatively similar to, but not exactly the same
as, that of a Schwarzschild black hole of the same mass.  The deviations are
controlled by the same throat impedance and mass--radius scaling discussed
in Sec.~\ref{subsec:acoustic-throat-geometry} and
App.~\ref{app:impedance_scaling}.

\subsection{Comparison with GR and observational windows}
\label{subsec:photon-GR-comparison}

The near-linear mass scaling of both the acoustic horizon radius \(r_H\) and
the photon-sphere radius \(r_{\text{ph}}\) implies that, for sufficiently
large and slowly rotating defects, the superfluid defect model will be
indistinguishable from GR in classic weak-field and moderately strong-field
tests.  In particular:
\begin{itemize}
  \item The 1PN observables considered in Papers~I and II (perihelion
        precession, light bending, Shapiro delay, gravitational redshift)
        are exactly matched by construction.
  \item The existence of an acoustic horizon and a photon sphere ensures
        that null rays experience an effective shadow region with size
        \(b_{\text{sh}} \propto M\), as in Schwarzschild.
  \item For impact parameters \(b \gg b_{\text{ph}}\), the bending angle is
        dominated by the same \(\mathcal{O}(G M / b c^2)\) terms that
        control standard weak-field lensing.
\end{itemize}
Deviations from GR are expected to appear in two closely related regimes:
\begin{enumerate}
  \item \textbf{Near-horizon strong-field lensing.}  For impact parameters
        \(b \sim b_{\text{ph}}\), the detailed shape of the index profile
        inside and just outside the acoustic horizon becomes important.
        The finite throat size and the transition from 3D to 4D flow
        introduce corrections to \(r_{\text{ph}}\) and \(b_{\text{ph}}\)
        that are not captured by a pure Schwarzschild metric.
  \item \textbf{Higher-order (2PN) corrections.}  The same fluid features
        that set the throat impedance and mass--radius scaling also control
        higher-order terms in the PN expansion, such as \(\Phi^2/c^2\) and
        \(v^2 \Phi^2 / c^4\) corrections.  These show up in precision
        timing, high-compactness orbital motion, and in the detailed shape
        of the lensing cross section near the shadow edge.
\end{enumerate}

From an observational perspective, the most promising windows for testing
these deviations are:
\begin{itemize}
  \item Very long baseline interferometry images of supermassive compact
        objects (EHT-style observations), where the measured shadow size and
        photon-ring structure can be compared with the nearly Schwarzschild
        but slightly modified predictions of the toy model.
  \item Precision timing of compact binaries in regimes where 2PN and higher
        corrections are measurable, especially if independent constraints on
        the mass--radius relation of the compact object are available.
  \item Numerical simulations of full superfluid dynamics in the \(n=5\)
        vacuum, using the PDE schemes discussed elsewhere in this series, to
        quantify how robust the predicted deviations are under more
        realistic, time-dependent flows.
\end{itemize}

In summary, the fixed \(n=5\), \(M \propto \rho\) fluid not only reproduces
the 1PN dynamics of GR but also predicts a specific pattern of small but
potentially observable deviations in strong-field lensing and near-horizon
structure.  These are natural targets for future observational and
numerical tests of the superfluid defect toy universe.

\section{Predictions, 2PN-Scale Deviations, and Tests}
\label{sec:predictions}

The previous sections have shown that the stiff \(n=5\), \(M \propto \rho\)
vacuum reproduces the 1PN dynamics of GR and supports an acoustic horizon
and photon sphere with approximately Schwarzschild-like scaling.  In this
section we summarize the main predictive levers of the model, focusing on
finite-size effects, the onset of 2PN-scale deviations, and possible
observational and numerical tests.

\subsection{Finite-size effects and aspect ratio \texorpdfstring{$L/a$}{L/a}}
\label{subsec:predictions-La}

In the throat picture of Paper~V, each defect is associated with a
four-dimensional bridge of minimal radius \(a\) and characteristic length
\(L\) in the bulk.  From the brane point of view, the far-field potential is
effectively monopolar, while the near-field structure depends on the
dimensionless aspect ratio
\begin{equation}
  \Lambda \equiv \frac{L}{a}.
  \label{eq:aspect-ratio}
\end{equation}
For \(\Lambda \gg 1\) the throat appears long and thin, and the near-field
region on the brane resembles a quasi-cylindrical cavity; for \(\Lambda \sim
\mathcal{O}(1)\) the throat is stubby and the transition from 3D to 4D flow
is more abrupt.

The inflow and impedance calculations in
\texttt{4d\_throat\_impedance\_scaling.wl} and
\texttt{black\_hole\_scaling.wl} indicate that the mass \(M\), horizon
radius \(r_H\), and throat radius \(a\) are related schematically by
\begin{equation}
  M \sim \mu\,a^{k},
  \qquad
  r_H \sim \kappa\,\frac{G M}{c^2},
  \label{eq:mass-scaling-generic}
\end{equation}
with \(k \lesssim 1\) and \(\mu, \kappa\) dimensionless coefficients that
depend on \(\Lambda\) and on the detailed boundary conditions in the bulk.
For an exactly Schwarzschild black hole one would have \(k = 1\) and a
strictly linear mass--radius relation; in the stiff \(n=5\) vacuum the
scaling is close to but not exactly linear, leaving room for small
finite-size corrections controlled by \(\Lambda\).

From the brane perspective, deviations from GR first enter through terms
that are sensitive to the finite size of the throat region, such as:
\begin{itemize}
  \item corrections to the far-field potential at order \(r^{-3}\) and
        higher, associated with the multipole structure of the 4D throat,
  \item modifications to the index profile \(N(r)\) inside and just outside
        the acoustic horizon due to the bending of flow into the bulk, and
  \item shifts in the effective horizon and photon-sphere radii as functions
        of \(\Lambda\).
\end{itemize}
These effects are naturally of order
\begin{equation}
  \epsilon_{\text{fs}}
  \sim \left( \frac{a}{r_H} \right)^p,
  \label{eq:finite-size-epsilon}
\end{equation}
with \(p \geq 1\) a model-dependent exponent.  Since \(r_H\) itself scales
with \(M\), Eq.~\eqref{eq:finite-size-epsilon} implies that finite-size
effects are most important for low-mass, high-compactness objects and become
progressively less visible for very massive defects with \(r_H \gg a\).

\subsection{Regimes of agreement and breakdown}
\label{subsec:predictions-regimes}

It is useful to organize the parameter space of the model in terms of the
dimensionless compactness
\begin{equation}
  \mathcal{C}
  \equiv \frac{G M}{r_H c^2}
  \sim \frac{1}{\kappa},
  \label{eq:compactness}
\end{equation}
and the strength of the potential and velocities probed by a given
experiment or observation.  Three qualitatively distinct regimes can be
identified:

\paragraph{1. Weak-field, slow-motion regime (1PN).}
Here \(|\Phi|/c^2 \ll 1\) and \(v^2/c^2 \ll 1\) everywhere of interest.  In
this regime:
\begin{itemize}
  \item The hybrid 1PN Lagrangian matches the EIH Lagrangian exactly by
        construction.
  \item Finite-size corrections are suppressed by powers of
        \(a/r \ll 1\), and are typically much smaller than the already tiny
        2PN corrections predicted by GR.
  \item All solar-system tests and classic binary-pulsar 1PN observables are
        reproduced.
\end{itemize}
In practice this regime covers most of the existing precision tests of GR.

\paragraph{2. Moderately strong-field, near-horizon regime.}
In this regime the potential reaches \(|\Phi|/c^2 \sim \mathcal{O}(0.1)\)
and orbital velocities approach \(v^2/c^2 \sim \mathcal{O}(0.1)\), but the
flow is still predominantly 3D and the throat is not fully resolved.  Here:
\begin{itemize}
  \item The effective index profile \(N(r)\) remains close to its
        Schwarzschild counterpart, and the photon-sphere and shadow sizes
        obey Eqs.~\eqref{eq:rph-scaling} and \eqref{eq:bph-scaling} with
        coefficients that differ from GR at the few to few-tens of percent
        level.
  \item 2PN corrections in the toy model are expected to track those of GR
        up to corrections of order \(\epsilon_{\text{fs}}\), making this a
        natural regime in which to look for small deviations.
\end{itemize}
Current EHT-style images of supermassive compact objects likely probe this
regime.

\paragraph{3. Strong-field, throat-resolving regime.}
Finally, in the immediate vicinity of the acoustic horizon and inside the
throat region, the flow is highly nonlinear and the effective potential and
index deviate significantly from the Schwarzschild form:
\begin{itemize}
  \item The transition from 3D inflow on the brane to 4D flow into the bulk
        leads to order-unity modifications of \(N(r)\) for \(r \lesssim
        \mathcal{O}(a)\).
  \item Higher-order terms in the PN expansion (2PN and beyond) become
        large, and the series itself may cease to be a useful organizing
        tool for the dynamics.
  \item Detailed predictions in this regime require full numerical solutions
        of the underlying superfluid PDEs in 3D+1 or 4D+1, rather than a
        purely PN treatment.
\end{itemize}
At present, this regime is more relevant for numerical experiments than for
direct astrophysical observations, though extreme compact binaries and
high-precision gravitational-wave measurements may eventually probe aspects
of it.

\subsection{Astrophysical and numerical tests}
\label{subsec:predictions-tests}

The superfluid defect toy model makes several concrete predictions that
distinguish it from GR once one goes beyond the strict 1PN regime.  These
can be grouped into astrophysical tests and numerical experiments.

\paragraph{Astrophysical tests.}

Promising observational channels include:
\begin{itemize}
  \item \textbf{Black-hole shadow size and shape.}
        The near-linear scaling of the shadow radius with mass is shared
        with Schwarzschild, but the proportionality constant is shifted by
        finite-size and throat effects.  Precision measurements of shadow
        size and photon-ring structure across a range of masses could
        constrain \(\kappa_{\text{ph}}\) and hence \(\Lambda\).
  \item \textbf{Strong-field lensing profiles.}
        Detailed measurements of lensing cross sections and caustic
        structures near the shadow edge could reveal deviations from the
        Schwarzschild bending angle at the few-percent level, particularly
        if combined with independent mass measurements.
  \item \textbf{High-compactness binaries and 2PN timing.}
        In compact binaries where 2PN terms are observationally relevant
        (for example, in pulsar timing or extreme mass-ratio inspirals),
        the modified 2PN coefficients of the toy model could manifest as
        small discrepancies in periastron advance, time delay, or waveform
        phasing relative to GR predictions.
\end{itemize}

\paragraph{Numerical experiments.}

On the modeling side, the most direct way to test and refine the predictions
of the superfluid defect model is through full PDE simulations:
\begin{itemize}
  \item \textbf{3D superfluid simulations of single defects.}
        Solving the nonlinear superfluid equations in 3D around a single
        throat with \(n=5\) would allow a direct measurement of the inflow
        profile \(u(r)\), the acoustic horizon radius \(r_H\), and the
        index profile \(N(r)\), providing an independent check of the
        semi-analytic and 1D radial approximations used here.
  \item \textbf{Binary and multi-defect configurations.}
        Extending these simulations to binary or multi-defect systems would
        permit a direct comparison between the emergent dynamics and the
        EIH predictions, testing the robustness of the 1PN matching and
        probing the onset of 2PN-scale deviations.
  \item \textbf{Ray tracing in simulated flows.}
        Given a numerically obtained flow and index profile, one can
        perform ray tracing to compute shadows, photon rings, and lensing
        patterns, and compare them directly with GR predictions and
        observations.
\end{itemize}

Together, these astrophysical and numerical tests provide a roadmap for
falsifying or refining the superfluid defect toy model.  The key point is
that, once the fluid is fixed by the 1PN matching to have \(n=5\) and
\(M \propto \rho\), the strong-field and 2PN-scale behavior is no longer a
free parameter: it is dictated by the same thermodynamic and geometric
structure that underlies the success of the model at 1PN.

\section{Discussion and Outlook}
\label{sec:discussion}

The superfluid defect toy universe was originally motivated by a simple
question: can a single structured vacuum medium reproduce the familiar
gravitational and electromagnetic phenomena of general relativity and
Maxwell theory, at least in the weak-field regime?  Papers~I--V answered
this question in the affirmative at first post--Newtonian order by
constructing an explicit model in which a compressible superfluid vacuum,
threaded by defect throats, reproduces Newtonian gravity, 1PN orbital
dynamics, light bending, Shapiro delay, Lense--Thirring precession, and an
effective electromagnetic sector.  However, that success came with several
apparent tunings and open questions: the special role of the stiff
polytrope \(n=5\), the need for an effective Lorentzian signature encoded in
\(\alpha^2 = -2/5\), and the status of horizons and strong--field structure
in the throat picture.

The present paper has addressed these issues from a thermodynamic and
geometric perspective.  By combining a polytropic equation of state,
\(P \propto \rho^n\), with a simple equivalence relation between defect mass
and vacuum density, \(M \propto \rho^q\), and by tracking both scalar
(density starvation) and vector (flow and refractive index) responses of the
medium, we showed that the requirement of 1PN consistency with the
Einstein--Infeld--Hoffmann dynamics acts as a uniqueness condition on the
fluid:
\begin{itemize}
  \item The polytropic index is fixed to \(n = 5\); the vacuum must be
        stiff in precisely the way anticipated by the optics paper.
  \item The defect mass must track the local vacuum density linearly,
        \(q = 1\), so that \(M \propto \rho\).
\end{itemize}
Within the class of models considered here, there is no alternative choice
of \((n,q)\) that reproduces both Newtonian gravity and the full EIH
interaction at 1PN order.  What appeared in earlier work as a collection of
successful but somewhat isolated tunings can thus be understood as the
emergent consequence of a single thermodynamic hypothesis and a single
equivalence assumption.

This hybrid analysis also sheds new light on the effective Lorentzian
signature parameter \(\alpha^2 = -2/5\) introduced in Paper~III.  In the
original formulation, matching the vector sector of the EIH Lagrangian
required the longitudinal part of the hydrodynamic energy functional to
carry a negative weight relative to the transverse part, encoded in
\(\alpha^2 < 0\).  Here, the same balance is seen to arise from the
competition between scalar and vector responses: density starvation reduces
the effective inertia in deep potential wells, while the flow and refractive
index increase the effective interaction energy.  When expressed in the
\(\alpha\)--parameterization, the unique \(n=5\), \(M \propto \rho\) fluid
automatically reproduces \(\alpha^2 = -2/5\).  The Lorentzian signature is
not an additional assumption but a derived property of the thermodynamically
fixed vacuum.

On the geometric side, we have shown that the same stiff fluid which
supports the 1PN dynamics also admits an acoustic horizon and photon
sphere in the near--defect region.  Steady inflow in the \(n=5\) vacuum
becomes transonic, defining an acoustic horizon at \(r = r_H\) where the
radial flow speed equals the local sound speed.  The location of this
horizon scales approximately linearly with mass,
\(r_H \sim \kappa G M / c^2\), and lies just outside the minimal throat
radius \(a\).  The associated refractive index profile supports a
photon--sphere--like orbit at \(r_{\text{ph}} \sim \kappa_{\text{ph}} G M /
c^2\) and a nearly Schwarzschild shadow, with small deviations controlled by
the throat impedance and aspect ratio \(L/a\).

Taken together, these results complete the conceptual arc of the 1PN
program:
\begin{itemize}
  \item The vacuum is not just any compressible medium; it is uniquely
        pinned to a stiff \(n=5\) polytrope with \(M \propto \rho\) by 1PN
        consistency.
  \item The ad hoc--seeming Lorentzian signature in the hydrodynamic energy
        functional emerges naturally from the hybrid scalar--vector response
        of this fluid.
  \item The same fixed fluid determines a specific class of horizon and
        throat geometries that are observationally accessible through
        strong--field lensing and shadow measurements.
\end{itemize}

\vspace{0.5em}
\noindent\textbf{Limitations and open questions.}
Despite these advances, several important limitations remain.  First, our
analysis is restricted to the 1PN expansion plus semi--analytic arguments
about the near--horizon flow.  We have not derived the full second
post--Newtonian (2PN) Lagrangian from the superfluid model, nor have we
included radiation reaction or gravitational waves.  While preliminary
symbolic calculations suggest that the \(n=5\) fluid does not immediately
contradict GR at 2PN order, a systematic matching program analogous to the
EIH analysis remains to be carried out.

Second, many of our strong--field conclusions rely on reduced models of the
flow (steady, spherically symmetric inflow, effective 1D radial equations)
and on numerical solutions of those reduced equations.  A fully dynamical,
three--dimensional simulation of the \(n=5\) superfluid, with realistic
initial conditions and defect motions, is needed to validate the existence
and robustness of the acoustic horizon, the precise scaling of \(r_H\) and
\(r_{\text{ph}}\), and the stability of the throat under time--dependent
perturbations.

Third, the electromagnetic sector, while conceptually unified with gravity
through the throat geometry, has been treated largely in the linearized,
long--wavelength limit.  The interplay between strong gravitational fields,
strong electromagnetic fields, and the underlying superfluid dynamics
remains to be explored in detail, especially in regimes relevant to compact
object magnetospheres and high-energy astrophysical phenomena.

\vspace{0.5em}
\noindent\textbf{Directions for future work.}
The natural next steps in this program can be grouped into three broad
categories:

\paragraph{1. Toward a full 2PN and radiation theory.}
A logical continuation of the present work is to extend the hybrid
scalar--vector analysis to second post--Newtonian order and beyond.  This
would involve:
\begin{itemize}
  \item computing the full 2PN expansion of the hybrid Lagrangian,
        including all \(\Phi^2/c^2\), \(v^2 \Phi^2 / c^4\), and
        multi--body cross terms,
  \item matching these coefficients to the standard 2PN and 2.5PN GR
        results, or identifying controlled deviations where the toy
        model predicts a different structure, and
  \item deriving the effective stress--energy carried by superfluid waves
        and vortices to understand gravitational radiation and its back
        reaction on defects.
\end{itemize}
Such a program would clarify whether the superfluid defect universe can
faithfully reproduce the rich phenomenology of binary inspirals and
gravitational waves, or whether it predicts specific 2PN--scale departures
that could be constrained by current and future observations.

\paragraph{2. Fully nonlinear numerical simulations.}
On the computational side, a major goal is to move from reduced radial
models to full 3D (or 4D) simulations of the superfluid equations.  With
modern GPU hardware and adaptive mesh refinement, it should be possible to:
\begin{itemize}
  \item simulate single and binary throats in the \(n=5\) vacuum,
        resolving both the near--throat and far--field regions,
  \item measure the emergent mass--radius relation, horizon structure,
        and index profile directly from the flow, and
  \item perform ray tracing and wave propagation in the numerically
        obtained backgrounds, providing parameter--free predictions for
        shadows, photon rings, and lensing.
\end{itemize}
Such simulations would also test the stability of the throats and the
self--healing properties of the vacuum under strong perturbations.

\paragraph{3. Phenomenology and observational confrontations.}
Finally, the model invites direct comparison with astrophysical data.  Key
targets include:
\begin{itemize}
  \item black--hole imaging (shadow size and shape, photon--ring
        structure) in systems where independent mass estimates are
        available,
  \item precision timing of compact binaries in regimes where 2PN and
        higher corrections are observable, and
  \item potential signatures in high--energy emission or polarization
        patterns arising from the unified superfluid description of
        gravity and electromagnetism.
\end{itemize}
In each case, the stiff \(n=5\) fluid and \(M \propto \rho\) relation
lead to concrete, quantitative predictions once the throat parameters
are fixed.

\vspace{0.5em}
\noindent In summary, this paper has shown that the 1PN success of the
superfluid defect toy universe is not the result of arbitrary tuning but
flows from a tightly constrained thermodynamic and geometric structure.  The
vacuum is forced into a specific corner of theory space, characterized by a
stiff \(n=5\) equation of state, a linear mass--density relation, and an
emergent Lorentzian signature in the longitudinal sector.  This same
structure determines the existence and properties of acoustic horizons and
photon spheres, providing clear targets for future numerical and
observational tests.  Whether the model ultimately survives these tests or
is ruled out, it offers a concrete example of how a self--healing vacuum
might mimic GR to high accuracy at 1PN order while leaving its own
fingerprints in the strong--field and 2PN regimes.

\appendix

\section{1PN Expansion and EIH Matching Details}
\label{app:1pn_details}

In this appendix we summarize the algebraic steps that connect the hybrid
scalar+vector Lagrangian of Sec.~\ref{sec:hybrid_uniqueness} to the
Einstein--Infeld--Hoffmann (EIH) coefficients at first post--Newtonian
order.  The goal is not to reproduce the full multi--body derivation of
Paper~III, but to make explicit how the polytropic index \(n\) and the
mass--density exponent \(q\) enter the 1PN coefficients, and how the
matching conditions select the unique solution
\((n,q) = (5,1)\).

\subsection{Setup and series expansion}
\label{app:1pn-setup}

We begin from the schematic hybrid test--body Lagrangian for a particle
moving in a static potential \(\Phi(\mathbf{r})\),
\begin{equation}
  L_{\text{hyb}}
  = -M(\Phi)\,c^2 \sqrt{1 - \frac{v^2}{c_s^2(\Phi)}}
    + L_{\text{flow}}[\Phi,\mathbf{v}],
  \label{eq:L-hyb-app}
\end{equation}
where:
\begin{itemize}
  \item \(M(\Phi)\) is the effective, potential dependent mass obtained from
        density starvation,
  \item \(c_s(\Phi)\) is the potential dependent sound speed arising from
        the polytropic equation of state, and
  \item \(L_{\text{flow}}\) encodes the interaction of the defect with the
        background flow and is responsible for the gravitomagnetic and
        multi--body vector terms.
\end{itemize}
In the weak field, slow motion regime we expand
\begin{align}
  M(\Phi)
  &= M_0 \left[
      1 + q\,\frac{\Phi}{c^2}
        + m_2\,\frac{\Phi^2}{c^4}
        + \mathcal{O}\!\left(\frac{\Phi^3}{c^6}\right)
    \right],
  \label{eq:M-expansion-app} \\
  \frac{c_s^2(\Phi)}{c^2}
  &= 1 + (n-1)\,\frac{\Phi}{c^2}
       + s_2\,\frac{\Phi^2}{c^4}
       + \mathcal{O}\!\left(\frac{\Phi^3}{c^6}\right),
  \label{eq:cs-expansion-app}
\end{align}
where \(m_2\) and \(s_2\) are coefficients that depend on the detailed
microphysics but do not affect the 1PN matching.  The squared speed is the
only place where \(n\) enters the kinematics at this order.

We also define the small expansion parameters
\begin{equation}
  \varepsilon_v \equiv \frac{v^2}{c^2},
  \qquad
  \varepsilon_\Phi \equiv \frac{|\Phi|}{c^2},
\end{equation}
and treat \(\varepsilon_v \sim \varepsilon_\Phi \equiv \varepsilon \ll 1\).
Terms of order \(\varepsilon\) are Newtonian, and terms of order
\(\varepsilon^2\) are 1PN.

\subsection{Single body 1PN expansion}
\label{app:1pn-single-body}

For the scalar and kinematic parts of the hybrid Lagrangian we write
\begin{equation}
  L_{\text{sc+kin}}
  \equiv -M(\Phi)\,c^2
          \sqrt{1 - \frac{v^2}{c_s^2(\Phi)}},
\end{equation}
and expand the square root using
\begin{equation}
  \sqrt{1 - x}
  = 1 - \frac{1}{2} x - \frac{1}{8} x^2
    + \mathcal{O}(x^3),
  \qquad x \equiv \frac{v^2}{c_s^2(\Phi)}.
\end{equation}
To 1PN order in \(\varepsilon\) we keep terms up to \(x^2\) and up to
first order in \(\Phi/c^2\) inside \(x\).  Using
Eqs.~\eqref{eq:M-expansion-app} and \eqref{eq:cs-expansion-app} we have
\begin{align}
  \frac{v^2}{c_s^2(\Phi)}
  &= \frac{v^2}{c^2}
     \left[
       1 - (n-1)\,\frac{\Phi}{c^2}
       + \mathcal{O}\!\left(\varepsilon^2\right)
     \right]
   = \varepsilon_v
     - (n-1)\,\varepsilon_v \varepsilon_\Phi
     + \mathcal{O}\!\left(\varepsilon^3\right),
\end{align}
so that
\begin{equation}
  x^2
  = \varepsilon_v^2
    + \mathcal{O}\!\left(\varepsilon^3\right),
\end{equation}
and
\begin{equation}
  \sqrt{1 - \frac{v^2}{c_s^2(\Phi)}}
  = 1 - \frac{1}{2} \varepsilon_v
    + \frac{1}{2}(n-1)\,\varepsilon_v \varepsilon_\Phi
    - \frac{1}{8} \varepsilon_v^2
    + \mathcal{O}\!\left(\varepsilon^3\right).
  \label{eq:sqrt-expanded-app}
\end{equation}

Multiplying Eqs.~\eqref{eq:M-expansion-app} and
\eqref{eq:sqrt-expanded-app}, and discarding terms of order
\(\varepsilon^3\), gives
\begin{align}
  L_{\text{sc+kin}}
  &= -M_0 c^2
     \left[
       1 + q\,\varepsilon_\Phi
     \right]
     \left[
       1 - \frac{1}{2}\varepsilon_v
         + \frac{1}{2}(n-1)\,\varepsilon_v \varepsilon_\Phi
         - \frac{1}{8}\varepsilon_v^2
     \right]
     + \mathcal{O}\!\left(\varepsilon^3\right) \nonumber \\
  &= -M_0 c^2
     - M_0 q\,\Phi
     + \frac{1}{2} M_0 v^2
     + \frac{1}{8} M_0 \frac{v^4}{c^2}
     + \frac{1}{2} M_0 \left[ q - (n-1) \right]
       \frac{\Phi v^2}{c^2}
     + \mathcal{O}\!\left(\varepsilon^3\right),
  \label{eq:L-sc-kin-1pn-app}
\end{align}
which reproduces Eq.~\eqref{eq:L-hybrid-expanded} when the constant rest
energy \(-M_0 c^2\) is dropped.  The key points are:
\begin{itemize}
  \item The Newtonian kinetic term \(+\tfrac{1}{2} M_0 v^2\) and the
        1PN kinematic correction \(+\tfrac{1}{8} M_0 v^4 / c^2\) are
        independent of \(n\) and \(q\).
  \item The Newtonian potential energy \(-M_0 q\,\Phi\) depends only on
        \(q\).
  \item The mixed 1PN term \(\propto \Phi v^2/c^2\) depends on the
        combination \(\tfrac{1}{2}[q - (n-1)]\), which we identified
        as \(C_{\text{hyb}}(n,q)\) in Eq.~\eqref{eq:C-hyb-def}.
\end{itemize}

\subsection{EIH template in the test--mass limit}
\label{app:1pn-eih-template}

For comparison, consider a nonspinning test mass \(M_0\) moving in the
static field of a much heavier source of mass \(M\).  In the standard EIH
gauge the 1PN Lagrangian for the test mass can be written schematically as
\begin{equation}
  L_{\text{EIH}}
  = \frac{1}{2} M_0 v^2
    - M_0 \Phi
    + A_{\text{EIH}}\,M_0 \frac{v^4}{c^2}
    + B_{\text{EIH}}\,M_0 \frac{\Phi v^2}{c^2}
    + D_{\text{EIH}}\,M_0 \frac{\Phi^2}{c^2}
    + \cdots,
  \label{eq:EIH-template-app}
\end{equation}
where \(\Phi = -G M / r\), and \(A_{\text{EIH}}\), \(B_{\text{EIH}}\),
and \(D_{\text{EIH}}\) are numerical coefficients fixed by GR.  The dots
denote higher order PN terms and, in the multi--body case, additional
velocity dependent interactions between bodies.

The important part for the present analysis is that:
\begin{itemize}
  \item the Newtonian terms have fixed coefficients \(+\tfrac{1}{2}\) and
        \(-1\) in front of \(M_0 v^2\) and \(M_0 \Phi\), and
  \item the 1PN mixed term has a fixed coefficient \(B_{\text{EIH}}\).
\end{itemize}
In particular, the GR value of \(B_{\text{EIH}}\) is gauge invariant within
the class of isotropic coordinate systems used in the previous papers, and
is the same coefficient that appears in the two--body EIH Lagrangian for the
pairwise terms of the form \(\Phi_{AB} v_A^2\).

\subsection{Matching conditions and solution for \texorpdfstring{$(n,q)$}{(n,q)}}
\label{app:1pn-matching}

Comparing Eqs.~\eqref{eq:L-sc-kin-1pn-app} and
\eqref{eq:EIH-template-app}, and recalling that \(L_{\text{flow}}\) does
not contribute to the single--body Newtonian terms, we obtain two algebraic
matching conditions at 1PN order:
\begin{align}
  \text{(Newtonian potential)}:
  \qquad & -M_0 q\,\Phi
           = -M_0 \Phi
           \quad\Rightarrow\quad
           q = 1,
  \label{eq:match-newtonian-app} \\
  \text{(mixed 1PN term)}:
  \qquad & C_{\text{hyb}}(n,q)\,M_0 \frac{\Phi v^2}{c^2}
            + C_{\text{flow}}(n)\,M_0 \frac{\Phi v^2}{c^2}
            = B_{\text{EIH}}\,M_0 \frac{\Phi v^2}{c^2},
  \label{eq:match-mixed-app}
\end{align}
where we have written the total mixed 1PN coefficient as the sum of the
hybrid scalar+kinematic contribution \(C_{\text{hyb}}(n,q)\) from
Eq.~\eqref{eq:C-hyb-def} and a pure flow contribution
\(C_{\text{flow}}(n)\) arising from integrating out the vector degrees of
freedom in \(L_{\text{flow}}\).

The Newtonian matching condition \eqref{eq:match-newtonian-app} fixes
\begin{equation}
  q = 1,
\end{equation}
as in Sec.~\ref{subsec:hybrid-q1}.  With this choice, the mixed 1PN
coefficient becomes
\begin{equation}
  C_{\text{hyb}}(n,q=1)
  = \frac{2 - n}{2},
\end{equation}
and Eq.~\eqref{eq:match-mixed-app} can be rewritten as
\begin{equation}
  \frac{2 - n}{2} + C_{\text{flow}}(n) = B_{\text{EIH}}.
  \label{eq:matching-eq-n}
\end{equation}

The compressible flow kernel that enters \(L_{\text{flow}}\) is the same
one used in Paper~III to parameterize the longitudinal and transverse
response in terms of \(\alpha\).  There, the EIH matching conditions for
the multi--body velocity dependent terms led to an algebraic relation
between \(\alpha\) and the fluid parameters that can be expressed as
\begin{equation}
  C_{\text{flow}}(n)
  = f_{\text{flow}}(n),
  \label{eq:Cflow-function}
\end{equation}
for some smooth function \(f_{\text{flow}}\) determined by the choice of
equation of state and by the longitudinal/transverse mixing.

In practice, the matching is carried out as follows:
\begin{enumerate}
  \item Construct the full two--body hybrid Lagrangian, including both
        scalar+kinematic and flow contributions, as a function of
        \((n,q)\).
  \item Expand this Lagrangian to 1PN order using symbolic manipulations
        (for example, in Mathematica; see the \texttt{fundamental\_fluid\_check.wl}
        and \texttt{density\_cross\_talk\_derivation.wl} scripts).
  \item Collect the coefficients of the independent 1PN structures:
        \(v_A^4\), \(v_B^4\), \(v_A^2 \Phi_B\),
        \(\mathbf{v}_A \cdot \mathbf{v}_B\),
        \((\mathbf{v}_A \cdot \mathbf{n}_{AB})
          (\mathbf{v}_B \cdot \mathbf{n}_{AB})\),
        and \(\Phi_A \Phi_B\).
  \item Equate these coefficients to the corresponding EIH coefficients
        obtained from GR.
\end{enumerate}
The resulting system of algebraic equations is overdetermined: there are
more coefficient conditions than free parameters.  Solving this system
symbolically yields a unique real solution for \((n,q)\):
\begin{equation}
  n = 5,
  \qquad
  q = 1.
\end{equation}
In particular, the equation \eqref{eq:matching-eq-n} for the mixed
\(\Phi v^2/c^2\) term is satisfied only at \(n = 5\), because the flow
contribution \(C_{\text{flow}}(n)\) that reproduces the remaining EIH
vector coefficients intersects the GR value \(B_{\text{EIH}}\) at that
single point.

\subsection{Summary}
\label{app:1pn-summary}

To summarize, the key outcomes of the 1PN expansion and EIH matching are:
\begin{itemize}
  \item The Newtonian limit fixes the mass--density exponent to
        \(q = 1\), ensuring that defect mass is proportional to the
        local vacuum density.
  \item The mixed 1PN coefficient \(\propto \Phi v^2/c^2\) depends on the
        combination
        \[
          C_{\text{hyb}}(n,q) + C_{\text{flow}}(n),
        \]
        where \(C_{\text{hyb}}(n,q)\) comes from scalar and kinematic
        effects and \(C_{\text{flow}}(n)\) comes from the vector flow.
  \item Matching this combination, together with the remaining 1PN
        coefficients, to the GR EIH values leads to a unique polytropic
        index \(n = 5\).
\end{itemize}
Thus, within the class of polytropic superfluid vacua with power--law
mass--density relations, the requirement of 1PN consistency with GR
selects a single fluid: a stiff \(n=5\) vacuum in which defect masses
track the local density, \(M \propto \rho\).

\section{Scalar--Vector Decomposition and Coefficient Tables}
\label{app:scalar_vector}

In this appendix we make explicit which pieces of the 1PN coefficients come
from the scalar (density starvation / variable mass) sector and which come
from the vector (sound speed / flow) sector.  The goal is to clarify how the
hybrid scalar+vector dynamics reproduces the Einstein--Infeld--Hoffmann
(EIH) coefficients and how the effective Lorentzian signature parameter
\(\alpha^2 = -2/5\) arises from the unique choice \((n,q) = (5,1)\).

Throughout, we work in the two--body case with labels \(A\) and \(B\) and
relative separation \(\mathbf{r}_{AB}\), and we restrict attention to the
velocity dependent terms at 1PN order.  The full symbolic derivations are
contained in the Mathematica scripts
\texttt{fundamental\_fluid\_check.wl} and
\texttt{density\_cross\_talk\_derivation.wl}; here we summarize the
structure of the results.

\subsection{Scalar contributions}
\label{app:scalar-contributions}

For a single body in a static potential, the scalar sector (variable mass
with fixed propagation speed) contributes the Lagrangian
\begin{equation}
  L_{\text{sc}}
  = -M(\Phi)\,c^2 \sqrt{1 - \frac{v^2}{c^2}},
  \qquad
  M(\Phi) = M_0 \left( 1 + q\,\frac{\Phi}{c^2} \right)
  + \mathcal{O}\!\left( \frac{\Phi^2}{c^4} \right),
\end{equation}
whose 1PN expansion is
\begin{equation}
  L_{\text{sc}}
  = \frac{1}{2} M_0 v^2
    - M_0 q\,\Phi
    + \frac{1}{8} M_0 \frac{v^4}{c^2}
    + \frac{1}{2} M_0 q\,\frac{\Phi v^2}{c^2}
    + \mathcal{O}\!\left( \frac{\Phi^2}{c^2}, \frac{\Phi v^4}{c^4} \right),
\end{equation}
as in Eq.~\eqref{eq:L-scalar-expansion}.

In a two--body system, the potential felt by body \(A\) is
\(\Phi_A = -G M_B / r_{AB}\) and vice versa.  To leading PN order, the
scalar sector contributes the following terms to the pairwise Lagrangian:
\begin{equation}
  L_{\text{sc}}^{(AB)}
  = - q \,\frac{G M_A M_B}{r_{AB}}
    + \frac{1}{2} q \,\frac{G M_A M_B}{r_{AB} c^2}
      \left( v_A^2 + v_B^2 \right)
    + \cdots,
  \label{eq:L-sc-AB}
\end{equation}
where the ellipsis denotes terms that are higher order in the PN expansion
or independent of \(q\).  There is no scalar--sector contribution to the
purely vector structures \(\mathbf{v}_A \cdot \mathbf{v}_B\) or
\((\mathbf{v}_A \cdot \mathbf{n}_{AB})
 (\mathbf{v}_B \cdot \mathbf{n}_{AB})\) at this order.

We can summarize the scalar contributions to the 1PN pairwise coefficients
in the schematic EIH form
\begin{equation}
  L_{AB}^{(1\text{PN})}
  = \frac{G M_A M_B}{r_{AB} c^2}
    \left[
      C_{1} (v_A^2 + v_B^2)
      + C_{2} \,\mathbf{v}_A \cdot \mathbf{v}_B
      + C_{3} \,
        (\mathbf{v}_A \cdot \mathbf{n}_{AB})
        (\mathbf{v}_B \cdot \mathbf{n}_{AB})
    \right] + \cdots,
  \label{eq:L-AB-coeffs-template}
\end{equation}
as
\begin{equation}
  C_{1}^{\text{(sc)}} = \frac{1}{2} q,
  \qquad
  C_{2}^{\text{(sc)}} = 0,
  \qquad
  C_{3}^{\text{(sc)}} = 0.
  \label{eq:C-sc}
\end{equation}

\subsection{Vector contributions}
\label{app:vector-contributions}

The vector sector includes both the dependence of the sound speed on the
potential and the flow--mediated interactions.  The kinematic piece with
variable sound speed but fixed mass has Lagrangian
\begin{equation}
  L_{\text{kin,vec}}
  = -M_0 c^2 \sqrt{1 - \frac{v^2}{c_s^2(\Phi)}},
  \qquad
  \frac{c_s^2(\Phi)}{c^2}
  = 1 + (n-1)\,\frac{\Phi}{c^2} + \cdots,
\end{equation}
with 1PN expansion
\begin{equation}
  L_{\text{kin,vec}}
  = \frac{1}{2} M_0 v^2
    + \frac{1}{8} M_0 \frac{v^4}{c^2}
    - \frac{1}{2} (n-1)\,M_0 \frac{\Phi v^2}{c^2}
    + \cdots,
\end{equation}
as in Eq.~\eqref{eq:L-kin-vec-expanded}.  In a two--body system this yields
kinematic contributions to \(C_1\) of the form
\begin{equation}
  C_{1}^{\text{(kin)}}
  = -\frac{1}{2} (n-1).
\end{equation}

The flow--induced vector potential \(L_{\text{flow}}\) encodes the overlap
of the velocity fields sourced by different defects.  In the far field, and
for slowly moving bodies, the resulting pairwise vector interaction can be
written in the same coefficient form as Eq.~\eqref{eq:L-AB-coeffs-template}
with
\begin{equation}
  C_{1}^{\text{(flow)}}(n),
  \qquad
  C_{2}^{\text{(flow)}}(n),
  \qquad
  C_{3}^{\text{(flow)}}(n),
\end{equation}
treated as functions of the polytropic index \(n\).  These functions are
determined by the compressible flow kernel and the longitudinal/transverse
mixing parameter \(\alpha\) introduced in Paper~III.

The detailed form of \(C_{i}^{\text{(flow)}}(n)\) is somewhat involved and
was obtained symbolically using the Mathematica scripts; here it is
sufficient to note that:
\begin{itemize}
  \item the same flow kernel reproduces the correct GR coefficients for the
        purely vector structures \(\mathbf{v}_A \cdot \mathbf{v}_B\) and
        \((\mathbf{v}_A \cdot \mathbf{n}_{AB})
          (\mathbf{v}_B \cdot \mathbf{n}_{AB})\) when expressed in terms
        of \(\alpha\), and
  \item the dependence of \(C_{1}^{\text{(flow)}}(n)\) on \(n\) is such that
        EIH matching yields a unique solution at \(n=5\).
\end{itemize}

Summarizing, the vector contributions decompose schematically as
\begin{align}
  C_{1}^{\text{(vec)}}(n)
  &= C_{1}^{\text{(kin)}}(n) + C_{1}^{\text{(flow)}}(n)
   = -\frac{1}{2} (n-1) + C_{1}^{\text{(flow)}}(n), \\
  C_{2}^{\text{(vec)}}(n)
  &= C_{2}^{\text{(flow)}}(n), \\
  C_{3}^{\text{(vec)}}(n)
  &= C_{3}^{\text{(flow)}}(n).
\end{align}

\subsection{Combined coefficients and comparison with GR}
\label{app:combined-coeffs}

Combining scalar and vector sectors, the total coefficients in
Eq.~\eqref{eq:L-AB-coeffs-template} take the form
\begin{align}
  C_{1}^{\text{(tot)}}(n,q)
  &= C_{1}^{\text{(sc)}}(q)
    + C_{1}^{\text{(vec)}}(n) \nonumber \\
  &= \frac{1}{2} q
    - \frac{1}{2} (n-1)
    + C_{1}^{\text{(flow)}}(n), \label{eq:C1-tot} \\
  C_{2}^{\text{(tot)}}(n)
  &= C_{2}^{\text{(vec)}}(n)
   = C_{2}^{\text{(flow)}}(n), \label{eq:C2-tot} \\
  C_{3}^{\text{(tot)}}(n)
  &= C_{3}^{\text{(vec)}}(n)
   = C_{3}^{\text{(flow)}}(n). \label{eq:C3-tot}
\end{align}
The Einstein--Infeld--Hoffmann Lagrangian for two nonspinning point masses
has fixed GR coefficients
\begin{equation}
  C_{1}^{\text{(GR)}},
  \qquad
  C_{2}^{\text{(GR)}},
  \qquad
  C_{3}^{\text{(GR)}},
\end{equation}
which can be read off from the standard EIH expressions.

We can summarize the matching problem in the table below:

\begin{center}
\begin{tabular}{c|c|c|c}
  term & scalar & vector (kin + flow) & GR target \\
  \hline
  \(v_A^2 + v_B^2\) &
  \(C_{1}^{\text{(sc)}} = \tfrac{1}{2} q\) &
  \(C_{1}^{\text{(vec)}} = -\tfrac{1}{2} (n-1) + C_{1}^{\text{(flow)}}(n)\) &
  \(C_{1}^{\text{(GR)}}\) \\
  \(\mathbf{v}_A \cdot \mathbf{v}_B\) &
  \(C_{2}^{\text{(sc)}} = 0\) &
  \(C_{2}^{\text{(vec)}} = C_{2}^{\text{(flow)}}(n)\) &
  \(C_{2}^{\text{(GR)}}\) \\
  \((\mathbf{v}_A \cdot \mathbf{n}_{AB})
    (\mathbf{v}_B \cdot \mathbf{n}_{AB})\) &
  \(C_{3}^{\text{(sc)}} = 0\) &
  \(C_{3}^{\text{(vec)}} = C_{3}^{\text{(flow)}}(n)\) &
  \(C_{3}^{\text{(GR)}}\)
\end{tabular}
\end{center}

Imposing the Newtonian limit \(q = 1\) fixes the scalar column.  The
remaining matching conditions are then
\begin{align}
  C_{1}^{\text{(tot)}}(n,q=1)
  &= C_{1}^{\text{(GR)}}, \\
  C_{2}^{\text{(tot)}}(n)
  &= C_{2}^{\text{(GR)}}, \\
  C_{3}^{\text{(tot)}}(n)
  &= C_{3}^{\text{(GR)}}.
\end{align}
Solving these equations with the explicit \(C_{i}^{\text{(flow)}}(n)\)
obtained from the compressible flow kernel yields a unique real solution
\begin{equation}
  n = 5,
  \qquad
  q = 1.
\end{equation}
Any other choice of \((n,q)\) fails to reproduce the GR values of the
velocity dependent coefficients.

\subsection{Relation to the \texorpdfstring{$\alpha$}{alpha} parameter}
\label{app:alpha-relation}

In Paper~III, the same flow kernel was parameterized directly in terms of
longitudinal and transverse contributions with a mixing parameter \(\alpha\).
The hydrodynamic energy functional was written schematically as
\begin{equation}
  E_{\text{hydro}}
  = E_{\perp}
    + \alpha^2\,E_{\parallel},
\end{equation}
where \(E_{\perp}\) and \(E_{\parallel}\) represent transverse and
longitudinal parts, respectively.  Matching the EIH Lagrangian at 1PN order
then led to the condition
\begin{equation}
  \alpha^2 = -\frac{2}{5},
\end{equation}
which was interpreted as an emergent Lorentzian signature in the
longitudinal sector.

The scalar--vector decomposition used in this paper provides a mapping
between \(n\), \(q\), and \(\alpha\).  Once \((n,q) = (5,1)\) is fixed by
the 1PN matching, the same coefficient matching that determines the
\(C_{i}^{\text{(flow)}}(n)\) can be re-expressed in terms of \(\alpha\).
Carrying out this reparameterization reproduces the condition
\(\alpha^2 = -2/5\) automatically.  In other words,
\begin{itemize}
  \item the scalar sector fixes the normalization of the potential and the
        Newtonian limit (\(q=1\)),
  \item the polytropic index \(n=5\) fixes the relative strength of the
        scalar and kinematic contributions to the mixed 1PN terms, and
  \item the flow kernel, when tuned to match the remaining EIH vector
        coefficients for \(n=5\), implies \(\alpha^2 = -2/5\).
\end{itemize}

Thus, the scalar--vector decomposition clarifies that the effective
Lorentzian signature condition \(\alpha^2 = -2/5\) is not an independent
assumption, but a derived consequence of the unique \((n,q) = (5,1)\) fluid
that reproduces the full set of 1PN EIH coefficients.

\section{Nonlinear Inflow and Transonic Solution}
\label{app:inflow}

This appendix collects the details of the nonlinear inflow model used to
define the acoustic horizon in Sec.~\ref{sec:acoustic_horizon}.  We work
with a steady, spherically symmetric accretion flow in the stiff
\(n=5\) vacuum, and we show how the transonic condition arises as a regularity
condition of the governing ordinary differential equation (ODE).  We also
outline the numerical method implemented in the
\texttt{nonlinear\_profile\_solver.wl} script.

\subsection{Governing equations}
\label{app:inflow-equations}

We consider a steady flow of the superfluid vacuum toward a throat of radius
\(a\).  The flow is taken to be spherically symmetric on the brane, with
radial velocity \(u(r) < 0\) (inward) and density \(\rho(r)\).  The basic
equations are:

\paragraph{Mass conservation.}
A constant mass flux \(\dot{M}_{\text{flux}}\) implies
\begin{equation}
  \dot{M}_{\text{flux}}
  = 4\pi r^2 \rho(r) u(r)
  = \text{const},
  \label{eq:inflow-mass-flux}
\end{equation}
which can be solved for \(\rho(r)\) as
\begin{equation}
  \rho(r)
  = \frac{\dot{M}_{\text{flux}}}{4\pi r^2 u(r)}.
  \label{eq:rho-from-u}
\end{equation}

\paragraph{Momentum balance (Euler equation).}
In the absence of viscosity, the radial Euler equation reads
\begin{equation}
  u(r) \frac{\mathrm{d}u}{\mathrm{d}r}
  = -\frac{1}{\rho(r)} \frac{\mathrm{d}P}{\mathrm{d}r}
    - \frac{\mathrm{d}\Phi_{\text{eff}}}{\mathrm{d}r},
  \label{eq:inflow-euler}
\end{equation}
where \(P(\rho)\) is the pressure and \(\Phi_{\text{eff}}(r)\) is an
effective potential that includes both the long--range monopolar
contribution \(-G M / r\) and near--throat corrections due to the 3D--to--4D
transition.  We parametrize
\begin{equation}
  \Phi_{\text{eff}}(r)
  = -\frac{G M}{r} + \Phi_{\text{throat}}(r; a),
  \label{eq:Phi-eff-def}
\end{equation}
where \(\Phi_{\text{throat}}(r; a)\) is a short--range potential that
decays rapidly for \(r \gg a\) and encodes the finite throat size.  Its
specific form is not needed for the PN analysis, but is fixed in the
\texttt{nonlinear\_profile\_solver.wl} script so that the 4D throat geometry
matches the construction of Paper~V.

\paragraph{Equation of state.}
The stiff superfluid vacuum is modeled as a polytrope
\begin{equation}
  P(\rho) = K \rho^n,
  \qquad
  n = 5,
\end{equation}
with sound speed
\begin{equation}
  c_s^2(\rho) = \frac{\mathrm{d}P}{\mathrm{d}\rho}
  = K n \rho^{n-1}.
\end{equation}
The background density \(\rho_0\) and the constant \(K\) are chosen so that
\(c_s(\rho_0) = c\) far from the throat.

\subsection{Reduction to a single ODE}
\label{app:inflow-ode}

Using the barotropic relation \(\mathrm{d}P/\mathrm{d}r
= c_s^2(\rho)\,\mathrm{d}\rho/\mathrm{d}r\) and eliminating
\(\mathrm{d}\rho/\mathrm{d}r\) in favor of \(\mathrm{d}u/\mathrm{d}r\) via
Eq.~\eqref{eq:rho-from-u}, we obtain a first--order ODE for \(u(r)\).
Differentiating Eq.~\eqref{eq:rho-from-u} gives
\begin{equation}
  \frac{\mathrm{d}\rho}{\mathrm{d}r}
  = \rho(r)
    \left[
      -\frac{2}{r}
      - \frac{1}{u(r)} \frac{\mathrm{d}u}{\mathrm{d}r}
    \right].
\end{equation}
Substituting into \(\mathrm{d}P/\mathrm{d}r\) and then into
Eq.~\eqref{eq:inflow-euler} yields
\begin{equation}
  u \frac{\mathrm{d}u}{\mathrm{d}r}
  = - c_s^2(\rho)
    \left[
      -\frac{2}{r}
      - \frac{1}{u} \frac{\mathrm{d}u}{\mathrm{d}r}
    \right]
    - \frac{\mathrm{d}\Phi_{\text{eff}}}{\mathrm{d}r},
\end{equation}
or, after rearranging,
\begin{equation}
  \left( u^2 - c_s^2 \right) \frac{\mathrm{d}u}{\mathrm{d}r}
  = u
    \left[
      \frac{2 c_s^2}{r}
      - \frac{\mathrm{d}\Phi_{\text{eff}}}{\mathrm{d}r}
    \right].
  \label{eq:inflow-ode-basic}
\end{equation}
Comparing with Eq.~\eqref{eq:du-dr-generic}, we see that the throat effects
enter through the effective potential:
\begin{equation}
  \mathcal{F}_{\text{throat}}(r; a, n)
  \equiv -\frac{\mathrm{d}\Phi_{\text{throat}}}{\mathrm{d}r},
\end{equation}
so that Eq.~\eqref{eq:inflow-ode-basic} can be written as
\begin{equation}
  \left( u^2 - c_s^2 \right) \frac{\mathrm{d}u}{\mathrm{d}r}
  = u
    \left[
      \frac{2 c_s^2}{r}
      - \frac{\mathrm{d}}{\mathrm{d}r}
        \left( -\frac{G M}{r} \right)
      + \mathcal{F}_{\text{throat}}(r; a, n)
    \right],
\end{equation}
which is exactly Eq.~\eqref{eq:du-dr-generic} in the main text.

It is often convenient to recast this ODE in terms of the Mach number
\(\mathcal{M}(r)\),
\begin{equation}
  \mathcal{M}(r) \equiv \frac{|u(r)|}{c_s(\rho(r))},
\end{equation}
and a dimensionless radius \(x \equiv r/a\).  In these variables, the ODE
takes the generic form
\begin{equation}
  \left( \mathcal{M}^2 - 1 \right)
  \frac{\mathrm{d}\log \mathcal{M}}{\mathrm{d}\log x}
  = \mathcal{G}(x; \mathcal{M}, \Lambda),
  \qquad
  \Lambda \equiv \frac{L}{a},
  \label{eq:Mach-ODE}
\end{equation}
where \(\mathcal{G}\) is a dimensionless function that encodes the detailed
dependence of the effective potential and sound speed on \(x\), and
\(\Lambda\) is the throat aspect ratio introduced in
Eq.~\eqref{eq:aspect-ratio}.  The explicit expression for
\(\mathcal{G}(x; \mathcal{M}, \Lambda)\) used in the numerical work is
recorded in the comments of \texttt{nonlinear\_profile\_solver.wl}.

\subsection{Sonic point and regularity condition}
\label{app:inflow-sonic}

The structure of Eq.~\eqref{eq:inflow-ode-basic} (or equivalently
Eq.~\eqref{eq:Mach-ODE}) implies the existence of a critical (sonic) point
where \(u^2 = c_s^2\) or \(\mathcal{M} = 1\).  At such a point the
coefficient of \(\mathrm{d}u/\mathrm{d}r\) (or
\(\mathrm{d}\mathcal{M}/\mathrm{d}x\)) in the ODE vanishes, and a smooth
solution requires the right--hand side to vanish as well.  This yields the
transonic conditions
\begin{equation}
  u^2(r_H) = c_s^2(r_H),
  \qquad
  \frac{2 c_s^2(r_H)}{r_H}
  - \frac{\mathrm{d}\Phi_{\text{eff}}}{\mathrm{d}r}\bigg|_{r_H}
  = 0,
  \label{eq:transonic-conditions-app}
\end{equation}
which coincide with Eq.~\eqref{eq:transonic-conditions} when
\(\Phi_{\text{eff}}\) is split as in Eq.~\eqref{eq:Phi-eff-def}.  In
dimensionless form this becomes
\begin{equation}
  \mathcal{M}(x_H) = 1,
  \qquad
  \mathcal{G}(x_H; \mathcal{M}=1, \Lambda) = 0,
  \label{eq:transonic-dimensionless}
\end{equation}
where \(x_H \equiv r_H/a\).

The sonic point is thus determined by a pair of conditions:
\begin{itemize}
  \item the Mach number equals unity, \(\mathcal{M} = 1\), and
  \item the effective driving term in the ODE vanishes at the same
        radius.
\end{itemize}
Physically, the first condition expresses the equality of advection speed
and signal speed, while the second expresses a balance between pressure
gradient, gravity, and throat--induced forces at the sonic point.

\subsection{Boundary conditions and numerical method}
\label{app:inflow-numerics}

The inflow ODE is a two--point boundary value problem.  The natural boundary
conditions are:
\begin{itemize}
  \item \textbf{Outer boundary.}  At large radii \(r \gg r_H\), the flow
        approaches a subsonic, nearly homogeneous background:
        \[
          u(r) \to u_{\infty}, \qquad
          \rho(r) \to \rho_0, \qquad
          c_s(r) \to c,
        \]
        with \(|u_{\infty}| \ll c\).  In practice, the numerical solution is
        initialized at a finite outer radius \(r_{\max}\) with
        \(\mathcal{M}(r_{\max}) \ll 1\).
  \item \textbf{Inner boundary.}  At the throat radius \(r \approx a\), the
        flow must match the imposed throat boundary condition.  This is
        encoded in the choice of \(\Phi_{\text{throat}}(r; a)\) and in the
        requirement that the mass flux \(\dot{M}_{\text{flux}}\) be finite.
\end{itemize}

The \texttt{nonlinear\_profile\_solver.wl} script solves the ODE by
treating the sonic radius \(r_H\) and the mass flux \(\dot{M}_{\text{flux}}\)
as shooting parameters.  The procedure is:
\begin{enumerate}
  \item Choose trial values for \(r_H\) and \(\dot{M}_{\text{flux}}\).
  \item Impose the sonic conditions \eqref{eq:transonic-conditions-app} at
        \(r = r_H\) to determine \(u(r_H)\) and \(\rho(r_H)\).
  \item Integrate the ODE outward from \(r_H\) to \(r_{\max}\) and inward
        from \(r_H\) to \(r \approx a\), using a standard ODE solver (for
        example, a Runge--Kutta method).
  \item Adjust \(r_H\) and \(\dot{M}_{\text{flux}}\) iteratively until the
        outer solution approaches the desired asymptotic state
        \((u,\rho) \to (u_{\infty},\rho_0)\) at \(r_{\max}\) and the inner
        solution remains regular at \(r \approx a\).
\end{enumerate}
For \(n=5\) and realistic choices of \(a\), \(\Lambda\), and the asymptotic
state, this procedure converges to a unique transonic solution in which the
flow is subsonic at large radii, crosses \(\mathcal{M} = 1\) at a radius
\(r_H \gtrsim a\), and becomes supersonic as it approaches the throat.

\subsection{Asymptotic behavior}
\label{app:inflow-asymptotics}

It is useful to record the leading asymptotic behavior of the solution in
the regimes \(r \gg r_H\) and \(r \approx r_H\).

\paragraph{Far field, \(r \gg r_H\).}
In the far field, the flow is weak and the density is close to \(\rho_0\),
so \(c_s \approx c\) and \(|u| \ll c\).  The ODE simplifies to
\begin{equation}
  u \frac{\mathrm{d}u}{\mathrm{d}r}
  \approx - \frac{\mathrm{d}\Phi_{\text{eff}}}{\mathrm{d}r}
  \approx -\frac{\mathrm{d}}{\mathrm{d}r}
          \left( -\frac{G M}{r} \right),
\end{equation}
which integrates to
\begin{equation}
  \frac{1}{2} u^2(r)
  \approx -\Phi(r) + \text{const}
  = \frac{G M}{r} + \text{const}.
\end{equation}
With the boundary condition \(u \to u_{\infty}\) as \(r \to \infty\), the
far--field velocity behaves as
\begin{equation}
  u(r)
  \approx u_{\infty}
    - \sqrt{\frac{2 G M}{r}}
    + \cdots,
\end{equation}
showing the expected acceleration into the potential well.

\paragraph{Near sonic point, \(r \approx r_H\).}
Expanding the ODE \eqref{eq:inflow-ode-basic} around \(r = r_H\) with
\(u^2(r_H) = c_s^2(r_H)\) and using the transonic conditions
\eqref{eq:transonic-conditions-app}, one finds that the derivative
\(\mathrm{d}u/\mathrm{d}r\) remains finite,
\begin{equation}
  \frac{\mathrm{d}u}{\mathrm{d}r}\bigg|_{r_H}
  = \left.
      \frac{u}{2 c_s^2}
      \left[
        \frac{2 c_s^2}{r}
        - \frac{\mathrm{d}\Phi_{\text{eff}}}{\mathrm{d}r}
        + \frac{\partial (c_s^2)}{\partial r}
      \right]
    \right|_{r = r_H},
\end{equation}
where the derivative of \(c_s^2\) with respect to \(r\) is computed via
\(\mathrm{d}c_s^2/\mathrm{d}r
= (\mathrm{d}c_s^2/\mathrm{d}\rho)(\mathrm{d}\rho/\mathrm{d}r)\) and
\(\mathrm{d}\rho/\mathrm{d}r\) is obtained from mass conservation.  This
regularity is the hallmark of a physically acceptable transonic solution and
is enforced in the numerical scheme by imposing the sonic conditions at
\(r_H\).

\subsection{Summary}
\label{app:inflow-summary}

The key points of this appendix are:
\begin{itemize}
  \item Steady, spherically symmetric inflow in the \(n=5\) vacuum reduces
        to a single nonlinear ODE for the radial velocity \(u(r)\), or
        equivalently for the Mach number \(\mathcal{M}(r)\).
  \item This ODE has a critical point where \(u^2 = c_s^2\), and a smooth
        solution requires the transonic conditions
        \eqref{eq:transonic-conditions-app} to hold at a radius \(r_H\).
  \item For physically relevant boundary conditions, there is a unique
        transonic solution in which the flow is subsonic at large radii and
        supersonic near the throat, with \(r_H \gtrsim a\).
\end{itemize}
These properties underlie the definition of the acoustic horizon in
Sec.~\ref{sec:acoustic_horizon} and feed directly into the index profiles
used in the lensing calculations of Sec.~\ref{sec:photon_sphere}.

\section{Throat Impedance and Mass--Radius Scaling}
\label{app:impedance_scaling}

This appendix summarizes the scaling arguments and numerical checks that
relate the throat radius \(a\), the acoustic horizon radius \(r_H\), and the
defect mass \(M\) in the stiff \(n=5\) superfluid vacuum.  The goal is to
clarify why the model predicts a nearly linear mass--radius relation,
\(r_H \propto M\), and how small deviations from exact linearity arise
through the finite impedance of the throat.

The calculations described here are implemented in the Mathematica scripts
\texttt{4d\_throat\_impedance\_scaling.wl} and
\texttt{black\_hole\_scaling.wl}.  We do not reproduce the full symbolic and
numerical code, but we record the key formulas and scaling relations.

\subsection{Throat impedance and mass flux}
\label{app:impedance-mass-flux}

In the throat picture of Paper~V, each defect is associated with a 4D
bridge of minimal radius \(a\) that connects the 3D brane to the bulk
superfluid.  The throat acts as a conduit for mass flux from the brane into
the bulk.  On the brane, this appears as an effective sink with mass \(M\);
in the bulk, it appears as a flux tube of cross sectional area
\(\pi a^2\).

The steady inflow solution of App.~\ref{app:inflow} implies a constant mass
flux from the brane into the throat,
\begin{equation}
  \dot{M}_{\text{flux}}
  = 4\pi r^2 \rho(r) u(r),
\end{equation}
which must match the flux through the throat in the bulk.  The effective
impedance of the throat can be defined as
\begin{equation}
  Z_{\text{th}} \equiv \frac{\Delta P}{\dot{M}_{\text{flux}}},
  \label{eq:Zth-def}
\end{equation}
where \(\Delta P\) is the pressure drop between the brane side and the bulk
side of the throat.  For a linear, quasi--static response one expects
\(\Delta P \sim \rho_0 c^2\) and a flux scaling like \(\dot{M}_{\text{flux}}
\sim \rho_0 c a^2\), so that
\begin{equation}
  Z_{\text{th}}
  \sim \frac{\rho_0 c^2}{\rho_0 c a^2}
  \sim \frac{c}{a^2}.
\end{equation}
More generally, the detailed 4D geometry modifies the prefactor and
introduces a weak dependence on the aspect ratio \(\Lambda = L/a\), but the
dominant scaling with \(a\) is expected to be \(Z_{\text{th}} \propto a^{-2}\).

The defect mass \(M\) is controlled by how much of the vacuum is drained
and stored in the throat and bulk.  A simple dimensional estimate, confirmed
by the numerical experiments in \texttt{black\_hole\_scaling.wl}, is
\begin{equation}
  M \sim \rho_0 a^3 \mathcal{F}(\Lambda),
  \label{eq:M-a3-schematic}
\end{equation}
where \(\mathcal{F}(\Lambda)\) is a dimensionless function that encodes the
effect of the throat length and bulk boundary conditions.  For a perfectly
spherical 3D defect one would have \(\mathcal{F} \sim \mathcal{O}(1)\) and a
strictly cubic scaling \(M \propto a^3\); for a 4D throat embedded in a
3D brane, the effective exponent is renormalized by the way flux lines
spread into the bulk.

\subsection{Effective mass--radius exponent}
\label{app:impedance-exponent}

To quantify this renormalization, the \texttt{4d\_throat\_impedance\_scaling.wl}
script computes an effective exponent \(k\) defined by
\begin{equation}
  k
  \equiv \frac{\mathrm{d}\log M}{\mathrm{d}\log a},
  \label{eq:k-def}
\end{equation}
using numerical solutions of the inflow and throat matching problem for a
range of parameter choices.  Operationally, one fits the relation between
\(M\) and \(a\) over a finite range to a power law \(M \propto a^{k}\) and
extracts \(k\).

The main findings can be summarized as:
\begin{itemize}
  \item For a wide range of aspect ratios \(\Lambda\) and boundary
        conditions, the effective exponent \(k\) lies in a narrow band
        around unity,
        \begin{equation}
          k \approx 1 \pm \delta_k,
          \qquad
          \delta_k \ll 1,
        \end{equation}
        with the precise value of \(\delta_k\) depending weakly on
        \(\Lambda\).
  \item In the limit of very long throats (\(\Lambda \gg 1\)), the exponent
        approaches \(k \to 1\) from above or below, depending on the bulk
        geometry and the imposed outer boundary conditions in the bulk.
  \item For \(\Lambda \sim \mathcal{O}(1)\), modest deviations from unity
        appear, reflecting the fact that the brane sees a compact, stubby
        object whose effective volume and cross section do not scale as a
        simple power of \(a\).
\end{itemize}

These results motivate the phenomenological scaling
\begin{equation}
  M
  \simeq \mu_*\,a^{k_*},
  \qquad
  k_* \lesssim 1,
  \label{eq:M-a-kstar}
\end{equation}
with \(\mu_*\) and \(k_*\) determined by the microphysics of the throat and
by the global topology of the bulk.  The near--linear scaling
\(k_* \approx 1\) is what enables the model to mimic the Schwarzschild
relation \(r_s = 2 G M / c^2\) over many orders of magnitude while still
allowing room for small finite--size corrections.

\subsection{Horizon radius and compactness}
\label{app:impedance-horizon-scaling}

The acoustic horizon radius \(r_H\) is determined by the transonic condition
\eqref{eq:transonic-conditions-app}, which involves both the gravitational
potential and the sound speed.  For the \(n=5\) vacuum, the combined
inflow--impedance calculations in \texttt{black\_hole\_scaling.wl} yield a
relation of the form
\begin{equation}
  r_H
  \simeq \kappa_* \frac{G M}{c^2}
       \simeq \kappa_* \frac{G \mu_*}{c^2} a^{k_*},
  \label{eq:rH-scaling-kstar}
\end{equation}
where \(\kappa_*\) is a dimensionless constant of order unity.  For
\(k_* = 1\) this reduces to a strictly linear relation between \(r_H\) and
\(M\); for \(k_* \neq 1\) the scaling is only approximately linear over a
finite range.

The dimensionless compactness parameter
\(\mathcal{C} = G M/(r_H c^2)\) defined in
Eq.~\eqref{eq:compactness} is then
\begin{equation}
  \mathcal{C}
  \simeq \frac{1}{\kappa_*} a^{1-k_*}.
\end{equation}
For \(k_* = 1\) the compactness is constant and the horizon radius is
exactly proportional to the mass, as in Schwarzschild.  For \(k_* \neq 1\),
the compactness acquires a mild dependence on \(a\), which can be interpreted
as a finite--size correction to the Schwarzschild relation.

Numerical scans in \texttt{black\_hole\_scaling.wl} support the following
qualitative picture:
\begin{itemize}
  \item For large defects (large \(a\) and \(M\)), the effective exponent
        \(k_*\) approaches unity and the compactness approaches a constant,
        so the model is effectively indistinguishable from GR at the level
        of the mass--radius relation and horizon scaling.
  \item For smaller defects, or for throats with \(\Lambda \sim \mathcal{O}(1)\),
        the deviations from \(k_* = 1\) become more pronounced, leading to
        small but potentially observable departures from the Schwarzschild
        compactness and shadow size.
\end{itemize}

\subsection{Implications for strong--field observables}
\label{app:impedance-implications}

The near--linear mass--radius scaling has direct implications for the
strong--field observables discussed in Sec.~\ref{sec:photon_sphere}.  In
particular:
\begin{itemize}
  \item The photon sphere radius \(r_{\text{ph}}\) inherits the same
        approximate linear scaling with \(M\),
        \[
          r_{\text{ph}}
          \simeq \kappa_{\text{ph}}(a)\, \frac{G M}{c^2},
        \]
        with \(\kappa_{\text{ph}}(a)\) varying only weakly with \(a\) when
        \(k_* \approx 1\).
  \item The critical impact parameter \(b_{\text{ph}}\) and hence the
        shadow radius likewise scale nearly linearly with \(M\), with small
        corrections governed by the same deviation of \(k_*\) from unity.
  \item The finite--size parameter \(\epsilon_{\text{fs}}\) introduced in
        Eq.~\eqref{eq:finite-size-epsilon} can be expressed in terms of the
        departure of \(k_*\) from unity and the ratio \(a/r_H\), providing a
        convenient way to parameterize deviations from Schwarzschild
        behavior in observational fits.
\end{itemize}

In summary, the throat impedance and inflow calculations show that, in the
stiff \(n=5\) vacuum, the defect mass \(M\), throat radius \(a\), and
horizon radius \(r_H\) are tied together by an almost linear mass--radius
relation.  This explains why the superfluid defect model can so closely
mimic the Schwarzschild geometry in the strong--field regime while still
predicting small, controlled deviations that are natural targets for future
numerical and observational tests.

\section{Lensing from Flow: Numerical Setup}
\label{app:lensing_numerics}

This appendix summarizes the numerical setup used to compute null-ray
trajectories, bending angles, and shadow radii in the fixed \(n=5\)
superfluid vacuum.  The calculations are implemented in the
\texttt{lensing\_from\_flow.wl} script, which takes as input the radial
flow and index profiles obtained from the nonlinear inflow solutions and
outputs ray trajectories and effective photon-sphere parameters.

\subsection{Effective index profile from the inflow}
\label{app:lensing-index}

The starting point is the spherically symmetric inflow solution
\(u(r)\), \(\rho(r)\) for the \(n=5\) vacuum, obtained as described in
App.~\ref{app:inflow}.  From \(\rho(r)\) we compute the local sound speed
using the polytropic equation of state
\begin{equation}
  c_s^2(r)
  = \frac{\mathrm{d}P}{\mathrm{d}\rho}
  = K n \rho^{n-1}(r),
  \qquad n = 5,
\end{equation}
with \(K\) chosen so that \(c_s(\rho_0) = c\) in the far field.  The
effective refractive index is then
\begin{equation}
  N(r) \equiv \frac{c}{c_s(r)}.
  \label{eq:N-of-r-full}
\end{equation}
In practice, the script works with a tabulated index profile
\(\{ r_i, N_i \}\), typically sampled on a logarithmic grid spanning from
\(r \approx a\) up to an outer radius \(r_{\max} \gg r_H\).  Between grid
points the profile is interpolated using a smooth spline to ensure that
\(N(r)\) and its first derivative are continuous.

The radial velocity \(u(r)\) is used primarily to locate the acoustic
horizon \(r_H\) via the condition \(|u(r_H)| = c_s(r_H)\).  The ray-tracing
itself is performed in the optical metric determined by \(N(r)\); the
explicit dependence on \(u(r)\) enters only through its influence on
\(c_s(r)\) and hence on \(N(r)\).

\subsection{Ray equations and impact parameter}
\label{app:lensing-ray-eqns}

On the brane, null rays propagate in the effective optical metric
\begin{equation}
  \mathrm{d}s_{\text{opt}}^2
  = -\frac{c^2}{N^2(r)}\,\mathrm{d}t^2
    + N^2(r)
      \left( \mathrm{d}r^2
      + r^2 \mathrm{d}\Omega^2 \right),
\end{equation}
as in Eq.~\eqref{eq:optical-metric-spherical}.  Restricting to the
equatorial plane \(\theta = \pi/2\), the line element reduces to
\begin{equation}
  \mathrm{d}s_{\text{opt}}^2
  = -\frac{c^2}{N^2(r)}\,\mathrm{d}t^2
    + N^2(r)
      \left( \mathrm{d}r^2
      + r^2 \mathrm{d}\phi^2 \right).
\end{equation}
The metric admits two Killing vectors, \(\partial_t\) and \(\partial_{\phi}\),
leading to conserved quantities
\begin{equation}
  E_{\text{opt}}
  = \frac{c^2}{N^2(r)} \frac{\mathrm{d}t}{\mathrm{d}\lambda},
  \qquad
  L_{\text{opt}}
  = N^2(r) r^2 \frac{\mathrm{d}\phi}{\mathrm{d}\lambda},
\end{equation}
where \(\lambda\) is an affine parameter.  The ratio
\begin{equation}
  b \equiv \frac{L_{\text{opt}}}{E_{\text{opt}}/c}
\end{equation}
plays the role of an impact parameter.

The null condition \(\mathrm{d}s_{\text{opt}}^2 = 0\) yields the radial
equation
\begin{equation}
  N^2(r) \left( \frac{\mathrm{d}r}{\mathrm{d}\lambda} \right)^2
  + N^2(r) r^2 \left( \frac{\mathrm{d}\phi}{\mathrm{d}\lambda} \right)^2
  = \frac{c^2}{N^2(r)}
    \left( \frac{\mathrm{d}t}{\mathrm{d}\lambda} \right)^2,
\end{equation}
which, after eliminating \(\mathrm{d}t/\mathrm{d}\lambda\) and
\(\mathrm{d}\phi/\mathrm{d}\lambda\) in favor of \(E_{\text{opt}}\),
\(L_{\text{opt}}\), and \(b\), can be written as
\begin{equation}
  \left( \frac{\mathrm{d}r}{\mathrm{d}\phi} \right)^2
  = \frac{r^4}{b^2}
    \left[
      \frac{N^2(r)}{1 - b^2 N^{-2}(r) / r^2}
    \right]^2
  \left[
    1 - \frac{b^2}{r^2 N^2(r)}
  \right].
\end{equation}
A more convenient form for numerical integration, equivalent to
Eq.~\eqref{eq:dphi-dr} in the main text, is
\begin{equation}
  \frac{\mathrm{d}\phi}{\mathrm{d}r}
  = \frac{b}{r^2}
    \frac{N^2(r)}{\sqrt{N^2(r) - b^2/r^2}}.
  \label{eq:dphi-dr-app}
\end{equation}
For a given impact parameter \(b\), this equation describes the angular
deflection of a ray as it passes through the index profile \(N(r)\).

\subsection{Boundary conditions and integration strategy}
\label{app:lensing-boundary}

The bending angle for a ray that comes in from infinity, reaches a minimum
radius \(r_{\min}\), and returns to infinity is
\begin{equation}
  \Delta\phi(b)
  = 2 \int_{r_{\min}}^{\infty}
      \frac{b}{r^2}
      \frac{N^2(r)}{\sqrt{N^2(r) - b^2/r^2}}
      \,\mathrm{d}r
    - \pi,
  \label{eq:bending-angle}
\end{equation}
where \(r_{\min}\) is defined implicitly by
\begin{equation}
  N^2(r_{\min})
  = \frac{b^2}{r_{\min}^2},
  \label{eq:rmin-condition}
\end{equation}
so that the denominator in Eq.~\eqref{eq:dphi-dr-app} changes sign.  For
impact parameters smaller than a critical value \(b_{\text{ph}}\), the
equation \(N^2(r) = b^2 / r^2\) has no real solution outside the horizon,
and the ray plunges into the throat; this defines the effective shadow.

The numerical integration proceeds as follows:
\begin{enumerate}
  \item For a given impact parameter \(b\), solve
        \(N^2(r) = b^2 / r^2\) for \(r_{\min}\) using a root-finding method
        (for example, Newton-Raphson), starting from an initial guess near
        the expected turning point.
  \item Evaluate the integral in Eq.~\eqref{eq:bending-angle} from
        \(r = r_{\min}\) up to an outer radius \(r_{\max}\) where
        \(N(r) \approx 1\) and the contribution to the deflection becomes
        negligible.  The integral is performed using an adaptive quadrature
        routine, with the integrand regularized near \(r_{\min}\) if
        necessary.
  \item Compute \(\Delta\phi(b)\) and record whether the ray escapes to
        infinity or is captured by the throat (no turning point outside
        \(r_H\)).
\end{enumerate}
By scanning over \(b\), one obtains the bending-angle function
\(\Delta\phi(b)\), the critical impact parameter \(b_{\text{ph}}\) at which
the bending angle diverges (corresponding to the photon sphere), and the
range of \(b\) for which the ray is captured.

\subsection{Photon sphere and shadow extraction}
\label{app:lensing-photon-shadow}

The photon sphere radius \(r_{\text{ph}}\) and critical impact parameter
\(b_{\text{ph}}\) are determined numerically by the conditions
\begin{equation}
  N^2(r_{\text{ph}})
  = \frac{b_{\text{ph}}^2}{r_{\text{ph}}^2},
  \qquad
  \frac{\mathrm{d}}{\mathrm{d}r}
  \left[ \frac{N^2(r)}{r^2} \right]_{r = r_{\text{ph}}}
  = 0,
\end{equation}
which are equivalent to Eqs.~\eqref{eq:photon-sphere-N} and
\eqref{eq:photon-sphere-slope} in the main text.  In practice, the script
implements this as a two-parameter root-finding problem in
\((r_{\text{ph}}, b_{\text{ph}})\), using the spline-interpolated
\(N(r)\) and its derivative.

Once \(b_{\text{ph}}\) is known, the effective shadow radius seen by a
distant observer is taken to be
\begin{equation}
  b_{\text{sh}} \approx b_{\text{ph}},
\end{equation}
with corrections from higher-order terms in the index profile typically
smaller than the uncertainties associated with the throat parameters
themselves.

The \texttt{lensing\_from\_flow.wl} script outputs:
\begin{itemize}
  \item The numerically determined \(r_{\text{ph}}\) and \(b_{\text{ph}}\)
        for a given set of throat and inflow parameters.
  \item The bending-angle function \(\Delta\phi(b)\) over a range of
        impact parameters, which can be compared with the corresponding
        Schwarzschild result.
  \item Sample ray trajectories \(r(\phi)\) illustrating whether rays with
        given \(b\) escape, orbit near the photon sphere, or are captured
        by the throat.
\end{itemize}

\subsection{Limitations and approximations}
\label{app:lensing-limitations}

Several simplifying assumptions are built into this numerical setup:
\begin{itemize}
  \item The flow is assumed to be steady and spherically symmetric on the
        brane; angular structure and time dependence are neglected.
  \item The index profile \(N(r)\) is taken to be determined entirely by
        the radial density profile \(\rho(r)\) in the \(n=5\) vacuum; any
        additional anisotropies or frequency dependence of the effective
        index are ignored.
  \item The rays are traced in the effective optical metric only on the
        brane; the full 4D throat geometry and bulk propagation are not
        modeled explicitly.
\end{itemize}
Within these approximations, the numerical scheme provides a controlled way
to connect the hydrodynamic inflow and throat structure to observable
lensing signatures, and to quantify the deviations from Schwarzschild
behavior induced by the fixed \(n=5\) superfluid vacuum.

\begin{thebibliography}{99}

\bibitem{Norris:Paper1}
Norris, T. (2025).
\newblock \emph{Newtonian and 1PN Orbital Dynamics from a Superfluid Defect Toy Model}.
\newblock Zenodo.
\newblock \href{https://doi.org/10.5281/zenodo.17759367}{doi:10.5281/zenodo.17759367}.

\bibitem{Norris:Paper2}
Norris, T. (2025).
\newblock \emph{Gravitational Optics and Soliton Geodesics in a Superfluid Defect Toy Model}.
\newblock Zenodo.
\newblock \href{https://doi.org/10.5281/zenodo.17794911}{doi:10.5281/zenodo.17794911}.

\bibitem{Norris:Paper3}
Norris, T. (2025).
\newblock \emph{Spin, Vorticity, and N-Body Dynamics in a Superfluid Defect Toy Model}.
\newblock Zenodo.
\newblock \href{https://doi.org/10.5281/zenodo.17822520}{doi:10.5281/zenodo.17822520}.

\bibitem{Norris:Paper4}
Norris, T. (2025).
\newblock \emph{Electromagnetic Fields and Charged Defects in a Superfluid Defect Toy Model}.
\newblock Zenodo.
\newblock \href{https://doi.org/10.5281/zenodo.17798569}{doi:10.5281/zenodo.17798569}.

\bibitem{Norris:Paper5}
Norris, T. (2025).
\newblock \emph{Brane-Bulk Throat Ontology for a Superfluid Defect Toy Universe}.
\newblock Zenodo.
\newblock \href{https://doi.org/10.5281/zenodo.17822761}{10.5281/zenodo.17822761}.

\end{thebibliography}

\end{document}
