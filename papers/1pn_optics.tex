\documentclass[11pt]{article}

% Basic packages
\usepackage[margin=1in]{geometry}
\usepackage{amsmath,amssymb,amsfonts}
\usepackage{bm}
\usepackage{graphicx}
\usepackage{hyperref}
\usepackage[numbers,sort&compress]{natbib}
\usepackage{authblk}

% Hyperref setup
\hypersetup{
  colorlinks=true,
  linkcolor=blue,
  citecolor=blue,
  urlcolor=blue
}

% Custom commands (tune / extend as needed)
\newcommand{\PN}{\mathrm{PN}}
\newcommand{\cS}{c_s}
\newcommand{\PhiP}{\Phi_{\mathrm{P}}}
\newcommand{\PhiL}{\Phi_{\mathrm{L}}}
\newcommand{\PhiTot}{\Phi}
\newcommand{\GM}{GM}
\newcommand{\ve}{\varepsilon}
\newcommand{\dd}{\mathrm{d}}

\title{Gravitational Optics and Soliton Geodesics\\
in a Superfluid Defect Toy Model}
\author{Trevor Norris}
\date{\today}

\begin{document}

\maketitle

\begin{abstract}
We extend a previously developed superfluid-defect toy model of gravity from
orbital dynamics to the full suite of classic 1PN tests involving light and
clocks.  The model treats the vacuum as a compressible superfluid and massive
bodies as flux-tube defects that drain this vacuum; Paper~1 showed that a scalar
lag field plus a position-dependent kinetic prefactor $\sigma(r)$ reproduces
the GR perihelion precession and fixes a single orbital parameter $\beta=3/2$.
Here we focus on gravitational optics and redshift.

We model the vacuum as a stiff ($n=5$) polytropic superfluid and show that a
flux-tube mass defect induces a $1/r$ pressure and density deficit that fixes a
refractive index profile $N(r)$ governing light bending and Shapiro delay.
Requiring this profile to reproduce the GR lensing coefficient uniquely selects
$n=5$ and implies a spatial curvature coefficient $2.0$ in the optical sector
($\gamma=1$).
If defects were to follow this bare acoustic metric, the predicted perihelion
advance would have coefficient $10$ instead of the GR value $6$.
We show that treating defects as hydrodynamically dressed solitons resolves
this tension: the static density deficit and the dipole flow around a moving
throat contribute $\kappa_\rho = 1$ and $\kappa_{\mathrm{add}} = 1/2$ to the
kinetic prefactor, yielding $\beta = 3/2$ and restoring the GR precession.
Within the 1PN, weak-field regime the combined orbital, optical, and redshift
sectors of the toy model are therefore indistinguishable from Schwarzschild,
even though light and matter couple differently to the underlying superfluid.
\end{abstract}

\section{Introduction}
\label{sec:intro}

\subsection{Motivation and overview}
\label{subsec:intro-motivation}

The classic solar-system tests of gravity---perihelion precession, light bending,
Shapiro time delay, and gravitational redshift---are often summarized as a single
statement: the Schwarzschild solution of General Relativity (GR) with post-Newtonian
parameters $\beta = \gamma = 1$ passes all known 1PN tests.
From that vantage point, any alternative description of gravity must either
explicitly reproduce the Schwarzschild metric in the appropriate limit or
offer a compelling, falsifiable mechanism by which the same observables arise.

This paper continues a different line of attack.
We treat gravity as an \emph{emergent} phenomenon in a ``toy universe'' where the
vacuum is a compressible superfluid and what we usually call ``matter'' is carried
by topological defects.
In this picture, a massive object is modeled as a flux-tube throat that drains
superfluid into a 4D interior, while circulation around that throat plays the role
of magnetic fields and charge.
Dyons---bound composites of throats and vortices---behave as charged, spinning
particles moving through the medium.
The hope is that, by insisting on a single underlying fluid with a small set of
rules, the familiar hierarchy of gravitational and electromagnetic effects can be
recovered as different facets of the same hydrodynamics.

Paper~1 in this series~\cite{Norris:2025Orbits} developed the \emph{orbital} sector
of this toy universe.
Starting from a scalar ``lag'' field that allows the bulk fluid to slip relative to
the defects, and a position-dependent kinetic prefactor that encodes how inertia
renormalizes in the throat background, the model reproduces Newtonian orbits and
a GR-like 1PN perihelion advance.
That work deliberately deferred all questions involving light, clocks, and
clock comparison experiments: it treated defects as massive test bodies only.

The central question of the present paper is therefore:

\begin{quote}
Can the same superfluid-defect toy model that reproduces GR-like orbital
precession at 1PN also reproduce the \emph{optical and clock} phenomenology
of GR---light bending, Shapiro delay, and gravitational redshift---using a
single, physically well-motivated modification of the vacuum state?
\end{quote}

Our answer is ``yes'' within the controlled regime of the model.
We show that there exists a unique, stiff polytropic vacuum ($n=5$ in the
language of polytropes) in which a flux-tube mass defect carves out a
$1/r$ pressure and density profile.
That profile, in turn, determines a refractive index $N(r) = c_0 / c_s(r)$
for small-amplitude waves in the medium.
Remarkably, the same $N(r)$ simultaneously accounts for:

\begin{itemize}
  \item the GR light-bending angle $\Delta\theta = 4GM/(b c^2)$ in the weak-field,
        small-angle limit;
  \item the GR Shapiro time delay with the correct $(1+\gamma)$ coefficient;
  \item a gravitational redshift consistent with the weak-field Schwarzschild
        potential, when clocks are realized as defect-bound oscillators.
\end{itemize}

The resulting picture is that of a single superfluid vacuum, characterized by a
specific equation of state and flux-tube geometry, whose background state around a
massive defect encodes \emph{both} the orbital dynamics of defects and the optics
of waves.
In the language of GR, the model is 1PN-equivalent to Schwarzschild for the classic
tests; in the language of analogue gravity, orbits and light rays are different
probes of the same emergent geometry.
Requiring the refractive index $N(r)$ of the stiff ($n=5$) vacuum to match
GR lensing fixes the optical sector and, if interpreted as a full spacetime
metric, would yield a perihelion advance with coefficient $10$ instead of
the GR value $6$.
We show that this apparent discrepancy is resolved once defects are treated
as hydrodynamically dressed solitons: the density perturbation and dipole
flow around a moving throat contribute $\kappa_\rho = 1$ and
$\kappa_{\mathrm{add}} = 1/2$ to the kinetic prefactor, giving
$\beta = 3/2$ and restoring the GR precession.

Throughout, we stress that this is an \emph{analog} construction, not a proposed
theory of our universe.
The medium defines a preferred frame; Lorentz invariance, where it appears, is
emergent and approximate.
The aim is not to compete with GR but to understand how far an effectively
hydrodynamic picture of vacuum and matter can be pushed before it fails.


\subsection{Summary of Paper~1 results}
\label{subsec:intro-paper1-summary}

For completeness, we briefly summarize the orbital-sector results from
Ref.~\cite{Norris:2025Orbits}, which we will treat as input data in what follows.

In the superfluid-defect toy model, a non-relativistic test defect of reference
mass $m$ moving in the field of a central mass $M$ is described by an effective
Lagrangian of the form
\begin{equation}
  L
  = \frac{1}{2}\bigl[1 + \sigma(r)\bigr]
      \bigl(\dot r^2 + r^2 \dot\varphi^2\bigr)
    - \Phi_{\mathrm{eff}}(r),
\end{equation}
where $\sigma(r)$ is a dimensionless kinetic prefactor that encodes how the
effective inertial mass depends on radius, and $\Phi_{\mathrm{eff}}$ is an
effective central potential.
The scalar lag field, which allows the bulk superfluid to slip relative to
defects, contributes a $1/r^3$ correction to the Newtonian potential,
so that
\begin{equation}
  \Phi_{\mathrm{eff}}(r)
  = -\frac{\mu}{r} - \frac{\mu^2}{2 \cS^2 r^2} \;,
  \qquad
  \mu = GM,
\end{equation}
with $\cS$ the characteristic wave speed (identified with $c$ in the
GR-matching limit).
This scalar-only sector produces exactly one half of the GR perihelion
advance for nearly Keplerian orbits:
\begin{equation}
  \Delta\varphi_{\text{scalar}}
  = \frac{1}{2}\,\Delta\varphi_{\text{GR}}
  = 3\,\frac{\pi \mu}{\cS^2 a(1-e^2)} \;,
\end{equation}
where $a$ and $e$ are the orbital semi-major axis and eccentricity.

To reach the full GR value, Paper~1 introduced a position-dependent kinetic
prefactor
\begin{equation}
  \sigma(r) = \beta\,\frac{\mu}{\cS^2 r},
\end{equation}
with a dimensionless coefficient $\beta$ determined by the hydrodynamics of the
throat and its near-field flows.
In the effective Lagrangian, $\bigl[1+\sigma(r)\bigr]$ plays the role of a
radial renormalization of inertia (or, equivalently, a perturbation to the
spatial metric experienced by the defect).
Solving the resulting equations of motion and matching to the standard 1PN
precession formula yields
\begin{equation}
  \Delta\varphi_{\text{tot}}
  = \bigl(3 + 2\beta\bigr)\,
    \frac{\pi \mu}{\cS^2 a(1-e^2)} \;.
\end{equation}
The hydrodynamic analysis of the throat decomposes $\beta$ into several
contributions (e.g. from density perturbations, additional flows, and possible
pseudoviscous terms).
The final, self-consistent choice sets one of these contributions to zero and
fixes
\begin{equation}
  \beta = \frac{3}{2},
\end{equation}
so that $\Delta\varphi_{\text{tot}}$ exactly reproduces the GR perihelion
advance at 1PN.

These analytic results were checked against two classes of numerical experiments:
(i) reduced-orbit integrations of the effective 2D ODE system, and
(ii) full 3D PDE simulations of the superfluid, scalar lag field, and defect
dynamics in a cubic domain.
Within their respective domains of validity, both classes of simulations confirm
that the scalar-only sector produces half of the GR precession and that, once the
kinetic prefactor with $\beta = 3/2$ is included, the total precession matches
the GR prediction to within the quoted numerical uncertainties.
In what follows we take the effective Lagrangian and the calibrated value of
$\beta$ as \emph{given} and focus on extending the model to light and clocks.


\subsection{Scope and roadmap of this paper}
\label{subsec:intro-roadmap}

The present paper builds on Ref.~\cite{Norris:2025Orbits} by developing the
gravitational optics and clock/redshift sector of the 1PN phenomenology in the
same superfluid-defect framework.
The central organizing idea is that a single, stiff superfluid vacuum with a
flux-tube mass defect defines a radial profile of density and sound speed, and
hence a refractive index $N(r)$, which simultaneously governs:

\begin{itemize}
  \item the trajectories of light-like rays (gravitational lensing),
  \item the time-of-flight of signals near a mass (Shapiro delay),
  \item and the rates of defect-based clocks (gravitational redshift).
\end{itemize}

Our strategy is as follows.
Section~\ref{sec:paper1-inputs} collects the orbital-sector inputs from Paper~1,
including the effective Lagrangian, the role of the scalar lag field, and the
interpretation of the kinetic prefactor $\sigma(r)$ and the parameter $\beta$.
Section~\ref{sec:n5-vacuum} constructs a stiff ($n=5$) polytropic vacuum with a
flux-tube mass defect and derives the resulting $1/r$ pressure and density
profiles, along with the associated sound-speed and refractive-index profiles
$N(r)$.
Section~\ref{sec:lensing} uses this $N(r)$ to compute the weak-field bending of
light, both analytically and via numerical ray tracing, and shows that the
deflection angle matches the GR result and fixes the post-Newtonian parameter
$\gamma = 1$ in this model.
Section~\ref{sec:shapiro} performs an analogous analysis for Shapiro time delay,
demonstrating that the same $N(r)$ reproduces the standard GR logarithmic delay
with the correct coefficient.

In Section~\ref{sec:redshift} we turn to gravitational redshift, treating clocks
as defect-bound oscillators whose frequencies track the local density---and hence
mass---of the defects.
We show that the density deficit induced by the flux-tube defect yields a
redshift consistent with the weak-field Schwarzschild potential, completing the
1PN optical and clock sector.
Section~\ref{sec:soliton-geodesics} then assembles these ingredients into a
``soliton geodesic'' picture: localized defects and wave packets follow
geodesics of an emergent acoustic metric determined by $c_s(r)$ and $\rho(r)$.
We discuss how the orbital and optical sectors jointly fix the effective PPN
parameters $\beta$ and $\gamma$, and how this resolves an apparent tension
between a naively constructed optical metric and the calibrated orbital dynamics.

Section~\ref{sec:degeneracies} analyzes several competing constructions---including
pure-drag and mixed drag/re-\\
fraction pictures---and argues that, within the
restricted class of spherically symmetric, polytropic superfluid vacua with
flux-tube defects, the $n=5$ pure-refraction branch is singled out by the 1PN
tests.
Finally, Section~\ref{sec:discussion} discusses the broader conceptual lessons and
limitations of the toy model and outlines directions for future work, including
the full electromagnetic sector, strong-field extensions, and cosmological
applications.

%---------------------------------------------------------------------------
\section{Inputs from Paper~1: orbital sector and $\beta$}
\label{sec:paper1-inputs}

In this section we collect the minimal orbital-sector ingredients from
Ref.~\cite{Norris:2025Orbits} that will be treated as inputs for the present
work.
Our goal is not to re-derive the results of Paper~1, but to make explicit
which structures are assumed, which parameters have already been fixed, and
how they will be used in the optical and clock sectors that follow.

\subsection{Effective Lagrangian and perihelion precession}
\label{subsec:paper1-lagrangian}

In the superfluid-defect toy model, the motion of a non-relativistic test
defect in the field of a central mass $M$ is described, after suitable
reductions, by an effective two-dimensional Lagrangian of the form
\begin{equation}
  L
  = \frac{1}{2}\bigl[1 + \sigma(r)\bigr]
      \bigl(\dot r^2 + r^2 \dot\varphi^2\bigr)
    - \Phi_{\mathrm{eff}}(r),
  \label{eq:paper1-L-eff}
\end{equation}
where $(r,\varphi)$ are polar coordinates in the orbital plane,
$\Phi_{\mathrm{eff}}$ is an effective central potential, and
$\sigma(r)$ is a dimensionless kinetic prefactor that encodes how the
effective inertial mass of the defect varies with radius.

The potential in Eq.~\eqref{eq:paper1-L-eff} contains two pieces.  The first
is the usual Newtonian term,
\begin{equation}
  \Phi_{\mathrm{N}}(r) = -\frac{\mu}{r},
  \qquad
  \mu \equiv GM,
\end{equation}
arising from the far-field pressure gradient in the superfluid vacuum.  The
second is a $1/r^2$ correction generated by the scalar ``lag'' field that
allows the bulk fluid to slip relative to the defects.
To leading post-Newtonian order this scalar sector contributes
\begin{equation}
  \Phi_{\mathrm{lag}}(r) = -\frac{\mu^2}{2 \cS^2 r^2},
\end{equation}
where $\cS$ is the characteristic wave speed in the medium (identified with
$c$ when comparing to GR).
The total effective potential is thus
\begin{equation}
  \Phi_{\mathrm{eff}}(r)
  = \Phi_{\mathrm{N}}(r) + \Phi_{\mathrm{lag}}(r)
  = -\frac{\mu}{r} - \frac{\mu^2}{2 \cS^2 r^2}.
  \label{eq:paper1-Phi-eff}
\end{equation}

If one temporarily sets $\sigma(r)=0$ and considers only the scalar-lag
correction to the Newtonian potential, the resulting nearly Keplerian orbits
exhibit a perihelion advance
\begin{equation}
  \Delta\varphi_{\text{scalar}}
  = 3\,\frac{\pi \mu}{\cS^2 a(1-e^2)}
  = \frac{1}{2}\,\Delta\varphi_{\text{GR}},
  \label{eq:paper1-precession-scalar}
\end{equation}
where $a$ and $e$ are the orbital semi-major axis and eccentricity, and
$\Delta\varphi_{\text{GR}}$ denotes the standard GR value
\begin{equation}
  \Delta\varphi_{\text{GR}}
  = 6\,\frac{\pi \mu}{c^2 a(1-e^2)}.
\end{equation}
Thus the scalar-lag potential alone reproduces exactly one half of the
Schwarzschild 1PN perihelion precession.

To recover the full GR precession, Paper~1 introduced a position-dependent
kinetic prefactor
\begin{equation}
  \sigma(r)
  = \beta\,\frac{\mu}{\cS^2 r},
  \label{eq:paper1-sigma}
\end{equation}
with $\beta$ a dimensionless parameter determined by the hydrodynamics of
the throat and its near-field flows.
Expanding Eq.~\eqref{eq:paper1-L-eff} to leading post-Newtonian order and
solving the resulting equations of motion yields an additional contribution
to the apsidal advance from the $\sigma(r)$ term.  The total perihelion
advance can be written as
\begin{equation}
  \Delta\varphi_{\text{tot}}
  = \bigl(3 + 2\beta\bigr)\,
    \frac{\pi \mu}{\cS^2 a(1-e^2)}.
  \label{eq:paper1-precession-total}
\end{equation}
Matching Eq.~\eqref{eq:paper1-precession-total} to the GR value forces
\begin{equation}
  3 + 2\beta = 6
  \quad\Rightarrow\quad
  \beta = \frac{3}{2}.
  \label{eq:paper1-beta-32}
\end{equation}

Equation~\eqref{eq:paper1-beta-32} is not an arbitrary fit: in
Ref.~\cite{Norris:2025Orbits} the same value emerges from an explicit
hydrodynamic decomposition of $\beta$ into several physically distinct
contributions, coupled with a null result for one of the candidate
contributions.
We summarize that interpretation next, as it will play an important role in
how we think about the emergent metric in the present paper.

\subsection{Hydrodynamic interpretation of $\sigma(r)$ and $\beta$}
\label{subsec:paper1-beta}

In the superfluid picture, the kinetic prefactor $1+\sigma(r)$ in
Eq.~\eqref{eq:paper1-L-eff} is not inserted by hand but arises from
coarse-graining the defect plus its near-field distortion of the fluid.
Intuitively, a defect moving through the medium must drag along some
co-moving volume of fluid and rearrange the surrounding flow pattern.  The
effective inertial mass that appears in its orbital dynamics is therefore a
renormalized quantity, sensitive to how the background pressure, density,
and flow fields vary with radius.

Paper~1 parameterized this renormalization in terms of a small set of
dimensionless coefficients that capture different hydrodynamic effects.
Writing
\begin{equation}
  \beta = \kappa_\rho + \kappa_{\text{add}} + \kappa_{\text{PV}},
  \label{eq:paper1-beta-decomp}
\end{equation}
one can interpret:
\begin{itemize}
  \item $\kappa_\rho$ as the contribution from the static density perturbation
        in the vicinity of the throat (mass deficit / excess in the co-moving
        volume);
  \item $\kappa_{\text{add}}$ as an additional contribution from coherent flow
        structures (e.g.\ the dipole-like ``cloud'' of velocity field around a
        translating void; see Appendix~\ref{app:dipole-derivation}) that accompany
        the defect;
  \item $\kappa_{\text{PV}}$ as a potential contribution from a pseudoviscous
        or phase-velocity mode associated with the scalar lag field.
\end{itemize}
Each term is computed by integrating appropriate quadratic combinations of the
fluid variables over a spherical shell around the defect, with the details
spelled out in Ref.~\cite{Norris:2025Orbits}.

Two features of this decomposition are important for the present work.
First, $\kappa_\rho$ and $\kappa_{\text{add}}$ are both strictly positive and
of order unity: they encode the fact that a defect carries along additional
``inertial dressing'' from the surrounding fluid, and that this dressing
grows as one approaches the throat.
Second, a direct symmetry analysis and explicit integration (via the
Mathematica calculations reported in Paper~1) show that the would-be
cross-term between the scalar lag mode and the primary flow modes vanishes,
and that $\kappa_{\text{PV}}$ itself is consistent with zero within the
accuracy of the model:
\begin{equation}
  \kappa_{\text{PV}} = 0,
  \qquad
  \beta = \kappa_\rho + \kappa_{\text{add}} = \frac{3}{2}.
\end{equation}
In particular, Appendix~\ref{app:dipole-derivation} shows explicitly that
a stiff spherical throat moving through the superfluid carries an added
mass $\kappa_{\mathrm{add}} = 1/2$ relative to the displaced fluid,
while $\kappa_\rho = 1$ arises from the static density deficit.
This ``null result'' for $\kappa_{\text{PV}}$ simplifies the orbital sector
and will be crucial when we match onto the optical metric in the PPN analysis.

From the perspective of the present paper, it is useful to think of
$\sigma(r)$ as a proxy for how the spatial part of the emergent metric
deviates from Euclidean form along defect worldlines, while
$\Phi_{\mathrm{eff}}(r)$ controls the time-time component.
The fact that $\beta$ is fixed by orbital dynamics and internal consistency
will constrain how we construct an effective acoustic metric in the optical
sector: we cannot freely tune spatial and temporal curvature to fit the
light and clock tests without spoiling the orbital fit.

\subsection{Numerical confirmation (pointer to Paper~1)}
\label{subsec:paper1-numerics}

The analytic results summarized above were tested in Paper~1 against two
classes of numerical experiments, which we briefly recall here.

The first class consists of reduced-orbit integrations of the Euler--Lagrange
equations derived from Eq.~\eqref{eq:paper1-L-eff}.
By rewriting the dynamics in terms of $(r,\varphi)$ and their conjugate
momenta, and using a symplectic integrator with adaptive step size, one can
track nearly Keplerian orbits over many periods while controlling numerical
drift.
For a range of eccentricities and semi-major axes relevant to Mercury-like
orbits, the measured apsidal advances satisfy
\begin{equation}
  \frac{\Delta\varphi_{\text{scalar}}}
       {\tfrac{1}{2}\Delta\varphi_{\text{GR}}}
  \approx 1,
  \qquad
  \frac{\Delta\varphi_{\text{tot}}}
       {\Delta\varphi_{\text{GR}}}
  \approx 1,
\end{equation}
with discrepancies well below the percent level and consistent with the
estimated numerical errors.
These runs confirm that the scalar-lag potential yields half of the GR
precession and that the inclusion of the kinetic prefactor with
$\beta = 3/2$ restores the full GR value.

The second class consists of full three-dimensional simulations of the
superfluid, scalar lag field, and a small ensemble of defects evolving in a
cubic domain with appropriate boundary conditions.
Here the governing equations are the discretized continuity and Euler-like
equations for the fluid, augmented by an evolution equation for the scalar
lag field and coupled defect trajectories.
These PDE simulations are more expensive but provide a useful cross-check
that the reduced-orbit description has not omitted any large collective
effects.
Within the range of parameters explored in Ref.~\cite{Norris:2025Orbits},
the PDE results agree with the reduced-orbit precession measurements and
support the identification $\cS \simeq c$ and $\beta = 3/2$ as the values
that best reproduce GR at 1PN.

In what follows we treat the effective Lagrangian~\eqref{eq:paper1-L-eff},
the potential~\eqref{eq:paper1-Phi-eff}, and the calibrated value
$\beta = 3/2$ as fixed inputs.
The task of the present paper is to show that, once the vacuum and flux-tube
structure are specified, the same superfluid-defect framework also reproduces
the 1PN optical and clock phenomenology of GR without further free parameters
in the gravitational sector.

%---------------------------------------------------------------------------
\section{Stiff $n=5$ vacuum and flux-tube mass defect}
\label{sec:n5-vacuum}

In this section we construct the background vacuum state that will
be used for the optical and clock sectors of the toy model.
The key ingredients are:

\begin{enumerate}
  \item a polytropic equation of state (EOS) for the superfluid vacuum,
  \item a flux-tube description of a mass defect as a sink in that vacuum,
  \item the resulting $1/r$ pressure and density profiles around the defect,
  \item and the induced variation of the local sound speed and refractive index
        $N(r) = c_0 / c_s(r)$.
\end{enumerate}

We will keep the derivation as simple as possible in the main text, working
with linear perturbations about a homogeneous background.  More general
polytropic indices $n$ and the detailed integral expressions are deferred to
Appendix~\ref{app:n5-optical}; here we focus on the stiff $n=5$ case that is
selected by the 1PN optical tests.


\subsection{Equation of state and hydrostatic balance}
\label{subsec:eos-hydrostatic}

We model the ``vacuum'' superfluid as a barotropic fluid with a
polytropic equation of state
\begin{equation}
  P = K \rho^n,
  \label{eq:eos-polytrope}
\end{equation}
where $P$ is the pressure, $\rho$ is the mass density, $K$ is a constant, and
$n$ is the polytropic index.
A homogeneous background state is characterized by $(\rho_0,P_0)$ satisfying
Eq.~\eqref{eq:eos-polytrope}.
Linear perturbations about this background propagate with sound speed
\begin{equation}
  c_0^2 \equiv \left.\frac{\partial P}{\partial \rho}\right|_{\rho_0}
  = n K \rho_0^{\,n-1}.
\end{equation}
We will ultimately identify $c_0$ with the effective light speed $c$ in the
GR-matching limit, but for now we keep the notation $c_0$ to emphasize that
it is a property of the vacuum state.

In the presence of a central mass $M$, the background superfluid is not
exactly homogeneous: the flux-tube defect that represents $M$ creates a
pressure deficit and a corresponding density deficit in the surrounding
vacuum.
Far from the throat, where the flow is slow and nearly static, the
equilibrium of the fluid is governed by hydrostatic balance:
\begin{equation}
  \frac{1}{\rho(r)}\,\frac{\mathrm{d}P}{\mathrm{d}r}
  = \frac{\mathrm{d}\Phi}{\mathrm{d}r},
  \label{eq:hydrostatic-balance}
\end{equation}
where $\Phi(r)$ is the effective gravitational potential generated by the
defect.
In the toy model, the far-field potential takes the usual Newtonian form
\begin{equation}
  \Phi(r) = -\frac{\mu}{r},
  \qquad
  \mu \equiv GM,
\end{equation}
so that
\begin{equation}
  \frac{\mathrm{d}\Phi}{\mathrm{d}r}
  = \frac{\mu}{r^2}.
\end{equation}
In the present toy model this $1/r$ law should be read as a
\emph{constitutive postulate}: we \emph{define} a flux-tube defect of mass
$M$ to be an object whose far-field coupling to the vacuum pressure
reproduces the Newtonian potential $\Phi = -GM/r$.
The hydrostatic relation Eq.~\eqref{eq:hydrostatic-balance} is therefore a
hydrodynamic repackaging of that assumption, not an attempt to derive
Newtonian gravity from the superfluid alone.

In the weak-field regime relevant for 1PN tests, the fractional density
perturbation $\delta\rho/\rho_0$ is small.
To leading order we may therefore replace $\rho(r)$ by $\rho_0$ on the left
hand side of Eq.~\eqref{eq:hydrostatic-balance}, obtaining
\begin{equation}
  \frac{1}{\rho_0}\,\frac{\mathrm{d}P}{\mathrm{d}r}
  \simeq \frac{\mu}{r^2}.
  \label{eq:hydrostatic-linear}
\end{equation}
Integrating outward from some reference radius and choosing the integration
constant so that the perturbation vanishes at spatial infinity
sets the radial dependence of the pressure perturbation.
We will use this to define the flux-tube ``mass defect'' in the next
subsection.


\subsection{Flux-tube defect and pressure profile}
\label{subsec:flux-tube}

In the superfluid-defect ontology, a mass $M$ is not a point source but a
flux tube: a narrow throat on the brane that connects to a 4D interior and
acts as a sink for superfluid.
On the brane, the time-averaged picture is that of a spherically symmetric
radial inflow toward the throat, with an associated pressure deficit and
density deficit in the surrounding vacuum.
The Newtonian $1/r^2$ force on nearby defects is encoded in the potential
$\Phi(r)$, while the ``vacuum structure'' appears as a modification of the
pressure and density profiles.

It is important to distinguish the internal topology of the defect from its
hydrodynamic profile on the brane.
While the defect extends as a flux tube into the 4D bulk, its intersection
with the 3D brane manifests as an effectively spherical point sink.
The added-mass calculation and shape-sensitivity analysis in
Appendix~\ref{app:dipole-derivation} show that this spherical assumption is
not innocuous: ellipsoidal or cylinder-like voids shift the hydrodynamic
added-mass coefficient away from $\kappa_{\mathrm{add}} = 1/2$.
In particular, a cylindrical cross-section would give
$\kappa_{\mathrm{add}} = 1$ (broadside) or $\kappa_{\mathrm{add}} = 0$
(end-on), corresponding to precession factors $7$ or $5$ rather than the
observed GR value $6$.
In what follows, the terminology ``flux tube'' therefore refers to the 4D
interior topology, while the 3D hydrodynamic cross-section on the brane is
assumed to be a stiff sphere.

Treating the far field as quasi-static and using the linearized hydrostatic
balance Eq.~\eqref{eq:hydrostatic-linear}, we integrate
\begin{equation}
  \frac{\mathrm{d}P}{\mathrm{d}r}
  \simeq \rho_0\,\frac{\mu}{r^2}
\end{equation}
from $r$ to $\infty$ and define the pressure perturbation
$\Delta P(r) \equiv P(r) - P_0$ relative to the asymptotic value $P_0$.
This yields
\begin{equation}
  \Delta P(r)
  = - \frac{\mu \rho_0}{r}
  = - \frac{GM \rho_0}{r}.
  \label{eq:deltaP}
\end{equation}
The negative sign reflects the fact that the throat creates tension or
suction in the surrounding vacuum: the pressure is lower near the defect
than in the far field.

Equation~\eqref{eq:deltaP} is the central ``mass defect'' profile for the
stiff vacuum.
It states that a flux-tube mass pulls down the pressure in a $1/r$ fashion,
with an amplitude fixed by $GM$ and the background density $\rho_0$.
In the next subsection we translate this pressure deficit into a density
deficit, a sound-speed deficit, and ultimately a refractive index profile
$N(r)$.
The electromagnetic flux-tube geometry (throat radius $a$, depth $L$, and
their ratio) will become important when we discuss the full EM sector, but
for the gravitational optics considered here only the far-field $1/r$
behavior is needed.


\subsection{Density, sound speed, and refractive index}
\label{subsec:rho-cs-N}

We now connect the pressure profile~\eqref{eq:deltaP} to the density and
sound-speed structure of the vacuum around the defect.

For small perturbations about the homogeneous background, the barotropic
relation between $P$ and $\rho$ implies
\begin{equation}
  \Delta P(r) \simeq c_0^2\,\Delta\rho(r),
\end{equation}
where $\Delta\rho(r) \equiv \rho(r) - \rho_0$ and $c_0$ is the background
sound speed.
Combining this with Eq.~\eqref{eq:deltaP} gives
\begin{equation}
  \frac{\Delta\rho(r)}{\rho_0}
  = \frac{\Delta P(r)}{\rho_0 c_0^2}
  = -\,\frac{\mu}{c_0^2 r}
  = -\,\frac{GM}{c_0^2 r}.
  \label{eq:deltarho}
\end{equation}
Thus the flux-tube defect carves out a $1/r$ density deficit in the stiff
vacuum.

The local sound speed is determined by the EOS via
\begin{equation}
  c_s^2(\rho)
  = \frac{\mathrm{d}P}{\mathrm{d}\rho}
  = n K \rho^{\,n-1}.
\end{equation}
Expanding about $\rho_0$ and keeping only linear terms in
$\Delta\rho/\rho_0$ yields
\begin{equation}
  c_s^2(\rho_0 + \Delta\rho)
  \simeq c_0^2\left[1 + (n-1)\frac{\Delta\rho}{\rho_0}\right].
\end{equation}
Taking a square root and linearizing once more, we find
\begin{equation}
  \frac{\Delta c_s}{c_0}
  \equiv \frac{c_s - c_0}{c_0}
  \simeq \frac{n-1}{2}\,\frac{\Delta\rho}{\rho_0}.
  \label{eq:deltacs-rho}
\end{equation}
Substituting Eq.~\eqref{eq:deltarho}, we obtain
\begin{equation}
  \frac{\Delta c_s}{c_0}
  \simeq -\,\frac{n-1}{2}\,\frac{GM}{c_0^2 r}.
  \label{eq:deltacs}
\end{equation}
For any polytropic index $n>1$, the sound speed is reduced near the defect
($\Delta c_s < 0$), reflecting the lower density in the throat's vicinity.

Light-like excitations in the analogue model follow rays whose local phase
speed is the sound speed $c_s(r)$.
It is therefore natural to define an effective refractive index
\begin{equation}
  N(r)
  \equiv \frac{c_0}{c_s(r)}.
\end{equation}
Inserting Eq.~\eqref{eq:deltacs} and expanding to first order in
$\Delta c_s/c_0$ yields
\begin{equation}
  N(r)
  \simeq 1 - \frac{\Delta c_s}{c_0}
  \simeq 1 + \frac{n-1}{2}\,\frac{GM}{c_0^2 r}.
  \label{eq:N-general-n}
\end{equation}
We see that the polytropic index $n$ controls the strength of the radial
gradient of $N(r)$: stiffer vacua (larger $n$) produce stronger refraction
for a given mass $M$.

The case of interest for this paper is the super-stiff $n=5$ polytrope.
Setting $n=5$ in Eq.~\eqref{eq:N-general-n} gives
\begin{equation}
  N_{n=5}(r)
  \simeq 1 + 2\,\frac{GM}{c_0^2 r}.
  \label{eq:N-n5}
\end{equation}
This is the refractive index profile we will use in the lensing and Shapiro
calculations of Sections~\ref{sec:lensing} and~\ref{sec:shapiro}.
In the 1PN matching limit we identify $c_0$ with $c$, so that the coefficient
in Eq.~\eqref{eq:N-n5} is directly comparable to the GR result.


\subsection{Physical interpretation}
\label{subsec:n5-physics-story}

The picture that emerges from this construction is simple to state.

We treat the vacuum as a super-stiff polytropic superfluid with
equation of state $P \propto \rho^5$.
A mass is represented by a flux-tube throat that drains fluid into a 4D
interior.
On the brane, this appears as a spherically symmetric sink whose
time-averaged effect is to pull down the pressure in a $1/r$ pattern:
$\Delta P(r) = - GM \rho_0 / r$.
The barotropic EOS then forces a matching $1/r$ density deficit,
$\Delta\rho/\rho_0 = -GM/(c_0^2 r)$, and a corresponding reduction in
the local sound speed, $\Delta c_s/c_0 \propto -GM/(c_0^2 r)$.
Because light-like excitations propagate at $c_s(r)$, they see an effective
refractive index
\begin{equation}
  N(r) = \frac{c_0}{c_s(r)}
  \simeq 1 + 2\,\frac{GM}{c_0^2 r}
\end{equation}
for the $n=5$ vacuum.

In the rest of the paper we will show that this single $1/r$ profile,
derived from the flux-tube pressure deficit in a stiff vacuum, is enough to
account for the classic 1PN optical and clock effects of GR.
Light bending and Shapiro delay follow from the way rays are deflected and
slowed by the radial gradient of $N(r)$, while gravitational redshift
follows from the way defect-based clocks respond to the same density
deficit.
Orbits, lenses, and clocks are thus different probes of the same underlying
vacuum structure.

%---------------------------------------------------------------------------
\section{Gravitational lensing from the refractive index profile}
\label{sec:lensing}

In this section we show how the refractive index profile $N(r)$ derived in
Section~\ref{sec:n5-vacuum} leads to the familiar GR light-bending angle
for a point mass.
We work in the weak-field, small-angle regime appropriate to solar-system
and quasar-lensing tests, treating lightlike excitations as rays propagating
through an inhomogeneous medium with phase speed $c_s(r)$ and refractive
index $N(r) = c_0 / c_s(r)$.
The calculation is standard in spirit---a Fermat-principle treatment of
light in a graded-index medium---but here the index profile is fixed by the
superfluid vacuum and flux-tube mass defect.


\subsection{Refractive index profile as input}
\label{subsec:lensing-N}

From Eq.~\eqref{eq:N-n5}, the stiff $n=5$ vacuum with a flux-tube mass $M$
produces, to first order in $GM/(c_0^2 r)$, a spherically symmetric
refractive index
\begin{equation}
  N(r)
  \simeq 1 + 2\,\frac{GM}{c_0^2 r},
  \qquad r = |\mathbf{x}|.
  \label{eq:lensing-N}
\end{equation}
Equivalently, it is often convenient to work with the logarithm,
\begin{equation}
  \ln N(r)
  \simeq 2\,\frac{GM}{c_0^2 r},
  \label{eq:lensing-lnN}
\end{equation}
since ray deflection is controlled by gradients of $\ln N$ rather than $N$
itself.

In the 1PN matching limit we identify $c_0$ with $c$, so we will replace
$c_0 \to c$ in what follows.
All of the lensing and Shapiro results in this section and the next are
understood to be accurate to leading order in $GM/(c^2 r)$, with higher
orders neglected.


\subsection{Thin-lens geometry and deflection integral}
\label{subsec:lensing-geometry}

Consider a light ray that passes the mass at a minimum distance (impact
parameter) $b$.
In the weak-field, small-angle limit, the actual curved trajectory can be
approximated as a straight line when computing the leading-order deflection.
We therefore adopt a coordinate system in which the unperturbed ray travels
along the $z$-axis, with closest approach at $z=0$, and the mass located at
the origin.
The radial distance from the mass to a point on the unperturbed ray is then
\begin{equation}
  r(z) = \sqrt{b^2 + z^2}.
\end{equation}

In a medium with refractive index $N(\mathbf{x})$ that varies slowly on a
wavelength scale, ray trajectories obey Fermat's principle; equivalently,
they satisfy the eikonal (geometric-optics) approximation.
To first order in the index perturbation, the total deflection angle
$\Delta\theta$ accumulated as the ray passes the mass can be written as
\begin{equation}
  \Delta\theta
  \simeq \int_{-\infty}^{+\infty} \nabla_\perp \ln N\bigl(r(z)\bigr)\,\mathrm{d}z,
  \label{eq:lensing-deflection-general}
\end{equation}
where $\nabla_\perp$ denotes the component of the gradient transverse to the
unperturbed ray direction (i.e.\ along the impact-parameter direction).
The derivation of Eq.~\eqref{eq:lensing-deflection-general} is standard:
one linearizes the ray equations in $\nabla N$ and integrates the transverse
acceleration along the unperturbed path.

For the spherically symmetric case at hand, the transverse gradient can be
expressed as
\begin{equation}
  \nabla_\perp \ln N
  = \frac{\partial \ln N}{\partial r}\,\frac{\partial r}{\partial b}\,
    \hat{\mathbf{e}}_\perp,
\end{equation}
where $\hat{\mathbf{e}}_\perp$ is a unit vector orthogonal to the ray
direction and
\begin{equation}
  \frac{\partial r}{\partial b}
  = \frac{b}{\sqrt{b^2 + z^2}}
  = \frac{b}{r}.
\end{equation}
The magnitude of the transverse gradient is therefore
\begin{equation}
  \left|\nabla_\perp \ln N\right|
  = \left|\frac{\mathrm{d}\ln N}{\mathrm{d}r}\right|\,\frac{b}{r}.
\end{equation}
Since the deflection angle is small, we may identify
$\Delta\theta$ with the integral of this magnitude along the path.


\subsection{Analytic evaluation of the bending angle}
\label{subsec:lensing-analytic}

Using the explicit form~\eqref{eq:lensing-lnN}, we have
\begin{equation}
  \frac{\mathrm{d}\ln N}{\mathrm{d}r}
  \simeq -2\,\frac{GM}{c^2}\,\frac{1}{r^2},
\end{equation}
and hence
\begin{equation}
  \left|\nabla_\perp \ln N\right|
  \simeq 2\,\frac{GM}{c^2}\,\frac{b}{r^3}.
\end{equation}
Substituting $r(z) = \sqrt{b^2 + z^2}$, we obtain
\begin{equation}
  \left|\nabla_\perp \ln N\bigl(r(z)\bigr)\right|
  \simeq 2\,\frac{GM}{c^2}\,
    \frac{b}{\bigl(b^2 + z^2\bigr)^{3/2}}.
\end{equation}
The total deflection angle is then
\begin{equation}
  \Delta\theta
  \simeq \int_{-\infty}^{+\infty}
          2\,\frac{GM}{c^2}\,
          \frac{b}{\bigl(b^2 + z^2\bigr)^{3/2}}
        \,\mathrm{d}z.
  \label{eq:lensing-deflection-integral}
\end{equation}

The integral in Eq.~\eqref{eq:lensing-deflection-integral} is elementary.
Define
\begin{equation}
  I(b)
  \equiv \int_{-\infty}^{+\infty}
          \frac{b}{\bigl(b^2 + z^2\bigr)^{3/2}}
        \,\mathrm{d}z.
\end{equation}
Making the substitution $z = b \tan\psi$ with $\psi \in (-\pi/2,\pi/2)$,
we have
\begin{equation}
  \mathrm{d}z = b \sec^2\psi\,\mathrm{d}\psi,
  \qquad
  b^2 + z^2 = b^2(1 + \tan^2\psi) = b^2 \sec^2\psi,
\end{equation}
and thus
\begin{equation}
  I(b)
  = \int_{-\pi/2}^{+\pi/2}
      \frac{b}{\bigl(b^2 \sec^2\psi\bigr)^{3/2}}
      \,b \sec^2\psi\,\mathrm{d}\psi
  = \int_{-\pi/2}^{+\pi/2}
      \frac{b^2 \sec^2\psi}{b^3 \sec^3\psi}
      \,\mathrm{d}\psi
  = \int_{-\pi/2}^{+\pi/2}
      \frac{1}{b}\cos\psi\,\mathrm{d}\psi.
\end{equation}
Evaluating the last integral gives
\begin{equation}
  I(b)
  = \frac{1}{b}\,\left[\sin\psi\right]_{-\pi/2}^{+\pi/2}
  = \frac{1}{b}\,\bigl(1 - (-1)\bigr)
  = \frac{2}{b}.
\end{equation}
Substituting back into Eq.~\eqref{eq:lensing-deflection-integral}, we obtain
the total deflection angle
\begin{equation}
  \Delta\theta
  \simeq 2\,\frac{GM}{c^2}\,I(b)
  = 2\,\frac{GM}{c^2}\,\frac{2}{b}
  = \frac{4GM}{b c^2}.
  \label{eq:lensing-dtheta-result}
\end{equation}

Equation~\eqref{eq:lensing-dtheta-result} is precisely the standard
Schwarzschild light-bending angle for a point mass in the weak-field,
small-angle limit.
It is important to emphasize that in the present construction the
coefficient $4GM/(b c^2)$ is not imposed by hand: it emerges from the
combination of (i) the flux-tube pressure deficit $\Delta P(r) \propto -1/r$
and (ii) the stiff $n=5$ EOS, which together determine the gradient of
$N(r)$.
For other polytropic indices $n$, the coefficient in
Eq.~\eqref{eq:lensing-dtheta-result} would differ, leading to
$\Delta\theta \propto (n-1)GM/(b c^2)$; the GR value singles out $n=5$
within this family.


\subsection{Mapping to PPN $\gamma$}
\label{subsec:lensing-ppn}

In the parametrized post-Newtonian (PPN) formalism, the leading-order
deflection angle of a light ray by a point mass $M$ is given by
\begin{equation}
  \Delta\theta_{\text{PPN}}
  = \frac{2(1+\gamma)GM}{b c^2},
\end{equation}
where $\gamma$ is the PPN parameter that measures the amount of space
curvature produced by unit rest mass.
General Relativity predicts $\gamma = 1$, yielding
$\Delta\theta_{\text{GR}} = 4GM/(b c^2)$.

Comparing Eq.~\eqref{eq:lensing-dtheta-result} with the PPN expression,
\begin{equation}
  \frac{4GM}{b c^2}
  \quad\stackrel{!}{=}\quad
  \frac{2(1+\gamma)GM}{b c^2},
\end{equation}
we immediately read off
\begin{equation}
  \gamma = 1.
\end{equation}
In other words, the \emph{PPN parameter inferred from the observed light
deflection} is the same as in GR.
Because the analogue model encodes curvature purely through the refractive
index---and hence through the spatial part of the optical metric, with
$g_{tt}^{\text{(opt)}} = -1$---the spatial metric component must carry
twice the usual Schwarzschild curvature in order to reproduce this
$\gamma = 1$ bending.
This trade-off between temporal and spatial curvature is made explicit in
the PPN mapping of Appendix~\ref{app:ppn-metric}.
Thus, insofar as the analogue model is concerned with light bending, the
stiff $n=5$ superfluid vacuum with a flux-tube mass defect is 1PN-equivalent
to Schwarzschild in the sense of reproducing the GR value of $\gamma$.

It is worth emphasizing that this conclusion rests only on the
\emph{optical} sector of the toy model: the derivation uses the refractive
index $N(r)$ and the geometry of rays, but does not make any assumptions
about the motion of massive defects beyond the identification $c_0 \to c$.
The orbital sector, and hence the parameter $\beta$, enter the story only
when we demand a single emergent metric that governs both massive and
massless probes, as discussed in Section~\ref{sec:soliton-geodesics} and
Appendix~\ref{app:ppn-metric}.


\subsection{Numerical ray-tracing}
\label{subsec:lensing-numerics}

For completeness, we briefly sketch how the analytic result
Eq.~\eqref{eq:lensing-dtheta-result} can be checked numerically within the
toy model.

In the geometric-optics limit, rays in an inhomogeneous medium with phase
speed $c_s(\mathbf{x})$ can be described by the Hamiltonian
\begin{equation}
  H(\mathbf{x},\mathbf{k})
  = c_s(\mathbf{x})\,|\mathbf{k}|,
\end{equation}
where $\mathbf{k}$ is the wavevector.
The ray equations are then
\begin{equation}
  \dot{\mathbf{x}} = \frac{\partial H}{\partial \mathbf{k}}
  = c_s(\mathbf{x})\,\frac{\mathbf{k}}{|\mathbf{k}|},
  \qquad
  \dot{\mathbf{k}} = -\frac{\partial H}{\partial \mathbf{x}}
  = -|\mathbf{k}|\,\nabla c_s(\mathbf{x}),
\end{equation}
with overdots denoting derivatives with respect to an affine parameter
along the ray.
Equivalently, one can work directly with $N(\mathbf{x}) = c_0 / c_s(\mathbf{x})$.

In the spherically symmetric case with $c_s(r)$ (or $N(r)$) given by
Eqs.~\eqref{eq:deltacs} and~\eqref{eq:lensing-N}, these equations reduce to
a two-dimensional system in the $(r,\varphi)$ plane.
Starting from initial conditions that approximate a plane wave incident from
$z = -\infty$ with impact parameter $b$, one integrates the ray equations
forward until the ray exits the region where $N(r)$ differs appreciably
from unity.
The asymptotic outgoing direction is then compared to the incoming direction
to extract the numerical deflection angle $\Delta\theta_{\text{num}}(b)$.

For a range of impact parameters $b$ large compared to the throat scale but
small enough that $GM/(b c^2)$ is not negligible, one finds that the ratio
\begin{equation}
  \frac{\Delta\theta_{\text{num}}(b)}
       {\Delta\theta_{\text{analytic}}(b)}
  \equiv
  \frac{\Delta\theta_{\text{num}}(b)}
       {4GM/(b c^2)}
\end{equation}
remains close to unity, with deviations consistent with the numerical
integration error and the neglect of higher-order terms in
$GM/(c^2 r)$.
This provides an internal consistency check that the geometric-optics
treatment of lensing in the $n=5$ vacuum is reliable and that no large
corrections arise from the approximations used in the analytic derivation.


\subsection{Assumptions and limitations}
\label{subsec:lensing-assumptions}

The analysis in this section rests on several important assumptions:

\begin{itemize}
  \item \textbf{Weak field and small angles.}
        We have assumed $GM/(c^2 r) \ll 1$ throughout the region where the
        ray passes, and we have linearized in this parameter.
        Strong-lensing phenomena, multiple images, and higher-order
        post-Newtonian corrections are beyond the scope of the present
        treatment.

  \item \textbf{Pure refraction.}
        The deflection has been attributed entirely to spatial gradients in
        the phase speed $c_s(r)$, encoded in $N(r)$.
        We have not included any explicit ``drag'' of rays by bulk flows of
        the superfluid; in the language of Section~\ref{sec:degeneracies},
        we are working in the pure-refraction branch of the model.

  \item \textbf{Static, spherically symmetric background.}
        The refractive index $N(r)$ was derived under the assumption of a
        static, spherically symmetric flux-tube defect.
        Time-dependent flows, rotating defects, or non-spherical mass
        distributions would require a more general treatment, potentially
        including vector and tensor perturbations of the effective metric.

  \item \textbf{Geometric-optics limit.}
        We have assumed that the wavelength of the excitations is much
        smaller than the length scale over which $N(r)$ varies, so that ray
        tracing is appropriate.
        Near the throat or in regions of rapidly varying density, wave
        effects such as diffraction and mode conversion could become
        important.
\end{itemize}

Within these assumptions, the stiff $n=5$ vacuum with a flux-tube mass
defect reproduces the GR prediction for gravitational lensing at 1PN.
In the next section we show that the same refractive index profile also
reproduces the GR Shapiro time delay with the correct coefficient, further
supporting the identification of this vacuum as the appropriate analogue of
the Schwarzschild geometry in the optical sector.

%---------------------------------------------------------------------------
\section{Shapiro delay from the same refractive index profile}
\label{sec:shapiro}

The Shapiro time delay is the excess travel time experienced by a light
signal that passes near a gravitating mass, relative to the time it would
take to traverse the same coordinate distance in flat space.
In the present toy model, this effect arises from the same refractive index
profile $N(r)$ that produced the light-bending angle in
Section~\ref{sec:lensing}: rays propagate more slowly in the region where
$c_s(r) < c_0$, and the integrated slow-down along the path yields a
logarithmic delay.


\subsection{Setup and geometry}
\label{subsec:shapiro-setup}

We consider a standard radar-type configuration: an emitter at some large
distance from the mass sends a pulse toward the mass, the pulse passes by
at a minimum impact parameter $b$, reflects off a distant target, and then
returns to the emitter.
For the analytic calculation it is convenient to treat the one-way trip
from emitter to receiver; the round-trip delay is then twice the one-way
result.

As in Section~\ref{sec:lensing}, we adopt a coordinate system in which the
unperturbed ray trajectory lies along the $z$-axis, with closest approach
to the mass at $z=0$ and impact parameter $b$ in the transverse direction.
The mass is located at the origin, and the radial distance from the mass is
\begin{equation}
  r(z) = \sqrt{b^2 + z^2}.
\end{equation}
Let $z = -Z_\mathrm{E}$ denote the emitter location and $z = +Z_\mathrm{R}$
the receiver location, with $Z_\mathrm{E}, Z_\mathrm{R} \gg b$.
The corresponding radial distances from the mass are approximately
\begin{equation}
  r_\mathrm{E} \simeq \sqrt{b^2 + Z_\mathrm{E}^2},
  \qquad
  r_\mathrm{R} \simeq \sqrt{b^2 + Z_\mathrm{R}^2}.
\end{equation}

In flat space, with a uniform propagation speed $c_0$, the time required
for the signal to travel from $z=-Z_\mathrm{E}$ to $z=+Z_\mathrm{R}$ is
\begin{equation}
  t_0
  = \frac{1}{c_0}\,\bigl(Z_\mathrm{E} + Z_\mathrm{R}\bigr).
\end{equation}
In the presence of the mass defect, the local propagation speed is
$c_s(r) = c_0 / N(r)$, so the actual travel time is modified.


\subsection{Time-of-flight integral}
\label{subsec:shapiro-integral}

In the geometric-optics limit, the travel time along a ray in a medium with
refractive index $N(\mathbf{x})$ is given by
\begin{equation}
  t
  = \frac{1}{c_0}\,\int N\bigl(\mathbf{x}(s)\bigr)\,\mathrm{d}s,
\end{equation}
where $s$ is the arclength along the ray.
To leading order in the index perturbation and for small deflection angles,
we can approximate the path as a straight line and replace $\mathrm{d}s$ by
$\mathrm{d}z$.
Using the stiff-vacuum refractive index profile
\begin{equation}
  N(r)
  \simeq 1 + 2\,\frac{GM}{c_0^2 r},
\end{equation}
the one-way travel time from emitter to receiver becomes
\begin{equation}
  t
  \simeq \frac{1}{c_0}
         \int_{-Z_\mathrm{E}}^{+Z_\mathrm{R}}
         \left[1 + 2\,\frac{GM}{c_0^2 r(z)}\right]
         \mathrm{d}z.
  \label{eq:shapiro-t-integral}
\end{equation}
Splitting off the flat part, we have
\begin{equation}
  t
  = t_0 + \Delta t,
\end{equation}
with the Shapiro delay
\begin{equation}
  \Delta t
  \equiv t - t_0
  \simeq \frac{2GM}{c_0^3}
         \int_{-Z_\mathrm{E}}^{+Z_\mathrm{R}}
         \frac{\mathrm{d}z}{r(z)}.
  \label{eq:shapiro-dt-integral}
\end{equation}
The remaining integral depends on the geometry through $b$, $Z_\mathrm{E}$,
and $Z_\mathrm{R}$.


\subsection{Analytic result and comparison to GR}
\label{subsec:shapiro-analytic}

The integral in Eq.~\eqref{eq:shapiro-dt-integral} is straightforward:
\begin{equation}
  I_\mathrm{tof}(b; Z_\mathrm{E}, Z_\mathrm{R})
  \equiv \int_{-Z_\mathrm{E}}^{+Z_\mathrm{R}}
         \frac{\mathrm{d}z}{\sqrt{b^2 + z^2}}.
\end{equation}
A convenient antiderivative is
\begin{equation}
  \int \frac{\mathrm{d}z}{\sqrt{b^2 + z^2}}
  = \ln\!\bigl(z + \sqrt{b^2 + z^2}\bigr) + C,
\end{equation}
so that
\begin{align}
  I_\mathrm{tof}(b; Z_\mathrm{E}, Z_\mathrm{R})
  &= \ln\!\bigl(Z_\mathrm{R} + \sqrt{b^2 + Z_\mathrm{R}^2}\bigr)
   - \ln\!\bigl(-Z_\mathrm{E} + \sqrt{b^2 + Z_\mathrm{E}^2}\bigr)
  \nonumber\\[0.5ex]
  &= \ln\!\left[
       \frac{Z_\mathrm{R} + \sqrt{b^2 + Z_\mathrm{R}^2}}
            {-Z_\mathrm{E} + \sqrt{b^2 + Z_\mathrm{E}^2}}
     \right].
  \label{eq:shapiro-I-exact}
\end{align}
In the regime of interest, the endpoints are far from the mass compared to
the impact parameter ($Z_\mathrm{E}, Z_\mathrm{R} \gg b$), so we can use
\begin{equation}
  \sqrt{b^2 + Z^2} \simeq |Z| + \frac{b^2}{2|Z|}
\end{equation}
to approximate the numerator and denominator of Eq.~\eqref{eq:shapiro-I-exact}.
Note that in our geometry $Z_\mathrm{R} > 0$ and $Z_\mathrm{E} < 0$, so
$|Z_\mathrm{R}| = Z_\mathrm{R}$ and $|Z_\mathrm{E}| = -Z_\mathrm{E}$, which
is used in the approximations below.
For $Z>0$ we have
\begin{equation}
  Z + \sqrt{b^2 + Z^2}
  \simeq Z + Z\left(1 + \frac{b^2}{2Z^2}\right)
  = 2Z + \mathcal{O}\!\left(\frac{b^2}{Z}\right),
\end{equation}
and for $-Z<0$ we similarly find
\begin{equation}
  -Z + \sqrt{b^2 + Z^2}
  \simeq -Z + Z\left(1 + \frac{b^2}{2Z^2}\right)
  = \frac{b^2}{2Z} + \mathcal{O}\!\left(\frac{b^4}{Z^3}\right).
\end{equation}
Applying these approximations to Eq.~\eqref{eq:shapiro-I-exact} yields
\begin{equation}
  I_\mathrm{tof}(b; Z_\mathrm{E}, Z_\mathrm{R})
  \simeq \ln\!\left[
           \frac{2Z_\mathrm{R}}{b^2/(2Z_\mathrm{E})}
         \right]
  = \ln\!\left(\frac{4 Z_\mathrm{E} Z_\mathrm{R}}{b^2}\right).
\end{equation}
Replacing $Z_\mathrm{E}$ and $Z_\mathrm{R}$ by the corresponding radial
distances $r_\mathrm{E}$ and $r_\mathrm{R}$ (to leading order they are
interchangeable in the logarithm), we have
\begin{equation}
  I_\mathrm{tof}(b; Z_\mathrm{E}, Z_\mathrm{R})
  \simeq \ln\!\left(
          \frac{4 r_\mathrm{E} r_\mathrm{R}}{b^2}
        \right).
\end{equation}
Substituting into Eq.~\eqref{eq:shapiro-dt-integral}, we obtain the one-way
Shapiro delay
\begin{equation}
  \Delta t
  \simeq \frac{2GM}{c_0^3}\,
         \ln\!\left(\frac{4 r_\mathrm{E} r_\mathrm{R}}{b^2}\right).
  \label{eq:shapiro-dt-result}
\end{equation}
In the 1PN matching limit $c_0 \to c$, this is precisely the standard GR
expression for the one-way Shapiro delay.

In the PPN formalism, the leading radial logarithmic contribution to the
one-way delay for a light signal traveling from $r_\mathrm{E}$ to
$r_\mathrm{R}$ with impact parameter $b$ is
\begin{equation}
  \Delta t_{\text{PPN}}
  = (1 + \gamma)\,\frac{GM}{c^3}\,
    \ln\!\left(\frac{4 r_\mathrm{E} r_\mathrm{R}}{b^2}\right),
\end{equation}
where, as in Section~\ref{sec:lensing}, $\gamma$ measures the amount of
space curvature per unit rest mass.
Comparing with Eq.~\eqref{eq:shapiro-dt-result},
\begin{equation}
  \frac{2GM}{c^3}
  \quad\stackrel{!}{=}\quad
  (1+\gamma)\,\frac{GM}{c^3},
\end{equation}
we again find
\begin{equation}
  \gamma = 1.
\end{equation}
Thus the same refractive index profile $N(r)$ that reproduces the GR
light-bending angle also reproduces the GR Shapiro delay with the correct
PPN coefficient.


\subsection{Numerical checks}
\label{subsec:shapiro-numerics}

As with lensing, the analytic Shapiro delay can be checked numerically
within the superfluid-defect model.

One approach is to integrate the time-of-flight integral
Eq.~\eqref{eq:shapiro-t-integral} directly on a discrete grid.
For a given impact parameter $b$ and endpoints $z=-Z_\mathrm{E}$,
$z=+Z_\mathrm{R}$, one evaluates
\begin{equation}
  t_\mathrm{num}
  = \frac{1}{c_0}\sum_i N\bigl(r(z_i)\bigr)\,\Delta z,
\end{equation}
with $r(z_i) = \sqrt{b^2 + z_i^2}$ and $\Delta z$ chosen small enough that
$N(r)$ varies slowly between adjacent points.
Subtracting the flat-space time $t_0$ yields a numerical delay
$\Delta t_\mathrm{num}$, which can be compared to the analytic prediction
Eq.~\eqref{eq:shapiro-dt-result}.

A more complete treatment, consistent with the ray-tracing picture of
Section~\ref{sec:lensing}, solves the Hamiltonian ray equations in the
$N(r)$ background and accumulates the elapsed coordinate time along the
ray:
\begin{equation}
  t_\mathrm{num}
  = \int \frac{N\bigl(\mathbf{x}(\lambda)\bigr)}{c_0}
         \left|\frac{\mathrm{d}\mathbf{x}}{\mathrm{d}\lambda}\right|
         \mathrm{d}\lambda,
\end{equation}
where $\lambda$ is an affine parameter along the ray.
For impact parameters in the weak-field regime and endpoints far from the
mass, one finds that the ratio
\begin{equation}
  \frac{\Delta t_\mathrm{num}}
       {\Delta t_{\text{analytic}}}
  \equiv
  \frac{\Delta t_\mathrm{num}}
       {2GM c^{-3}
        \ln(4 r_\mathrm{E} r_\mathrm{R} / b^2)}
\end{equation}
remains close to unity, with deviations consistent with numerical
discretization errors and the neglect of higher-order terms in
$GM/(c^2 r)$.

These numerical checks reinforce the analytic conclusion: in the stiff
$n=5$ vacuum with a flux-tube mass defect, the same refractive index
profile that yields the GR light-bending angle also reproduces the GR
Shapiro delay, with no additional freedom once $N(r)$ is fixed by the
superfluid equation of state and the flux-tube geometry.

%---------------------------------------------------------------------------
\section{Gravitational redshift and clock rates}
\label{sec:redshift}

Having established how the stiff $n=5$ vacuum and flux-tube defect reshape
the pressure, density, and sound speed of the superfluid, we now turn to
gravitational redshift.
In the toy model, clocks tick slower near a mass not because ``time itself''
is stretched, but because the defects that make up the clock become
slightly lighter in the lower-density vacuum.
The key idea is that any clock whose frequency is set by a mass scale $m$
will experience a fractional shift
\begin{equation}
  \frac{\delta\omega}{\omega_0}
  = \frac{\delta m}{m_0}
  = \frac{\delta\rho}{\rho_0},
\end{equation}
and the density deficit induced by the flux-tube defect is
$\delta\rho/\rho_0 \simeq -GM/(r c^2)$ in the weak-field regime.
This reproduces the standard GR weak-field redshift.


\subsection{Density perturbations and defect mass}
\label{subsec:redshift-density}

From the hydrostatic analysis in Section~\ref{sec:n5-vacuum}, the presence
of a flux-tube mass $M$ creates a pressure deficit
$\Delta P(r) = -GM\rho_0/r$ and, in the weak-field limit, a corresponding
density deficit
\begin{equation}
  \frac{\delta\rho(r)}{\rho_0}
  \equiv \frac{\rho(r) - \rho_0}{\rho_0}
  = -\frac{GM}{r c^2}
  + \mathcal{O}\!\left(\frac{G^2M^2}{c^4 r^2}\right),
  \label{eq:redshift-deltarho}
\end{equation}
where we have set $c_0 \to c$ for comparison with GR.  This is the same
profile that underlies lensing and Shapiro delay.

The defects that represent matter are not rigid ``beads'' of fixed mass:
they are cavitation structures that displace vacuum and trap field energy.
In the simplified energy bookkeeping we use here, the rest energy of a
defect scales with the local vacuum density, e.g.
\begin{equation}
  E_{\text{rest}}(r)
  \sim m(r)c^2
  \propto \rho_{\text{local}}(r)\,V_{\text{cav}}\,c^2,
\end{equation}
with $V_{\text{cav}}$ the defect volume.  To leading order we treat
$V_{\text{cav}}$ and the relevant mode structure as fixed, so that the
defect mass is proportional to the local density
\begin{equation}
  m(r) \propto \rho_{\text{local}}(r)
  \quad\Rightarrow\quad
  \frac{\delta m(r)}{m_0}
  = \frac{\delta\rho(r)}{\rho_0}.
  \label{eq:redshift-dm}
\end{equation}
Substituting Eq.~\eqref{eq:redshift-deltarho} gives
\begin{equation}
  \frac{\delta m(r)}{m_0}
  = -\frac{GM}{r c^2}
  + \mathcal{O}\!\left(\frac{G^2M^2}{c^4 r^2}\right).
  \label{eq:redshift-dm-final}
\end{equation}
Near a mass, defects---and therefore any clocks built from them---are
slightly lighter than in the asymptotic vacuum.


\subsection{Clock frequency scaling}
\label{subsec:redshift-frequency}

We now connect the position-dependent defect mass to the ticking rate of
clocks.
Consider any clock whose characteristic frequency $\omega$ is set by a mass
scale $m$.
A concrete example is an atomic clock based on a Bohr-like bound state:
for a hydrogenic atom the level spacing scales roughly as
\begin{equation}
  \omega \sim \frac{m_e e^4}{\hbar^3}
  \propto m_e,
\end{equation}
so a fractional change in the electron mass $m_e$ produces the same
fractional change in the clock frequency.

In the toy model we abstract this into the simple scaling relation
\begin{equation}
  \omega \propto m
  \quad\Rightarrow\quad
  \frac{\delta\omega}{\omega_0}
  = \frac{\delta m}{m_0}.
  \label{eq:redshift-omega-m}
\end{equation}
Combining Eqs.~\eqref{eq:redshift-dm} and~\eqref{eq:redshift-omega-m} with
the density profile~\eqref{eq:redshift-deltarho}, we obtain the chain
\begin{equation}
  \frac{\delta\omega(r)}{\omega_0}
  = \frac{\delta m(r)}{m_0}
  = \frac{\delta\rho(r)}{\rho_0}
  = -\frac{GM}{r c^2}
  + \mathcal{O}\!\left(\frac{G^2M^2}{c^4 r^2}\right).
  \label{eq:redshift-full-chain}
\end{equation}
Clocks located deeper in the potential well (smaller $r$) therefore tick
more slowly: their local frequency is reduced relative to an identical
clock at infinity by an amount proportional to $GM/(r c^2)$.

To connect with standard notation, we can write the fractional shift in
observed frequency as
\begin{equation}
  \frac{\Delta\nu}{\nu}
  \equiv \frac{\nu_{\infty} - \nu(r)}{\nu_{\infty}}
  \simeq -\frac{\delta\omega(r)}{\omega_0}
  = \frac{GM}{r c^2},
\end{equation}
so that $\Delta\nu/\nu > 0$ corresponds to a redshift (photons lose
frequency climbing out of the potential).


\subsection{Relation to the refractive index $N(r)$}
\label{subsec:redshift-N}

The gravitational redshift in this model is thus governed by the same
density profile $\delta\rho(r)$ that underlies the optical sector.
The lensing and Shapiro calculations only required the way the sound speed
responds to the density deficit: $c_s(\rho)$ and hence
$N(r) = c_0 / c_s(r)$.
For the $n=5$ EOS we found
\begin{equation}
  N(r)
  \simeq 1 + 2\,\frac{GM}{r c^2},
\end{equation}
and from this we derived the GR light-bending angle and Shapiro delay.
The redshift derivation, by contrast, depends only on the fact that the
defect mass tracks the density,
$m(r) \propto \rho_{\text{local}}(r)$, and does not require the explicit
$n=5$ form of $N(r)$; it needs only the density profile that the gravity
sector has already fixed.

In this sense, redshift is the ``$1GM$'' member of the hierarchy
\[
  \{\text{redshift},\ \text{Shapiro},\ \text{lensing}\}
  \sim \{1GM,\ 2GM,\ 4GM\},
\]
all of which trace back to the single $1/r$ density deficit created by the
flux-tube defect in a stiff vacuum: clocks slow because $m\propto\rho$
drops, signals are delayed because $c_s(\rho)$ drops, and rays bend because
$\nabla N(r)$ is nonzero.


\subsection{Limitations and observational regime}
\label{subsec:redshift-limitations}

Several caveats are important when interpreting the redshift result
Eq.~\eqref{eq:redshift-full-chain}:

\begin{itemize}
  \item \textbf{Weak-field / 1PN regime.}
        The derivation assumes $GM/(r c^2) \ll 1$ and keeps only the
        leading term in this expansion.
        Strong-field environments (near horizons or very compact objects)
        would require a non-linear treatment of both the EOS and the defect
        structure.
        In particular, for the stiff $n=5$ EOS used here one has
        $c_s \propto \rho^2$, so any region where the density is driven
        toward zero (for example near a deeply cavitating throat) would
        also drive the sound speed toward zero and suggest the emergence
        of a ``sonic horizon''; exploring that regime lies beyond the
        present 1PN analysis.

  \item \textbf{Clock model dependence.}
        The simple scaling $\omega \propto m$ is appropriate for a wide
        class of mass-based oscillators (atomic clocks, nuclear clocks,
        solid-state resonators), but not for all conceivable clocks.
        The claim is that \emph{any} clock whose characteristic frequency
        is proportional to a defect mass will exhibit the GR-like
        redshift; more exotic clocks may probe additional structure.

  \item \textbf{Use of $m(r)$ is internal only.}
        The identification $m(r) \propto \rho(r)$ is used here solely for
        the internal dynamics of clocks, not for the center-of-mass motion
        of test bodies.
        For orbital dynamics, matter is treated as solitons following
        geodesics of the emergent metric, not as Newtonian particles with
        position-dependent mass in $F = m(r)a$.
        This resolves the ``mass scaling trilemma'' that would otherwise
        arise from combining a density-dependent mass with the orbital
        sector.

  \item \textbf{No direct test of $n=5$ from redshift alone.}
        As noted above, the redshift derivation depends only on
        $\delta\rho/\rho$ and $m\propto\rho$, not on the detailed $n=5$
        scaling of $c_s(\rho)$.
        In the full model, it is the combination of lensing, Shapiro, and
        redshift that selects the stiff $n=5$ vacuum; redshift by itself
        is compatible with a wider class of EOS.
\end{itemize}

Within these limitations, the toy model reproduces the standard GR
weak-field redshift formula
\begin{equation}
  \frac{\Delta\nu}{\nu}
  = -\frac{\Delta\Phi}{c^2}
  = -\frac{GM}{r c^2}
  + \mathcal{O}\!\left(\frac{G^2M^2}{c^4 r^2}\right),
\end{equation}
but interprets it as a consequence of density-induced mass reduction of
defects, rather than as a fundamental stretching of time.

%---------------------------------------------------------------------------
\section{Soliton geodesics and the effective metric}
\label{sec:soliton-geodesics}

The results of the previous sections can be summarized as follows.
On the one hand, Paper~1 showed that the orbital motion of defects in the
superfluid-defect toy model is accurately described by an effective
Lagrangian of the form
\begin{equation}
  L
  = \frac{1}{2}\bigl[1 + \sigma(r)\bigr]
      \bigl(\dot r^2 + r^2 \dot\varphi^2\bigr)
    - \Phi_{\mathrm{eff}}(r),
  \label{eq:SG-L-eff-again}
\end{equation}
with $\Phi_{\mathrm{eff}}(r)$ and $\sigma(r)$ fixed by the scalar lag field
and the hydrodynamics of the throat.
On the other hand, the present paper has shown that the same throat and
vacuum structure produce a refractive index $N(r) = c_0/c_s(r)$ that
reproduces the GR values of the light-bending angle, Shapiro delay, and
weak-field redshift.

Taken together, these results suggest a unified geometric interpretation:
both massive defects and light-like excitations probe a common superfluid
vacuum, but they couple to different combinations of the underlying
variables $c_s(r)$, $\rho(r)$, and the flow field.
In the 1PN regime these couplings can be summarized by an effective metric
whose null sector reproduces the calibrated optics and whose timelike
sector reproduces the calibrated orbital dynamics.
In this section we make that interpretation explicit at the level of
1PN phenomenology.


\subsection{Soliton hypothesis and an emergent equivalence principle}
\label{subsec:soliton-hypothesis}

The starting point is the \emph{soliton hypothesis} for matter.
Defects in the toy model are not point particles; they are localized,
topologically protected excitations of the underlying fields---vortex
throats threaded by flux, with bound near-field structure.
To leading order, a defect can be treated as a coherent wave packet of
underlying degrees of freedom moving through the superfluid vacuum.

The dynamics of such a wave packet is governed by an eikonal or
WKB-like approximation: its center of mass follows the characteristic
curves (rays) of the underlying wave equation, just as a photon wave packet
follows null geodesics of the electromagnetic eikonal equation in a curved
spacetime.
In the analogue-gravity literature, the characteristic curves of acoustic
waves in a moving fluid are often interpreted as geodesics of an emergent
``acoustic metric''.
We adopt the same viewpoint here and elevate it to a toy-model analogue of
the strong equivalence principle:

\begin{quote}
  \emph{Localized solitonic excitations of the superfluid---defects,
  composite defects, and linear wave packets---propagate in the same
  background vacuum state, but couple to different combinations of the
  underlying fluid fields.
  Acoustic wave packets probe the bare sound-speed geometry encoded in
  $c_s(r)$, whereas defects feel a hydrodynamically dressed geometry
  built from $\rho(r)$ and the dipole flow that accompanies a moving
  void.
  In the 1PN regime both types of probe can be described as following
  geodesics of a single effective metric that reproduces the calibrated
  orbital and optical tests.}
\end{quote}

In the weak-field limit this distinction can be quantified in terms of the
spatial potentials: light rays see
$\Psi_{\text{opt}}(r) = 2\,GM/r$ from the $n=5$ refractive profile,
whereas solitonic defects see
$\Psi_{\text{orb}}(r) = (\beta/2)\,GM/r = (3/4)\,GM/r$ via the mapping
$1+\sigma(r) \leftrightarrow 1+2\Psi(r)/c^2$ with $\beta = 3/2$.
The ratio
\[
  \frac{\Psi_{\text{orb}}}{\Psi_{\text{opt}}} = \frac{3}{8}
\]
captures how the hydrodynamic dressing weakens the effective spatial
curvature sampled by massive defects relative to light.

This hypothesis does not assert that the metric is fundamental; it simply
codifies the observation that both massive and massless probes in the toy
model respond to the same superfluid background.
The task is to identify that metric and show that, at 1PN, it reproduces
both the orbital sector (with its calibrated $\beta$) and the optical sector
(with its $\gamma=1$).
What differs is \emph{how} they sample the background: light is sensitive
primarily to $c_s(r)$ (the acoustic metric), whereas defects are sensitive
to the combination of density and added mass that appears in $\sigma(r)$
(see Appendix~\ref{app:dipole-derivation}).


\subsection{Acoustic metric and Hamiltonian description}
\label{subsec:acoustic-metric}

In a generic barotropic, irrotational, inviscid fluid, small perturbations
satisfy a wave equation that can be written in the form
\begin{equation}
  \Box_{\text{acoustic}} \psi = 0,
\end{equation}
where $\Box_{\text{acoustic}}$ is the d'Alembertian associated with an
effective acoustic metric $g^{\text{(ac)}}_{\mu\nu}$ built from the
background velocity field $\mathbf{v}(\mathbf{x})$, density
$\rho(\mathbf{x})$, and sound speed $c_s(\mathbf{x})$.
The explicit form of $g^{\text{(ac)}}_{\mu\nu}$ is standard and will not be
repeated here; what matters for our purposes is its Hamiltonian
(eikonal) limit.  

In the geometric-optics approximation, wave packets of a given mode
propagate according to a Hamiltonian
\begin{equation}
  H(\mathbf{x},\mathbf{k})
  = \omega(\mathbf{x},\mathbf{k}),
\end{equation}
where $\omega$ is the local dispersion relation.
In the simplest, non-dispersive case relevant for the present work, the
dispersion relation for sound-like modes in the rest frame of the fluid is
\begin{equation}
  \omega(\mathbf{x},\mathbf{k})
  = c_s(\mathbf{x})\,|\mathbf{k}|,
\end{equation}
so that
\begin{equation}
  H(\mathbf{x},\mathbf{k})
  = c_s(\mathbf{x})\,|\mathbf{k}|.
\end{equation}
The ray equations
\begin{equation}
  \dot{\mathbf{x}} = \frac{\partial H}{\partial \mathbf{k}},
  \qquad
  \dot{\mathbf{k}} = -\frac{\partial H}{\partial \mathbf{x}},
  \label{eq:SG-Hamilton}
\end{equation}
then describe the motion of both ``light-like'' excitations and, by the
soliton hypothesis, the centers of massive defects in the appropriate
limit.

For a static, spherically symmetric background with $c_s = c_s(r)$ and
vanishing bulk flow, the Hamiltonian~\eqref{eq:SG-Hamilton} defines a
family of null geodesics in an effective optical metric of the form
\begin{equation}
  \mathrm{d}s^2
  = -c^2 \mathrm{d}t^2
    + \frac{1}{N^2(r)}\,\bigl(\mathrm{d}r^2 + r^2 \mathrm{d}\Omega^2\bigr),
  \label{eq:SG-optical-metric}
\end{equation}
with $N(r) = c_0 / c_s(r)$.
This is just a convenient way of packaging the statement that spatial
distances are stretched by a factor $N(r)$ for the purposes of ray
propagation.
The lensing and Shapiro calculations of the previous sections can be viewed
as calculations of null geodesics in Eq.~\eqref{eq:SG-optical-metric}.

The central question is how to extend this optical metric to a full
effective spacetime metric that also governs the motion of massive
defects and reproduces the orbital Lagrangian~\eqref{eq:SG-L-eff-again} at
1PN.


\subsection{Connection to the orbital Lagrangian}
\label{subsec:lagrangian-connection}

A convenient way to make contact between the emergent metric picture and
the orbital sector is to start from a generic static, spherically symmetric
line element in isotropic coordinates,
\begin{equation}
  \mathrm{d}s^2
  = -\bigl[1 + 2\Phi_{\text{eff}}(r)/c^2 + \dots\bigr] c^2 \mathrm{d}t^2
    + \bigl[1 + 2\Psi(r)/c^2 + \dots\bigr]
      \bigl(\mathrm{d}r^2 + r^2 \mathrm{d}\varphi^2\bigr),
  \label{eq:SG-metric-PN}
\end{equation}
where $\Phi_{\text{eff}}(r)$ is the effective Newtonian potential and
$\Psi(r)$ encodes the leading spatial curvature (the dots denote higher
post-Newtonian corrections and angular directions orthogonal to the orbital
plane).
The action for a test body of rest mass $m$ moving in this geometry is
\begin{equation}
  S = -m c \int \sqrt{-g_{\mu\nu} \dot{x}^\mu \dot{x}^\nu}\,\mathrm{d}\lambda,
\end{equation}
with $\lambda$ an affine parameter.
Expanding in powers of $v^2/c^2$ and discarding an overall constant yields a
non-relativistic Lagrangian of the form
\begin{equation}
  L_{\text{geo}}
  = \frac{1}{2}\bigl[1 + 2\Psi(r)/c^2\bigr]
      \bigl(\dot r^2 + r^2 \dot\varphi^2\bigr)
    - \Phi_{\text{eff}}(r)
    + \mathcal{O}\!\left(\frac{v^4}{c^2}\right).
  \label{eq:SG-L-geo}
\end{equation}
Comparing Eq.~\eqref{eq:SG-L-geo} with the effective Lagrangian
Eq.~\eqref{eq:SG-L-eff-again}, we see that they match if we identify
\begin{equation}
  1 + \sigma(r)
  \;\longleftrightarrow\;
  1 + 2\Psi(r)/c^2,
  \qquad
  \Phi_{\text{eff}}(r)
  \;\text{as in Paper~1}.
\end{equation}
To leading order in $GM/(c^2 r)$, the kinetic prefactor $\sigma(r)$ thus
plays the role of an effective spatial curvature potential
$\Psi(r)$, with
\begin{equation}
  \sigma(r)
  \simeq \frac{2\Psi(r)}{c^2}.
\end{equation}

From Paper~1 we know that $\sigma(r)$ is of the form
\begin{equation}
  \sigma(r)
  = \beta\,\frac{GM}{c^2 r},
\end{equation}
with $\beta = 3/2$ fixed by precession.
This implies
\begin{equation}
  \Psi(r)
  = \frac{1}{2}\sigma(r) c^2
  = \frac{\beta}{2}\,\frac{GM}{r}
  = \frac{3}{4}\,\frac{GM}{r}.
\end{equation}
It is useful to compare this orbital spatial potential with the one
inferred directly from the optical sector.
For the stiff $n=5$ vacuum we found that the refractive index
$N(r)$ implies an optical spatial potential
$\Psi_{\text{opt}}(r) \propto 2 GM/r$ when interpreted via the acoustic
metric.
By contrast, the hydrodynamically dressed defects that generate
$\sigma(r)$ respond to an effective spatial potential
$\Psi_{\text{orb}}(r) = (\beta/2) GM/r = (3/4) GM/r$, corresponding to
a spatial coupling coefficient $1.5$ rather than $2.0$.
This difference reflects the fact that massive solitons do not trace the
bare acoustic geometry of phonons; they feel the combination of density
and added-mass inertia encoded in $\sigma(r)$.
At the same time, the optical sector---through $N(r)$ and the light-bending
calculation---has already fixed the combination of metric functions that
controls null geodesics.
There is therefore a non-trivial consistency condition: the metric that
reproduces the orbital Lagrangian and the metric inferred from the optical
sector must agree at the level of PPN parameters.
We examine this next.


\subsection{PPN interpretation and 1PN equivalence}
\label{subsec:ppn-equivalence}

In the PPN formalism for static, spherically symmetric fields, the metric
can be written as
\begin{align}
  g_{tt}
  &= -\left(1 - 2\frac{GM}{r c^2}
              + 2\beta \frac{G^2M^2}{r^2 c^4}
              + \dots\right),
  \\
  g_{rr}
  &= 1 + 2\gamma \frac{GM}{r c^2} + \dots,
\end{align}
with similar expressions for angular components.
General Relativity predicts $\beta = \gamma = 1$.
For our purposes, the key points are:

\begin{itemize}
  \item The \emph{light-bending} and \emph{Shapiro} tests depend only on
        $\gamma$ (and the Newtonian $GM/r$), through the way null geodesics
        respond to spatial curvature.
  \item The \emph{perihelion precession} depends on both $\beta$ and
        $\gamma$, through the combined effect of spatial curvature and
        non-linearities in the effective potential.
\end{itemize}

Sections~\ref{sec:lensing} and~\ref{sec:shapiro} have already shown that,
for the stiff $n=5$ vacuum, the effective refractive index $N(r)$ yields
$\gamma = 1$ when interpreted through the optical metric
Eq.~\eqref{eq:SG-optical-metric}.
Paper~1 showed that, once the scalar lag sector and $\sigma(r)$ with
$\beta=3/2$ are included, the orbital dynamics reproduce the GR
perihelion advance.
When these two pieces are combined into a single metric of the form
Eq.~\eqref{eq:SG-metric-PN}, one finds that the effective PPN parameters
satisfy
\begin{equation}
  \beta_{\text{eff}} = 1,
  \qquad
  \gamma_{\text{eff}} = 1,
\end{equation}
up to 1PN corrections.
In other words, the emergent metric is 1PN-equivalent to Schwarzschild:
for all classic weak-field tests, it is indistinguishable from the GR
metric.

A subtlety arises if one attempts to build the metric \emph{only} from the
optical sector (i.e.\ from $N(r)$) and then treats defects as point test
particles in that optical metric.
Such a construction typically yields the correct light bending but an
incorrect perihelion precession (e.g.\ a ``$10$'' instead of ``$6$'' in the
coefficient of $GM/(a c^2)$).
The kinetic prefactor $\sigma(r)$ and the scalar lag term in
$\Phi_{\mathrm{eff}}(r)$ precisely repair this mismatch.
This is one way of seeing that defects are not test particles on the pure
optical metric; they are solitons whose inertial dressing modifies how they
sample the metric, leading to the calibrated $\beta=3/2$ in the effective
Lagrangian.
A more detailed accounting of this PPN mapping is deferred to
Appendix~\ref{app:ppn-metric}.


\subsection{Mass-scaling trilemma and its resolution}
\label{subsec:trilemma}

The emergent-metric picture also helps resolve a potential inconsistency in
how mass enters the toy model.

Naively, one might try to combine the following three statements:

\begin{enumerate}
  \item Defects are cavitation structures whose mass scales with local
        density: $m(r) \propto \rho_{\text{local}}(r)$ (the $\kappa_\rho$
        contribution in Eq.~\eqref{eq:paper1-beta-decomp}).
  \item The equation of motion for a test body is $F = m(r)\,a$, with
        $m(r)$ appearing directly in the inertial term.
  \item The effective gravitational force on a test body is
        $F = -m(r)\,\nabla\Phi$, with $\Phi(r)$ fixed by the flux-tube
        defect.
\end{enumerate}

If one takes all three at face value, the $m(r)$ factors cancel in
$F = m(r)a$, suggesting that the center-of-mass motion should be
\emph{insensitive} to the density dependence of the mass, in apparent
tension with the need for a non-trivial $\sigma(r)$ and $\beta$ to match
1PN precession.
This is the ``mass-scaling trilemma'': the same $m(r)$ seems to be both
dynamically irrelevant (in $F=ma$) and dynamically essential (in the
orbital sector and redshift).

The resolution in the present framework is simply that statement (ii)
above is not the right way to formulate the dynamics.
Defects are not point particles obeying Newton's second law with a
position-dependent mass; they are solitons whose centers follow geodesics
of the emergent metric.
The role of $m(r)$ in the orbital sector is encoded indirectly, through the
way it modifies the effective metric coefficients (via the kinetic
prefactor and scalar lag field), not through a literal $F = m(r)a$.
By contrast, clocks are sensitive to $m(r)$ in a \emph{local} way, through
their internal frequencies, which is why $m(r)\propto\rho(r)$ is the right
input for redshift.
Once this distinction is made, the trilemma evaporates: orbital motion,
lensing, Shapiro delay, and redshift all become different aspects of
geodesic motion in the same emergent geometry.

In particular, the effective spatial coupling in the orbital sector is
controlled not by $m(r)$ alone but by the combination
$\beta = \kappa_\rho + \kappa_{\mathrm{add}}$.
The static density profile around the throat contributes
$\kappa_\rho = 1$, while the dipole ``cloud'' of superfluid flow
required to transport a stiff spherical void contributes an added mass
$\kappa_{\mathrm{add}} = 1/2$, as derived in
Appendix~\ref{app:dipole-derivation}.
This hydrodynamic dressing is what allows the same vacuum configuration
to reconcile the ``$10$ vs $6$'' tension: light probes the optical
coefficient $2.0$, whereas defects feel the reduced orbital coefficient
$1.5$ that yields the correct GR precession (the naive optical
test-particle model overpredicts the GR coefficient by a factor $5/3$).


\subsection{Emergent vs fundamental metric}
\label{subsec:emergent-metric}

Finally, it is worth emphasizing the status of the metric in this toy
model.
The line element~\eqref{eq:SG-metric-PN} (or any more complete version
built from $c_s$, $\rho$, and $\mathbf{v}$) is \emph{emergent}, not
fundamental.
It is a convenient, coarse-grained description of how wave packets and
solitons propagate through the superfluid vacuum, valid in the 1PN,
weak-field regime where wavelengths are much smaller than the scale over
which the background varies.
The fundamental degrees of freedom are the fluid variables and the fields
that define the defects and the scalar lag mode.

This perspective has two important consequences:

\begin{itemize}
  \item It explains why the metric can be ``right'' at 1PN---in the sense of
        matching Schwarzschild for the classic tests—without being exact in
        strong fields or at very short distances.
        Deviations from GR in those regimes would reflect the breakdown of
        the hydrodynamic and eikonal approximations, not a failure of the
        emergent metric where it is meant to apply.

  \item It clarifies the role of analogue gravity in this context.
        The goal is not to derive GR from a superfluid, but to exhibit a
        controlled regime in which a simple, physically transparent fluid
        model reproduces the same 1PN phenomenology.
        The emergent metric is a bookkeeping device for that regime, not a
        claim about the ontological status of spacetime.
\end{itemize}

In summary, the soliton-geodesic interpretation provides a unified
framework in which the orbital sector (with its calibrated $\beta$) and
the optical/clock sector (with its $\gamma=1$ and redshift) are different
projections of the same emergent geometry.
The next section turns to the question of uniqueness: to what extent is the
stiff $n=5$ pure-refraction vacuum singled out by these requirements within
the broader space of superfluid vacua and defect constructions?

%---------------------------------------------------------------------------
\section{Degeneracies and the choice of $n=5$ pure refraction}
\label{sec:degeneracies}

The construction in Sections~\ref{sec:n5-vacuum}--\ref{sec:redshift} picks
out a particular vacuum and defect configuration: a stiff $n=5$ polytropic
superfluid with a flux-tube mass defect, in which all of the 1PN optical
and clock effects are attributed to refraction, i.e.\ to gradients of the
sound speed $c_s(r)$ encoded in $N(r)$.
From the internal point of view of the toy model this is a highly
non-trivial choice: other, apparently reasonable constructions of the
superfluid and its flows are possible.
In this section we briefly survey these competing branches and explain why
we ultimately adopt the $n=5$ pure-refraction branch as the preferred
solution within the restricted model space.


\subsection{Competing constructions}
\label{subsec:degeneracies-menu}

Broadly speaking, there are three qualitatively different ways to obtain
the same 1PN observables (light bending, Shapiro delay, redshift,
perihelion precession) from a superfluid with defects:

\begin{enumerate}
  \item[(A)] \textbf{Fast-brane / pure drag.}
    The vacuum sound speed remains essentially uniform, $c_s \simeq c_0$,
    so $N(r) \simeq 1$ everywhere.
    Gravitational effects on light and matter are attributed almost
    entirely to \emph{drag} by a fast background flow $\mathbf{v}(r)$
    toward the throat (a moving-medium analogue of the ``river model''
    of black holes).

  \item[(B)] \textbf{Split $n=3$ model (drag + refraction).}
    The vacuum has a softer ($n=3$) polytropic EOS.
    Both bulk flows and sound-speed gradients are present, and the total
    1PN effect is decomposed into a drag contribution and a refraction
    contribution, each responsible for roughly half of the GR signal.

  \item[(C)] \textbf{Stiff $n=5$ pure refraction (preferred).}
    The vacuum is super-stiff ($n=5$).
    The far-field flow is slow in the brane frame; the dominant 1PN
    effects arise from the $1/r$ pressure and density deficits and the
    resulting $N(r)$ profile.
    Drag plays a subleading role in the 1PN sector.
\end{enumerate}

All three branches can be tuned to recover the Newtonian potential and a
GR-like 1PN expansion for at least one observable.
The question is which branch, if any, survives when we demand a common
vacuum and defect structure that simultaneously matches \emph{all} of the
1PN tests and is compatible with the defect-based EM picture.


\subsection{Rejection of fast-brane / pure drag}
\label{subsec:degeneracies-fast-brane}

The fast-brane / pure-drag branch (A) is conceptually attractive at first
glance.
In a moving-medium analogue of gravity, light rays propagating through a
flow with $\mathbf{v}(r)$ experience an effective spacetime geometry even
if $c_s$ is constant.
One can therefore imagine a model in which the Newtonian potential,
lensing, and Shapiro delay are all realized by a suitably chosen
radial inflow toward the flux-tube throat, with little or no role for
refraction.

Upon closer inspection, however, this branch faces several difficulties:

\begin{itemize}
  \item \textbf{Continuity and energetics.}
        To reproduce the observed magnitude of the 1PN effects with
        $N(r)\simeq 1$, the inflow velocity must approach the escape
        velocity on large scales, $|\mathbf{v}(r)| \sim \sqrt{2GM/r}$,
        over an extended region.
        Maintaining such a fast, steady inflow over astronomical distances
        strains the continuity equation and the global energetics of the
        superfluid, especially if similar flows are required around many
        defects.

  \item \textbf{Optical vs orbital tuning.}
        In a pure-drag picture, the same flow profile must account for
        both the bending of light and the precession of orbits.
        Matching the coefficients in both sectors simultaneously is
        possible, but requires a more delicate tuning of $\mathbf{v}(r)$
        than in the refraction-based construction.

  \item \textbf{EM compatibility.}
        The fast-brane picture does not mesh cleanly with the EM sector,
        where the flux tube is already committed to carrying electric
        flux and setting the throat geometry.
        Requiring the same structure to sustain a large-scale, nearly
        free-fall inflow introduces additional constraints that are
        difficult to reconcile with the EM requirements.
\end{itemize}

For these reasons we regard the pure-drag branch as disfavored in the
present toy model.
It remains a useful conceptual foil, but not a robust realization of the
1PN phenomenology once all sectors are taken into account.


\subsection{Rejection of split $n=3$ model}
\label{subsec:degeneracies-n3}

The split $n=3$ model (branch B) occupies an intermediate position.
Here the superfluid vacuum obeys a softer polytropic EOS, and both drag
and refraction contribute appreciably to the 1PN effects.
A simple implementation is to arrange, by hand, that half of the GR
lensing and Shapiro signals come from bulk flow and half from the
sound-speed gradient, with an $n=3$ EOS tuned so that
\[
  N(r) \simeq 1 + \alpha_{n=3}\,\frac{GM}{r c^2}
\]
produces the desired ``$2GM$'' share of the effect.

This branch avoids the extreme flows of the pure-drag picture and makes
non-trivial use of refraction.
Nonetheless, it is also disfavored on several grounds:

\begin{itemize}
  \item \textbf{Split bookkeeping.}
        The decomposition of the 1PN effects into drag and refraction is
        an internal bookkeeping choice of the model; observables only see
        the total.
        In the split $n=3$ branch, the relative weights of drag and
        refraction must be tuned to agree simultaneously with both
        orbital and optical data, which is a less economical use of the
        available degrees of freedom.

  \item \textbf{EOS–geometry tension.}
        A softer EOS implies a weaker dependence of $c_s$ on $\rho$ and
        thus a smaller refraction coefficient.
        Compensating for this by increasing the amplitude of the density
        deficit pushes against the hydrostatic and EM constraints on the
        flux-tube geometry.

  \item \textbf{Lack of clean unification.}
        In the stiff $n=5$ branch, all of the 1PN optical and clock
        effects can be read off from a single $1/r$ profile and its
        derivatives.
        In the split $n=3$ branch, the interpretation is more muddled:
        some fraction of the signal is ``in space'' (refraction), some is
        ``in time'' (drag), and the split depends on internal modeling
        choices.
\end{itemize}

From the perspective of clarity and falsifiability, a branch in which the
entire 1PN optical and clock sector is tied to a single function $N(r)$,
fixed by the EOS and flux-tube geometry, is preferable.


\subsection{Preferred $n=5$ branch and uniqueness claims}
\label{subsec:degeneracies-n5}

The stiff $n=5$ pure-refraction branch (C) avoids the pitfalls of the
other branches and offers a remarkably compact picture:

\begin{itemize}
  \item The flux-tube mass defect carves out a $1/r$ pressure and density
        deficit, with amplitude fixed by $GM$ and $\rho_0$.
  \item The $n=5$ EOS implies $c_s(\rho)$ such that
        $N(r) \simeq 1 + 2GM/(r c^2)$.
  \item This $N(r)$ alone yields the GR light-bending angle and Shapiro
        delay (fixing $\gamma=1$), while the same density deficit yields
        the GR weak-field redshift.
  \item The orbital sector, with its scalar lag and
        $\sigma(r) = \beta GM/(c^2 r)$, supplies the remaining structure
        needed to match the 1PN precession and fix $\beta=3/2$.
\end{itemize}

Within the restricted model space we have considered—spherically symmetric
flux-tube defects in a barotropic, polytropic superfluid vacuum—the stiff
$n=5$ branch is \emph{effectively unique} in the following sense:

\begin{enumerate}
  \item For generic $n$, the coefficient of $GM/(r c^2)$ in $N(r)$ is
        proportional to $(n-1)$.
        Demanding the GR light-bending angle fixes this coefficient and
        hence selects $n=5$.
  \item Once $N(r)$ is fixed in this way, the Shapiro delay and redshift
        follow automatically; there is no remaining freedom in the optical
        and clock sector at 1PN.
  \item The orbital sector already fixed $\beta$ and the structure of
        $\Phi_{\mathrm{eff}}(r)$ in Paper~1; these are insensitive to the
        choice of $n$ in the weak field, provided the far-field potential
        remains $-GM/r$.
\end{enumerate}

Our uniqueness claim is therefore modest but concrete:
\emph{among spherically symmetric, polytropic superfluid vacua with
flux-tube mass defects that share the same Newtonian limit, the stiff
$n=5$ branch is singled out by the requirement that a single $N(r)$ profile
reproduce the GR values of light bending, Shapiro delay, and redshift at
1PN.}
Within that branch, the orbital sector then fixes $\beta$ and
$\Phi_{\mathrm{eff}}(r)$, completing the 1PN match.


\subsection{Relation to the orbital $\beta$ and throat geometry}
\label{subsec:degeneracies-beta-L}

It is useful to separate clearly what is and is not constrained by the
choice of $n=5$ pure refraction.

\begin{itemize}
  \item \textbf{Fixed by optics and clocks.}
        The 1PN optical and clock tests fix:
        \begin{itemize}
          \item the combination of EOS and flux-tube structure that leads
                to $N(r)\simeq 1 + 2GM/(r c^2)$,
          \item and hence the stiff $n=5$ vacuum in the polytropic family.
        \end{itemize}
        Once this is chosen, the coefficients in lensing, Shapiro, and
        redshift are determined.

  \item \textbf{Fixed by orbits.}
        The orbital sector, together with the scalar lag field, fixes:
        \begin{itemize}
          \item the effective potential $\Phi_{\mathrm{eff}}(r)$ at 1PN,
          \item the kinetic prefactor $\sigma(r)$ and the combination of
                hydrodynamic contributions that make up $\beta$,
          \item and the identification $\beta = 3/2$ required by
                precession.
        \end{itemize}
        These constraints arise from the way the throat and its near-field
        flows dress the defects; they are largely insensitive to $n$ at
        the level of the far-field EOS.

  \item \textbf{Constrained but not fixed: throat micro-geometry.}
        The detailed throat geometry—its radius $a$, depth $L$, and the
        ratio $L/a$—is constrained by a combination of orbital and EM
        considerations, but not fully fixed by the 1PN gravity sector
        alone.
        In particular, the 1PN optical tests are sensitive only to the
        far-field $1/r$ profile, not to the internal structure of the
        throat.  The latter will be further constrained in the full EM
        analysis.
\end{itemize}

In summary, the choice of $n=5$ pure refraction is driven by the optical
and clock sector and feeds back into the orbital sector only indirectly,
through the requirement of a single emergent metric.
The detailed throat geometry remains a degree of freedom to be fixed by
the electromagnetic sector and higher-precision observables, which we
leave to future work.

%---------------------------------------------------------------------------
\section{Discussion and outlook}
\label{sec:discussion}

\subsection{Summary of results}
\label{subsec:discussion-summary}

The goal of this paper was to extend the superfluid-defect toy model of
Ref.~\cite{Norris:2025Orbits} from the orbital sector to the full set of
classic 1PN tests, and to understand how these tests constrain the
structure of the vacuum and the defects.

On the orbital side, we began from the effective Lagrangian
\begin{equation}
  L
  = \frac{1}{2}\bigl[1 + \sigma(r)\bigr]
      \bigl(\dot r^2 + r^2 \dot\varphi^2\bigr)
    - \Phi_{\mathrm{eff}}(r),
\end{equation}
in which a scalar ``lag'' field produces a $1/r^2$ correction to the
Newtonian potential and a kinetic prefactor $\sigma(r)$ encodes a
position-dependent renormalization of inertia.
As reviewed in Section~\ref{sec:paper1-inputs}, the scalar sector accounts
for one half of the GR perihelion precession, while $\sigma(r)$ with
$\beta = 3/2$ accounts for the other half, yielding a total precession
equal to the Schwarzschild 1PN result.

On the vacuum side, we constructed a stiff ($n=5$) polytropic superfluid
with a flux-tube mass defect.
Hydrostatic balance in this vacuum implies a $1/r$ pressure deficit
$\Delta P(r) = -GM\rho_0/r$ around the defect and a corresponding $1/r$
density deficit, which in turn reduces the local sound speed and induces
a refractive index profile
\begin{equation}
  N(r)
  = \frac{c_0}{c_s(r)}
  \simeq 1 + 2\,\frac{GM}{r c^2}
\end{equation}
in the weak-field limit.

Treating light-like excitations as rays in this graded-index medium, we
showed in Section~\ref{sec:lensing} that the total bending angle for a ray
with impact parameter $b$ is
\begin{equation}
  \Delta\theta = \frac{4GM}{b c^2},
\end{equation}
in agreement with the GR prediction and corresponding to PPN
$\gamma = 1$.
In Section~\ref{sec:shapiro} we used the same $N(r)$ to compute the
Shapiro time delay for signals passing near the mass, finding the standard
logarithmic form with the correct $(1+\gamma)$ coefficient.
In Section~\ref{sec:redshift} we then argued that defect-based clocks,
whose frequencies scale with a local mass $m(r)\propto\rho(r)$, exhibit a
weak-field redshift
\begin{equation}
  \frac{\Delta\nu}{\nu}
  \simeq -\,\frac{GM}{r c^2},
\end{equation}
again matching the GR result when $\Delta\nu/\nu$ is defined as the shift of
a clock deeper in the potential relative to one at infinity.

These results admit a unified interpretation in terms of \emph{soliton
geodesics} in an emergent metric, as discussed in
Section~\ref{sec:soliton-geodesics}.
The same flux-tube defect and stiff vacuum that produced the orbital
Lagrangian and its calibrated $\beta$ also define an acoustic/optical
metric that governs null rays and soliton centers.
Within the 1PN, weak-field regime, the effective PPN parameters of this
emergent metric satisfy
\begin{equation}
  \beta_{\mathrm{eff}} = 1,
  \qquad
  \gamma_{\mathrm{eff}} = 1,
\end{equation}
so that, for the classic solar-system tests, the toy model is
indistinguishable from Schwarzschild.

Finally, in Section~\ref{sec:degeneracies} we compared this stiff
$n=5$ pure-refraction branch to alternative constructions based on drag or
mixed drag/refraction, and argued that, within the restricted class of
spherically symmetric polytropic vacua with flux-tube defects, the
$n=5$ branch is singled out by the requirement that a single refractive
index profile $N(r)$ reproduce GR-like lensing, Shapiro delay, and
redshift at 1PN.


\subsection{Conceptual lessons}
\label{subsec:discussion-conceptual}

Beyond the technical match to GR, the toy model offers several conceptual
lessons about how gravity, optics, and inertia can emerge from a
hydrodynamic substrate.

First, the model realizes a concrete version of the idea that
\emph{gravity is a statement about the state of the vacuum}.
Here the ``vacuum'' is a compressible superfluid, and what GR would
describe as curvature near a mass is instead encoded in a modest depletion
of pressure and density and a corresponding reduction in the local sound
speed.
The $1/r$ profiles that appear in the effective metric arise not from a
fundamental spacetime field but from the way a flux-tube defect distorts
the surrounding medium.

Second, the model illustrates how a single background profile can unify
effects that are often treated separately.
Lensing, Shapiro delay, and redshift all trace back to the same $1/r$
density deficit: rays slow and bend because $c_s(\rho)$ decreases, and
clocks slow because defect masses $m(\rho)$ decrease.
The orbital sector then probes the same background through the way it
renormalizes kinetic and potential terms.
From this perspective, the classic 1PN tests are less a collection of
independent phenomena and more a coordinated set of probes of one
underlying function of radius.

Third, the soliton-geodesic viewpoint helps clarify what it means for an
emergent metric to be ``real''.
In the toy model, the metric is not a fundamental field; it is a
bookkeeping device that captures how localized excitations propagate
through a particular fluid state.
Yet, within its regime of validity, this emergent metric obeys the same
rules as a genuine spacetime metric: it defines geodesics, it has PPN
parameters, and it can be tested against experiments.
This suggests that many of the familiar geometric structures of GR may be
robust features of any theory in which excitations propagate on a
background with a small number of state variables, rather than unique to a
specific microscopic completion.

Finally, the analysis sharpens the distinction between \emph{local} and
\emph{global} uses of mass in an emergent picture.
Locally, the mass of a defect affects its internal frequencies and hence
clock rates, and it can depend on position through the density profile.
Globally, the motion of defects is governed not by $F = m(r) a$ with
position-dependent inertial mass, but by geodesic equations in an emergent
metric whose coefficients have already absorbed the effects of $m(r)$.
Recognizing this distinction resolves apparent contradictions and helps
organize the model in a way that respects both the equivalence principle
and the hydrodynamic origin of inertia.

\subsection{Physical constraints: stiffness and universality}
\label{subsec:discussion-stiffness}

The 1PN calibration of the orbital sector relies on the throat behaving hydrodynamically as a stiff spherical obstacle with respect to its translational inertia. While the internal topology controls the electromagnetic and vector sectors (to be discussed in future work), the translational added mass is dominated by the spherical displacement envelope of the vacuum.
In the hydrodynamic picture this is encoded in the added-mass coefficient
$\kappa_{\mathrm{add}} = 1/2$ for a sphere moving through the superfluid,
which, together with $\kappa_\rho = 1$, yields the kinetic prefactor
$\beta = \kappa_\rho + \kappa_{\mathrm{add}} = 3/2$ and restores the GR
perihelion precession.
The shape-sensitivity analysis of Appendix~\ref{app:dipole-derivation}
shows that this assumption is quantitatively important: deforming the void
by only $10\%$ into an oblate spheroid (a ``pancake'' with $b/a = 1.1$)
shifts $\kappa_{\mathrm{add}}$ from $0.5$ to $\simeq 0.56$, and would change
the precession factor $3 + 2\beta$ from $6.0$ to $\simeq 6.12$, a
$\sim 2\%$ deviation from the GR value.
Solar-system bounds on perihelion precession thus translate into a
``stiffness'' constraint: the topological surface tension that holds the
throat open must be large enough that the brane intersection remains
approximately spherical under orbital accelerations, with departures from
sphericity limited to the few-percent level.

A second constraint arises from the universality of free fall for composite
bodies.
In the added-mass framework there are two idealized limits.
If the superfluid vacuum treats a planet as an impermeable solid body of
macroscopic density $\rho_{\mathrm{matter}}$, the effective acceleration in
an external field depends on that density via the ratio
$\rho_{\mathrm{matter}} / (\rho_{\mathrm{matter}} + \kappa_{\mathrm{macro}}\rho_{\mathrm{vac}})$,
where $\kappa_{\mathrm{macro}}$ is the added-mass coefficient of the
macroscopic obstacle and $\rho_{\mathrm{vac}}$ is the vacuum density.
By contrast, if the vacuum permeates bulk matter and flows around each
microscopic throat, the inertial response of a composite planet is just the
sum of the individual added masses, and its free-fall acceleration matches
that of a single defect exactly, independent of the number of constituents.
A simple symbolic calculation shows that the solid-body and permeable
cases agree only if
\begin{equation}
  \rho_{\mathrm{matter}}
  = \frac{\kappa_{\mathrm{macro}}}{\kappa_{\mathrm{single}}}\,
    \frac{m_{\mathrm{defect}}}{v_{\mathrm{defect}}},
\end{equation}
i.e. if $\rho_{\mathrm{matter}}$ is tuned to a special value proportional
to the effective defect density $m_{\mathrm{defect}}/v_{\mathrm{defect}}$.
In the toy model we therefore adopt \emph{permeation} as a structural
assumption: the superfluid vacuum must be effectively ``ghost-like'' to
bulk matter, flowing through ordinary materials rather than around them, so
that universality of free fall for composite bodies is automatic rather
than the result of fine-tuning.


\subsection{Limitations and future work}
\label{subsec:discussion-future}

The toy model is deliberately modest in scope.
It is worth summarizing its main limitations, both to keep the claims in
check and to highlight directions for future work.

\paragraph{1PN and weak-field regime.}
All of our calculations have been performed in the weak-field, 1PN regime,
with $GM/(r c^2) \ll 1$ and velocities small compared to $c$.
The emergent metric is only required to match Schwarzschild at this order,
and nothing in the present analysis guarantees that strong-field phenomena
(horizons, innermost stable orbits, gravitational waves) will behave like
their GR counterparts.
Exploring the strong-field limit would require a non-linear treatment of
the equation of state, the defect core, and the scalar lag field.

\paragraph{Spherical symmetry and isolated defects.}
We have assumed a single, static, spherically symmetric flux-tube defect in
an otherwise homogeneous vacuum.
Realistic astrophysical systems involve multiple bodies, rotation, and
non-spherical structures.
It would be interesting to generalize the construction to include
spinning defects, defect binaries, and extended mass distributions, and to
ask how the effective metric responds in these more complex settings.

\paragraph{Microscopic underpinnings.}
The model treats the superfluid and defects at an effective, continuum
level, with an EOS and flux-tube structure chosen to satisfy macroscopic
constraints.
A more complete theory would specify the microscopic excitations and
interactions that give rise to this EOS and topology, and would derive the
scalar lag mode and hydrodynamic coefficients (such as $\beta$) from
first principles.
This would also clarify the domain of validity of the hydrodynamic and
eikonal approximations and might reveal additional constraints or
corrections.

\paragraph{Electromagnetic sector.}
Throughout this paper we have largely bracketed the electromagnetic sector,
treating the flux-tube as a gravitational and hydrodynamic object without
specifying its EM couplings.
A natural next step is to work out the full EM construction in the same
superfluid-defect framework, tying together throat geometry, charge,
magnetic fields, and the $n=5$ vacuum.
Among other things, this would constrain the throat radius and depth, test
the consistency of the flux-tube picture with known EM phenomena, and
potentially relate the gravitational and electromagnetic sectors more
tightly.

\paragraph{Cosmological and galactic scales.}
The analysis here is local, centered on a single defect.
Extending the model to cosmological or galactic scales raises new
questions: how do many defects back-react on the vacuum?
Do large-scale flows or density variations emerge that could mimic dark
matter or dark energy effects?
Is it possible to construct a statistically homogeneous and isotropic
vacuum filled with defects that yields a viable large-scale expansion
history?
These questions go far beyond the remit of the present paper, but the
superfluid-defect language is well suited to posing them.

\paragraph{Numerical exploration.}
A systematic numerical campaign—paralleling the orbital and PDE numerics
of Paper~1—would be valuable both as a check on the analytic approximations
and as a way to explore regimes (e.g.\ intermediate field strengths,
non-trivial geometries) where closed-form expressions are unavailable.

\medskip

Taken together, Paper~1 and the present work suggest that a relatively
simple superfluid-defect toy model is capable of reproducing the full
suite of classic 1PN tests of gravity, with a clear and unified physical
interpretation in terms of vacuum structure, flux tubes, and soliton
geodesics.
Whether this framework can be extended to encompass electromagnetism,
strong-field gravity, and cosmology in a similarly coherent way remains an
open and intriguing question.

%---------------------------------------------------------------------------
\appendix

\section{Stiff $n=5$ superfluid vacuum and optical sector details}
\label{app:n5-optical}

In this appendix we collect the derivations underlying the optical sector
of the model.
We work with a general polytropic index $n$ and only specialize to the
stiff case $n=5$ at the end.
This makes it clear how the light-bending and Shapiro coefficients select
the $n=5$ branch within the polytropic family.


\subsection{Polytropic EOS and density/sound-speed perturbations}
\label{subsec:app-eos}

We start from the barotropic polytropic equation of state,
\begin{equation}
  P = K \rho^n,
\end{equation}
with homogeneous background $(\rho_0, P_0)$ and background sound speed
\begin{equation}
  c_0^2 \equiv
  \left.\frac{\partial P}{\partial \rho}\right|_{\rho_0}
  = n K \rho_0^{\,n-1}.
\end{equation}
Consider a weak, static perturbation of the vacuum induced by a central
mass $M$.
In the far-field, quasi-static regime the fluid satisfies hydrostatic
balance,
\begin{equation}
  \frac{1}{\rho(r)}\frac{\mathrm{d}P}{\mathrm{d}r}
  = \frac{\mathrm{d}\Phi}{\mathrm{d}r},
\end{equation}
with $\Phi(r) = -GM/r$ the effective Newtonian potential.

In the weak-field limit we linearize about the homogeneous background,
writing
\begin{equation}
  \rho(r) = \rho_0 + \Delta\rho(r),
  \qquad
  P(r) = P_0 + \Delta P(r),
\end{equation}
with $|\Delta\rho| \ll \rho_0$ and $|\Delta P| \ll P_0$, and approximate
$\rho(r) \simeq \rho_0$ in the hydrostatic equation.
This yields
\begin{equation}
  \frac{1}{\rho_0}\frac{\mathrm{d}P}{\mathrm{d}r}
  \simeq \frac{\mathrm{d}\Phi}{\mathrm{d}r}
  = \frac{GM}{r^2}.
\end{equation}
Integrating from $r$ to $\infty$ and choosing the integration constant so
that $\Delta P(\infty)=0$ gives
\begin{equation}
  \Delta P(r)
  = -\rho_0\int_r^\infty \frac{GM}{r'^2}\,\mathrm{d}r'
  = -\frac{GM\rho_0}{r}.
  \label{eq:app-deltaP}
\end{equation}

To relate $\Delta P$ to $\Delta\rho$, we expand the EOS to first order:
\begin{equation}
  P(\rho_0 + \Delta\rho)
  = P_0 + \left.\frac{\partial P}{\partial \rho}\right|_{\rho_0}\Delta\rho
        + \mathcal{O}(\Delta\rho^2)
  = P_0 + c_0^2\,\Delta\rho + \mathcal{O}(\Delta\rho^2),
\end{equation}
so that
\begin{equation}
  \Delta P(r) \simeq c_0^2\,\Delta\rho(r).
\end{equation}
Combining with Eq.~\eqref{eq:app-deltaP}, we obtain
\begin{equation}
  \Delta\rho(r)
  = \frac{\Delta P(r)}{c_0^2}
  = -\frac{GM\rho_0}{c_0^2 r},
\end{equation}
or in fractional form
\begin{equation}
  \frac{\Delta\rho(r)}{\rho_0}
  = -\frac{GM}{c_0^2 r}
  + \mathcal{O}\!\left(\frac{G^2M^2}{c_0^4 r^2}\right).
  \label{eq:app-deltarho}
\end{equation}
Note that this leading-order density deficit is \emph{independent} of the
polytropic index $n$; $n$ enters only through the way $c_s$ responds to
$\Delta\rho$.

The local sound speed is given by
\begin{equation}
  c_s^2(\rho)
  = \frac{\mathrm{d}P}{\mathrm{d}\rho}
  = n K \rho^{\,n-1}.
\end{equation}
Expanding around $\rho_0$,
\begin{equation}
  c_s^2(\rho_0 + \Delta\rho)
  = c_0^2\left(\frac{\rho_0 + \Delta\rho}{\rho_0}\right)^{n-1}
  \simeq c_0^2\left[1 + (n-1)\frac{\Delta\rho}{\rho_0}\right],
\end{equation}
where we have used $(1+x)^{n-1} \simeq 1 + (n-1)x$ for $|x|\ll1$.
Taking the square root and linearizing again,
\begin{equation}
  c_s(\rho_0 + \Delta\rho)
  = c_0\sqrt{1 + (n-1)\frac{\Delta\rho}{\rho_0}}
  \simeq c_0\left[1 + \frac{n-1}{2}\frac{\Delta\rho}{\rho_0}\right].
\end{equation}
Thus
\begin{equation}
  \frac{\Delta c_s(r)}{c_0}
  \equiv \frac{c_s(r) - c_0}{c_0}
  \simeq \frac{n-1}{2}\frac{\Delta\rho(r)}{\rho_0}.
  \label{eq:app-deltacs-rho}
\end{equation}
Substituting Eq.~\eqref{eq:app-deltarho} yields
\begin{equation}
  \frac{\Delta c_s(r)}{c_0}
  \simeq -\frac{n-1}{2}\frac{GM}{c_0^2 r},
  \label{eq:app-deltacs}
\end{equation}
which shows that $c_s$ is reduced near the defect for any $n>1$, with a
strength proportional to $(n-1)$.


\subsection{General-$n$ refractive index and lensing}
\label{subsec:app-general-n-lensing}

The effective refractive index for sound-like excitations is
\begin{equation}
  N(r)
  \equiv \frac{c_0}{c_s(r)}.
\end{equation}
Using Eq.~\eqref{eq:app-deltacs} and expanding to first order,
\begin{equation}
  N(r)
  = \frac{1}{1 + \Delta c_s(r)/c_0}
  \simeq 1 - \frac{\Delta c_s(r)}{c_0}
  \simeq 1 + \frac{n-1}{2}\frac{GM}{c_0^2 r}.
\end{equation}
It is convenient to define
\begin{equation}
  \alpha_n \equiv \frac{n-1}{2},
\end{equation}
so that
\begin{equation}
  N(r)
  \simeq 1 + \alpha_n \frac{GM}{c_0^2 r}.
  \label{eq:app-N-general}
\end{equation}
To the same order we may write
\begin{equation}
  \ln N(r)
  \simeq \alpha_n \frac{GM}{c_0^2 r},
  \label{eq:app-lnN-general}
\end{equation}
since $\ln(1+x)\simeq x$ for $|x|\ll1$.

We now compute the light-bending angle for general $n$.
As in the main text, we approximate the ray path by a straight line with
impact parameter $b$, parameterized by $z$.
The radial distance from the mass is $r(z) = \sqrt{b^2 + z^2}$.
To leading order in the index perturbation, the total deflection angle is
\begin{equation}
  \Delta\theta
  \simeq \int_{-\infty}^{+\infty}
         \nabla_\perp \ln N\bigl(r(z)\bigr)\,\mathrm{d}z,
\end{equation}
where $\nabla_\perp$ is the gradient transverse to the unperturbed ray
direction.
For a spherically symmetric $\ln N(r)$,
\begin{equation}
  \left|\nabla_\perp \ln N\right|
  = \left|\frac{\mathrm{d}\ln N}{\mathrm{d}r}\right|\frac{b}{r}.
\end{equation}
Using Eq.~\eqref{eq:app-lnN-general},
\begin{equation}
  \frac{\mathrm{d}\ln N}{\mathrm{d}r}
  \simeq -\alpha_n \frac{GM}{c_0^2}\frac{1}{r^2},
\end{equation}
so that
\begin{equation}
  \left|\nabla_\perp \ln N\bigl(r(z)\bigr)\right|
  \simeq \alpha_n \frac{GM}{c_0^2}\frac{b}{r^3}
  = \alpha_n \frac{GM}{c_0^2}
    \frac{b}{(b^2+z^2)^{3/2}}.
\end{equation}
The deflection angle is therefore
\begin{equation}
  \Delta\theta
  \simeq \alpha_n \frac{GM}{c_0^2}
         \int_{-\infty}^{+\infty}
         \frac{b}{(b^2+z^2)^{3/2}}\,\mathrm{d}z.
\end{equation}
The integral is the same as in the main text,
\begin{equation}
  \int_{-\infty}^{+\infty}
  \frac{b}{(b^2+z^2)^{3/2}}\,\mathrm{d}z
  = \frac{2}{b},
\end{equation}
so we obtain
\begin{equation}
  \Delta\theta
  \simeq \alpha_n \frac{GM}{c_0^2}\frac{2}{b}
  = \frac{2\alpha_n GM}{b c_0^2}.
\end{equation}
Specializing to the 1PN matching limit $c_0\to c$ gives
\begin{equation}
  \Delta\theta_n
  = \frac{2\alpha_n GM}{b c^2}
  = \frac{(n-1)GM}{b c^2}.
  \label{eq:app-dtheta-n}
\end{equation}
In the PPN formalism, the corresponding deflection angle is
\begin{equation}
  \Delta\theta_{\text{PPN}}
  = \frac{2(1+\gamma)GM}{b c^2}.
\end{equation}
Equating Eq.~\eqref{eq:app-dtheta-n} with $\Delta\theta_{\text{PPN}}$ yields
\begin{equation}
  2(1+\gamma) = n-1
  \quad\Rightarrow\quad
  \gamma = \frac{n-3}{2}.
\end{equation}
Demanding $\gamma = 1$ (the GR value) therefore selects
\begin{equation}
  n-3 = 2
  \quad\Rightarrow\quad
  n = 5.
\end{equation}
Thus, within this polytropic family, the light-bending data single out the
stiff $n=5$ vacuum.


\subsection{Shapiro delay and redshift integrals}
\label{subsec:app-shapiro-redshift}

We now repeat the Shapiro delay derivation in a way that keeps the
dependence on $\alpha_n$ explicit, and then collect the redshift
integrals for reference.

\paragraph{General-$n$ Shapiro delay.}

Using the same setup as in Section~\ref{sec:shapiro}, the one-way
time-of-flight from $z=-Z_\mathrm{E}$ to $z=+Z_\mathrm{R}$ along a
straight-line path with impact parameter $b$ is
\begin{equation}
  t
  \simeq \frac{1}{c_0}
         \int_{-Z_\mathrm{E}}^{+Z_\mathrm{R}}
         \left[1 + \alpha_n \frac{GM}{c_0^2 r(z)}\right]\mathrm{d}z,
\end{equation}
where $r(z) = \sqrt{b^2+z^2}$.
Subtracting the flat-space time
$t_0 = (Z_\mathrm{E}+Z_\mathrm{R})/c_0$ gives the delay
\begin{equation}
  \Delta t_n
  = t - t_0
  \simeq \alpha_n \frac{GM}{c_0^3}
         \int_{-Z_\mathrm{E}}^{+Z_\mathrm{R}}
         \frac{\mathrm{d}z}{\sqrt{b^2+z^2}}.
\end{equation}
As in the main text, the integral evaluates to
\begin{equation}
  \int_{-Z_\mathrm{E}}^{+Z_\mathrm{R}}
  \frac{\mathrm{d}z}{\sqrt{b^2+z^2}}
  = \ln\!\left[
      \frac{Z_\mathrm{R} + \sqrt{b^2 + Z_\mathrm{R}^2}}
           {-Z_\mathrm{E} + \sqrt{b^2 + Z_\mathrm{E}^2}}
    \right].
\end{equation}
For $Z_\mathrm{E},Z_\mathrm{R}\gg b$, this becomes
\begin{equation}
  \int_{-Z_\mathrm{E}}^{+Z_\mathrm{R}}
  \frac{\mathrm{d}z}{\sqrt{b^2+z^2}}
  \simeq \ln\!\left(\frac{4 r_\mathrm{E} r_\mathrm{R}}{b^2}\right),
\end{equation}
where $r_\mathrm{E} \simeq Z_\mathrm{E}$ and $r_\mathrm{R}\simeq Z_\mathrm{R}$.
Thus
\begin{equation}
  \Delta t_n
  \simeq \alpha_n \frac{GM}{c_0^3}
         \ln\!\left(\frac{4 r_\mathrm{E} r_\mathrm{R}}{b^2}\right).
\end{equation}
Setting $c_0\to c$,
\begin{equation}
  \Delta t_n
  = \alpha_n \frac{GM}{c^3}
    \ln\!\left(\frac{4 r_\mathrm{E} r_\mathrm{R}}{b^2}\right).
  \label{eq:app-dt-n}
\end{equation}

In the PPN formalism,
\begin{equation}
  \Delta t_{\text{PPN}}
  = (1+\gamma)\frac{GM}{c^3}
    \ln\!\left(\frac{4 r_\mathrm{E} r_\mathrm{R}}{b^2}\right).
\end{equation}
Comparing with Eq.~\eqref{eq:app-dt-n} shows that
\begin{equation}
  \alpha_n = 1 + \gamma.
\end{equation}
Using $\alpha_n = (n-1)/2$ and $\gamma = 1$ again yields $n=5$.
Thus the Shapiro delay provides the same constraint on $n$ as the
lensing calculation.
Equivalently, once $N(r)$ is fixed by the EOS, the Shapiro coefficient
is not an independent input: it is a derived quantity.


\paragraph{Redshift integrals.}

For redshift, the relevant quantity is the fractional density deficit,
Eq.~\eqref{eq:app-deltarho},
\begin{equation}
  \frac{\Delta\rho(r)}{\rho_0}
  = -\frac{GM}{c_0^2 r},
\end{equation}
which, as noted above, is independent of $n$ at leading order.
If the rest mass of a defect is proportional to the local density,
\begin{equation}
  m(r) \propto \rho_{\text{local}}(r),
\end{equation}
then
\begin{equation}
  \frac{\Delta m(r)}{m_0}
  = \frac{\Delta\rho(r)}{\rho_0}
  = -\frac{GM}{c_0^2 r},
\end{equation}
and any clock whose characteristic frequency satisfies $\omega\propto m$
obeys
\begin{equation}
  \frac{\Delta\omega(r)}{\omega_0}
  = -\frac{GM}{c_0^2 r}.
\end{equation}
In the 1PN matching limit $c_0\to c$ this becomes
\begin{equation}
  \frac{\Delta\omega(r)}{\omega_0}
  = -\frac{GM}{c^2 r},
\end{equation}
which reproduces the standard weak-field gravitational redshift.
As emphasized in the main text, the redshift constraint alone does not
fix $n$, because $\Delta\rho/\rho_0$ does not depend on the EOS in the
linear regime; it is the combination of redshift, Shapiro, and lensing
that singles out $n=5$.

%---------------------------------------------------------------------------
\section{PPN mapping and metric consistency}
\label{app:ppn-metric}

In this appendix we make explicit how the toy model maps onto the standard
parametrized post-Newtonian (PPN) framework at 1PN, and why a naive
identification of the ``optical metric'' with the full spacetime metric
fails to reproduce the correct perihelion precession.
We organize the discussion into three parts:
(i) the optical metric inferred directly from $N(r)$,
(ii) the perihelion precession one would obtain by treating defects as
test particles on that optical metric, and
(iii) how the scalar lag sector and kinetic prefactor $\sigma(r)$ repair
this mismatch and restore the GR value of the PPN parameter $\beta$.
From the viewpoint of the toy model, this is the sharpest internal
consistency check: the same vacuum configuration that fixes the optical
sector must also yield a total perihelion advance with coefficient $6$,
not $10$, once the scalar lag and hydrodynamic dressing encoded in
$\sigma(r)$ are taken into account.


\subsection{Optical metric from $N(r)$}
\label{subsec:app-optical-metric}

The refractive index profile for the stiff $n=5$ vacuum derived in the main
text is
\begin{equation}
  N(r)
  \equiv \frac{c_0}{c_s(r)}
  \simeq 1 + 2\,\frac{GM}{r c^2},
  \qquad
  r = |\mathbf{x}|,
  \label{eq:app-ppn-N}
\end{equation}
where we have identified $c_0\to c$ in the 1PN matching limit.
In geometric optics, this can be represented by an \emph{optical metric}
in which spatial distances are stretched by a factor $N(r)$ for the
purpose of ray propagation.
A natural choice in isotropic coordinates is
\begin{equation}
  \mathrm{d}s^2_{\text{opt}}
  = -c^2 \mathrm{d}t^2
    + N^2(r)\,
      \bigl(\mathrm{d}r^2 + r^2 \mathrm{d}\Omega^2\bigr),
  \label{eq:app-ppn-optical-metric}
\end{equation}
where $\mathrm{d}\Omega^2$ is the metric on the unit 2-sphere.

Expanding $N^2$ to first order in $GM/(r c^2)$,
\begin{equation}
  N^2(r)
  = \bigl(1 + 2GM/(r c^2)\bigr)^2
  \simeq 1 + 4\,\frac{GM}{r c^2}
  + \mathcal{O}\!\left(\frac{G^2M^2}{c^4 r^2}\right),
\end{equation}
so that
\begin{equation}
  \mathrm{d}s^2_{\text{opt}}
  \simeq -c^2 \mathrm{d}t^2
    + \left[1 + 4\,\frac{GM}{r c^2}\right]
      \bigl(\mathrm{d}r^2 + r^2 \mathrm{d}\Omega^2\bigr).
  \label{eq:app-ppn-optical-metric-expanded}
\end{equation}
Comparing this with the usual 1PN isotropic form,
\begin{equation}
  \mathrm{d}s^2
  = -\left(1 - 2\frac{GM}{r c^2} + \dots\right)c^2 \mathrm{d}t^2
    + \left(1 + 2\gamma\frac{GM}{r c^2} + \dots\right)
      \bigl(\mathrm{d}r^2 + r^2 \mathrm{d}\Omega^2\bigr),
  \label{eq:app-ppn-general-metric}
\end{equation}
we see that the optical metric~\eqref{eq:app-ppn-optical-metric-expanded}
corresponds to the choice
\begin{equation}
  g_{tt}^{\text{(opt)}} = -1,
  \qquad
  g_{rr}^{\text{(opt)}} = 1 + 4\frac{GM}{r c^2},
\end{equation}
i.e.\ to an effective
\begin{equation}
  \gamma_{\text{optical}}
  = 2,
  \qquad
  \Phi_{\text{optical}}(r) = 0,
\end{equation}
if one tries to read off PPN parameters naively.

This mismatch is, of course, an artifact of interpreting the purely
spatial optical metric as a full spacetime metric: in
Eq.~\eqref{eq:app-ppn-optical-metric} we have \emph{by construction} left
$g_{tt}$ unperturbed while packing all of the refractive information into
$g_{ij}$.
For null rays this is harmless: what matters is the combination of $g_{tt}$
and $g_{ij}$ that controls the spatial projections of null geodesics, and
this combination gives the correct light-bending and Shapiro coefficients
when $N(r)$ has the form~\eqref{eq:app-ppn-N}.
However, if one now takes Eq.~\eqref{eq:app-ppn-optical-metric-expanded}
and uses it as the \emph{full} spacetime metric for massive test bodies,
one does not recover the correct perihelion precession.


\subsection{Precession from the optical metric}
\label{subsec:app-optical-precession}

To see the problem explicitly, consider a hypothetical model in which
\begin{quote}
  (i) the spacetime metric is taken to be
  Eq.~\eqref{eq:app-ppn-optical-metric-expanded}, and \\
  (ii) massive defects are treated as point test particles following
  timelike geodesics of this metric.
\end{quote}
We refer to this as the \emph{optical test-particle} model.

In isotropic coordinates, the expanded optical metric
Eq.~\eqref{eq:app-ppn-optical-metric-expanded} reads, to first order in
$GM/(r c^2)$,
\begin{equation}
  \mathrm{d}s^2_{\text{opt}}
  \simeq -c^2 \mathrm{d}t^2
    + \left[1 + 4\,\frac{GM}{r c^2}\right]
      \bigl(\mathrm{d}r^2 + r^2 \mathrm{d}\varphi^2\bigr).
\end{equation}
Comparing this with the usual 1PN isotropic form
\begin{equation}
  \mathrm{d}s^2
  = -\left(1 - 2\frac{GM}{r c^2} + \dots\right)c^2 \mathrm{d}t^2
    + \left(1 + 2\gamma\frac{GM}{r c^2} + \dots\right)
      \bigl(\mathrm{d}r^2 + r^2 \mathrm{d}\Omega^2\bigr),
  \label{eq:app-ppn-general-metric-again}
\end{equation}
we see that the optical metric has
\begin{equation}
  g_{tt}^{\text{(opt)}} = -1,
  \qquad
  g_{rr}^{\text{(opt)}} = 1 + 4\frac{GM}{r c^2}
  \;\Rightarrow\;
  \gamma_{\text{opt}} = 2,
  \qquad
  \Phi_{\text{opt}}(r) = 0.
\end{equation}
To recover the Newtonian limit for massive bodies one must therefore
\emph{manually} add a Newtonian potential
\begin{equation}
  \Phi_{\text{opt}}(r) = -\frac{GM}{r},
\end{equation}
and, in PPN language, take $\beta_{\text{opt}} = 1$ while keeping
$\gamma_{\text{opt}} = 2$ from the optical sector.

For bound orbits in a static, spherically symmetric field, the standard
1PN PPN expression for the perihelion advance is
\begin{equation}
  \Delta\varphi
  = \frac{6\pi GM}{a(1-e^2)c^2}
    \left(\frac{2 - \beta + 2\gamma}{3}\right),
\end{equation}
where $a$ is the semi-major axis and $e$ the eccentricity of the orbit.
Inserting the ``naive'' optical test-particle parameters
\begin{equation}
  \beta_{\text{opt}} = 1,
  \qquad
  \gamma_{\text{opt}} = 2,
\end{equation}
Here $\beta_{\text{opt}}$ and $\gamma_{\text{opt}}$ are PPN parameters; the
hydrodynamic parameter $\beta$ used in the main text is a distinct
inertia-renormalization coefficient fixed to $\beta=3/2$ by Paper~1.
yields
\begin{equation}
  \Delta\varphi_{\text{opt,PN}}
  = \frac{6\pi GM}{a(1-e^2)c^2}
    \left(\frac{2 - 1 + 2\times 2}{3}\right)
  = 10\,\frac{\pi GM}{a(1-e^2)c^2},
\end{equation}
i.e.\ a coefficient $10$ rather than the GR value $6$.
Equivalently, the optical test-particle model overpredicts the perihelion
precession by a factor $10/6 = 5/3$.

This is the revised version of the optical vs GR discrepancy mentioned in
the main text: if one attempts to use the optical metric alone as the full
spacetime metric for massive defects, while keeping the Newtonian limit
intact, the perihelion precession comes out too large by a factor $5/3$.
The optical test-particle model is therefore \emph{not} 1PN-equivalent to
Schwarzschild, even though it reproduces the correct light-bending
coefficient.
In the language of the main text, this hypothetical construction
corresponds to forcing defects to follow the bare acoustic geometry of
phonons, ignoring the hydrodynamic dressing that appears in
$\beta = \kappa_\rho + \kappa_{\mathrm{add}}$.
Sections~\ref{sec:soliton-geodesics} and~\ref{subsec:trilemma} explain
why this is not the correct dynamics for solitonic defects.


\subsection{Role of scalar lag and $\beta$}
\label{subsec:app-role-beta}

The actual superfluid-defect model avoids this problem in two ways:

\begin{enumerate}
  \item Massive defects are not treated as point test particles on the
        pure optical metric; instead, their dynamics are governed by the
        effective Lagrangian constructed in Paper~1,
        \begin{equation}
          L
          = \frac{1}{2}\bigl[1 + \sigma(r)\bigr]
              \bigl(\dot r^2 + r^2 \dot\varphi^2\bigr)
            - \Phi_{\mathrm{eff}}(r),
          \label{eq:app-ppn-L-eff-again}
        \end{equation}
        with
        \begin{equation}
          \Phi_{\mathrm{eff}}(r)
          = -\frac{GM}{r} - \frac{G^2M^2}{2 c^2 r^2},
          \qquad
          \sigma(r) = \beta\,\frac{GM}{c^2 r}.
        \end{equation}

  \item The scalar lag field and the kinetic prefactor $\sigma(r)$ are
        interpreted as different components of an \emph{emergent} metric,
        rather than as ad hoc corrections to Newtonian dynamics.
\end{enumerate}

To make the connection explicit, compare the geodesic Lagrangian in a
generic static, spherically symmetric metric written in isotropic
coordinates,
\begin{equation}
  \mathrm{d}s^2
  = -\left[1 + 2\Phi(r)/c^2 + \dots\right]c^2 \mathrm{d}t^2
    + \left[1 + 2\Psi(r)/c^2 + \dots\right]
      \bigl(\mathrm{d}r^2 + r^2 \mathrm{d}\varphi^2\bigr).
\end{equation}
Expanding the point-particle action to order $v^2/c^2$ gives
\begin{equation}
  L_{\text{geo}}
  = \frac{1}{2}\bigl[1 + 2\Psi(r)/c^2\bigr]
      \bigl(\dot r^2 + r^2 \dot\varphi^2\bigr)
    - \Phi(r)
    + \mathcal{O}\!\left(\frac{v^4}{c^2}\right).
  \label{eq:app-ppn-L-geo-again}
\end{equation}
Comparing Eqs.~\eqref{eq:app-ppn-L-eff-again}
and~\eqref{eq:app-ppn-L-geo-again}, we identify
\begin{equation}
  \Phi(r) \leftrightarrow \Phi_{\mathrm{eff}}(r),
  \qquad
  1 + \sigma(r) \leftrightarrow 1 + 2\Psi(r)/c^2.
\end{equation}
To leading order in $GM/(r c^2)$, this implies
\begin{equation}
  \Psi(r)
  = \frac{1}{2}\sigma(r) c^2
  = \frac{\beta}{2}\frac{GM}{r}.
\end{equation}

From Paper~1 we know that the combination of the scalar lag potential
and the $\sigma(r)$ correction produces a total apsidal advance
\begin{equation}
  \Delta\varphi_{\text{tot}}
  = \bigl(3 + 2\beta\bigr)\,
    \frac{\pi GM}{a(1-e^2)c^2}.
\end{equation}
Requiring agreement with the Schwarzschild result,
\begin{equation}
  \Delta\varphi_{\text{GR}}
  = 6\,\frac{\pi GM}{a(1-e^2)c^2},
\end{equation}
fixes
\begin{equation}
  3 + 2\beta = 6
  \quad\Rightarrow\quad
  \beta = \frac{3}{2}.
\end{equation}
Using the decomposition~\eqref{eq:paper1-beta-decomp} and the explicit
added-mass calculation in Appendix~\ref{app:dipole-derivation}, this can
be written as
\begin{equation}
  \beta = \kappa_\rho + \kappa_{\mathrm{add}}
        = 1 + \frac{1}{2}.
\end{equation}
Thus the same hydrodynamic coefficients that control the inertial
dressing of a moving throat also control the PPN parameter that repairs
the ``$10$ vs $6$'' discrepancy in the optical test-particle model.
In terms of the metric functions, this corresponds to
\begin{equation}
  \Psi(r)
  = \frac{3}{4}\,\frac{GM}{r},
  \qquad
  \Phi(r)
  = -\frac{GM}{r}
      - \frac{G^2M^2}{2 c^2 r^2},
\end{equation}
which, when mapped onto the PPN parameters, yields
$\beta_{\text{eff}} = 1$ and $\gamma_{\text{eff}} = 1$ at 1PN.

The crucial point is that the coefficient of the $GM/(r c^2)$ term in
$\Psi(r)$ is no longer tied directly to the index profile $N(r)$ as it was
in the pure optical metric; instead, it is determined by the hydrodynamic
structure of the defect (through $\beta$) and by the scalar lag sector.
Light rays remain governed by the refractive index $N(r)$, which fixes the
combination of $\Phi$ and $\Psi$ that controls null geodesics and thus
$\gamma=1$, while massive solitons see the full effective metric built out
of $\Phi$ and $\Psi$.
The interplay of these two pieces is what repairs the ``$10$ vs $6$''
discrepancy (optical test particles overpredict by $5/3$) and yields a
single emergent metric that is 1PN-equivalent to
Schwarzschild.

In summary:

\begin{itemize}
  \item The \emph{optical metric alone}, interpreted as a full spacetime
        metric, gives the right light-bending coefficient but an incorrect
        perihelion precession ($10$ instead of $6$, i.e.\ an overestimate
        by a factor $5/3$).
  \item The \emph{full emergent metric}, which incorporates both the
        scalar lag potential and the kinetic prefactor $\sigma(r)$ as
        determined in Paper~1, yields the correct precession and remains
        consistent with the optical sector, with effective
        $\beta_{\text{eff}} = \gamma_{\text{eff}} = 1$ at 1PN.
  \item This reinforces the interpretation that defects are solitons
        following geodesics in an emergent metric, not test particles on
        the pure optical metric derived from $N(r)$ alone.
\end{itemize}

%---------------------------------------------------------------------------
\section{Hydrodynamic added mass of a moving throat}
\label{app:dipole-derivation}

In this appendix we derive the added mass coefficient
$\kappa_{\mathrm{add}} = 1/2$ for a stiff spherical throat moving through
the superfluid vacuum.  The defect is modeled as a massless spherical void
of radius $R$ that moves at constant velocity $v$ through an otherwise
static, incompressible, inviscid fluid of density $\rho_0$.  The void displaces
fluid, but has no bulk mass of its own; any inertia associated with its motion
must therefore arise from the kinetic energy of the induced flow field.

We work in the potential-flow regime, with velocity potential $\phi$ and
velocity field $\mathbf{u} = \nabla\phi$.
We neglect viscosity, so the flow is purely potential and the forces are
conservative; viscous drag and dissipation lie outside the added-mass
picture used here.
For a sphere moving along the $z$-axis, the leading disturbance at large $r$
is a dipole.  We therefore take the ansatz
\begin{equation}
  \phi(r,\theta) = -\,\frac{\mu_{\mathrm{dip}}}{r^2}\cos\theta,
\end{equation}
where $(r,\theta)$ are spherical coordinates with $\theta=0$ in the
direction of motion and $\mu_{\mathrm{dip}}$ is a dipole moment to be fixed
by the boundary condition.

The void is assumed to be \emph{stiff}: its spherical topology is maintained
by topological tension, so the fluid cannot cross the boundary.
The radial velocity of the fluid at the surface must therefore match the
radial velocity of the surface itself,
\begin{equation}
  u_r\big|_{r=R} = v\cos\theta,
\end{equation}
where $v$ is the speed of the throat in the lab frame.
From the potential we have
\begin{equation}
  u_r = \frac{\partial\phi}{\partial r}
      = \frac{2\mu_{\mathrm{dip}}}{r^3}\cos\theta,
\end{equation}
so the boundary condition at $r=R$ gives
\begin{equation}
  \frac{2\mu_{\mathrm{dip}}}{R^3}\cos\theta = v\cos\theta
  \quad\Rightarrow\quad
  \mu_{\mathrm{dip}} = \frac{1}{2} v R^3.
\end{equation}
Substituting this back into the potential,
\begin{equation}
  \phi(r,\theta) = -\,\frac{v R^3}{2 r^2}\cos\theta.
\end{equation}

The kinetic energy density of the flow is
\begin{equation}
  \mathcal{E}_{\mathrm{kin}} = \frac{1}{2}\rho_0\,|\mathbf{u}|^2,
\end{equation}
with
\begin{equation}
  u_r = \frac{\partial\phi}{\partial r}
      = v R^3 \frac{\cos\theta}{r^3},
  \qquad
  u_\theta = \frac{1}{r}\frac{\partial\phi}{\partial\theta}
      = \frac{v R^3}{2 r^3}\sin\theta.
\end{equation}
The magnitude squared is
\begin{equation}
  |\mathbf{u}|^2
  = u_r^2 + u_\theta^2
  = v^2 R^6 \frac{\cos^2\theta + \tfrac{1}{4}\sin^2\theta}{r^6}.
\end{equation}
Integrating over the volume outside the sphere,
\begin{equation}
  E_{\mathrm{cloud}}
  = \int_{r>R} \mathcal{E}_{\mathrm{kin}}\,\mathrm{d}V
  = \frac{1}{2}\rho_0 v^2 R^6
    \int_R^\infty \frac{\mathrm{d}r}{r^4}
    \int_0^\pi \mathrm{d}\theta \sin\theta
      \left(\cos^2\theta + \frac{1}{4}\sin^2\theta\right)
    \int_0^{2\pi}\mathrm{d}\varphi.
\end{equation}
The angular integrals give a numerical factor of order unity, and the radial
integral converges at infinity.
Carrying out the integrals explicitly one finds
\begin{equation}
  E_{\mathrm{cloud}}
  = \frac{1}{4} m_{\mathrm{disp}} v^2,
\end{equation}
where
\begin{equation}
  m_{\mathrm{disp}} = \frac{4\pi}{3}\rho_0 R^3
\end{equation}
is the mass of the displaced fluid.
Equating this to the kinetic energy of an effective added mass,
\begin{equation}
  E_{\mathrm{cloud}} = \frac{1}{2} m_{\mathrm{add}} v^2,
\end{equation}
we obtain
\begin{equation}
  m_{\mathrm{add}} = \frac{1}{2} m_{\mathrm{disp}}
  \quad\Rightarrow\quad
  \kappa_{\mathrm{add}} = \frac{m_{\mathrm{add}}}{m_{\mathrm{disp}}}
  = \frac{1}{2}.
\end{equation}

Thus, a stiff spherical throat moving through the superfluid carries an
effective inertial mass equal to half the mass of the fluid it displaces.
This \emph{hydrodynamic added mass} justifies the choice
$\kappa_{\mathrm{add}} = 1/2$ used in the main text and, together with
$\kappa_\rho = 1$, yields the total kinetic prefactor
$\beta = \kappa_\rho + \kappa_{\mathrm{add}} = 3/2$ that restores the GR
perihelion precession.

(We note that this derivation assumes a spherical displacement envelope; this is compatible with the composite 'Dyon' topological structures required for the gravitomagnetic sector, provided the effective cross-section remains spherical.)

This result is specific to the spherical displacement geometry assumed here.
Applying the same method to ellipsoidal voids shows that deviations from
sphericity generally increase (oblate, ``pancake'') or decrease (prolate,
``cigar'') the added-mass coefficient relative to $\kappa_{\mathrm{add}}=1/2$.
For example, a $10\%$ oblate deformation already shifts
$\kappa_{\mathrm{add}}$ by $\sim 12\%$, which translates into a
$\sim 2\%$ error in the predicted perihelion precession.
The observational agreement with GR at the 1PN level therefore constrains
the moving throat to behave as a stiff, approximately spherical obstacle on
the brane.

%---------------------------------------------------------------------------
\section{Numerical methods for the optics sector}
\label{app:numerics-optics}

This appendix documents the numerical methods intended for checking the
analytic results of Sections~\ref{sec:lensing} and~\ref{sec:shapiro}.
The goal is to provide enough detail that the ray-tracing and
time-of-flight codes can be implemented, debugged, and reproduced without
revisiting the main text.

\subsection{Ray-tracing algorithms}
\label{subsec:app-ray-tracing}

The ray-tracing problem is to integrate the geometric-optics equations for
light-like excitations in the refractive index field
\begin{equation}
  N(r)
  \simeq 1 + 2\,\frac{GM}{r c^2}
\end{equation}
derived in Section~\ref{sec:n5-vacuum}.
We treat rays as trajectories in phase space $(\mathbf{x},\mathbf{k})$.

\paragraph{Hamiltonian formulation.}

The local dispersion relation in the rest frame of the fluid is
\begin{equation}
  \omega(\mathbf{x},\mathbf{k})
  = c_s(\mathbf{x})\,|\mathbf{k}|
  = \frac{c}{N(\mathbf{x})}\,|\mathbf{k}|,
\end{equation}
so we take the Hamiltonian
\begin{equation}
  H(\mathbf{x},\mathbf{k})
  = c_s(\mathbf{x})\,|\mathbf{k}|
  = \frac{c}{N(\mathbf{x})}\,|\mathbf{k}|.
\end{equation}
The ray equations are then
\begin{align}
  \dot{\mathbf{x}}
  &= \frac{\partial H}{\partial \mathbf{k}}
   = c_s(\mathbf{x})\,\frac{\mathbf{k}}{|\mathbf{k}|},
  \\
  \dot{\mathbf{k}}
  &= -\frac{\partial H}{\partial \mathbf{x}}
   = -|\mathbf{k}|\,\nabla c_s(\mathbf{x}),
\end{align}
with dots denoting derivatives with respect to an affine parameter
$\lambda$ (not necessarily physical time).

Because $H$ is homogeneous of degree one in $\mathbf{k}$, there is a
redundancy in the choice of $|\mathbf{k}|$: we may fix
$|\mathbf{k}| = k_0$ at the boundary and track only the direction of
$\mathbf{k}$, or we may choose a reparametrization in which $H$ is
constant along the ray.
In practice it is often simplest numerically to fix
$|\mathbf{k}| = 1$ initially and renormalize periodically to control
drift.

\paragraph{Dimensional reduction and coordinate choices.}

For a static, spherically symmetric $N(r)$ with zero background flow, the
ray trajectories are planar.
We can therefore restrict attention to the equatorial plane and work in
2D cylindrical coordinates $(R,z)$ or polar coordinates $(r,\varphi)$.
A convenient choice for lensing is:

\begin{itemize}
  \item Use Cartesian coordinates $(x,z)$ in the ray plane, with the mass
        at the origin and the unperturbed ray direction along $+z$.
  \item Initialize rays at $z = z_\mathrm{in} \ll 0$ with transverse
        offset $x = b$ (the impact parameter) and initial wavevector
        approximately aligned with $+z$.
\end{itemize}

The refractive index and its gradient are then
\begin{align}
  r(x,z) &= \sqrt{x^2 + z^2},\\
  N(x,z) &= 1 + 2\,\frac{GM}{r(x,z) c^2},\\
  \nabla c_s &= -\frac{c}{N^2}\nabla N,
\end{align}
with $\nabla N$ computed either analytically or by finite differences
on a grid.

\paragraph{Integration scheme.}

We recommend a standard 4th–5th order adaptive Runge--Kutta method (such
as Dormand–Prince) or a symplectic integrator with velocity-Verlet-type
updates, depending on the desired balance between accuracy and long-term
stability.
A minimal implementation is:

\begin{itemize}
  \item State vector
        $\mathbf{y} = (x,z,k_x,k_z)$.
  \item Right-hand side $\dot{\mathbf{y}} = f(\mathbf{y})$ defined by the
        Hamiltonian equations above.
  \item Adaptive step size $\Delta\lambda$ chosen so that the local
        truncation error remains below a specified tolerance (e.g.\
        $10^{-8}$ in dimensionless units).
  \item Integration domain $z \in [z_\mathrm{in}, z_\mathrm{out}]$ with
        $|z_\mathrm{in}|, |z_\mathrm{out}| \gg b$ so that
        $N \simeq 1$ and the trajectory has asymptoted to a straight line.
\end{itemize}

At the end of each integration, the outgoing ray direction is read off
from the asymptotic wavevector $\mathbf{k}_\mathrm{out}$, and the
deflection angle is computed as
\begin{equation}
  \Delta\theta_{\text{num}}(b)
  = \arccos\!\left(
      \frac{\mathbf{k}_\mathrm{in}\cdot\mathbf{k}_\mathrm{out}}
           {|\mathbf{k}_\mathrm{in}||\mathbf{k}_\mathrm{out}|}
    \right),
\end{equation}
with $\mathbf{k}_\mathrm{in}$ the initial wavevector.

\paragraph{Units and nondimensionalization.}

For numerical stability it is convenient to work with dimensionless
variables.
One natural choice is:

\begin{itemize}
  \item Use $r_0$ as a characteristic length scale (e.g.\ $r_0 = b$ or
        $r_0 = GM/c^2$).
  \item Define dimensionless coordinates
        $\tilde{\mathbf{x}} = \mathbf{x}/r_0$ and parameters
        $\epsilon = GM/(c^2 r_0)$.
  \item Express $N(\tilde{r}) \simeq 1 + 2\epsilon/\tilde{r}$.
\end{itemize}

The ray equations are then written in terms of $(\tilde{\mathbf{x}},
\tilde{\mathbf{k}})$ with $\epsilon\ll1$.
This makes it easier to scan over parameter ranges relevant to different
regimes (e.g.\ solar-system vs.\ strong-field).


\subsection{Time-of-flight integration}
\label{subsec:app-tof}

The Shapiro delay calculations require computing travel times in the same
$N(r)$ background.
Two complementary numerical strategies can be used.

\paragraph{Direct line integral.}

In the small-deflection approximation, we treat the ray as following a
straight-line path with fixed impact parameter $b$.
The one-way travel time from $z=-Z_\mathrm{E}$ to $z=+Z_\mathrm{R}$ is
approximated as
\begin{equation}
  t_\mathrm{num}
  = \frac{1}{c}\int_{-Z_\mathrm{E}}^{+Z_\mathrm{R}}
     N\bigl(r(z)\bigr)\,\mathrm{d}z,
  \qquad
  r(z) = \sqrt{b^2 + z^2}.
\end{equation}
In discrete form,
\begin{equation}
  t_\mathrm{num}
  \simeq \frac{1}{c}\sum_{i=0}^{N_z} N\bigl(r(z_i)\bigr)\,\Delta z,
\end{equation}
with a uniform or adaptive grid in $z$.

The flat-space time is
\begin{equation}
  t_0
  = \frac{Z_\mathrm{E} + Z_\mathrm{R}}{c},
\end{equation}
so the numerical Shapiro delay is
\begin{equation}
  \Delta t_\mathrm{num}
  = t_\mathrm{num} - t_0.
\end{equation}

Convergence is checked by halving $\Delta z$ and verifying that
$\Delta t_\mathrm{num}$ changes by less than a specified tolerance.

\paragraph{Time accumulation along numerical rays.}

For consistency with the full ray-tracing, one can also accumulate travel
time along the numerically computed curved ray:
\begin{equation}
  t_\mathrm{num}
  = \int \frac{N\bigl(\mathbf{x}(\lambda)\bigr)}{c}
         \left|\frac{\mathrm{d}\mathbf{x}}{\mathrm{d}\lambda}\right|
         \mathrm{d}\lambda.
\end{equation}
In practice, during the integration of the ray equations we maintain an
accumulator $t(\lambda)$ with update rule
\begin{equation}
  \dot{t}
  = \frac{N\bigl(\mathbf{x}(\lambda)\bigr)}{c}
    \left|\dot{\mathbf{x}}(\lambda)\right|.
\end{equation}
A simple discretization is
\begin{equation}
  t_{n+1}
  = t_n
    + \frac{N(\mathbf{x}_n) + N(\mathbf{x}_{n+1})}{2c}
      \left|\mathbf{x}_{n+1} - \mathbf{x}_n\right|,
\end{equation}
with $(\mathbf{x}_n)$ the positions along the numerical ray.
Subtracting the flat-space time between the same endpoints yields the
numerical delay.

This method automatically captures any corrections due to the slight
bending of the ray away from a straight line, but is more expensive than
the direct line integral.
In the weak-field regime, the two methods should agree within numerical
error.


\subsection{Convergence tests and error estimates}
\label{subsec:app-convergence}

\paragraph{Resolution scans.}

For each observable (deflection angle, Shapiro delay), we perform:

\begin{itemize}
  \item A step-size scan for the ray integrator:
        \begin{itemize}
          \item Run the integration with step sizes
                $\Delta\lambda$, $\Delta\lambda/2$, $\Delta\lambda/4$.
          \item Measure the change in $\Delta\theta_\mathrm{num}$ and
                $\Delta t_\mathrm{num}$ between successive refinements.
          \item Infer an empirical convergence rate (e.g.\ consistent with
                4th-order for RK4).
        \end{itemize}
  \item A grid-resolution scan for the direct line integrals:
        \begin{itemize}
          \item Halve $\Delta z$ and compare the resulting
                $\Delta t_\mathrm{num}$.
          \item Confirm that the change scales as $\mathcal{O}(\Delta z^p)$
                with $p\simeq2$ (trapezoidal) or $p\simeq4$ (Simpson).
        \end{itemize}
\end{itemize}

\paragraph{Domain-size dependence.}

The analytic formulas assume $Z_\mathrm{E},Z_\mathrm{R}\gg b$ so that the
logarithmic factors can be approximated with $r_\mathrm{E}\simeq Z_\mathrm{E}$,
$r_\mathrm{R}\simeq Z_\mathrm{R}$.
Numerically, we check:

\begin{itemize}
  \item How $\Delta\theta_\mathrm{num}(b)$ and
        $\Delta t_\mathrm{num}(b)$ change as we move the integration
        boundaries $z_\mathrm{in}$, $z_\mathrm{out}$ further out.
  \item That the results converge once $|z_\mathrm{in,out}|/b$ exceeds a
        threshold (e.g.\ $50$ or $100$).
\end{itemize}

\paragraph{Comparison to analytic benchmarks.}

For each observable we define a dimensionless ratio:

\begin{align}
  \mathcal{R}_\theta(b)
  &= \frac{\Delta\theta_\mathrm{num}(b)}
          {4GM/(b c^2)},\\[0.5ex]
  \mathcal{R}_t(b; r_\mathrm{E}, r_\mathrm{R})
  &= \frac{\Delta t_\mathrm{num}(b; r_\mathrm{E}, r_\mathrm{R})}
          {2GM c^{-3}\ln(4 r_\mathrm{E} r_\mathrm{R}/b^2)}.
\end{align}

We will quantify the numerical error as
\begin{equation}
  \delta_\theta
  = \max_b\bigl|\mathcal{R}_\theta(b) - 1\bigr|,
  \qquad
  \delta_t
  = \max_{b,r_\mathrm{E},r_\mathrm{R}}
      \bigl|\mathcal{R}_t(b; r_\mathrm{E}, r_\mathrm{R}) - 1\bigr|,
\end{equation}
over the parameter ranges of interest.
Target tolerances (e.g.\ $\delta_\theta,\delta_t \lesssim 10^{-3}$) can
then be used to set the step sizes and grid resolutions in production
runs.

%---------------------------------------------------------------------------
\begin{thebibliography}{99}

\bibitem{Norris:2025Orbits}
T.~Norris,
``Newtonian and 1PN Orbital Dynamics from a Superfluid Defect Toy Model,''
Zenodo, 2025. \url{https://doi.org/10.5281/zenodo.17759367}

\bibitem{Einstein:1915}
A.~Einstein,
``Zur allgemeinen Relativit\"atstheorie,''
\textit{Sitzungsber. Preuss. Akad. Wiss. Berlin (Math. Phys.)} (1915) 778–786.

\bibitem{Einstein:1916}
A.~Einstein,
``Die Grundlage der allgemeinen Relativit\"atstheorie,''
\textit{Annalen der Physik} \textbf{49} (1916) 769–822.

\bibitem{Eddington:1920}
A.~S.~Eddington,
\textit{Space, Time and Gravitation: An Outline of the General Relativity Theory},
Cambridge University Press, 1920.

\bibitem{Shapiro:1964}
I.~I.~Shapiro,
``Fourth Test of General Relativity,''
\textit{Phys. Rev. Lett.} \textbf{13} (1964) 789–791.

\bibitem{Will:1993}
C.~M.~Will,
\textit{Theory and Experiment in Gravitational Physics},
Cambridge University Press, 2nd ed., 1993.

\bibitem{Will:2014}
C.~M.~Will,
``The Confrontation between General Relativity and Experiment,''
\textit{Living Rev. Relativ.} \textbf{17} (2014) 4.

\bibitem{Weinberg:1972}
S.~Weinberg,
\textit{Gravitation and Cosmology: Principles and Applications of the General Theory of Relativity},
Wiley, 1972.

\bibitem{MisnerThorneWheeler}
C.~W.~Misner, K.~S.~Thorne and J.~A.~Wheeler,
\textit{Gravitation},
W.~H.~Freeman, 1973.

\bibitem{Barcelo:2005}
C.~Barcel\'o, S.~Liberati and M.~Visser,
``Analogue Gravity,''
\textit{Living Rev. Relativ.} \textbf{8} (2005) 12.

\bibitem{Unruh:1981}
W.~G.~Unruh,
``Experimental Black-Hole Evaporation?,''
\textit{Phys. Rev. Lett.} \textbf{46} (1981) 1351–1353.

\bibitem{Visser:1997}
M.~Visser,
``Acoustic Black Holes: Horizons, Ergospheres, and Hawking Radiation,''
\textit{Class. Quant. Grav.} \textbf{15} (1998) 1767–1791.

\bibitem{LandauLifshitz:Fluids}
L.~D.~Landau and E.~M.~Lifshitz,
\textit{Fluid Mechanics},
Pergamon Press, 2nd ed., 1987.

\bibitem{Chandrasekhar:1939}
S.~Chandrasekhar,
\textit{An Introduction to the Study of Stellar Structure},
University of Chicago Press, 1939.

\bibitem{Tooper:1965}
R.~F.~Tooper,
``Adiabatic Fluid Spheres in General Relativity,''
\textit{Astrophys. J.} \textbf{142} (1965) 1541–1562.

\bibitem{PoissonWill:2014}
E.~Poisson and C.~M.~Will,
\textit{Gravity: Newtonian, Post-Newtonian, Relativistic},
Cambridge University Press, 2014.

\bibitem{Blanchet:2014}
L.~Blanchet,
``Gravitational Radiation from Post-Newtonian Sources and Inspiralling Compact Binaries,''
\textit{Living Rev. Relativ.} \textbf{17} (2014) 2.

\bibitem{Visser:2002}
M.~Visser,
``Essential and Inessential Features of Hawking Radiation,''
\textit{Int. J. Mod. Phys. D} \textbf{12} (2003) 649–661.

\bibitem{AnalogGravityBook}
D.~Faccio, F.~Belgiorno, S.~Cacciatori, V.~Gorini, S.~Liberati and U.~Moschella (eds.),
\textit{Analogue Gravity Phenomenology},
Springer, 2013.

\end{thebibliography}

\end{document}
