\documentclass[11pt]{article}

% Basic packages
\usepackage[margin=1in]{geometry}
\usepackage{amsmath,amssymb,amsfonts}
\usepackage{bm}
\usepackage{graphicx}
\usepackage{hyperref}
\usepackage[numbers,sort&compress]{natbib}
\usepackage{authblk}

% Hyperref setup
\hypersetup{
  colorlinks=true,
  linkcolor=blue,
  citecolor=blue,
  urlcolor=blue
}

% Custom commands (tune / extend as needed)
\newcommand{\PN}{\mathrm{PN}}
\newcommand{\cS}{c_s}
\newcommand{\PhiP}{\Phi_{\mathrm{P}}}
\newcommand{\PhiL}{\Phi_{\mathrm{L}}}
\newcommand{\PhiTot}{\Phi}
\newcommand{\GM}{GM}
\newcommand{\ve}{\varepsilon}
\newcommand{\dd}{\mathrm{d}}
\newcommand{\vfluid}{\mathbf{v}_{\mathrm{fluid}}}
\newcommand{\Afluid}{\mathbf{A}_{\mathrm{fluid}}}
\newcommand{\Gam}{\Gamma}

\title{Spin, Vorticity, and N-Body Dynamics\\
in a Superfluid Defect Toy Model}

\author{Trevor Norris}
\date{\today}

\begin{document}

\maketitle

\begin{abstract}
We extend our superfluid--defect toy universe to spin and full $N$--body dynamics at 1PN order.
The vacuum is a compressible superfluid and massive bodies are flux--tube ``throat'' defects that exchange mass/flux with the medium (effective sinks in the 3D description).
Earlier papers fixed the orbital sector ($\beta=3$)~\cite{Norris:2025Orbits} and selected a stiff $n=5$ equation of state from optical 1PN observables~\cite{Norris:Paper2}.

Here we promote defects to composite ``dyons'' (sink + vortex ring).
Their far--field circulation defines a gravitomagnetic vector potential with the correct $J/r^3$ scaling, reproducing the Lense--Thirring effect and fixing the vortex--spin calibration by matching the Kerr weak--field limit.
In the $N$--body problem, density--dependent masses reproduce the static $G^2$ three--body term of the Einstein--Infeld--Hoffmann (EIH) Lagrangian.
The remaining velocity--dependent EIH cross terms arise from overlap energy of translational wakes.
Using an isotropic projector decomposition into longitudinal and transverse solenoidal (vortical) components (with an optional helical mode), we match the full EIH cross--term tensor with real parameters: $a_H=0$, $\alpha^2=3/4$, and $K=2/\pi^2$.
The positive value $\alpha^2=3/4$ makes the quadratic wake functional positive--definite, showing that earlier imaginary couplings were artifacts of an incomplete wake basis.
Subject to the dynamical requirement that moving sink defects generate a long--range wake $u\propto 1/r^2$ (equivalently $u(\mathbf{k})\propto 1/k$), the model reproduces GR's 1PN scalar, optical, spin, and vector dynamics while remaining consistent with the calibrated orbital sector and optical tests.
\end{abstract}

\section{Introduction}
\label{sec:intro}

\subsection{Motivation and overview}
\label{subsec:intro-motivation}

The classic solar-system tests of gravity---perihelion precession, light bending,
Shapiro time delay, gravitational redshift, and the dynamics of weakly bound
$N$-body systems---are often summarized in a single statement:
the Schwarzschild solution of General Relativity (GR) with post-Newtonian (PN)
parameters $\beta_{\mathrm{PPN}} = \gamma_{\mathrm{PPN}} = 1$ passes all currently accessible 1PN tests.
Throughout this series we use $\beta_{\mathrm{PPN}}$ and $\gamma_{\mathrm{PPN}}$
as shorthand for the values inferred from the standard 1PN observables.
For optical tests we denote the optically inferred value by $\gamma_{\mathrm{obs}}$
and identify $\gamma_{\mathrm{PPN}}\equiv\gamma_{\mathrm{obs}}$.
In the toy model this observational equivalence does not require a single
underlying bare metric: light and matter couple to distinct acoustic and
hydrodynamic projections that are calibrated to agree at 1PN.
From this vantage point, any alternative description of gravity must either
reproduce the Schwarzschild metric in the appropriate weak-field limit or offer
a comparably constrained and falsifiable mechanism by which the same observables
arise.

This paper continues a different line of attack.
Instead of starting from a Lorentzian spacetime and quantizing perturbations of
the metric, we treat gravity as an \emph{emergent} phenomenon in a ``toy
universe'' where the vacuum is a compressible superfluid and massive bodies are
flux-tube defects that drain this vacuum.
The effective gravitational dynamics experienced by defects are then encoded in
the density, pressure, and flow of the surrounding superfluid, together with a
small number of phenomenological parameters that characterize the throat
geometry and equation of state.
In this language, the usual PN ``potential'' and its higher-order corrections
are realized as different facets of a single hydrodynamic configuration.

Most analogue-gravity constructions are qualitative: they reproduce some aspect
of GR kinematics (for example, horizon structure or redshift) without attempting
a quantitative match to the full 1PN phenomenology of the solar system.
In this series of papers we pursue a more aggressive question in a deliberately
simple setting:
\medskip

\begin{quote}
\emph{How far toward the full 1PN phenomenology of GR can one push a
minimal, classical hydrodynamic toy model of defects in a compressible
superfluid?}
\end{quote}

\medskip
Paper~I and Paper~II showed that, with a suitable choice of scalar lag dynamics
and equation of state, the toy model can already reproduce the standard scalar
and optical 1PN tests.
The present work addresses the remaining ingredients: spin-induced
gravitomagnetism (Lense--Thirring) and the full $N$-body interaction encoded in
the Einstein--Infeld--Hoffmann (EIH) Lagrangian.
Our goal is not only to match the GR coefficients but to understand which
features of the emergent superfluid description are \emph{forced} by that
match.

\subsection{Summary of Papers I and II}
\label{subsec:intro-summary-papers12}

Paper~I in this series~\cite{Norris:2025Orbits} developed the \emph{orbital} sector
of the toy universe.
The vacuum was modeled as a compressible superfluid, and massive bodies as
flux-tube ``throats'' of radius $a$ and depth $L$ that drain the surrounding
fluid.
A scalar ``lag'' field allowed the bulk fluid to slip relative to the defects,
and a position-dependent kinetic prefactor $\sigma(r)$ encoded how defect
inertia is renormalized in the throat background.
The long-range field sourced by a defect naturally split into two scalar
contributions: an instantaneous Poisson sector that reproduces the Newtonian
$1/r$ potential, and a retarded lag sector that supplies the finite-propagation
effects.
In the updated calibration, the scalar lag field relaxes in the static
single–body limit, so the 1PN perihelion precession is generated entirely by
the inertia profile $\sigma(r)$.
Demanding that nearly Keplerian orbits around an isolated defect reproduce the
GR 1PN perihelion advance fixes a single dimensionless parameter $\beta=3$ in
the defect Lagrangian.
The same cavitation analysis that ties the defect mass to the surrounding
density implies a density–depletion coefficient $\kappa_\rho = 1$, which
controls the static $G^2$ ``gravity gravitates'' term in the $N$–body problem.

Paper~II~\cite{Norris:Paper2} extended the toy universe to the \emph{optical}
sector.
Treating the vacuum as a stiff ($n=5$) polytropic superfluid endowed the medium
with a density-dependent sound speed and hence a radial refractive index
profile $N(r)$ around a defect.
By constructing an effective optical metric from $N(r)$ and comparing to the
Schwarzschild metric, the analysis showed that $n=5$ is uniquely selected
(within the class considered) by the requirements of matching 1PN light
bending, Shapiro time delay, and gravitational redshift.
As emphasized in Paper~II, this agreement is enforced at the level of
calibrated 1PN observables (light bending, Shapiro delay, redshift, and
perihelion precession), while the underlying acoustic (light) and
hydrodynamic (matter) projections can differ (bi-metric structure).
Once the scalar and optical contributions are combined and calibrated, the
inferred PPN parameters satisfy $\beta_{\mathrm{PPN}}=\gamma_{\mathrm{PPN}}=1$,
so the toy model reproduces both the orbital and optical 1PN tests with no new
free parameters beyond those already fixed by the scalar and optical sectors.

Taken together, Papers I and II establish that a single superfluid-defect toy
model can account for the scalar and optical 1PN tests of gravity with a small
set of tightly constrained parameters: the throat aspect ratio $L/a$, the scalar
renormalization parameter $\beta$, and the polytropic index $n$.
What remains, and is addressed in the present work, are the \emph{vector}
phenomena associated with spin and $N$-body dynamics.

\subsection{Scope and roadmap}
\label{subsec:intro-roadmap}

At the 1PN level, the missing pieces fall into two closely related categories.
First, a spinning gravitating body generates a \emph{gravitomagnetic} field:
in GR this is encoded in the off-diagonal metric components $g_{0i}$ and
observed as the Lense--Thirring precession of gyroscopes and orbital planes
around rotating masses.
Second, the dynamics of multiple bodies at 1PN order are governed by the
Einstein--Infeld--Hoffmann Lagrangian, which contains not only the Newtonian
pairwise potential but also static $G^2$ three-body terms and velocity-dependent
interaction terms with a very specific tensor structure.
Reproducing these ingredients is a stringent test of any emergent-gravity model.

In the superfluid language, both effects are naturally associated with
\emph{flow}.
We model spinning defects as \emph{dyons}: composite objects in which a
flux-tube sink (mass) is bound to a localized vortical core (spin) in the surrounding
superfluid.
The far-zone flow is taken to have a dipole-like (vector-potential) structure
$\mathbf{v}\propto (\mathbf{J}\times\hat{\mathbf{r}})/r^2$ (so Cartesian components scale as
$1/r^2$); through the metric dictionary it
reproduces the weak-field Kerr gravitomagnetic potential and the observable
Lense--Thirring precession $\bm{\Omega}_{\text{LT}}\propto 1/r^3$.
At the same time, the overlapping velocity fields of multiple dyons give rise
to an effective vector interaction that scales as $1/r$ and can, in principle,
be matched to the EIH velocity-dependent terms.

Section~\ref{sec:inputs} reviews the ingredients we import from Papers I and II:
the scalar lag field, the position-dependent inertia, the $n=5$ stiff equation
of state, and the dictionary that maps density and flow to metric components.
Section~\ref{sec:spin} introduces the dyon construction and shows that it
reproduces the GR Lense--Thirring effect with a fixed calibration between the
vortex strength and the physical angular momentum $J$.
Section~\ref{sec:nbody} then turns to the $N$-body problem: we show how the
density-dependent mass generates the static $G^2$ three-body term, construct
the vector interaction from overlapping translational wakes, and derive the
conditions under which the resulting tensor structure matches the EIH
Lagrangian.
In parallel with the scalar ($q=1$)~\cite{Norris:2025Orbits} and optical ($n=5$)~\cite{Norris:Paper2}
sectors fixed in Papers~I and~II, the EIH cross-term coefficients alone fix the
translational wake that supplies the mixed-velocity terms: the isotropic
projector decomposition yields a real wake mixing $\alpha^2=3/4$ (with minimal
$a_H=0$) and overall coupling $K=2/\pi^2$.
Rather than signaling an instability, the multi-body sector thus selects a
specific hydrodynamic response for the vacuum.

Finally, Section~\ref{sec:discussion} summarizes how Papers~I and~II together
with the present work reproduce the full 1PN phenomenology of GR within the toy
superfluid universe, discusses the limitations of this construction, and
outlines directions for extending the model to higher PN orders, radiative
effects, and the electromagnetic sector.

\section{Inputs from Papers I and II: Scalar Sector and Metric Dictionary}
\label{sec:inputs}

In this section we collect the minimal ingredients from the orbital
and optical analyses of Papers~I and~II that will be treated as inputs
for the present work.
Our goal is not to re-derive those results, but to make explicit
which structures are assumed, which parameters have already been fixed,
and how they combine into an effective metric dictionary that will be
extended to include spin and $N$-body dynamics in the sections that follow.

\subsection{Superfluid defect toy model recap}
\label{subsec:inputs-model}

The underlying ontology of the toy universe is unchanged from
Papers~I and~II.
The fundamental medium is a homogeneous, compressible superfluid
with bulk mass density $\rho_0$ and characteristic wave speed $\cS$.
Matter is represented by localized \emph{defects} that act as sinks
of the superfluid: each defect removes fluid from the bulk and routes
it along a narrow ``throat'' of radius $a$ and depth $L$, before
returning it to the ambient medium.
On scales large compared to $a$ and $L$ the details of the throat
geometry are coarse-grained into an effective point-like source of
strength
\begin{equation}
  \mu \equiv GM,
\end{equation}
where $M$ is the inertial mass associated with the defect and $G$
is the effective gravitational constant in the toy model.

The superfluid bulk is described by a density field $\rho(\mathbf{x},t)$,
a pressure $p(\mathbf{x},t)$, and a velocity field
$\mathbf{v}(\mathbf{x},t)$ which obey the usual continuity and Euler
equations, augmented by sink terms localized on the defect cores.
In the simplest, non-rotating sector considered in Paper~I the flow
is irrotational and can be written in terms of a scalar potential
$\Phi(\mathbf{x},t)$ whose gradient gives the acceleration of a test
defect,
\begin{equation}
  \mathbf{a}(\mathbf{x},t) = -\bm{\nabla} \Phi(\mathbf{x},t).
\end{equation}
For a static defect the far-field solution reduces to the familiar
Newtonian form $\Phi(r) \simeq -\mu/r$.
More generally, time dependence and finite propagation speed in the
medium give rise to a retarded scalar contribution which behaves as
a post-Newtonian correction to the effective potential.

In the full toy universe a complete defect (a ``dyon'') can in principle
carry both a scalar sink and a vector vortex; in Papers~I and~II the
focus was on the scalar and optical sectors, and the vector (spin)
structure was left implicit.
In the present work we will restore the dyon picture explicitly,
but the scalar fields and equation of state remain exactly those
determined in the earlier papers.

\subsection{Scalar lag field, inertia profile, and \texorpdfstring{$\beta$}{beta}}
\label{subsec:inputs-scalar-beta}

The scalar sector of the toy model contains two distinct pieces.
The first is an instantaneous Poisson contribution
$\Phi_{\mathrm{N}}(r) = -\mu/r$ sourced by the defect, which plays
the role of the Newtonian potential.
The second is a retarded ``lag'' contribution $\Phi_{\mathrm{lag}}$
arising from finite propagation speed in the medium, which obeys a
wave equation of the schematic form
\begin{equation}
  \frac{\partial^2 \Phi}{\partial t^2}(\mathbf{x},t)
  = \cS^2 \bigl[\nabla^2 \Phi(\mathbf{x},t) - 4\pi G\,\rho(\mathbf{x},t)\bigr].
\end{equation}
For slowly moving sources this retarded contribution reduces, at
leading post-Newtonian order, to a spherically symmetric $1/r^2$
correction,
\begin{equation}
  \Phi_{\mathrm{lag}}(r)
  = -\frac{\mu^2}{2 \cS^2 r^2},
\end{equation}
so that the total effective potential seen by a non-relativistic
test defect is
\begin{equation}
  \Phi_{\mathrm{eff}}(r)
  = \Phi_{\mathrm{N}}(r) + \Phi_{\mathrm{lag}}(r)
  = -\frac{\mu}{r} - \frac{\mu^2}{2 \cS^2 r^2}.
  \label{eq:Phi-eff-paper1}
\end{equation}

Paper~I further introduced a position-dependent kinetic prefactor
$\sigma(r)$ which encodes how the inertial response of a defect is
renormalized by the throat background.
At the level of an effective point-particle description, the kinetic
term acquires a multiplicative factor $[1+\sigma(r)]$, and the
inertial mass $m$ appearing in the Lagrangian is replaced by
\begin{equation}
  m_{\mathrm{eff}}(r) = m\,[1+\sigma(r)],
\end{equation}
with
\begin{equation}
  \sigma(r) = \beta\,\frac{\mu}{\cS^2 r},
  \label{eq:sigma-paper1}
\end{equation}
where $\beta$ is a dimensionless constant.
One can think of $\sigma(r)$ as a simple model for a radial
dependence of the spatial metric components experienced by the
defect, or equivalently as a phenomenological way of encoding
hydrodynamic inertia in the coarse-grained description.

When the Lagrangian with $\Phi_{\mathrm{eff}}(r)$ and $\sigma(r)$ is
expanded consistently to 1PN order and the resulting equations of
motion are applied to nearly Keplerian orbits, the scalar lag field
relaxes in the static single–body limit and does not contribute to
the perihelion advance.
The entire 1PN precession is generated by the position-dependent
inertial dressing encoded in $\sigma(r)$, yielding a total
coefficient
\begin{equation}
  \Delta\varphi_{\text{tot}}
  = 6\,\frac{\pi \mu}{\cS^2 a(1-e^2)},
\end{equation}
with $a$ and $e$ the orbital semi-major axis and eccentricity,
provided
\begin{equation}
  \beta = 3.
\end{equation}
In this sense, Paper~I fixed the single scalar parameter $\beta$ by
requiring that the toy model reproduce the GR 1PN perihelion
precession purely through inertial renormalization.
This same value of $\beta$ will be assumed throughout the present
work, and will enter the $N$-body analysis via the density-dependent
mass and static $G^2$ term in the EIH Lagrangian.

\subsection{Optical sector and the stiff \texorpdfstring{$n=5$}{n=5} vacuum}
\label{subsec:inputs-optics}

Paper~II turned to the optical and clock sectors of the toy model.
There the superfluid vacuum was endowed with a polytropic equation
of state
\begin{equation}
  p = K\,\rho^{1+1/n},
\end{equation}
with $n$ the polytropic index and $K$ a constant.
A mass defect was again modeled as a flux-tube sink embedded in this
vacuum, with the same coarse-grained parameter $\mu = GM$ describing
its far-field strength.
Hydrostatic balance in the presence of the effective potential
$\Phi(r)$ then implies a radial density and pressure deficit around
the defect, which in turn modify the local sound speed
$c_s(r)$ and define a refractive index profile
\begin{equation}
  N(r) = \frac{c_0}{c_s(r)},
\end{equation}
where $c_0$ is the sound speed in the far-field vacuum.

For a general polytropic index $n$ one obtains, in the weak-field
regime, a refractive index of the form
\begin{equation}
  N(r) \simeq 1 + \alpha_n\,\frac{GM}{c_0^2 r},
\end{equation}
with a coefficient $\alpha_n$ that depends on $n$.
Paper~II showed that the stiff $n=5$ branch is singled out by the
1PN optical tests.
Specializing to $n=5$ yields
\begin{equation}
  N_{n=5}(r)
  \simeq 1 + 2\,\frac{GM}{c_0^2 r},
  \label{eq:N-n5-paper2}
\end{equation}
which is the profile used in the lensing and Shapiro-delay
calculations.
In the 1PN matching limit one identifies $c_0$ with $c$, so the
coefficient in Eq.~\eqref{eq:N-n5-paper2} is directly comparable to
the GR result.

Treating lightlike excitations as rays propagating through an
inhomogeneous medium with refractive index $N(r)$, Paper~II computed
the weak-field bending angle for a point mass and showed that
Eq.~\eqref{eq:N-n5-paper2} reproduces the standard GR deflection
angle with the correct coefficient.
An analogous analysis of signal propagation time in the same index
profile yielded the familiar logarithmic Shapiro delay, again with
the GR coefficient.
Finally, by constructing an effective optical metric from $N(r)$ and
relating it to the PN expansion of the spacetime metric, Paper~II
showed that the combined scalar and optical sectors correspond to
PPN parameters
\begin{equation}
  \beta_{\mathrm{PPN}} = 1,
  \qquad
  \gamma_{\mathrm{PPN}} = 1,
\end{equation}
when all contributions are accounted for.
The toy model therefore reproduces both the orbital and optical 1PN
tests of GR with a single choice of throat geometry and equation of
state: a flux-tube defect with a fixed aspect ratio $L/a$, scalar
renormalization parameter $\beta=3$ in the defect Lagrangian, and a
stiff $n=5$ polytropic vacuum.

\subsection{Remaining 1PN tasks}
\label{subsec:inputs-missing}

The inputs summarized above completely determine the scalar and
optical sectors of the toy universe at 1PN order.
The Newtonian potential and its scalar lag correction fix the
effective $g_{00}$ components relevant for slow-motion dynamics,
while the refractive index profile in the $n=5$ vacuum captures the
optical manifestations of the spatial metric and yields $\gamma=1$
when compared to GR.
From the PPN point of view, the model already behaves like GR in the
monopole, spherically symmetric sector.

What remains are the genuinely \emph{vector} phenomena associated
with flow and vorticity.
A spinning mass generates a gravitomagnetic field, encoded in GR by
the off-diagonal metric components $g_{0i}$ and observed as
Lense--Thirring precession.
In an $N$-body system, the full 1PN dynamics are captured by the
Einstein--Infeld--Hoffmann Lagrangian, which contains static $G^2$
three-body terms and velocity-dependent pairwise terms with a highly
constrained tensor structure.
In the superfluid language these effects arise from the velocity
field $\mathbf{v}(\mathbf{x},t)$ sourced by spinning defects and
from the interaction energy of overlapping flows.

The present paper builds on the scalar potential $\Phi_{\mathrm{eff}}$,
the inertia profile $\sigma(r)$, and the $n=5$ refractive index
$N(r)$ summarized above, and extends the toy model to include spin
and $N$-body dynamics.
In particular, we will construct composite defects (dyons) whose
scalar sink and vortex ring together reproduce the GR Lense--Thirring
field, and we will show how the interaction of their flows generates
the full EIH 1PN Lagrangian, fixing the translational wake parameters to
($a_H=0$, $\alpha^2=3/4$, $K=2/\pi^2$) once the $q=1$ scalar background from
Paper~I~\cite{Norris:2025Orbits} and stiff $n=5$ optical sector from
Paper~II~\cite{Norris:Paper2} are taken into account.

\section{Spin and the Lense--Thirring Effect}
\label{sec:spin}

In Papers~I and~II the defects were treated as non-spinning sources:
only the scalar sink structure of the throat entered the analysis.
From the GR point of view this corresponds to working with a
Schwarzschild-like sector where the metric is diagonal and the
off–diagonal components $g_{0i}$ vanish.
The next natural step is to endow the defects with spin and ask
whether the same superfluid toy universe can reproduce the
gravitomagnetic phenomena associated with rotating masses, most
notably the Lense--Thirring effect.
In this section we show that a composite ``dyon'' defect---a flux-tube
sink bound to a vortex ring in the surrounding superfluid---does
exactly that.

\subsection{The radial mismatch problem}
\label{subsec:spin-mismatch}

In GR, the dominant spin effect of an isolated, slowly rotating body
with angular momentum vector $\mathbf{J}$ is described by the
Lense--Thirring precession.
To leading order in $J$ and in the weak-field limit, the metric can
be written as
\begin{equation}
  g_{0i}^{\text{(GR)}}
  = -\frac{2G}{c^3}\,
    \epsilon_{ijk}\,\frac{J^j x^k}{r^3}
  + \mathcal{O}(J^2),
\end{equation}
with $r = |\mathbf{x}|$.
The precession of a gyroscope at position $\mathbf{r}$ is then
governed by the gravitomagnetic precession vector
\begin{equation}
  \bm{\Omega}_{\text{LT}}(\mathbf{r})
  = \frac{G}{c^2 r^3}
    \left[3(\mathbf{J}\cdot\hat{\mathbf{r}})\,\hat{\mathbf{r}}
          - \mathbf{J}\right],
  \label{eq:Omega-LT-GR}
\end{equation}
which scales as $J/r^3$.
Any viable emergent-gravity model must reproduce this $1/r^3$
behavior together with the angular dependence encoded in
Eq.~\eqref{eq:Omega-LT-GR}.

A natural first guess in a superfluid picture is to model a spinning
defect as a line vortex aligned with $\mathbf{J}$.
For a straight vortex along the $z$–axis the azimuthal velocity
in cylindrical coordinates $(r_\perp,\phi,z)$ is
\begin{equation}
  v_\phi(r_\perp)
  = \frac{\Gamma}{2\pi r_\perp},
\end{equation}
where $\Gamma$ is the circulation.
The local angular velocity of the fluid around the vortex is then
$\omega \sim v_\phi / r_\perp \propto 1/r_\perp^2$.
This has the wrong radial dependence: the induced precession falls as
$1/r^2$ rather than $1/r^3$ and cannot be made compatible with the
Lense--Thirring scaling at large distances.
Moreover, the flow is topologically constrained to be purely
azimuthal; it does not reproduce the dipolar angular structure in
Eq.~\eqref{eq:Omega-LT-GR}.

A different classical construction is to consider a slowly rotating
sphere of radius $R$ and angular velocity $\bm{\Omega}$ embedded in
an otherwise static fluid.
Continuity then demands a compensating backflow around the sphere,
and in the far field the velocity field takes the form of a dipole:
\begin{equation}
  \mathbf{v}_{\text{backflow}}(\mathbf{r})
  \sim \frac{R^3}{r^3}\,\bm{\Omega}\times\mathbf{r},
  \qquad r \gg R.
  \label{eq:backflow}
\end{equation}
This has the desired dipolar structure $\propto \bm{\Omega}\times\mathbf{r}/r^3$
(Cartesian components $\propto 1/r^2$) and the right
$\bm{\Omega}\times\mathbf{r}$ angular dependence to mimic the GR
gravitomagnetic vector potential locally.
However, when used as the building block for an $N$-body interaction
it behaves too much like a rigid rotation of the bulk: the overlap
energy of two such backflows decays as $1/r^3$ and therefore induces
an interaction that falls too rapidly with separation to match the
Einstein--Infeld--Hoffmann (EIH) vector term, which requires an
effective $1/r$ potential.
We will return to this tension in Sec.~\ref{sec:nbody}; for now it
suffices to note that neither a simple line vortex nor the rigid
backflow of a spinning sphere provides a satisfactory starting point
for a unified description of spin and $N$-body dynamics.

\subsection{The dyon solution}
\label{subsec:spin-dyon}

The construction that succeeds in the toy universe is a composite
defect, or \emph{dyon}, in which a scalar sink and a vortex ring are
bound together on the same throat.
The scalar sink is exactly the one used in Papers~I and~II: it
removes fluid from the bulk and sources the effective potential
$\Phi(r)$.
The new ingredient is a localized vortical core that carries the
defect's spin and sources a long-range azimuthal flow.
In this paper we take the far-zone behavior of that flow as a
phenomenological requirement imposed by matching the weak-field Kerr
gravitomagnetic potential (equivalently the Lense--Thirring observable);
the microscopic/topological origin of such a defect is deferred.
In spherical coordinates $(r,\theta,\phi)$ with the $z$–axis chosen
along the angular momentum $\mathbf{J}$, we define the far-zone
\emph{physical} azimuthal speed by
\begin{equation}
  v_\phi(r,\theta)
  = \frac{D}{r^2}\,\sin\theta,
  \qquad (r\gg a),
  \label{eq:vphi-dyon}
\end{equation}
where $D$ is an effective spin-flow strength fixed by matching the Kerr
weak-field gravitomagnetic potential.
In these coordinates one should also distinguish physical and covariant
azimuthal components: $v_{\phi}^{\text{(cov)}} = r\sin\theta\,
v_{\phi}^{\text{(phys)}}$, so $v_\phi^{\text{(phys)}}\propto 1/r^2$
implies $v_{\phi}^{\text{(cov)}}\propto 1/r$, consistent with the Kerr
weak-field scaling $g_{0\phi}\propto \sin^2\theta/r$ and hence
$\Omega\propto 1/r^3$.

To connect this flow to gravitomagnetism we use the same acoustic
metric dictionary as in the scalar and optical sectors.
At leading order in the flow speed, the effective line element for
test defects moving in the superfluid can be written schematically as
\begin{equation}
  \dd s^2
  = -\left(1 + \frac{2\Phi}{c^2}\right)c^2 \dd t^2
    - \frac{4}{c^3}\,\mathbf{A}_{\text{eff}}\cdot\dd\mathbf{x}\,\dd t
    + \left(1 - \frac{2\Psi}{c^2}\right)\dd\mathbf{x}^2,
  \label{eq:acoustic-metric}
\end{equation}
where $\Phi$ and $\Psi$ are the scalar potentials already fixed by
Papers~I and~II, and $\mathbf{A}_{\text{eff}}$ is an effective vector
potential proportional to the bulk flow velocity $\mathbf{v}$.
For an irrotational flow $\mathbf{A}_{\text{eff}}$ can be removed by
a gauge transformation; for the vortical dyon flow in
Eq.~\eqref{eq:vphi-dyon} it carries physical content and directly
encodes $g_{0i}$ at 1PN order.

Writing
\begin{equation}
  \mathbf{A}_{\text{eff}}(\mathbf{r})
  = \kappa\,\rho_0\,\mathbf{v}(\mathbf{r}),
\end{equation}
with $\rho_0$ the far-field density and $\kappa$ a constant of
proportionality determined by the underlying hydrodynamics, and
inserting Eq.~\eqref{eq:vphi-dyon} into the off–diagonal part of
Eq.~\eqref{eq:acoustic-metric}, one finds a gravitomagnetic potential
of the form
\begin{equation}
  g_{0i}^{\text{(dyon)}}
  = -\frac{2G}{c^3}\,
    \epsilon_{ijk}\,\frac{\tilde{J}^j x^k}{r^3},
  \qquad
  \tilde{\mathbf{J}}
  = \alpha_D\,D\,\hat{\mathbf{z}},
\end{equation}
for some dimensionless constant $\alpha_D$.
Matching this to the GR expression fixes the relation between the
dyon spin-flow strength $D$ and the physical angular momentum
$\mathbf{J}$:
\begin{equation}
  D = \frac{4G}{c^2}\,J,
  \label{eq:D-calibration}
\end{equation}
up to the same sign conventions used to orient the circulation and
the spin.
In other words, once the inertial mass $M$ and spin $J$ of a defect
are specified, the strength of its vortex ring is not a new free
parameter: it is fixed by the requirement that the far-field
gravitomagnetic potential coincide with the 1PN Kerr limit.

\subsection{Acoustic metric and observable spin tests}
\label{subsec:spin-observables}

Given the calibration in Eq.~\eqref{eq:D-calibration}, the dyon
construction reproduces not only the form of $g_{0i}$ but also the
standard spin precession observables of GR.
The precession of a gyroscope with spin $\mathbf{S}$ at position
$\mathbf{r}$ in a stationary spacetime with metric $g_{\mu\nu}$ is
governed, to 1PN order, by the equation
\begin{equation}
  \frac{\dd\mathbf{S}}{\dd t}
  = \bm{\Omega}\times\mathbf{S},
\end{equation}
where $\bm{\Omega}$ receives a contribution from the
gravitomagnetic potential $g_{0i}$.
Inserting the dyon-induced $g_{0i}$ into the standard PN precession
formula yields exactly the Lense--Thirring vector
$\bm{\Omega}_{\text{LT}}(\mathbf{r})$ of Eq.~\eqref{eq:Omega-LT-GR}
with $\mathbf{J}$ identified as the defect spin.

Similarly, the precession of the orbital plane of a test defect
moving in the field of a spinning dyon reproduces the GR nodal
precession rate.
For a nearly circular orbit of radius $r$ around a central dyon of
mass $M$ and spin $J$ aligned with the $z$–axis, the rate of change
of the longitude of the ascending node is
\begin{equation}
  \dot{\Omega}_{\text{node}}
  = \frac{2GJ}{c^2 r^3},
\end{equation}
to leading order in $J$.
This matches the classic GR Lense--Thirring result for satellites
around the Earth, and is the quantity measured by experiments such
as LAGEOS and Gravity Probe~B.
In the toy universe, once $J$ is specified, this precession rate is
an output of the dyon flow; there is no room to independently adjust
the strength of the gravitomagnetic coupling.

\subsection{Falsifiability in the spin sector}
\label{subsec:spin-falsifiability}

From the standpoint of the toy model, the spin sector introduces no
new continuous parameters beyond the physical angular momentum
$\mathbf{J}$ of each defect.
The mapping between $J$ and the dyon spin-flow strength $D$ in
Eq.~\eqref{eq:D-calibration} is fixed by the requirement that the
far-field gravitomagnetic potential match the Kerr limit of GR.
Given this calibration, the same dyon construction determines:

\begin{itemize}
  \item the Lense--Thirring precession of gyroscopes in orbit around
        a spinning mass,
  \item the nodal precession of orbital planes for test defects,
  \item and, as we will see in Sec.~\ref{sec:nbody}, the vector part
        of the $N$-body interaction encoded in the EIH Lagrangian via
        translational wakes of moving defects.
\end{itemize}

This rigidity makes the spin sector sharply falsifiable.
If future measurements of frame-dragging around rotating bodies were
to deviate from the GR Lense--Thirring predictions, the dyon
construction in its present form would fail.
Conversely, the fact that a single composite defect---a flux-tube
sink bound to a vortex ring---can reproduce both the scalar and
spin-induced 1PN phenomenology with no additional tuning is a
non-trivial consistency check of the superfluid toy universe.
In the next section we extend the construction to translational wakes
sourced by moving defects to address the full $N$-body problem and the
structure of the EIH Lagrangian.

\section{N-Body Dynamics and the EIH Lagrangian}
\label{sec:nbody}

The scalar and optical sectors of the toy universe already reproduce
the 1PN tests that probe a single, essentially isolated mass: the
Newtonian potential and its scalar correction fix the perihelion
advance, while the $n=5$ refractive index recovers light bending,
Shapiro delay, and gravitational redshift with the GR coefficients.
To complete the 1PN picture one must address the dynamics of multiple
bodies.
In GR this is encoded in the Einstein--Infeld--Hoffmann (EIH)
Lagrangian, which describes the relative motion of point masses in
the weak-field, slow-motion regime.
In this section we show how the same superfluid toy model gives rise
to the EIH structure, and what constraints this imposes on the
underlying hydrodynamics.

\subsection{The EIH target and sector decomposition}
\label{subsec:nbody-target}

For an $N$-body system of point masses $\{m_A\}$ with positions
$\{\mathbf{x}_A\}$ and velocities $\{\mathbf{v}_A\}$, the EIH
Lagrangian at 1PN order can be written schematically as
\begin{equation}
  L_{\text{EIH}}
  = L_{\text{N}}
    + \frac{1}{c^2} L_{1\PN}
  + \mathcal{O}\!\left(\frac{1}{c^4}\right),
\end{equation}
with a Newtonian part
\begin{equation}
  L_{\text{N}}
  = \sum_A \frac{1}{2} m_A v_A^2
    + \frac{1}{2}\sum_{A\neq B}
      \frac{G m_A m_B}{r_{AB}},
\end{equation}
and a 1PN correction that splits into three qualitatively distinct
pieces:
\begin{equation}
  L_{1\PN}
  = L_{\text{kin}}
    + L_{\text{stat}}
    + L_{\text{vec}}.
\end{equation}
Here $L_{\text{kin}}$ is a purely kinetic correction of order $v^4$,
\begin{equation}
  L_{\text{kin}}
  = \sum_A \frac{1}{8} m_A v_A^4,
\end{equation}
$L_{\text{stat}}$ collects the static nonlinear terms proportional
to $G^2$ which couple three masses at a time,
\begin{equation}
  L_{\text{stat}}
  \sim \sum_{A\neq B\neq C}
       \frac{G^2 m_A m_B m_C}{r_{AB} r_{AC}},
\end{equation}
and $L_{\text{vec}}$ contains the velocity-dependent pairwise
interaction terms.
For two bodies $A$ and $B$, the latter can be written in the form
\begin{equation}
  L_{\text{vec}}^{(AB)}
  = \frac{G m_A m_B}{r_{AB}}
    \left[
      \frac{3}{2}\,(v_A^2 + v_B^2)
      - \frac{7}{2}\,\mathbf{v}_A\cdot\mathbf{v}_B
      - \frac{1}{2}\,
        (\mathbf{v}_A\cdot\mathbf{n}_{AB})
        (\mathbf{v}_B\cdot\mathbf{n}_{AB})
    \right],
  \label{eq:EIH_target}
\end{equation}
where $\mathbf{n}_{AB} = (\mathbf{x}_A-\mathbf{x}_B)/r_{AB}$ is the
unit separation vector.
The three coefficients in Eq.~\eqref{eq:EIH_target} are highly
constrained: they encode, in a compact way, the vector and tensor
structure implied by the underlying metric theory.

In the toy superfluid universe, these three pieces have natural
interpretations:

\begin{itemize}
  \item The $v^4$ term $L_{\text{kin}}$ arises from the relativistic
        expansion of the defect kinetic energy in the effective
        metric fixed by Papers~I and~II.
  \item The static nonlinear term $L_{\text{stat}}$ reflects the
        fact that the defect mass depends on the local pressure and
        density; this is essentially the statement that ``gravity
        gravitates'' in the scalar sector.
  \item The velocity-dependent term $L_{\text{vec}}$ is generated by
        the interaction energy of overlapping dyon flows.
        Its detailed tensor structure depends on how the transverse
        (vortical) and longitudinal (compressible) components of
        the flow are coupled.
\end{itemize}

In what follows we focus on the static nonlinear and vector pieces,
which contain the genuinely new physics from the perspective of the
toy model.  The purely kinetic correction will be assumed to take its
standard relativistic form in the emergent metric.

\subsection{Static non-linearity (cavitation)}
\label{subsec:nbody-static}

In the superfluid picture, a defect does not carry a rigid, fixed
mass independent of its environment.
Instead, its effective mass is a property of the throat immersed in
the surrounding vacuum: it depends on the local pressure, density,
and potential.
In Paper~I this was encoded in a position-dependent kinetic
prefactor $\sigma(r)$, which can be rephrased as a density-dependent
mass $m(\rho)$.
For a defect labeled $A$ we can write, to leading order in the
perturbation of the vacuum,
\begin{equation}
  m_A(\mathbf{x}_A)
  = m_{A,0}\left[
      1 + \kappa_\rho\,
          \frac{\Phi_{\text{loc}}(\mathbf{x}_A)}{c^2}
      + \mathcal{O}\!\left(\frac{\Phi^2}{c^4}\right)
    \right],
  \label{eq:mass-density-dep}
\end{equation}
where $m_{A,0}$ is the bare mass, $\Phi_{\text{loc}}$ is the local
effective potential generated by all other defects and the background
vacuum, and $\kappa_\rho$ is a dimensionless density–depletion
coefficient fixed by the cavitation analysis of Paper~I.
In the calibration used throughout Papers~I and~II and the present work one has
$\kappa_\rho = 1$, which ensures that the resulting static $G^2$
three–body term matches the Einstein--Infeld--Hoffmann coefficient
associated with the PPN parameter $\beta_{\mathrm{PPN}}=1$.
The same cavitation analysis used in Paper~I to calibrate the throat
aspect ratio $L/a$ also decomposes the inertial parameter $\beta=3$
into density and pressure--volume contributions, with $\kappa_\rho$
capturing the density-depletion piece.

To see how this generates the static $G^2$ term, consider the
Newtonian potential energy between bodies $A$ and $B$,
\begin{equation}
  V_{AB}^{\text{N}}
  = -\frac{G\,m_A(\mathbf{x}_A)\,m_B(\mathbf{x}_B)}{r_{AB}}.
\end{equation}
Inserting Eq.~\eqref{eq:mass-density-dep} for each mass and expanding
to first order in $\Phi_{\text{loc}}/c^2$ produces correction terms of
order $G^2/c^2$.
For instance, the mass of body $A$ picks up a contribution from the
potential generated by a third body $C$,
\begin{equation}
  \Phi_{\text{loc}}(\mathbf{x}_A)
  \supset -\frac{G m_C}{r_{AC}},
\end{equation}
so that the $AB$ interaction energy acquires a correction
\begin{equation}
  \delta V_{AB}^{(C)}
  = -\frac{G}{r_{AB}}\,
    \left[\kappa_\rho \frac{m_{A,0} m_{B,0}}{c^2}\,
          \Phi_{\text{loc}}(\mathbf{x}_A)
          + (A\leftrightarrow B)
    \right]
  \;\supset\;
  \kappa_\rho\,
  \frac{G^2 m_{A,0} m_{B,0} m_C}{c^2 r_{AB} r_{AC}}.
\end{equation}
Summing over all triplets $(A,B,C)$ and symmetrizing produces a
three-body interaction energy of the schematic form
\begin{equation}
  V_{\text{stat}}^{(3)}
  = -\sum_{A\neq B\neq C}
     \frac{G^2 m_A m_B m_C}{c^2}\,
     F_{\text{stat}}(r_{AB},r_{AC},r_{BC}),
\end{equation}
with
\begin{equation}
  F_{\text{stat}}(r_{AB},r_{AC},r_{BC})
  \propto \frac{1}{r_{AB} r_{AC}}
  + \text{permutations}.
\end{equation}
The precise numerical coefficient of this term is determined by
$\kappa_\rho$ and the details of how $\Phi_{\text{loc}}$ is
assembled from the scalar and optical sectors; the Mathematica
analysis shows that, with $\kappa_\rho=1$ fixed by the single-body
cavitation analysis, the resulting three-body term agrees with the
static $G^2$ part of the EIH Lagrangian.

Conceptually, this mechanism is nothing but ``cavitation'' in the
vacuum: defects are holes that displace fluid, and the amount of
fluid displaced depends on the local pressure and potential, which in
turn are sourced by other defects.
Gravity gravitates because defects feel not only the potential, but
also the potential of the potential, through their density-dependent
mass.

\subsection{Vector interaction: translational wakes}
\label{subsec:nbody-vector}

We now turn to the velocity-dependent \emph{cross} terms in the
Einstein--Infeld--Hoffmann (EIH) Lagrangian, i.e.\ the coefficients of
$\mathbf{v}_A\!\cdot\!\mathbf{v}_B$ and
$(\mathbf{v}_A\!\cdot\!\mathbf{n}_{AB})(\mathbf{v}_B\!\cdot\!\mathbf{n}_{AB})$
in Eq.~\eqref{eq:EIH_target}.
In the toy universe these terms arise from the overlap energy of the
\emph{translational wakes} sourced by \emph{moving} defects.

It is crucial to distinguish this translation-driven wake from the
rotation-driven flow used in the spin sector (Sec.~\ref{sec:spin}).
A spinning defect sources a vortical far field calibrated by matching to
Lense--Thirring ($J/r^3$), whereas the EIH cross terms depend on linear
velocities $\mathbf{v}_A,\mathbf{v}_B$ and must be sourced by the defect's
\emph{translational} motion.

We retain the overlap-energy definition
\begin{equation}
  V_{\text{vec}}^{(AB)}
  = \rho_0 \int \dd^3 x\,
    \mathbf{u}_A(\mathbf{x})\cdot\mathbf{u}_B(\mathbf{x}),
  \label{eq:Vvec-overlap}
\end{equation}
but now interpret $\mathbf{u}_{A,B}$ as translational wake fields.
To obtain a pairwise interaction that falls off as $1/r_{AB}$ at 1PN order,
the wake must scale as $u(\mathbf{k})\propto 1/k$ in Fourier space so that
$u_A\cdot u_B\propto 1/k^2$.

We therefore parametrize the most general isotropic linear-response wake as a
sum of transverse and longitudinal projector pieces (and an optional helical
transverse mode):
\begin{align}
  \mathcal{P}_L^{ij}(\mathbf{k}) &= \frac{k^i k^j}{k^2},
  &
  \mathcal{P}_T^{ij}(\mathbf{k}) &= \delta^{ij} - \frac{k^i k^j}{k^2},
  \\
  \mathbf{u}_{\text{trans}}(\mathbf{k};\mathbf{v})
  &=
  i\,\frac{K}{k}\,\mathcal{P}_T(\mathbf{k})\,\mathbf{v}
  \;+\;
  i\,\frac{K\,a_H}{k^2}\,(\mathbf{k}\times \mathbf{v})
  \;+\;
  i\,\frac{K\,\alpha}{k^3}\,\mathbf{k}\,(\mathbf{k}\cdot\mathbf{v}),
  \label{eq:wake-ansatz}
\end{align}
where $K$ is an overall coupling, $\alpha$ controls the longitudinal
(compressible) wake admixture, and $a_H$ parametrizes a helical transverse
component (set to zero in the minimal EIH match).

Evaluating the overlap integral with the wake ansatz
\eqref{eq:wake-ansatz} produces a $1/r_{AB}$ pair interaction of the EIH form,
\begin{equation}
  V_{\text{vec}}^{(AB)}
  = \frac{G m_A m_B}{c^2 r_{AB}}
    \left[
      C_\parallel(\alpha,a_H)\,
      \mathbf{v}_A\cdot\mathbf{v}_B
      + C_{\text{L}}(\alpha,a_H)\,
      (\mathbf{v}_A\cdot\mathbf{n}_{AB})
      (\mathbf{v}_B\cdot\mathbf{n}_{AB})
      + \cdots
    \right],
  \label{eq:Vvec-general}
\end{equation}
where the ellipsis denotes terms that renormalize into the self-velocity
channel and/or the already-calibrated kinetic prefactors.
The nontrivial information for the EIH tensor match lies in the cross-term
coefficients $C_\parallel$ and $C_{\text{L}}$, derived in
Sec.~\ref{subsec:nbody-derivation} and Appendix~\ref{app:alpha-tuning}.

\subsection{Derivation walkthrough: matching the EIH cross tensor}
\label{subsec:nbody-derivation}

Using the translational wake ansatz \eqref{eq:wake-ansatz} in the overlap
integral \eqref{eq:Vvec-overlap}, the EIH cross-term coefficients take the
universal form (Appendix~\ref{app:alpha-tuning})
\begin{align}
  C_\parallel(\alpha,a_H) &= K\,\pi^2\bigl(-1 + a_H^2 - \alpha^2\bigr),
  \label{eq:Cpara-wake}
  \\
  C_{\text{L}}(\alpha,a_H) &= K\,\pi^2\bigl(-1 + a_H^2 + \alpha^2\bigr),
  \label{eq:CL-wake}
\end{align}
up to the overall coupling $K$.

The EIH targets for the cross terms in Eq.~\eqref{eq:EIH_target} are
\begin{equation}
  C_\parallel^{\text{EIH}}=-\frac{7}{2},
  \qquad
  C_{\text{L}}^{\text{EIH}}=-\frac{1}{2}.
\end{equation}
Taking the ratio eliminates $K$ and yields
\begin{equation}
  \frac{C_{\text{L}}}{C_\parallel}
  =
  \frac{-1 + a_H^2 + \alpha^2}{-1 + a_H^2 - \alpha^2}
  =
  \frac{1}{7},
\end{equation}
which implies
\begin{equation}
  \alpha^2=\frac{3}{4}\,(1-a_H^2).
\end{equation}
The minimal (least structured) wake match sets $a_H=0$, giving the real value
\begin{equation}
  \alpha^2=\frac{3}{4}.
  \label{eq:alpha-squared}
\end{equation}
With this choice, the overall coupling is fixed by the parallel coefficient
\eqref{eq:Cpara-wake}:
\begin{equation}
  K=\frac{2}{\pi^2}.
  \label{eq:K-wake}
\end{equation}
Together these reproduce the EIH cross coefficients exactly:
\begin{equation}
  C_\parallel=-\frac{7}{2},
  \qquad
  C_{\text{L}}=-\frac{1}{2}.
\end{equation}

In particular, no imaginary compressibility parameter is required.
The earlier appearance of $\alpha^2<0$ was an artifact of restricting the
translation wake to an incomplete transverse basis; once the full isotropic
projector decomposition is used, the EIH tensor is obtained with real
parameters.
The coefficient forms and tuning are verified symbolically in the accompanying
Mathematica script \texttt{mathematica/1pn\_spin\_and\_nbody/n\_body.wl}.

\subsection{Stability and interpretation of the wake parameters}
\label{subsec:nbody-signature}

The matching in Sec.~\ref{subsec:nbody-derivation} fixes the relative weight
of the longitudinal (compressible) and transverse (solenoidal/vortical) components of the
\emph{translational} wake.\footnote{Here ``transverse'' is meant in the Fourier--projector sense, $\mathbf{k}\cdot\mathbf{u}_T=0$: a solenoidal (vortical) wake component sourced by defect circulation, not an elastic shear modulus.}  The required value $\alpha^2=3/4$ is real, so the
corresponding quadratic wake functional is positive-definite in the usual
Euclidean hydrodynamic sense.

Equally importantly, the minimal match sets the optional helical transverse
component to zero ($a_H=0$).  This cleanly separates the sectors of the toy
model: spin observables are governed by the rotational/vortical flow
calibrated in Sec.~\ref{sec:spin}, whereas the EIH velocity-dependent cross
terms are governed by the translational wake response.

The remaining physical hypothesis is dynamical rather than algebraic:
the medium must support a long-range translational wake with Fourier scaling
$u(\mathbf{k})\propto 1/k$ (equivalently, $u(\mathbf{x})\propto 1/r^2$) so that
the overlap energy produces a $1/r$ interaction at 1PN order.
Unlike the $1/r^3$ dipolar disturbance produced by a translating impermeable rigid
inclusion in potential flow (schematically $\delta\mathbf{v}\sim a^3\mathbf{U}/r^3$), this
$1/r^2$ scaling is consistent with a defect that continuously exchanges mass/flux
with the vacuum---an effective sink, i.e.\ a monopole-type far-field flow in the effective 3D description.
In the throat picture, this corresponds to exchange with a bulk reservoir rather
than literal mass nonconservation in a closed fluid.
Whether a stiff vacuum equation of state (such as the $n=5$ polytrope inferred in
Paper~II~\cite{Norris:Paper2}) naturally generates this response is a question
for the PDE-level simulations.  In the present work we show that \emph{if} such
wakes exist, their tensor structure can reproduce the EIH interaction with
real, stable parameters.

\section{Discussion and Outlook}
\label{sec:discussion}

\subsection{Summary: 1PN completion}
\label{subsec:discussion-summary}

Taken together, the three papers in this series show how a single,
minimal superfluid--defect toy model can reproduce the full suite of
1PN solar-system tests usually attributed to the Schwarzschild and
Kerr solutions of GR with PPN parameters
$\beta_{\mathrm{PPN}}=\gamma_{\mathrm{PPN}}=1$.
It is useful to summarize the structure of the construction and the
status of the free parameters.

In Paper~I~\cite{Norris:2025Orbits}, the focus was on orbital dynamics in
the static, spherically symmetric sector.
A scalar lag field and a position-dependent kinetic prefactor
$\sigma(r)$ were introduced to model how defects move relative to the
bulk vacuum and how their inertia is renormalized by the throat
background.
In the static single–body limit the lag field relaxes, so the 1PN
perihelion advance is generated entirely by the inertia profile
$\sigma(r)$.
By demanding that nearly Keplerian orbits around an isolated defect
reproduce the observed 1PN perihelion advance, the analysis fixed a
single dimensionless parameter $\beta=3$ and constrains the throat
aspect ratio $L/a$ through the pressure--volume coefficient.
The same cavitation analysis ties the defect mass to the surrounding
density and selects $\kappa_\rho=1$, which sets the strength of the
static $G^2$ term in the $N$--body problem.
Within this calibration, the scalar sector of the model yields an
effective $g_{00}$ component that matches GR at 1PN order.

Paper~II~\cite{Norris:Paper2} extended the same toy universe to the
optical and clock sectors.
Treating the vacuum as a stiff ($n=5$) polytropic superfluid endowed
the medium with a density-dependent sound speed and hence a radial
refractive index profile $N(r)$ around a defect.
By constructing an effective optical metric from $N(r)$ and comparing
to the Schwarzschild metric, the analysis showed that $n=5$ is
uniquely selected (within the class considered) by the requirements
of matching 1PN light bending, Shapiro time delay, and gravitational
redshift.
The resulting effective spacetime has PPN parameters
$\beta_{\mathrm{PPN}}=\gamma_{\mathrm{PPN}}=1$
once the scalar and optical contributions are combined, so the toy
model reproduces both the orbital and optical 1PN tests with no new
free parameters beyond those already fixed in Paper~I.

The present work adds the remaining ingredients: spin-induced
gravitomagnetism and the full $N$-body structure encoded in the
Einstein--Infeld--Hoffmann Lagrangian.
By promoting defects to composite dyons (flux-tube sinks bound to vortex rings),
we showed that the far-field vorticity reproduces the Lense--Thirring
gravitomagnetic potential with the correct $J/r^3$ scaling, fixing the
relation between vortex strength and angular momentum $J$ by matching to the
Kerr weak-field limit.
Separately, the velocity-dependent \emph{cross} terms in the EIH Lagrangian are
reproduced by the overlap energy of translational wakes sourced by moving
defects.  Using the isotropic transverse/longitudinal projector decomposition,
the EIH cross-term tensor fixes a real wake mixing $\alpha^2=3/4$ (with minimal
$a_H=0$) and an overall coupling $K=2/\pi^2$.

With this choice, the 1PN dynamics of the toy universe---including
single-body orbits, light propagation, spin precession, and general
$N$-body motion---are fully specified by a small, tightly constrained
set of parameters:
\begin{itemize}
  \item the throat geometry (a fixed aspect ratio $L/a$),
\item the scalar renormalization parameter ($\beta=3$),
  \item the stiff polytropic index ($n=5$),
  \item and the dyon spin calibration ($D\propto GJ/c^2$) together
        with the translational wake tuning ($a_H=0$, $\alpha^2=3/4$, and $K=2/\pi^2$).
\end{itemize}
No further continuous freedom remains at 1PN order within this
framework.

\subsection{Emergent metric and effective field theory viewpoint}
\label{subsec:discussion-eft}

Although the toy universe is formulated in hydrodynamic language, it
is best understood as an emergent metric theory.
The mapping can be summarized schematically as follows:
\begin{itemize}
  \item The \emph{defects} (flux-tube throats) and their geometry
        encode the rest masses of gravitating bodies and fix the
        scale $\mu=GM$ that appears in the effective potentials.
  \item The scalar lag field and density deficit determine
        $\Phi(r)$ and the inertial renormalization $\sigma(r)$, and
        thereby fix the effective $g_{00}$ component relevant for
        slow-motion dynamics.
  \item The refractive index profile $N(r)$ of the stiff $n=5$
        vacuum captures the optical manifestations of the spatial
        metric, leading to an effective $g_{ij}$ with $\gamma=1$ in
        the PPN expansion.
  \item The bulk flow and vorticity generated by spinning dyons
        furnish an effective vector potential and hence the
        off–diagonal $g_{0i}$ components responsible for
        gravitomagnetism and the vector part of the EIH interaction.
\end{itemize}
In this sense the superfluid variables $(\rho,p,\mathbf{v})$ provide
a particular parameterization of the metric and its first PN
corrections, constrained by the throat microphysics and the equation
of state.

From an effective field theory (EFT) perspective, what the present
series accomplishes is a highly nontrivial matching.
A generic metric EFT with a symmetry structure comparable to GR
contains many a priori independent coefficients at 1PN order, which
are usually encoded in the PPN parameters and higher-derivative
operators.
Here, by contrast, all of the 1PN coefficients are determined by a
handful of hydrodynamic response parameters of a single medium.
The fact that these parameters can be chosen once and for all to
reproduce the GR phenomenology suggests that the superfluid-defect
picture captures a nontrivial subset of metric EFTs.
At the same time, the requirement that the scalar ($q=1$)~\cite{Norris:2025Orbits}
and stiff optical ($n=5$)~\cite{Norris:Paper2} sectors dovetail with the
translational wake tuning ($a_H=0$, $\alpha^2=3/4$, $K=2/\pi^2$) indicates that
not every classical fluid admits such an interpretation: the vacuum
thermodynamics and linear response are tightly constrained.

\subsection{Limitations and open questions}
\label{subsec:discussion-limitations}

The superfluid toy universe is deliberately minimal, and its domain
of applicability is correspondingly limited.
Several caveats and open questions are worth highlighting.

First, the analysis is restricted to the weak-field, slow-motion
regime in which a 1PN expansion is valid.
We have not attempted to extend the construction to strong-field
situations where horizons, ergoregions, or large curvature effects
are important.
Whether the same superfluid medium can support analogues of black
holes or compact binaries that reproduce GR predictions beyond 1PN is
an open question.

Second, radiation and dissipation have been neglected.
The present treatment describes conservative dynamics: there is no
gravitational-wave emission, no radiation reaction, and no associated
energy loss from the system.
In a genuine emergent-gravity scenario one would expect the
superfluid to support wave excitations that play the role of
gravitational radiation, and the dyon dynamics would have to be
augmented to account for their backreaction.
Developing a wave sector consistent with the 1PN matching presented
here is an important next step.

Third, the internal structure of the defects has been treated in a
highly compressed way.
Throat geometry enters only through a small number of coarse-grained
parameters ($a$, $L$, and the circulation $\Gamma$), and the dyon
parameters $(\alpha,a_H,K)$ summarize the longitudinal/transverse/helical
translational wake response.
A more microscopic model of the throat---for example, one rooted in a
condensed-matter analogue or in a specific quantum field theory of
the vacuum---could clarify whether the required thermodynamic choices
($q=1$, $n=5$)~\cite{Norris:2025Orbits,Norris:Paper2} and wake tuning
($a_H=0$, $\alpha^2=3/4$, $K=2/\pi^2$) arise naturally or must be imposed by
hand.
Future work will investigate whether these specific values---especially $n=5$ and $\alpha^2=3/4$---arise from throat boundary conditions in a higher-dimensional (brane--bulk) construction, with defect mass and spin expressed as flux integrals of the underlying geometry.

Fourth, we have so far considered only uncharged, purely gravitational
defects.
The electromagnetic sector is conspicuously absent.
In a fully unified analogue model, electric charge and the Maxwell
field would emerge as additional collective modes of the same vacuum
medium, possibly tied to different components of the throat or its
winding structure.
How the EM sector couples to the superfluid gravity described here,
and whether it preserves the successful 1PN matching, remains to be
seen.

Finally, the connection to cosmology has been left unexplored.
The present work treats the vacuum medium as homogeneous and static
on large scales, with no global expansion or background flow.
Embedding the superfluid-defect construction into an expanding
cosmological background would raise new questions about the role of
defects in structure formation, the effective cosmological constant,
and possible departures from GR on large scales.

\subsection{Observational handles}
\label{subsec:discussion-observables}

Although the toy universe is not intended as a replacement for GR,
it does offer a well-defined set of observational handles.
By construction, any deviation from the 1PN phenomenology of GR in
the regimes considered here would falsify the specific parameter
choices that underlie the model.

In the scalar and optical sectors, the primary tests are the classic
solar-system measurements:
perihelion precession, light deflection by the Sun, Shapiro time
delay, and gravitational redshift.
These observables were used to fix $\beta$, $n$, and $L/a$, so they
serve more as consistency checks than as independent predictions.
Nevertheless, any future refinement that revealed tension among these
constraints would feed back into the allowed parameter space of the
superfluid description.

In the spin sector, frame-dragging measurements provide a sharper
probe.
Experiments such as Gravity Probe~B and the analysis of LAGEOS
satellite orbits already constrain the Lense--Thirring precession
around the Earth to within a few tens of percent.
Within the toy model, the same calibration that matches Kerr at 1PN
fixes the relation between spin $J$ and vortex strength $D$, so the
frame-dragging rates of all spinning defects are tightly linked.
Any confirmed discrepancy between observed and GR-predicted
frame-dragging in the weak-field regime would either rule out the
dyon construction or force a revision of the underlying hydrodynamic
dictionary.

In the $N$-body sector, the EIH Lagrangian governs the dynamics of
weakly bound systems ranging from planetary orbits to wide binaries.
Precision ephemerides in the solar system and timing observations of
pulsar binaries already test the EIH tensor structure to high
accuracy.
Since the superfluid model reproduces this structure only when the
scalar ($q=1$)~\cite{Norris:2025Orbits} and optical ($n=5$)~\cite{Norris:Paper2} sectors dovetail
with the translational wake tuning ($a_H=0$, $\alpha^2=3/4$, $K=2/\pi^2$),
improved measurements of velocity-dependent effects in multi-body systems can
be interpreted as tests of this specific hydrodynamic response.

Looking ahead, the most discriminating tests are likely to arise
beyond the strict 1PN regime: in systems where 2PN corrections, spin
couplings, and radiation reaction all play important roles.
Extending the superfluid-defect toy universe into those domains would
either uncover qualitative departures from GR---providing concrete
targets for observation---or further cement the correspondence
between hydrodynamic and geometric descriptions of gravity.
In either case, the present 1PN completion offers a useful baseline:
it demonstrates that, at least in principle, a remarkably small
amount of structured ``fluid'' can mimic the familiar phenomenology
of curved spacetime.

\appendix

\section{Dyon far-field flow and gravitomagnetic calibration}
\label{app:dyon-flow}

This appendix records only an effective far-field definition and calibration of
the dyon spin flow used in Sec.~\ref{sec:spin}.
In this paper we define the dyon's far-zone behavior by matching to the GR
weak-field Kerr/Lense--Thirring sector; the microscopic or topological origin
of such a defect is deferred.

\subsection{GR target}

The weak-field Kerr gravitomagnetic potential can be written as
\begin{equation}
  g_{0i}^{\rm (GR)}
  = -\frac{2G}{c^3}\,
    \epsilon_{ijk}\,\frac{J^j x^k}{r^3},
  \label{eq:app-g0i-GR}
\end{equation}
which implies the familiar spherical-component scaling
$g_{0\phi}\propto \sin^2\theta/r$ and hence
$\bm{\Omega}_{\text{LT}}\propto 1/r^3$.

\subsection{Acoustic mapping + required flow}

As in the main text, we take the off-diagonal metric term to be sourced by an
effective vector potential $\mathbf{A}_{\text{eff}}$ proportional to the bulk
flow $\mathbf{v}$,
\begin{equation}
  \dd s^2
  \supset - \frac{4}{c^3}\,\mathbf{A}_{\text{eff}}\cdot\dd\mathbf{x}\,\dd t,
  \qquad
  \mathbf{A}_{\text{eff}}(\mathbf{r})=\kappa\,\rho_0\,\mathbf{v}(\mathbf{r}),
\end{equation}
with $\rho_0$ the far-field density and $\kappa$ a constant fixed by the
hydrodynamic dictionary used throughout the paper.

The phenomenological far-zone azimuthal flow used for the dyon is
\begin{equation}
  v_\phi(r,\theta)=\frac{D}{r^2}\sin\theta,
\end{equation}
where $v_\phi$ is the \emph{physical} azimuthal speed and $D$ is an effective
spin-flow strength.
In spherical coordinates the corresponding covariant component carries an
additional factor,
\begin{equation}
  v_{\phi}^{\text{(cov)}} = r\sin\theta\, v_{\phi}^{\text{(phys)}},
\end{equation}
so $v_\phi^{\text{(phys)}}\propto 1/r^2$ implies
$v_{\phi}^{\text{(cov)}}\propto 1/r$, consistent with the Kerr scaling
$g_{0\phi}\propto \sin^2\theta/r$.

\subsection{Calibration}

With the above dictionary, matching the dyon-induced $g_{0i}$ to the GR target
\eqref{eq:app-g0i-GR} fixes the same spin calibration used in the main text,
\begin{equation}
  D=\frac{4G}{c^2}J,
\end{equation}
so the dyon spin-flow strength is not a new free parameter once the physical
spin $J$ is specified.
This appendix therefore serves only to state the effective far-field definition
and calibration; deeper microscopic explanations are left to future work.

\section{Static non-linearity derivation}
\label{app:static-derivation}

In this appendix we make the derivation in
Sec.~\ref{subsec:nbody-static} explicit.
The goal is to show how a density--dependent defect mass
$m_A(\rho)$ generates a three-body static term of order $G^2$ with
the characteristic EIH structure
\begin{equation}
  L_{\text{stat}}
  \;\sim\;
  \sum_{A\neq B\neq C}
  \frac{G^2 m_A m_B m_C}{c^2\, r_{AB} r_{AC}},
\end{equation}
up to a numerical coefficient fixed by the same pressure--volume
analysis used in Paper~I to calibrate $L/a$ and to decompose the
inertial parameter $\beta=3$ into density and pressure--volume
contributions.

\subsection{Density-dependent mass and local potential}

The key physical input is that the effective mass of a defect is not
a fixed constant but depends on the local properties of the vacuum
medium.
In the coarse-grained description this dependence can be parametrized
by the local effective potential $\Phi_{\text{loc}}(\mathbf{x})$,
which encodes both the scalar lag field and the pressure--volume
response of the throat.
To leading order one may write
\begin{equation}
  m_A(\mathbf{x}_A)
  = m_{A,0}\left[
      1 + \kappa_\rho\,
          \frac{\Phi_{\text{loc}}(\mathbf{x}_A)}{c^2}
      + \mathcal{O}\!\left(\frac{\Phi^2}{c^4}\right)
    \right],
  \label{eq:app-mA}
\end{equation}
where $m_{A,0}$ is the bare mass parameter of defect $A$,
$\Phi_{\text{loc}}(\mathbf{x}_A)$ is the effective potential at the
defect location, and $\kappa_\rho$ is a dimensionless
density–response coefficient which, in the calibration of Paper~I,
satisfies $\kappa_\rho=1$.
In the main text this is Eq.~\eqref{eq:mass-density-dep}.

For a configuration of $N$ defects, the local potential can be
split into contributions from individual sources,
\begin{equation}
  \Phi_{\text{loc}}(\mathbf{x}_A)
  = \sum_{B\neq A} \Phi_{B}(\mathbf{x}_A)
    + \Phi_{\text{vac}}(\mathbf{x}_A),
\end{equation}
where $\Phi_{B}$ is the field due to defect $B$ and
$\Phi_{\text{vac}}$ is the (slowly varying) background contribution.
At the level of the static $G^2$ term we may drop the background,
since it does not generate the $1/r_{AB}r_{AC}$ structure of
interest.
In the weak-field regime the field of an isolated defect is
\begin{equation}
  \Phi_{B}(\mathbf{x})
  = -\frac{G m_{B,0}}{|\mathbf{x}-\mathbf{x}_B|}
  + \mathcal{O}\!\left(\frac{G^2 m_{B,0}^2}{c^2 r^2}\right),
\end{equation}
so that, to Newtonian accuracy,
\begin{equation}
  \Phi_{\text{loc}}(\mathbf{x}_A)
  \simeq -\sum_{C\neq A}
          \frac{G m_{C,0}}{r_{AC}},
  \qquad
  r_{AC} = |\mathbf{x}_A-\mathbf{x}_C|.
  \label{eq:app-Phi-loc}
\end{equation}

Inserting Eq.~\eqref{eq:app-Phi-loc} into Eq.~\eqref{eq:app-mA} and
keeping terms up to $\mathcal{O}(G/c^2)$ gives
\begin{equation}
  m_A(\mathbf{x}_A)
  = m_{A,0}
    - \kappa_\rho\,
      \frac{G m_{A,0}}{c^2}
      \sum_{C\neq A}
      \frac{m_{C,0}}{r_{AC}}
    + \mathcal{O}\!\left(\frac{G^2}{c^4}\right).
  \label{eq:app-mA-expanded}
\end{equation}

\subsection{Newtonian pair potential with dressed masses}

The Newtonian potential energy of an $N$--body configuration with
positions $\{\mathbf{x}_A\}$ and effective masses $\{m_A\}$ is
\begin{equation}
  V_{\text{N}}
  = -\frac{1}{2}
    \sum_{A\neq B}
    \frac{G\,m_A(\mathbf{x}_A)\,m_B(\mathbf{x}_B)}{r_{AB}},
  \label{eq:app-VN}
\end{equation}
where the factor of $1/2$ avoids double counting of pairs.
Substituting Eq.~\eqref{eq:app-mA-expanded} for each mass and
expanding to first order in $\kappa_\rho G/c^2$ yields
\begin{align}
V_{\text{N}}
  &= -\frac{1}{2}
     \sum_{A\neq B}
     \frac{G}{r_{AB}}\,
     \Biggl\{
       m_{A,0} m_{B,0}
       - \kappa_\rho\,
         \frac{G m_{A,0} m_{B,0}}{c^2}
         \sum_{C\neq A}
         \frac{m_{C,0}}{r_{AC}}
       \nonumber\\
  &\hspace{7.5em}
       - \kappa_\rho\,
         \frac{G m_{A,0} m_{B,0}}{c^2}
         \sum_{D\neq B}
         \frac{m_{D,0}}{r_{BD}}
       + \mathcal{O}\!\left(\frac{G^2}{c^4}\right)
     \Biggr\}.
     \label{eq:app-VN-expanded}
\end{align}
The first term in braces reproduces the usual Newtonian potential
energy,
\begin{equation}
  V_{\text{N}}^{(0)}
  = -\frac{1}{2}
    \sum_{A\neq B}
    \frac{G m_{A,0} m_{B,0}}{r_{AB}}.
\end{equation}
The remaining terms are of order $G^2/c^2$ and generate the static
three-body interaction that we wish to isolate.

It is convenient to focus on the correction associated with a
particular triple $(A,B,C)$.
Consider the piece of $V_{\text{N}}$ in which the mass of $A$ is
dressed by the potential of $C$:
\begin{equation}
  \delta V_{AB}^{(C)}
  = -\frac{1}{2}
    \sum_{A\neq B}
    \frac{G}{r_{AB}}\,
    \left[
      -\kappa_\rho\,
       \frac{G m_{A,0} m_{B,0}}{c^2}
       \sum_{C\neq A}
       \frac{m_{C,0}}{r_{AC}}
    \right].
\end{equation}
Extracting the contribution with a specific $C$ and suppressing the
subscript $0$ on the bare masses for readability, we can write
\begin{equation}
  \delta V_{AB}^{(C)}
  = \frac{\kappa_\rho G^2}{2 c^2}
    \sum_{A\neq B}
    \sum_{C\neq A}
    \frac{m_A m_B m_C}{r_{AB} r_{AC}}.
  \label{eq:app-dV-ABC}
\end{equation}
A completely analogous term arises from dressing the mass of $B$ by
the potential of a defect $D$,
\begin{equation}
  \delta V_{BA}^{(D)}
  = \frac{\kappa_\rho G^2}{2 c^2}
    \sum_{A\neq B}
    \sum_{D\neq B}
    \frac{m_A m_B m_D}{r_{AB} r_{BD}}.
\end{equation}

\subsection{Symmetrization over triples and EIH form}

The sum in Eq.~\eqref{eq:app-dV-ABC} runs over all ordered pairs
$(A,B)$ and, for each $A$, over all $C\neq A$.
To make contact with the usual EIH notation it is helpful to rewrite
the result as a fully symmetric sum over unordered triples.
Define the triple sum
\begin{equation}
  \sum_{A,B,C}^{\prime}
  \equiv
  \sum_{\substack{A,B,C \\ \text{all distinct}}},
\end{equation}
so that each unordered set $\{A,B,C\}$ appears $3!=6$ times in the
primed sum.
Then the combined three-body correction from dressing both $A$ and
$B$ can be written schematically as
\begin{equation}
  V_{\text{stat}}^{(3)}
  = \frac{\kappa_\rho G^2}{2 c^2}
    \sum_{A,B,C}^{\prime}
    m_A m_B m_C
    \left(
      \frac{1}{r_{AB} r_{AC}}
      + \frac{1}{r_{AB} r_{BC}}
    \right),
  \label{eq:app-V3-pre-sym}
\end{equation}
where the two terms in parentheses correspond to the two ways in
which the potential of the third body can dress the masses in the
pair.
Since the primed sum is fully symmetric in $(A,B,C)$, the structure
in Eq.~\eqref{eq:app-V3-pre-sym} can be reorganized into the more
familiar EIH form
\begin{equation}
  V_{\text{stat}}^{(3)}
  = -C_{\text{stat}}\,
    \frac{G^2}{c^2}
    \sum_{A,B,C}^{\prime}
    \frac{m_A m_B m_C}{r_{AB} r_{AC}},
  \label{eq:app-V3-EIH-form}
\end{equation}
for some positive coefficient $C_{\text{stat}}$ proportional to
$\kappa_\rho$.
The overall minus sign reflects the convention that $V_{\text{N}}$
is negative for an attractive interaction.

In the calibration used in Papers~I and II, the density–depletion
coefficient $\kappa_\rho$ is fixed to unity by the cavitation analysis
of a single defect in the background vacuum.
With this choice, and including additional scalar contributions
subleading in the single-body analysis but of the same order in
$G^2/c^2$, the Mathematica implementation of the full scalar sector
confirms that the three-body term above matches exactly the static
$G^2$ piece of the Einstein--Infeld--Hoffmann Lagrangian, i.e. the
``gravity gravitates'' term controlled by the PPN parameter
$\beta_{\mathrm{PPN}}=1$.

\subsection{Interpretation}

From the hydrodynamic perspective, the derivation above can be
summarized in a simple slogan: \emph{gravity gravitates because
defects are cavities whose mass depends on the ambient potential}.
Each defect displaces a volume of vacuum whose mass content is
modulated by the local pressure and density.
When one defect sits in the field of others, the amount of vacuum it
displaces---and hence its effective inertial/gravitational mass---is
altered.
This dependence shows up as a correction to the pairwise Newtonian
potential energy that couples three masses at a time and falls off as
$1/(r_{AB} r_{AC})$, just as required by the static part of the EIH
Lagrangian.

Crucially, the coefficient of this term is not an independent
parameter: it is fixed by the same pressure--volume response of the
throat that was already constrained by the single-body 1PN analysis.
Once the scalar sector is calibrated to reproduce the perihelion
advance and the pressure--volume coefficient, the static $G^2$
three-body interaction is an unavoidable consequence of the model.

\section{Vector kernel and \texorpdfstring{$\alpha$}{alpha} tuning}
\label{app:alpha-tuning}

In this appendix we derive the translational wake overlap coefficients used in
Sec.~\ref{subsec:nbody-vector} and show how the EIH cross-term tensor fixes the
wake parameters to real values.

\subsection{Fourier-space representation of the translational wake}

We assume that a moving defect sources a long-range translational wake whose
linear response can be decomposed into transverse and longitudinal projector
components (and, optionally, a helical transverse component).
With $\mathcal{P}_T^{ij}=\delta^{ij}-k^i k^j/k^2$ and
$\mathcal{P}_L^{ij}=k^i k^j/k^2$, we write
\begin{equation}
  \mathbf{u}_{\text{trans}}(\mathbf{k};\mathbf{v})
  =
  i\,\frac{K}{k}\,\mathcal{P}_T(\mathbf{k})\,\mathbf{v}
  \;+\;
  i\,\frac{K\,a_H}{k^2}\,(\mathbf{k}\times \mathbf{v})
  \;+\;
  i\,\frac{K\,\alpha}{k^3}\,\mathbf{k}\,(\mathbf{k}\cdot\mathbf{v}),
\end{equation}
which is Eq.~\eqref{eq:wake-ansatz} in the main text.

\subsection{Overlap integral and coefficient extraction}

For two bodies $A$ and $B$, the interaction energy is
\begin{equation}
  V_{\text{vec}}^{(AB)}
  = \rho_0 \int \dd^3 x\,
    \mathbf{u}_A(\mathbf{x})\cdot\mathbf{u}_B(\mathbf{x})
  =
  \rho_0 \int \frac{\dd^3 k}{(2\pi)^3}\,
    \mathbf{u}_A(-\mathbf{k})\cdot\mathbf{u}_B(\mathbf{k})\,
    e^{i\mathbf{k}\cdot\mathbf{r}_{AB}}.
\end{equation}
Performing the angular integrals yields a $1/r_{AB}$ interaction of the form
\eqref{eq:Vvec-general} with cross-term coefficients
\begin{align}
  C_\parallel(\alpha,a_H) &= K\,\pi^2\bigl(-1 + a_H^2 - \alpha^2\bigr),
  \\
  C_{\text{L}}(\alpha,a_H) &= K\,\pi^2\bigl(-1 + a_H^2 + \alpha^2\bigr).
\end{align}

\subsection{EIH matching and tuned parameters}

Matching the EIH cross-term coefficients in Eq.~\eqref{eq:EIH_target} requires
\begin{equation}
  C_\parallel=-\frac{7}{2},
  \qquad
  C_{\text{L}}=-\frac{1}{2}.
\end{equation}
The ratio constraint gives
\begin{equation}
  \alpha^2=\frac{3}{4}\,(1-a_H^2),
\end{equation}
and the minimal match sets $a_H=0$, yielding $\alpha^2=3/4$
(Eq.~\eqref{eq:alpha-squared}).
The overall coupling is then fixed by $C_\parallel=-7/2$ to
\begin{equation}
  K=\frac{2}{\pi^2},
\end{equation}
as in Eq.~\eqref{eq:K-wake}.
With these choices the wake overlap reproduces the EIH cross tensor exactly.

\subsection{Summary of the tuning procedure}

For clarity, the translational-wake tuning proceeds as follows:
\begin{enumerate}
  \item Write the isotropic wake response in projector form
        (Eq.~\eqref{eq:wake-ansatz}), with longitudinal mixing parameter $\alpha$
        and optional helical amplitude $a_H$.
  \item Evaluate the overlap integral to obtain $C_\parallel(\alpha,a_H)$ and
        $C_{\text{L}}(\alpha,a_H)$.
  \item Impose the EIH cross-term targets $(-7/2,-1/2)$ to fix $\alpha^2$ and
        then fix the overall coupling $K$.
\end{enumerate}

\section{EIH--metric mapping}
\label{app:eih-metric}

In this appendix we briefly summarize how the effective metric of the
superfluid toy universe maps onto the Einstein--Infeld--Hoffmann
(EIH) Lagrangian, and how the parameter choices made across
Papers~I and~II and the present work ensure consistency with the standard post-Newtonian
(PN) expansion of GR.
The goal is not to reproduce the full EIH derivation, but to show how
the scalar, optical, and vector sectors combine into a consistent set of
effective 1PN metric data (an effective PN metric for the calibrated dynamics),
even though the underlying acoustic (optical) and hydrodynamic (matter)
projections need not coincide (bi-metric structure).

\subsection{Metric ansatz and PN expansion}

In the weak-field, slow-motion regime, it is convenient to write the
metric in the usual PN form
\begin{align}
g_{00}
  &= -1 + \frac{2U}{c^2} - \frac{2\beta_{\mathrm{PPN}} U^2}{c^4}
     + \mathcal{O}\!\left(\frac{1}{c^6}\right),
  \label{eq:app-g00-PN}\\
  g_{0i}
  &= -\frac{4V_i}{c^3}
     + \mathcal{O}\!\left(\frac{1}{c^5}\right),
  \label{eq:app-g0i-PN}\\
g_{ij}
  &= \left(1 + \frac{2\gamma_{\mathrm{PPN}} U}{c^2}\right)\delta_{ij}
     + \mathcal{O}\!\left(\frac{1}{c^4}\right),
  \label{eq:app-gij-PN}
\end{align}
where $U$ is the Newtonian potential and $V_i$ is the gravitomagnetic
vector potential generated by moving and spinning masses.
The constants $\beta_{\mathrm{PPN}}$ and $\gamma_{\mathrm{PPN}}$ are
the standard PPN parameters;
in GR one has $\beta_{\mathrm{PPN}}=\gamma_{\mathrm{PPN}}=1$.

In the superfluid defect model, the effective metric arises from the
scalar potential $\Phi$ and inertia profile $\sigma(r)$ (Papers~I),
the refractive index $N(r)$ of the $n=5$ vacuum (Paper~II), and the
dyon flow $\mathbf{v}(\mathbf{x})$ (this paper).
Schematically, the mapping can be written as
\begin{align}
  g_{00}
  &= -\left[ 1 + 2\frac{\Phi_{\text{eff}}}{c^2}
              + \mathcal{F}(\Phi_{\text{eff}}^2)
              + \cdots
      \right],
  \\
  g_{0i}
  &= -\frac{4}{c^3} A_{\text{eff},i}(\mathbf{x}),
  \\
  g_{ij}
  &= \left[1 - 2\frac{\Psi_{\text{eff}}}{c^2}\right]\delta_{ij}
     + \cdots,
\end{align}
where $\Phi_{\text{eff}}$ is the total scalar potential including
lag corrections, $\Psi_{\text{eff}}$ encodes the spatial curvature
inherited from the $n=5$ vacuum, and $\mathbf{A}_{\text{eff}}$ is
proportional to the bulk flow velocity $\mathbf{v}$ of the superfluid
around dyons.

Matching to Eqs.~\eqref{eq:app-g00-PN}--\eqref{eq:app-gij-PN}
identifies
\begin{align}
  U &\equiv -\Phi_{\text{eff}},
  \\
  V_i &\equiv A_{\text{eff},i},
\end{align}
 and expresses the effective PPN parameters $\beta_{\mathrm{PPN}}$
and $\gamma_{\mathrm{PPN}}$ in terms of the scalar and optical
response functions.
Paper~II showed explicitly that, once the scalar lag and $n=5$
refractive index are combined with the pressure--volume constraint
from Paper~I, one obtains
\begin{equation}
  \beta_{\mathrm{PPN}} = 1,
  \qquad
  \gamma_{\mathrm{PPN}} = 1.
\end{equation}
(Here $\gamma_{\mathrm{PPN}}$ denotes the value inferred from optical
observables after applying the dressing relation.)
The present work extends this dictionary to include $g_{0i}$ via the
dyon flow and thereby fixes the vector potential $V_i$ that appears
in the EIH Lagrangian.

\subsection{From metric to EIH Lagrangian}

Given a metric of the form
Eqs.~\eqref{eq:app-g00-PN}--\eqref{eq:app-gij-PN}, the
EIH Lagrangian follows from expanding the point-particle action
\begin{equation}
  S
  = -\sum_A m_A c
      \int \sqrt{-g_{\mu\nu}\,
                  \frac{\dd x_A^\mu}{\dd t}
                  \frac{\dd x_A^\nu}{\dd t}}\,
           \dd t,
\end{equation}
in powers of $1/c$ up to $\mathcal{O}(1/c^2)$.
The result can be written in the schematic form
\begin{equation}
  L_{\text{EIH}}
  = \sum_A \frac{1}{2} m_A v_A^2
    + \frac{1}{2}\sum_{A\neq B}
       \frac{G m_A m_B}{r_{AB}}
    + \frac{1}{c^2} L_{1\PN}[U,V_i]
    + \mathcal{O}\!\left(\frac{1}{c^4}\right),
\end{equation}
where $L_{1\PN}$ contains the $v^4$ kinetic term, the static $G^2$
three-body term, and the velocity-dependent pairwise interactions.
The detailed form of $L_{1\PN}$ is determined entirely by $U$,
$V_i$, and the PPN parameters.

Using the standard PN bookkeeping, one finds that:

\begin{itemize}
  \item The $v^4$ kinetic term and the $v^2 U$ terms come from
        expanding $\sqrt{-g_{00} - 2 g_{0i} v_A^i/c - g_{ij}
        v_A^i v_A^j/c^2}$ in powers of $v_A/c$.
  \item The static $G^2$ term arises from the quadratic dependence
        of $U$ on the masses (``gravity gravitates'') together with
        the $U^2$ contribution to $g_{00}$ proportional to $\beta$.
  \item The velocity-dependent pairwise terms come from the cross
        couplings $g_{0i} v_A^i$ and from the spatial metric
        $g_{ij}$, and their tensor structure is governed by the
        combination of $U$, $V_i$, and $\gamma$.
\end{itemize}

For $\beta_{\mathrm{PPN}}=\gamma_{\mathrm{PPN}}=1$ and for $U$ and
$V_i$ satisfying the usual
Poisson equations
\begin{align}
  \nabla^2 U &= -4\pi G \sum_A m_A \delta^{(3)}(\mathbf{x}-\mathbf{x}_A),
  \\
  \nabla^2 V_i &= -4\pi G \sum_A m_A v_A^i
                     \delta^{(3)}(\mathbf{x}-\mathbf{x}_A),
\end{align}
one recovers the standard EIH Lagrangian with the coefficients shown
in Eq.~\eqref{eq:EIH_target}.

In the superfluid model, the scalar and optical sectors ensure that
$U$ behaves as in GR at 1PN order, while the dyon construction
ensures that $V_i$ has the dipolar structure
$V_i\propto \epsilon_{ijk}J^j x^k/r^3$ (Cartesian components $\propto 1/r^2$) around
individual spinning masses.
The nontrivial work in this paper is to show that, once the vector
interaction arising from overlapping translational wakes is projected onto an
$1/r_{AB}$ kernel, the resulting coefficients of
$\mathbf{v}_A\cdot\mathbf{v}_B$ and
$(\mathbf{v}_A\cdot\mathbf{n}_{AB})(\mathbf{v}_B\cdot\mathbf{n}_{AB})$
match those of the EIH Lagrangian when the wake parameters are tuned to
$a_H=0$, $\alpha^2=3/4$, and $K=2/\pi^2$; the scalar and optical sectors are
fixed independently in Papers~I and~II.

\subsection{PPN parameters and internal consistency}

The PPN formalism provides a convenient way to check the internal
consistency of the emergent metric across different sectors:

\begin{itemize}
  \item Paper~I fixed $\beta_{\text{eff}}$ by demanding that the
        perihelion precession of nearly Keplerian orbits match GR.
        This depends primarily on the structure of $g_{00}$ and its
        $U^2$ term.
  \item Paper~II fixed $\gamma_{\text{eff}}$ by matching light
        bending, Shapiro delay, and redshift, which depend on the
        combination of $g_{00}$ and $g_{ij}$ through the optical
        metric constructed from $N(r)$.
  \item This paper effectively fixes the vector-sector PPN parameter
        (the analogue of $\alpha_1$ in some PPN conventions) by
        matching the EIH velocity-dependent tensor structure, which
        depends on $g_{0i}$ and its relation to the flow.
\end{itemize}

In all three cases the matching conditions are applied to the
\emph{same} underlying medium response, whose acoustic (optical) and hydrodynamic
(matter) projections can be packaged into consistent effective 1PN
metric data once calibrated, parameterized by a handful of
hydrodynamic response coefficients: the throat geometry ($L/a$), the
scalar renormalization parameter ($\beta$ in the defect Lagrangian),
the polytropic index ($n$), and the longitudinal/transverse mixing in
the dyon flow ($\alpha$).
The fact that a single choice of these parameters yields
$\beta_{\text{eff}}=\gamma_{\text{eff}}=1$ and reproduces the full
EIH Lagrangian at 1PN order is the main consistency result of the
series.

\subsection{Sector-by-sector check}

For completeness, we summarize the sector-by-sector mapping between
metric components and EIH terms:

\begin{description}
  \item[Scalar sector ($g_{00}$):]
    The combination of the Newtonian potential, lag correction, and
    density-dependent mass $m(\Phi_{\text{loc}})$ fixes $g_{00}$ and
    generates both the 1PN perihelion advance and the static $G^2$
    three-body term in $L_{\text{EIH}}$, with $\beta_{\text{eff}}=1$.

  \item[Optical sector ($g_{ij}$):]
    The $n=5$ refractive index profile $N(r)$ yields an effective
    spatial metric with $\gamma_{\text{eff}}=1$ inferred from the optical
    metric/observables, reproducing the 1PN light-bending and
    Shapiro-delay coefficients and entering the $v^2 U$ terms in the
    EIH Lagrangian.

  \item[Vector sector ($g_{0i}$):]
    The translational wake and its overlap kernel define the vector potential
    $V_i$ and hence $g_{0i}$. Once the scalar ($q=1$)~\cite{Norris:2025Orbits} and optical
    ($n=5$)~\cite{Norris:Paper2} contributions are included, the EIH matching conditions
    fix the wake parameters to $a_H=0$, $\alpha^2=3/4$, and $K=2/\pi^2$, as
    summarized in Sec.~\ref{subsec:nbody-derivation}. With this choice, the
    resulting velocity-dependent interaction matches the EIH tensor structure
    in Eq.~\eqref{eq:EIH_target}, including the relative coefficients of
    $\mathbf{v}_A\cdot\mathbf{v}_B$ and
    $(\mathbf{v}_A\cdot\mathbf{n}_{AB})(\mathbf{v}_B\cdot\mathbf{n}_{AB})$.
\end{description}

Thus, starting from a hydrodynamic description of a single medium,
one recovers the same metric data $(U,V_i,\beta,\gamma)$ that
underlie the standard EIH derivation in GR.
The emergent metric of the superfluid toy universe is therefore
EIH-equivalent to the GR metric at 1PN order, within the regimes and
approximations considered in this series.

\begin{thebibliography}{99}

\bibitem{Norris:2025Orbits}
Norris, T. (2025).
\newblock \emph{Newtonian and 1PN Orbital Dynamics from a Superfluid Defect Toy Model}
\newblock Zenodo.
\newblock \href{https://doi.org/10.5281/zenodo.17875005}{doi:10.5281/zenodo.17875005}.

\bibitem{Norris:Paper2}
Norris, T. (2025).
\newblock \emph{Gravitational Optics and Soliton Geodesics in a Superfluid Defect Toy Model}.
\newblock Zenodo.
\newblock \href{https://doi.org/10.5281/zenodo.17918131}{doi:10.5281/zenodo.17918131}.

\end{thebibliography}

\end{document}
