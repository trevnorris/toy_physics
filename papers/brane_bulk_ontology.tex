\documentclass[12pt]{article}

\usepackage[margin=1in]{geometry}
\usepackage{amsmath,amssymb,amsfonts}
\usepackage{graphicx}
\usepackage{hyperref}
\usepackage{bm}
\usepackage{physics}
\usepackage{mathtools}
\usepackage{caption}
\usepackage{subcaption}
\usepackage{tensor}
\usepackage{authblk}

\hypersetup{
  colorlinks=true,
  linkcolor=blue,
  citecolor=blue,
  urlcolor=blue
}

\title{Brane--Bulk Throat Ontology\\ for a Superfluid Defect Toy Universe}
\author{Trevor Norris}

\date{\today}

\begin{document}

\maketitle

\begin{abstract}
This paper, the fifth in a series developing a ``superfluid defect toy universe,''
resolves the geometric tension between the model's gravitational and
electromagnetic sectors. Previous work established that reproducing 1PN
gravity requires defects to behave as spherical sinks, whereas the
electromagnetic sector requires cylindrical resonant cavities. We resolve
this apparent contradiction by promoting defects to brane--bulk throats
connecting the observable 3D brane to a 4D superfluid bulk. Dimensional
reduction reveals that the far--field potential on the brane is monopolar,
with angular corrections from the throat geometry suppressed by $(a/r)^2$.
Numerical integration of a representative hard--bounded ``rounded funnel'' throat geometry
confirms this suppression, yielding a leading quadrupole coefficient
$\alpha_2 \approx 1.4\times 10^{-2}$ (for $L/a=2$ and a $10\%$ quadrupolar anisotropy),
thereby robustly recovering the spherical sink approximation for 1PN gravity. Inside the
throat, however, 4D acoustic modes separate into cylindrical Bessel profiles;
enthalpy minimization of these modes at fixed charge selects the preferred
aspect ratio $L/a \approx 1.85$ found in the electromagnetic construction.
We further show that long--range magnetostatics can arise without bulk vorticity by identifying the magnetic field with the vorticity of a brane--confined transverse wake sourced by the motion of a charged throat. Finally, we incorporate the corrected wake--mixing constraint $\alpha^2 = 3/4$ from Paper~III as a real, positive--definite longitudinal/transverse weighting in the vector sector once the wake basis is completed.  This ontology constrains
finite--size effects, laying the groundwork for falsifiable 2PN predictions.
\end{abstract}

\section{Introduction}
\label{sec:intro}

\subsection{Motivation and the sphere--cylinder tension}
\label{subsec:intro_tension}

This paper is the fifth in a series developing a ``superfluid defect toy
universe'' in which gravity and electromagnetism emerge from the dynamics
of a compressible fluid.\footnote{For clarity, we will refer to the previous
papers as Papers~I--IV\cite{Norris:2025Orbits,Norris:Paper2,Norris:Paper3,Norris:Paper4} throughout.}  In this model the vacuum is treated as a
stiff superfluid, and massive bodies are modeled as localized defects that
drain and stir this medium.  Far from any defect, the flow is slow and
approximately irrotational; near the defect, nonlinear and topological
effects become important.  The central claim of the series so far has been
that, with a suitable choice of hydrodynamic energy functional, this
superfluid picture reproduces the standard post–Newtonian (PN) expansion of
general relativity (GR) at first PN order (1PN) for isolated bodies and
$N$--body systems, while also supporting an emergent electromagnetism sector.

Paper~I (the orbital paper) showed that if a defect behaves as a spherical
sink in a three–dimensional superfluid, then the resulting potential flow
around the defect reproduces Newtonian gravity and the 1PN perihelion
precession of GR.  In that construction the background flow is spherically
symmetric and purely radial on spatial slices, and the defect induces both
a Newtonian $1/r$ potential and a set of scalar inertia renormalizations that
sum to the calibrated orbital parameter $\beta$.  In the notation of
Paper~I these contributions are: density depletion $\kappa_{\rho}=1$
(fixing the static $G_2$ ``gravity gravitates'' interaction), pressure--volume
work $\kappa_{\mathrm{PV}}=3/2$ (the energy cost of displacing fluid against
the ambient pressure), and added mass $\kappa_{\mathrm{add}}=1/2$ (the
hydrodynamic inertia of entrained fluid).  Together,
\begin{equation}
  \beta = \kappa_{\rho}+\kappa_{\mathrm{PV}}+\kappa_{\mathrm{add}} = 3,
\end{equation}
matching the GR prediction for perihelion advance.  Crucially, the calculation assumes that
the defect is effectively spherical: when one tries to repeat the same
analysis with a cylindrical defect in three dimensions, the resulting PN
coefficients become orientation–dependent and do not reproduce GR in a
rotationally invariant way.

Paper~II (the optics paper) recast the same superfluid model in terms of an
effective optical metric.  Light propagation in the gravitational field of a
single defect was described in terms of refraction in a stiff polytropic
superfluid, and it was shown that matching the GR lensing coefficient fixes
a specific polytropic index and curvature coefficient in the optical sector.
Again, the gravitational field is sourced by a spherically symmetric sink:
from the point of view of the optical metric, the defect is a point–like,
effectively spherical object.

Paper~III (the spin and $N$--body paper) extended the model to include
spin--induced frame dragging and general $N$--body interactions.  There the
hydrodynamic wake functional was generalized to include both longitudinal and
transverse projector components, and the resulting 1PN dynamics were matched
to the Einstein--Infeld--Hoffmann (EIH) Lagrangian.  The matching uniquely
constrained the relative weighting of the longitudinal and transverse pieces,
selecting the real wake--mixing ratio
\begin{equation}
  \alpha^2 = \frac{3}{4}.
\end{equation}
With the completed wake basis this value is compatible with an everywhere
positive--definite quadratic functional; the GR vector coefficients arise from
the projector structure of the interaction kernel rather than from an
indefinite (``Lorentzian'') metric on mode space.

Importantly for the present work, the spin and $N$--body
analysis still assumes a spherical defect: attempts to model the defect as a
3D cylinder again fail at 1PN, with significant orientation–dependent
errors in the predicted precession and spin couplings.

Paper~IV (the EM paper) shifted focus to the electromagnetic sector.  There
the defect was modeled as a resonant cavity in the superfluid, and a
hydrodynamic–electromagnetic dictionary was constructed that maps certain
mode structures of the fluid onto electric and magnetic fields.  Stabilizing
a charged defect required a cylindrical cavity of radius $a$ and length
$L$, supporting a fundamental mode with radial profile
$J_0(x_{01} r/a)$ and a standing wave along the axial direction.  Minimizing
the enthalpy at fixed ``charge'' picked out a preferred aspect ratio,
\begin{equation}
  \frac{L}{a} = \frac{\sqrt{2}\,\pi}{x_{01}} \approx 1.85,
\end{equation}
where $x_{01}$ is the first zero of $J_0$.  In this construction, charge is
associated with circulation and vorticity in and around a cylindrical
defect, and the cylindrical geometry is not an optional detail: it is needed
to obtain the correct mode structure and a stable charged configuration.

Taken at face value, these results appear to ascribe contradictory
geometries to the same object.  The gravitational papers (I--III) insist
that the defect must behave as a spherical sink in three dimensions in order
to reproduce 1PN gravity.  The electromagnetic paper (IV) insists that the
defect must behave as a cylindrical resonant cavity in order to reproduce
electromagnetism.  One cannot simply declare the defect to be ``both a
sphere and a cylinder'' within a purely three–dimensional picture without
spoiling either the gravitational or the electromagnetic sector.  This
sphere–cylinder tension is an internal inconsistency in the toy model, and
it forces us to look for a deeper geometric ontology in which both
descriptions can be simultaneously true.

\subsection{Brane--bulk throat ontology as a resolution}
\label{subsec:intro_ontology}

The central proposal of this paper is that the apparent contradiction is an
artifact of insisting on a purely 3D description of an intrinsically
higher–dimensional defect.  We will argue that the defect should instead be
understood as a \emph{throat} connecting a three–dimensional brane to a
four–dimensional superfluid bulk.

Concretely, we introduce coordinates $(t,\mathbf{x},w)$, where
$\mathbf{x} = (x,y,z)$ label directions along the brane and $w$ is a bulk
coordinate transverse to the brane.  The physical world accessible to
ordinary observers is modeled as a hypersurface at $w=0$, which we will
refer to as the brane.  The superfluid fills the full four–dimensional
space, and its density and velocity fields extend smoothly into the bulk.
A defect is then represented not as a compact three–dimensional object,
but as a localized region where the brane pinches into the bulk, forming a
throat of radius $a$ and depth $L$ along $w$.

From the point of view of an observer confined to the brane, the intersection
of the throat with $w=0$ is a nearly spherical opening of radius $a$.  The
effective flow induced on the brane by drainage into the throat is then
approximately that of a spherical sink located at the center of this
opening.  At distances $r \gg a$ along the brane, the flow is radially
symmetric and the corresponding effective gravitational potential is
indistinguishable from that of a point mass.  In this sense, the 1PN
gravitational sector of Papers~I--III only probes the \emph{mouth} of the
throat, and it is naturally described by a spherical sink in an effective
3D fluid.

From the point of view of modes living \emph{inside} the throat, however,
the relevant geometry is entirely different.  Away from the mouth the
throat is approximately a cylinder of radius $a$ and length $L$ embedded in
the bulk.  Acoustic modes confined within this cylindrical region obey a
4D wave equation whose radial and axial dependence factorize.  Imposing
appropriate boundary conditions at $r=a$ and at the ends $w=0$ and $w=L$
selects Bessel–type radial profiles and sine– or cosine–type axial profiles,
precisely of the form used in Paper~IV.  Enthalpy minimization at fixed
``charge'' then selects a preferred aspect ratio $L/a \approx 1.85$ for the
throat.  In this picture, the electromagnetic sector is not telling us that
the whole defect is a 3D cylinder; it is telling us that the \emph{interior
of the throat in the bulk} behaves as an approximately cylindrical cavity.

The apparent sphere–cylinder contradiction is thus resolved once we
distinguish between the brane projection and the bulk interior.  Gravity is
governed by the effective three–dimensional projection of the bulk flow onto
the brane and is therefore sensitive primarily to the spherical mouth of the
throat.  Electromagnetism is governed by cavity modes that live inside the
throat and therefore probe the cylindrical interior.  The two descriptions
are compatible because they refer to different slices of the same
higher–dimensional geometry.

In the same spirit, we will argue that the wake--mixing ratio
$\alpha^2=3/4$ fixed in Paper~III admits a natural geometric reading in the
throat ontology: motion of a brane--anchored defect excites both transverse
(brane--parallel) and longitudinal (bulk--directed) wake response, and the
EIH matching selects a specific real weighting between these components.
With the completed wake basis the underlying quadratic functional is
positive--definite; the GR sign structure arises in the effective mapping of
hydrodynamic potentials to 1PN observables rather than from an indefinite
``Lorentzian'' metric on wake space.

\subsection{Scope and goals of this paper}
\label{subsec:intro_scope}

The goal of this paper is to make the brane--bulk throat ontology precise
enough that the previous results of Papers~I--IV can be seen as different
projections of a single geometric picture, and to lay the groundwork for a
systematic analysis of higher–order PN corrections.

On the constructive side, we will:

\begin{itemize}
  \item Formulate a four–dimensional superfluid framework with a brane at
  $w=0$ and defects represented as throats of radius $a$ and depth $L$.

  \item Show how dimensional reduction from 4D to the brane organizes the
  far–field effective theory in terms of monopole and higher multipole
  moments, with angular corrections suppressed by $(a/r)^2$.

  \item Demonstrate that, at 1PN order, the far–field behavior of a throat
  on the brane is indistinguishable from that of the spherical sinks used in
  Papers~I--III, explaining why those papers were successful despite their
  purely 3D language.

  \item Reinterpret the electromagnetic cavity of Paper~IV as the interior
  of the throat, and the enthalpy–selected aspect ratio $L/a \approx 1.85$
  as a geometric property of the throat in the bulk rather than an arbitrary
  3D cylinder.

  \item Provide a simple two--mode toy model illustrating how the wake--mixing
  ratio $\alpha^2 = 3/4$ can arise as a real longitudinal/transverse
  weighting in a positive--definite quadratic form (and how earlier negative
  values reflected an incomplete wake basis).
\end{itemize}

On the interpretive side, we will:

\begin{itemize}
  \item Argue that finite–size effects associated with the throat, including
  the detailed structure of the transition region near the mouth, naturally
  appear as $(a/r)^2$ corrections, i.e. as 2PN and higher–order terms.

  \item Outline how a future 2PN calculation could be organized around the
  throat geometry, turning the model into a falsifiable framework whose
  predictions can be confronted with precision tests of gravity and
  electromagnetism.

  \item Clarify in what sense the present construction resembles brane–world
  and superfluid–vacuum scenarios in the broader literature, and in what
  ways it differs.
\end{itemize}

What this paper does \emph{not} do is equally important.  We will not
attempt a full 2PN derivation of the effective gravitational Lagrangian or
the complete electromagnetic sector.  We will not specify the microscopic
origin or stabilization mechanism of the brane itself, nor will we attempt
to simulate strong–field or highly dynamical processes such as mergers or
collapse.  Our aim is more modest: to provide a coherent geometric ontology
that removes the internal tensions of the existing toy model and strongly
constrains the form of any future higher–order calculations.

\subsection{Structure of the paper}
\label{subsec:intro_structure}

The remainder of the paper is organized as follows.  In
Sec.~\ref{sec:bulk_framework} we introduce the four–dimensional superfluid
framework and the brane embedding.  We define the bulk density, velocity,
and enthalpy fields, specify the equation of state, and write down the 4D
continuity and Euler equations.  We then describe the brane as a
hypersurface at $w=0$, discuss the induced three–dimensional fields on the
brane, and introduce a simple geometric model of a throat of radius $a$ and
depth $L$ connecting the brane to the bulk.

In Sec.~\ref{sec:dimensional_reduction} we turn to dimensional reduction and
the far–field structure.  We define effective 3D sources obtained by
integrating over the bulk coordinate $w$, and we study a Gaussian toy model
for a localized throat.  This model allows us to compute the effective
monopole mass and quadrupole moment, and to show explicitly that angular
corrections to the potential are suppressed by $(a/r)^2$.  We explain why
the 1PN results of Papers~I--III are insensitive to these finite–size
effects and therefore naturally see only a spherical source.

Sec.~\ref{sec:throat_em} focuses on the interior of the throat and the
electromagnetic sector.  Starting from the 4D acoustic wave equation inside
the throat, we perform a separation–of–variables analysis to recover the
cylindrical cavity modes of Paper~IV, including the fundamental mode with
radial profile $J_0(x_{01} r/a)$ and appropriate axial dependence along $w$.
We then show how enthalpy minimization at fixed ``charge'' selects the
aspect ratio $L/a \approx 1.85$, and we reinterpret this result as a
geometric constraint on the throat.  We also review how charge and mass can
be viewed as different projections of the same throat geometry: charge as
vorticity flux through the throat cross–section, and mass as volume flux
into the throat.

In Sec.~\ref{sec:near_field} we examine the near–field transition region
where streamlines bend from radial inflow on the brane to axial flow into
the bulk.  We describe the resulting stress patterns and construct a
multipole expansion of the effective 3D source associated with this region.
We argue that the leading corrections are quadrupolar and scale as
$(a/r)^2$, providing a geometric interpretation of expected 2PN finite–size
effects and a link to the shape–sensitivity tests performed in
Paper~III.

Sec.~\ref{sec:wake_mixing_constraint} revisits the wake mixing
$\alpha^2 = 3/4$ in the 1PN vector sector.  We briefly recap the relevant
structure of the EIH matching from Paper~III and then introduce a simple
two–mode toy model in which transverse (brane–parallel) and longitudinal
(bulk–directed) modes are coupled.  We show how an effective mixing
between these modes arises and interpret this as an effective projector
signature in mode space induced by the brane--bulk geometry of the throat.

In Sec.~\ref{sec:2pn_predictions} we outline how a full 2PN analysis might
proceed within this ontology.  We review the standard PN counting, identify
the classes of corrections that are most sensitive to the throat geometry,
and discuss possible observational channels through which the model could be
tested or ruled out.  We emphasize that the ontology developed here sharply
constrains the allowed structure of 2PN terms, turning the toy model into a
framework that is, in principle, falsifiable.

Finally, in Sec.~\ref{sec:discussion_outlook} we summarize the main elements
of the brane--bulk throat picture and its relation to Papers~I--IV, discuss
possible scenarios for what happens at the bottom of the throat and for the
stabilization of the brane, and speculate about the behavior of multiple
defects interacting through the bulk.  Several technical details, including
explicit Gaussian multipole integrals, mode separation in the throat, and
the two–mode quadratic form underlying $\alpha^2 = 3/4$, are collected in
the appendices.

\section{4D Superfluid Framework and Brane Embedding}
\label{sec:bulk_framework}

In this section we set up the higher--dimensional fluid framework that will
underlie the rest of the paper.  We treat the vacuum as a compressible
superfluid living in four spatial dimensions plus time, with the observable
universe represented as a three--dimensional brane embedded in this bulk.
The goal is not to derive everything from first principles, but to lay out a
minimal, consistent kinematic and thermodynamic setup that can be used to
interpret the results of Papers~I--IV in a unified geometric language.

\subsection{Bulk fields, equation of state, and hydrodynamic equations}
\label{subsec:bulk_fields}

We work with coordinates
\begin{equation}
  (t,\mathbf{X}) = (t,\mathbf{x},w),
  \qquad
  \mathbf{x} = (x,y,z),
\end{equation}
where $\mathbf{x}$ are the spatial coordinates along the brane and $w$ is a
single bulk coordinate transverse to the brane.  The full superfluid lives
in the four--dimensional spatial manifold parametrized by
$\mathbf{X} = (x,y,z,w)$.

The state of the bulk fluid is described by
\begin{itemize}
  \item a mass density $\rho(\mathbf{x},w,t)$,
  \item a velocity field
    $\mathbf{v}_4(\mathbf{x},w,t)$ with components
    \begin{equation}
      \mathbf{v}_4
      = (v_x,v_y,v_z,v_w)
      \equiv (\mathbf{v},v_w),
    \end{equation}
    where $\mathbf{v}$ denotes the three components tangent to the brane and
    $v_w$ is the bulk component, and
  \item a pressure $P(\mathbf{x},w,t)$.
\end{itemize}

We assume a barotropic equation of state of polytropic form,
\begin{equation}
  P = K \rho^n,
  \label{eq:poly_eos}
\end{equation}
with $K>0$ and polytropic index $n=5$.  This is the same stiff superfluid
equation of state that was singled out in Paper~II by matching gravitational
lensing and Shapiro delay; here we simply adopt it as part of the bulk
ontology.  The background is taken to be homogeneous with density $\rho_0$
and pressure $P_0 = K\rho_0^n$ in the absence of defects, and we will work
with small perturbations around this background,
\begin{equation}
  \rho = \rho_0 + \delta\rho,
  \qquad
  P = P_0 + \delta P,
  \qquad
  \abs{\delta\rho} \ll \rho_0,
  \quad
  \abs{\delta P} \ll P_0.
\end{equation}

The bulk dynamics are governed by the usual continuity and Euler equations
generalized to four spatial dimensions.  In Cartesian coordinates the
continuity equation reads
\begin{equation}
  \partial_t \rho
  + \nabla_4 \cdot \big(\rho\,\mathbf{v}_4\big)
  = 0,
  \label{eq:bulk_continuity}
\end{equation}
where
\begin{equation}
  \nabla_4 = (\partial_x,\partial_y,\partial_z,\partial_w),
  \qquad
  \nabla_4\cdot\mathbf{v}_4
  = \partial_x v_x + \partial_y v_y + \partial_z v_z + \partial_w v_w.
\end{equation}
The Euler equation for an inviscid barotropic fluid is
\begin{equation}
  \rho\,\Big(\partial_t \mathbf{v}_4
  + (\mathbf{v}_4\cdot\nabla_4)\mathbf{v}_4\Big)
  = - \nabla_4 P + \mathbf{f}_{\mathrm{ext}},
  \label{eq:bulk_euler}
\end{equation}
where $\mathbf{f}_{\mathrm{ext}}$ denotes any external or defect--induced
body forces.  In the far field of a localized defect the flow is slow and
nearly irrotational, the nonlinear advective term can be treated
perturbatively, and we may take $\mathbf{f}_{\mathrm{ext}}\simeq 0$ for the
purpose of deriving the acoustic modes.

It is convenient to introduce the specific enthalpy
\begin{equation}
  h(\rho)
  \equiv \int^{\rho} \frac{\mathrm{d}P}{\rho'}
  = \frac{nK}{n-1}\,\rho^{n-1},
  \label{eq:enthalpy_def}
\end{equation}
for the polytropic equation of state~\eqref{eq:poly_eos}.  Linearizing
$h(\rho)$ around $\rho_0$ yields
\begin{equation}
  h(\rho_0 + \delta\rho)
  \simeq h_0
  + \left.\frac{\mathrm{d}h}{\mathrm{d}\rho}\right|_{\rho_0}\delta\rho
  = h_0 + \frac{c_s^2}{\rho_0}\,\delta\rho,
\end{equation}
where
\begin{equation}
  c_s^2
  \equiv \left.\frac{\mathrm{d}P}{\mathrm{d}\rho}\right|_{\rho_0}
  = nK\rho_0^{n-1}
\end{equation}
is the squared sound speed.  We will henceforth drop the constant offset
$h_0$ and use $h$ to denote the enthalpy perturbation,
\begin{equation}
  h(\mathbf{x},w,t)
  \simeq \frac{c_s^2}{\rho_0}\,\delta\rho(\mathbf{x},w,t).
  \label{eq:enthalpy_perturbation}
\end{equation}

In the far field of a defect and for small--amplitude motions we may linearize
Eqs.~\eqref{eq:bulk_continuity}--\eqref{eq:bulk_euler} about the homogeneous
background.  Writing $\rho = \rho_0 + \delta\rho$ and
$\mathbf{v}_4 = \delta\mathbf{v}_4$ and neglecting quadratic terms in the
perturbations, the continuity equation becomes
\begin{equation}
  \partial_t \delta\rho
  + \rho_0\,\nabla_4\cdot\delta\mathbf{v}_4
  = 0,
  \label{eq:lin_cont}
\end{equation}
while the Euler equation reduces to
\begin{equation}
  \rho_0\,\partial_t \delta\mathbf{v}_4
  = - \nabla_4 \delta P
  = - \rho_0\,\nabla_4 h.
  \label{eq:lin_euler}
\end{equation}
Taking the divergence of Eq.~\eqref{eq:lin_euler} and using
Eq.~\eqref{eq:lin_cont} to eliminate $\nabla_4\cdot\delta\mathbf{v}_4$ gives
a wave equation for $h$,
\begin{equation}
  \partial_t^2 h
  - c_s^2\,\nabla_4^2 h
  = 0,
  \qquad
  \nabla_4^2
  = \partial_x^2 + \partial_y^2 + \partial_z^2 + \partial_w^2.
  \label{eq:4d_acoustic}
\end{equation}
Equivalently,
\begin{equation}
  -\frac{1}{c_s^2}\,\partial_t^2 h
  + \nabla_4^2 h
  = 0.
\end{equation}
This 4D acoustic wave equation is the master equation governing linear
perturbations in the bulk.  In later sections we will specialize it to the
interior of a throat and use it to derive the cavity modes relevant for the
electromagnetic sector.

\subsection{The brane as a hypersurface at \texorpdfstring{$w=0$}{w = 0}}
\label{subsec:brane_hypersurface}

We now embed a three--dimensional brane in the 4D spatial bulk to represent
the observable universe.  The brane is defined as the hypersurface
\begin{equation}
  \mathcal{B}:\quad w = 0.
\end{equation}
All of the effective 3D physics described in Papers~I--IV is understood as
arising from the behavior of the bulk fields restricted to, or projected
onto, this hypersurface.  In particular:
\begin{itemize}
  \item The mouths of defects are localized regions on $\mathcal{B}$.
  \item Test bodies and light rays follow trajectories $(t,\mathbf{x}(t))$
    that lie on or very near $w=0$.
\end{itemize}

To make contact with the previous 3D description it is useful to define
effective brane fields obtained by integrating or sampling the bulk fields
in the $w$--direction.  One simple construction is to introduce a weighting
kernel $K(w)$ peaked around $w=0$ and define
\begin{align}
  \rho_{\mathrm{3D}}(\mathbf{x},t)
  &\equiv \int_{-\infty}^{+\infty} \mathrm{d}w\,
    K(w)\,\rho(\mathbf{x},w,t),
    \label{eq:rho3d_def}
  \\
  \mathbf{v}_{\mathrm{3D}}(\mathbf{x},t)
  &\equiv
    \frac{1}{N}
    \int_{-\infty}^{+\infty} \mathrm{d}w\,
    K(w)\,\Pi_{\parallel}\,\mathbf{v}_4(\mathbf{x},w,t),
    \label{eq:v3d_def}
\end{align}
where $\Pi_{\parallel}$ projects onto the $x,y,z$ components and
$N = \int \mathrm{d}w\,K(w)$ is a normalization factor.  For the purposes of
this paper we will not commit to a specific functional form for $K(w)$;
one may think of it as either a narrow bump of width comparable to the
microscopic brane thickness, or simply as $K(w)=1$ with the understanding
that only the behavior near $w=0$ contributes significantly.

The fields $\rho_{\mathrm{3D}}$ and $\mathbf{v}_{\mathrm{3D}}$ play the role
of the effective density and velocity fields in the three--dimensional toy
models of Papers~I--III.  When we speak of a ``spherical sink'' on the brane
we are referring to the behavior of $\rho_{\mathrm{3D}}$ and
$\mathbf{v}_{\mathrm{3D}}$ as functions of $\mathbf{x}$, induced by the
presence of a throat in the bulk.

At the level of equations of motion, the projections
\eqref{eq:rho3d_def}--\eqref{eq:v3d_def} induce an effective 3D continuity
equation and a Poisson--like equation for a Newtonian potential
$\Phi(\mathbf{x},t)$ on the brane.  Schematically,
\begin{align}
  \partial_t \rho_{\mathrm{3D}}
  + \nabla \cdot (\rho_{\mathrm{3D}}\mathbf{v}_{\mathrm{3D}})
  &= 0,
  \\
  \nabla^2 \Phi
  &= 4\pi G_{\mathrm{eff}}\,\rho_{\mathrm{3D}},
\end{align}
where $\nabla$ is the 3D gradient with respect to $\mathbf{x}$ and
$G_{\mathrm{eff}}$ is an effective gravitational constant determined by the
bulk parameters.  We will not attempt to derive $G_{\mathrm{eff}}$ from
first principles here; instead we treat it as calibrated by the matching to
Newtonian gravity and 1PN corrections in Papers~I--III.

Crucially, the effective brane velocity need not be purely longitudinal.  On
$\mathcal{B}$ we can decompose
\begin{equation}
  \mathbf{v}_{\mathrm{3D}} = \nabla\Phi + \mathbf{v}_T,
  \qquad \nabla\cdot \mathbf{v}_T = 0,
  \label{eq:v3d_decomposition}
\end{equation}
into an irrotational component that controls the scalar potential sector used
in the orbital and optical matching, and a divergence--free transverse
component confined to the brane.  In the gravitational regime we may take
$\mathbf{v}_T$ negligible so that $\mathbf{v}_{\mathrm{3D}}\approx\nabla\Phi$,
but in the electromagnetic sector the transverse mode provides the natural
carrier of a vector potential; in Sec.~\ref{subsec:brane_magnetostatics} we
identify $\mathbf{A}\equiv\kappa_A \mathbf{v}_T$ and
$\mathbf{B}=\nabla\times\mathbf{A}$.



The brane also provides natural boundary conditions on the bulk flow.
Physically, the brane may be thought of as a locus where some microscopic
mechanism pins the fluid or changes its properties, but in this toy model we
capture that information simply by specifying the behavior of
$\rho, \mathbf{v}_4,$ and $h$ at $w=0$.  For example, in the absence of a
defect we may assume that the normal velocity vanishes at the brane,
\begin{equation}
  v_w(\mathbf{x},w=0,t) = 0
\end{equation}
so that there is no net flux of fluid across $\mathcal{B}$.  In the presence
of a defect mouth, by contrast, there is a localized region on the brane
where $v_w$ is nonzero and negative, corresponding to drainage into the
throat.  The details of this boundary condition will matter for the
near--field structure and for the relationship between the defect's mass and
its drainage rate, but the broad picture is simple: the brane is a
distinguished hypersurface whose intersection with a throat appears as a
localized sink to brane--bound observers.

\subsection{Throat topology and geometry}
\label{subsec:throat_geometry}

We are now ready to formalize the geometric picture of a defect as a
brane--bulk throat.  Intuitively, the brane at $w=0$ is locally deformed and
pinched into the bulk, forming a tube--like region filled with the same
superfluid.  We model this region as a cylindrical throat of radius $a$ and
depth $L$ embedded in the four--dimensional spatial bulk.

Let the throat domain $\mathcal{T}$ be a connected region of space defined
approximately by
\begin{equation}
  \mathcal{T}
  \simeq \left\{
    (\mathbf{x},w)\,\middle|\,
    0 \le w \le L,\;
    \sqrt{x^2 + y^2 + z^2} \lesssim a
  \right\},
  \label{eq:throat_domain}
\end{equation}
with the understanding that near $w=0$ and $w=L$ the geometry may deviate
from an exact cylinder.  For most of the paper we will only need a
coarse--grained description at scales larger than any microscopic brane
thickness, so we treat $a$ and $L$ as effective parameters characterizing
the radius and depth of the throat.

From the perspective of an observer living on the brane $\mathcal{B}$, the
intersection of the throat with $w=0$ is a two--sphere of radius $a$,
\begin{equation}
  \partial\mathcal{T}\cap\mathcal{B}
  \simeq \left\{
    \mathbf{x}\,\middle|\,
    \abs{\mathbf{x}} = a
  \right\}.
\end{equation}
This spherical boundary is what appears, in the effective 3D description, as
the ``surface'' of a spherical defect.  The region $\abs{\mathbf{x}}<a$ on
the brane is not filled with solid matter, but is instead the mouth of the
throat opening into the bulk.  This is the geometry implicitly assumed in
Papers~I--III when modeling a defect as a spherical sink: at distances
$r=\abs{\mathbf{x}}\gg a$, the details of the throat interior are invisible
and only the total drainage rate through this mouth matters.

From the perspective of the bulk, by contrast, the region
$0 < w < L$ with $\abs{\mathbf{x}}\lesssim a$ is approximately a straight
4D cylinder of cross--sectional area $\pi a^2$.  This cylindrical interior
is the domain in which the cavity modes of the electromagnetic sector live.
In Sec.~\ref{sec:throat_em} we will impose boundary conditions on $h$ at
$r=\abs{\mathbf{x}}=a$ and at $w=0$ and $w=L$ and solve the 4D acoustic wave
equation~\eqref{eq:4d_acoustic} inside $\mathcal{T}$ to recover the
Bessel--type radial modes and the preferred aspect ratio $L/a$ found in
Paper~IV.

At the bottom of the throat, around $w=L$, we deliberately leave the
geometry unspecified.  Several possibilities are conceivable:
\begin{itemize}
  \item the throat could close off smoothly, forming a finite cavity;
  \item it could open into a larger bulk region, allowing radiation or
    flow into the deep bulk;
  \item or it could connect to other throats, forming a network of
    defects linked through the bulk.
\end{itemize}
The choice among these possibilities affects the detailed mode spectrum and
the global topology of the model, but it will not play a direct role in the
1PN considerations of this paper.  We return to these questions in
Sec.~\ref{sec:discussion_outlook}.

Finally, it is worth noting that the throat geometry naturally supports
topological quantities associated with circulation and vorticity.  Loops
encircling the throat mouth on the brane can carry quantized circulation,
and vorticity lines can thread the throat interior, connecting the brane to
the bulk.  In the electromagnetic sector these quantities will be related to
electric charge and magnetic flux; for now we simply flag the fact that the
topology of $\mathcal{T}$ provides the right sort of structure to encode
such conserved charges.

With this 4D superfluid framework and throat geometry in place, we can now
turn to dimensional reduction and the far--field structure of the effective
3D theory on the brane.

\section{Dimensional Reduction and Far-Field Structure}
\label{sec:dimensional_reduction}

In this section we show how a localized throat in the 4D bulk generates an
effective three--dimensional source on the brane.  The key points are:
(i) integrating over the bulk coordinate $w$ produces an effective 3D
density whose leading contribution is a monopole mass $M\sim \rho_0 L a^3$,
and (ii) angular deviations from spherical symmetry are naturally suppressed
by $(a/r)^2$ at large radii $r$ on the brane.  We illustrate these ideas
with a simple Gaussian toy model and then connect the resulting far--field
potential to the 1PN structure used in Papers~I--III.

\subsection{General dimensional reduction from 4D to 3D}
\label{subsec:dim_red_general}

Given a bulk density $\rho(\mathbf{x},w,t)$ and velocity
$\mathbf{v}_4(\mathbf{x},w,t)$, the effective 3D fields seen by brane--bound
observers are obtained by integrating over the bulk coordinate $w$ with some
weighting kernel $K(w)$ localized near the brane:
\begin{align}
  \rho_{\mathrm{3D}}(\mathbf{x},t)
  &\equiv \int_{-\infty}^{+\infty} \mathrm{d}w\,
    K(w)\,\rho(\mathbf{x},w,t),
    \label{eq:rho3d_general}
  \\
  \mathbf{v}_{\mathrm{3D}}(\mathbf{x},t)
  &\equiv
    \frac{1}{N}
    \int_{-\infty}^{+\infty} \mathrm{d}w\,
    K(w)\,\Pi_{\parallel}\,\mathbf{v}_4(\mathbf{x},w,t),
    \label{eq:v3d_general}
\end{align}
with $N = \int \mathrm{d}w\,K(w)$ and $\Pi_{\parallel}$ projecting onto the
brane--parallel components $(v_x,v_y,v_z)$.  For the purposes of this paper
we may take $K(w)=1$ and understand that the integrals are dominated by the
region where the throat lives; more refined choices would only change
numerical prefactors.

Once $\rho_{\mathrm{3D}}$ and $\mathbf{v}_{\mathrm{3D}}$ are defined, their
dynamics on the brane are governed by an effective 3D continuity equation
and a Poisson--like equation for a Newtonian potential $\Phi(\mathbf{x},t)$,
\begin{align}
  \partial_t \rho_{\mathrm{3D}}
  + \nabla \cdot (\rho_{\mathrm{3D}}\mathbf{v}_{\mathrm{3D}})
  &= 0,
  \label{eq:3d_continuity}
  \\
  \nabla^2 \Phi
  &= 4\pi G_{\mathrm{eff}}\,\rho_{\mathrm{3D}},
  \label{eq:3d_poisson}
\end{align}
where $\nabla$ is the 3D gradient with respect to $\mathbf{x}$ and
$G_{\mathrm{eff}}$ is an effective gravitational constant determined by the
bulk parameters and the details of the projection.
In practice $G_{\mathrm{eff}}$ is fixed by matching Eq.~\eqref{eq:3d_poisson}
to Newtonian gravity at large distances, as done implicitly in Papers~I--III.

For a localized defect represented as a throat of radius $a$ and depth $L$,
$\rho(\mathbf{x},w,t)$ is strongly peaked near $\abs{\mathbf{x}}\lesssim a$
and $0 < w < L$.  At distances $r=\abs{\mathbf{x}}\gg a,L$ on the brane,
$\rho_{\mathrm{3D}}$ therefore looks like a nearly point--like source plus
small angular corrections.  The total effective mass is
\begin{equation}
  M
  \equiv \int \rho_{\mathrm{3D}}(\mathbf{x},t)\,\mathrm{d}^3x
  = \int \mathrm{d}^3x\int \mathrm{d}w\,
      K(w)\,\rho(\mathbf{x},w,t),
\end{equation}
and the higher multipole moments of $\rho_{\mathrm{3D}}$ encode finite--size
effects associated with the throat geometry and the transition region near
its mouth.

\subsection{Gaussian toy model for a localized throat}
\label{subsec:gaussian_toy_model}

To make these statements concrete, we now introduce a simple toy model for
the bulk density perturbation associated with a single throat.  We work in
spherical coordinates $(r,\theta,\phi)$ on the brane and use $w$ for the
bulk coordinate.  The toy 4D density profile is taken to be
\begin{equation}
  \rho_4(r,\theta,w)
  = \rho_0
    \exp\!\left(-\frac{r^2}{a^2}\right)
    \exp\!\left(-\frac{w^2}{L^2}\right)
    \big[1 + \varepsilon P_2(\cos\theta)\big],
  \label{eq:rho4_gaussian}
\end{equation}
where $P_2(\cos\theta) = (3\cos^2\theta - 1)/2$ is the $\ell=2$ Legendre
polynomial, $a$ is the throat radius, $L$ is the throat depth, and
$\varepsilon$ is a small dimensionless parameter controlling the degree of
angular asymmetry.  The exponential factors localize the defect within a
region of size $\sim a$ on the brane and $\sim L$ in the bulk; the
$\varepsilon P_2$ term models the mild asphericity associated with the
transition region where streamlines bend from radial to axial flow.

For simplicity we choose $K(w)=1$ in the projection onto the brane.
Integrating Eq.~\eqref{eq:rho4_gaussian} over $w$ gives the effective 3D
density
\begin{equation}
  \rho_{\mathrm{3D}}(r,\theta)
  = \int_{-\infty}^{+\infty}
    \rho_4(r,\theta,w)\,\mathrm{d}w
  = \sqrt{\pi}\,L\rho_0\,\exp\!\left(-\frac{r^2}{a^2}\right)
    \big[1 + \varepsilon P_2(\cos\theta)\big].
  \label{eq:rho3d_gaussian}
\end{equation}
The total mass of the defect is then
\begin{equation}
  M
  = \int \rho_{\mathrm{3D}}(r,\theta)\,\mathrm{d}^3x
  = \pi^2 L a^3 \rho_0,
  \label{eq:gaussian_mass}
\end{equation}
where the $\varepsilon$ term integrates to zero by angular symmetry.
This scaling is exactly what we expect for a throat of radius $a$ and depth
$L$: $M$ is proportional to the background density times the effective
throat volume.

To quantify the leading angular deviation from sphericity we consider the
standard quadrupole--like scalar moment
\begin{equation}
  Q_{20}
  \propto \int \rho_{\mathrm{3D}}(r,\theta)\,
           r^2 P_2(\cos\theta)\,\mathrm{d}^3x.
  \label{eq:q20_def}
\end{equation}
Carrying out the angular integrals and the Gaussian radial integral (details
are relegated to Appendix~\ref{app:gaussian_multipoles}), we obtain
\begin{equation}
  Q_{20}
  = \frac{3}{10}\,\pi^2 L a^5 \varepsilon \rho_0,
  \label{eq:q20_result}
\end{equation}
and hence
\begin{equation}
  \frac{Q_{20}}{M}
  = \frac{3}{10}\,\varepsilon a^2.
  \label{eq:q_over_m_scaling}
\end{equation}
The precise numerical prefactor $3/10$ is not important for our purposes;
the crucial point is the scaling
\begin{equation}
  \frac{Q}{M} \sim \varepsilon a^2,
\end{equation}
which is generic for a localized, mildly aspherical source of size $a$ on
the brane.

\subsection{Far-field potential and \texorpdfstring{$(a/r)^2$}{(a/r)^2} corrections}
\label{subsec:far_field_potential}

The effective Newtonian potential $\Phi(r,\theta)$ generated on the brane by
a localized mass distribution $\rho_{\mathrm{3D}}$ admits the usual
multipole expansion at large radii $r\gg a,L$:
\begin{equation}
  \Phi(r,\theta)
  \approx -\frac{GM}{r}
  - G\,\frac{Q}{r^3}\,P_2(\cos\theta)
  + \cdots,
  \label{eq:phi_multipole}
\end{equation}
where $M$ is the monopole mass, $Q$ is the quadrupole--like scalar defined
in Eq.~\eqref{eq:q20_def} (up to conventional normalization factors), and
the ellipsis denotes higher multipoles suppressed by further powers of
$a/r$.

Substituting the scaling $Q/M \sim \varepsilon a^2$ from the Gaussian toy
model, Eq.~\eqref{eq:q_over_m_scaling}, into Eq.~\eqref{eq:phi_multipole}
gives
\begin{equation}
  \Phi(r,\theta)
  \approx -\frac{GM}{r}
  \left[
    1
    + \mathcal{O}\!\left(
        \varepsilon \frac{a^2}{r^2}
      \right) P_2(\cos\theta)
    + \cdots
  \right].
  \label{eq:phi_farfield_scaling}
\end{equation}
Thus the leading anisotropic correction to the Newtonian potential is
suppressed by $\varepsilon(a/r)^2$.  In the language of the throat ontology,
this correction encodes the imprint of the transition region near the mouth
of the throat, where the flow is neither purely radial on the brane nor
purely axial in the bulk, but bends from one into the other.

The main takeaway of this section is therefore:
\begin{itemize}
  \item Dimensional reduction of a localized 4D throat--like density
        produces an effective 3D source with a monopole mass
        $M \sim \rho_0 L a^3$ and a tower of finite--size corrections.
  \item The leading angular distortion is quadrupolar and suppressed by
        $(a/r)^2$ at large radii.
  \item These are precisely the types of terms that, in a full post--Newtonian
        treatment, would be expected to appear at 2PN order as finite--size
        corrections associated with the physical size $a$ of the body.
\end{itemize}

\subsection{Connection to Papers I--III and why 1PN sees a sphere}
\label{subsec:connection_papers1to3}

We now connect this dimensional reduction picture to the 1PN results of
Papers~I--III.  Those papers effectively treat each defect as a point--like
or perfectly spherical source on the brane, described by a monopolar
potential $\Phi \sim -GM/r$ plus velocity--dependent 1PN corrections.
Finite--size structure of the defect is either neglected or, in the case of
the shape--sensitivity tests in Paper~III, shown to have only a weak effect
on the 1PN precession coefficient for modest deformations.

From the standpoint of the throat ontology, this is exactly what one should
expect.  At distances $r\gg a,L$, the effective 3D density
$\rho_{\mathrm{3D}}$ produced by dimensional reduction is dominated by its
monopole component.  Angular structure such as the quadrupole is suppressed
by $(a/r)^2$, and higher multipoles by even higher powers of $a/r$.  As a
result:
\begin{itemize}
  \item The leading Newtonian potential and the 1PN corrections derived in
        Papers~I--III are insensitive to the detailed throat geometry, as
        long as the mass $M$ is held fixed.
  \item The defect therefore looks like a spherical sink at 1PN order, even
        though its interior is cylindrically structured in the bulk.
  \item The modest shape sensitivity found in Paper~III (e.g.\ $\sim 10\%$
        oblateness leading to $\sim 2\%$ shifts in the precession
        coefficient) is naturally interpreted as a small leakage of these
        $(a/r)^2$ corrections into the 1PN observables used in that
        analysis, and as a hint that the model has a geometrically rich
        interior that will matter more at 2PN and beyond.
\end{itemize}

This perspective also clarifies how the gravitational and electromagnetic
sectors can coexist without contradiction.  Papers~I--III operate in the
far--field regime on the brane, where only the monopole and velocity--dependent
interaction terms matter, so the defect appears spherical.  Paper~IV probes
modes confined inside the throat, where the cylindrical interior and the
aspect ratio $L/a$ are crucial.  Dimensional reduction shows that these
descriptions are simply different projections of the same 4D geometry: the
spherical mouth seen by gravity and the cylindrical interior seen by
electromagnetism are both encoded in the same throat.

In the next section we turn from this far--field point of view to the
interior of the throat itself, deriving the cavity modes of the 4D acoustic
equation and reinterpreting the enthalpy--selected aspect ratio
$L/a \approx 1.85$ as a geometric property of the throat in the bulk.

\section{Throat Geometry and the Electromagnetic Sector}
\label{sec:throat_em}

We now turn from the far--field effective description on the brane to the
interior of the throat itself.  The goal is to show how the cylindrical
cavity picture of Paper~IV arises naturally once we treat the defect as a
brane--bulk throat, and to reinterpret the enthalpy--selected aspect ratio
$L/a$ and the notion of ``charge'' in purely geometric terms.  Throughout
this section we work with the linear 4D acoustic equation
\begin{equation}
  \partial_t^2 h
  - c_s^2 \,\nabla_4^2 h
  = 0,
  \label{eq:4d_acoustic_recall}
\end{equation}
introduced in Sec.~\ref{subsec:bulk_fields}, but now restricted to the
throat domain $\mathcal{T}$.

\subsection{4D cavity modes in the throat}
\label{subsec:throat_modes}

Inside the throat, the geometry is approximately that of a straight tube of
radius $a$ and depth $L$ extending into the bulk.  It is convenient to adopt
coordinates adapted to this tube: we take $w$ along the throat and introduce
a radial coordinate $r$ measuring distance from the center of the throat
within the brane directions, together with angular coordinates on the
$(x,y,z)$ directions.  For the lowest--lying modes we are interested in,
there is no dependence on the azimuthal angles or on any internal structure
within the cross--section; the modes are effectively axisymmetric about the
throat center.

In this approximation the 4D Laplacian inside the throat separates as
\begin{equation}
  \nabla_4^2
  \simeq \nabla_{\perp}^2 + \partial_w^2,
\end{equation}
where $\nabla_{\perp}^2$ is the Laplacian in the radial direction $r$ (and
its associated angular coordinate), restricted to axisymmetric configurations.
We seek separated solutions of the form
\begin{equation}
  h(t,r,w) = \Re\!\left\{ H(r)\,W(w)\,e^{-i\omega t} \right\}.
  \label{eq:separated_ansatz}
\end{equation}
Substituting Eq.~\eqref{eq:separated_ansatz} into
Eq.~\eqref{eq:4d_acoustic_recall} yields
\begin{equation}
  \left[
    \omega^2
    - c_s^2\left(
        \frac{1}{H}\nabla_{\perp}^2 H
      + \frac{1}{W}\partial_w^2 W
      \right)
  \right] H W = 0.
\end{equation}
Dividing by $HW$ and rearranging gives
\begin{equation}
  \frac{1}{H}\nabla_{\perp}^2 H
  + \frac{1}{W}\partial_w^2 W
  = \frac{\omega^2}{c_s^2},
\end{equation}
which we can separate by setting each side equal to a constant.  Introducing
separation constants $-k_r^2$ and $-k_w^2$, we obtain
\begin{align}
  \nabla_{\perp}^2 H + k_r^2 H &= 0,
  \label{eq:radial_eq_general}
  \\
  \partial_w^2 W + k_w^2 W &= 0,
  \label{eq:axial_eq_general}
\end{align}
with the dispersion relation
\begin{equation}
  \omega^2 = c_s^2\,(k_r^2 + k_w^2).
  \label{eq:dispersion_relation}
\end{equation}

For axisymmetric modes, $\nabla_{\perp}^2$ reduces to the radial part of the
Laplacian in cylindrical--like coordinates,
\begin{equation}
  \nabla_{\perp}^2 H
  = \frac{1}{r^2}\partial_r\!\left(r^2 \partial_r H\right),
\end{equation}
or, in the thin--throat limit where the cross--section is effectively
two--dimensional, to the standard Bessel form
\begin{equation}
  \nabla_{\perp}^2 H
  = \frac{1}{r}\partial_r\!\left(r \partial_r H\right).
\end{equation}
In either case the regular, axisymmetric solutions of
Eq.~\eqref{eq:radial_eq_general} at $r=0$ are Bessel functions of the first
kind:
\begin{equation}
  H(r) \propto J_0(k_r r),
\end{equation}
with $k_r$ quantized by boundary conditions at the throat wall $r=a$.

In keeping with the cavity analysis of Paper~IV, we impose a boundary
condition that the enthalpy fluctuation vanishes at the wall,
\begin{equation}
  h(r=a,w,t) = 0 \quad\Rightarrow\quad H(a) = 0,
\end{equation}
corresponding physically to a \emph{pinned phase boundary} in the stiff
superfluid vacuum.  The order parameter is topologically fixed at the throat
wall, so small enthalpy (pressure) perturbations are forced to zero there.
Mathematically, this Dirichlet condition selects the zeros of $J_0$,
\begin{equation}
  k_r = \frac{x_{0n}}{a},
\end{equation}
where $x_{0n}$ is the $n$th zero of $J_0$.  We will be primarily interested
in the fundamental radial mode,
\begin{equation}
  k_r = \frac{x_{01}}{a},
\end{equation}
with radial profile $J_0(x_{01}r/a)$.

Along the throat direction $w$ we impose boundary conditions at $w=0$ and
$w=L$.  The simplest idealization is to take $h$ to vanish at both ends,
\begin{equation}
  h(r,w=0,t) = h(r,w=L,t) = 0 \quad\Rightarrow\quad W(0)=W(L)=0,
\end{equation}
corresponding to nodes at the mouth and bottom of the throat.  This leads to
standing--wave solutions
\begin{equation}
  W_n(w) \propto \sin\!\left(\frac{n\pi w}{L}\right),
  \qquad
  k_w = \frac{n\pi}{L},
  \quad n=1,2,\dots
\end{equation}
The fundamental axial mode has $n=1$, so the full fundamental throat mode is
\begin{equation}
  h_1(t,r,w)
  \propto J_0\!\left(\frac{x_{01} r}{a}\right)
          \sin\!\left(\frac{\pi w}{L}\right)
          \cos(\omega t),
  \label{eq:fundamental_mode}
\end{equation}
with
\begin{equation}
  \omega^2
  = c_s^2\left(
      \frac{x_{01}^2}{a^2}
      + \frac{\pi^2}{L^2}
    \right).
  \label{eq:fundamental_dispersion}
\end{equation}
This is the same $J_0$--Bessel / standing--wave structure used in the
cylindrical cavity analysis of Paper~IV, now understood as the fundamental
mode of the brane--bulk throat.

\subsection{Enthalpy minimization and the aspect ratio \texorpdfstring{$L/a$}{L/a}}
\label{subsec:aspect_ratio}

In Paper~IV the key electromagnetic result was that minimizing the enthalpy
of the cavity at fixed ``charge'' picked out a preferred aspect ratio
\begin{equation}
  \frac{L}{a} = \frac{\sqrt{2}\,\pi}{x_{01}} \approx 1.85.
  \label{eq:aspect_ratio_result}
\end{equation}
We now sketch how this emerges from the throat picture.

The relevant functional is the total perturbation energy (or enthalpy) of
the mode in the throat.  For a linear acoustic mode $h(t,r,w)$ with
frequency $\omega$, the time--averaged energy stored in the perturbation can
be written schematically as
\begin{equation}
  \mathcal{E}[h]
  \sim \int_{\mathcal{T}}\!\mathrm{d}^3x\,\mathrm{d}w\;
       \rho_0
       \left[
         \frac{1}{2c_s^2}\,(\partial_t h)^2
         + \frac{1}{2}\,(\nabla_4 h)^2
       \right],
  \label{eq:enthalpy_functional}
\end{equation}
where we have suppressed numerical prefactors that do not depend on $a$ or
$L$.  For a separated mode of the form
$h(t,r,w) = A H(r)W(w)\cos(\omega t)$ the time average of $(\partial_t h)^2$
contributes $\omega^2$, while the spatial gradients contribute $k_r^2$ and
$k_w^2$:
\begin{equation}
  \langle (\partial_t h)^2 \rangle
  \propto \omega^2 A^2 H^2 W^2,
  \qquad
  \langle (\nabla_4 h)^2 \rangle
  \propto (k_r^2 + k_w^2) A^2 H^2 W^2.
\end{equation}
Using the dispersion relation~\eqref{eq:fundamental_dispersion}, the total
time--averaged energy carried by the fundamental mode is therefore
proportional to
\begin{equation}
  \mathcal{E}
  \propto A^2 (k_r^2 + k_w^2) \int_{\mathcal{T}} H^2 W^2,
  \label{eq:energy_kw}
\end{equation}
where the integral over $H^2 W^2$ supplies a factor of order $a^2 L$ for the
fundamental mode.

To define a variational problem we must specify which quantity is held fixed
when minimizing $\mathcal{E}$.  Following Paper~IV, we identify a conserved
``charge'' $\mathcal{Q}$ associated with the mode amplitude, which depends
on the same integral of $H^2W^2$ but not on $k_r^2+k_w^2$.  Schematically,
\begin{equation}
  \mathcal{Q}
  \propto A^2 \int_{\mathcal{T}} H^2 W^2,
  \label{eq:charge_functional}
\end{equation}
so that, at fixed $\mathcal{Q}$, the energy scales as
\begin{equation}
  \mathcal{E}
  \propto (k_r^2 + k_w^2)\,\mathcal{Q}.
\end{equation}
Minimizing $\mathcal{E}$ at fixed $\mathcal{Q}$ therefore amounts to
minimizing $k_r^2 + k_w^2$.

For the fundamental throat mode we have
\begin{equation}
  k_r^2 + k_w^2
  = \frac{x_{01}^2}{a^2}
    + \frac{\pi^2}{L^2}.
\end{equation}
Varying $a$ and $L$ while holding the effective cross--sectional area
$\pi a^2$ and depth $L$ in a constrained way (reflecting the fact that the
total mass $M\sim \rho_0 L a^3$ is fixed) leads to a minimum of
$k_r^2 + k_w^2$ at a particular ratio $L/a$.  The detailed calculation
follows the same steps as in Paper~IV; here we simply quote the result:
\begin{equation}
  \frac{L}{a}
  = \frac{\sqrt{2}\,\pi}{x_{01}}.
\end{equation}
At this aspect ratio, the contributions of radial and axial gradients are in
a particular balance that extremizes the energy per unit charge stored in
the cavity.  In the throat ontology, Eq.~\eqref{eq:aspect_ratio_result} is
not an arbitrary parameter choice for a 3D cylinder, but a statement about
the geometry of the 4D throat: for a given mass and charge, the throat
relaxes to a preferred ratio of depth to radius.

\subsection{Charge as circulation and vorticity flux through the throat}
\label{subsec:charge_flux}

So far we have treated the ``charge'' $\mathcal{Q}$ as an abstract conserved
quantity associated with the cavity mode amplitude.  In the superfluid
picture, $\mathcal{Q}$ has a more concrete interpretation in terms of
circulation and vorticity.

Consider a closed loop $\mathcal{C}$ on the brane encircling the throat
mouth, for example a circle of radius $r>a$ in a plane intersecting the
brane.  The circulation of the superfluid velocity around this loop is
\begin{equation}
  \Gamma
  = \oint_{\mathcal{C}} \mathbf{v}\cdot\mathrm{d}\boldsymbol{\ell}.
\end{equation}
In a quantum fluid, $\Gamma$ would be quantized; in the present classical
toy model we treat it as an integer--valued conserved input label,
$\Gamma\in\mathbb{Z}$, in the absence of vorticity creation or reconnection
events.  In the hydrodynamic--electromagnetic dictionary of Paper~IV the
effective charge is proportional to this circulation (up to a fixed
normalization), $\mathcal{Q}\propto\Gamma$.  By Stokes' theorem, $\Gamma$ is
equivalently the flux of vorticity through any surface $\mathcal{S}$
bounded by $\mathcal{C}$,
\begin{equation}
  \Gamma
  = \int_{\mathcal{S}}
      (\nabla\times\mathbf{v})\cdot\mathrm{d}\mathbf{S}.
\end{equation}
If the only significant vorticity near $\mathcal{C}$ is carried by the
throat, then this flux is, to a good approximation, the vorticity threading
the throat cross--section.

In the hydrodynamic--electromagnetic dictionary of Paper~IV, the electric
charge of the defect is identified (up to a proportionality constant) with
this circulation or vorticity flux.  The field variables that play the role
of the electric and magnetic fields, $\mathbf{E}$ and $\mathbf{B}$, are
constructed from the enthalpy gradients and the velocity field in such a way
that Maxwell's equations emerge as effective equations in the near field of
the defect.  In that language, $\mathcal{Q}$ is essentially the integral of
the mode amplitude over the throat cross--section, which is in turn
proportional to the vorticity flux through that cross--section.

The throat picture makes this geometric: the vorticity lines thread the
throat like the flux lines of a solenoid threading a torus.  The conserved
charge is the flux of vorticity through the ``hole'' of the throat, i.e.\ 
through the surface of area $\pi a^2$ that spans the spherical mouth on the
brane.  Different values of charge correspond to different amounts of
vorticity threading the throat, and the cavity mode adjusts its amplitude to
accommodate this flux while keeping the enthalpy as low as possible.  The
preferred aspect ratio $L/a$ then reflects how the throat geometry arranges
itself to support a given amount of vorticity flux at minimum energetic
cost.


\subsection{Magnetostatics from brane transverse wakes}
\label{subsec:brane_magnetostatics}

A persistent obstruction in potential--flow pictures is that a translating
defect in an irrotational bulk produces a dipolar velocity potential
$\phi\sim (\mathbf{u}\cdot \mathbf{r})/r^3$, whose curl vanishes identically
outside the core.  If one attempts to identify the magnetic field directly
with bulk vorticity, this would imply $\mathbf{B}=0$ in the far zone, in
conflict with Maxwell magnetostatics.

The brane projection resolves this: the same defect that sources the
longitudinal gravitational sector can also excite a brane--confined transverse
mode $\mathbf{v}_T$ in the decomposition \eqref{eq:v3d_decomposition}.  We
treat $\mathbf{v}_T$ as a surface--current (vortex--sheet) degree of freedom
whose governing equation on $\mathcal{B}$ is a vector Poisson problem in
Coulomb gauge.  For a localized moving throat centered at
$\mathbf{x}_0(t)$, define $\mathbf{r}\equiv \mathbf{x}-\mathbf{x}_0$.  We
write
\begin{equation}
  \nabla\cdot\mathbf{A}=0, \qquad
  \nabla^2 \mathbf{A} = -\kappa_A\,\mathcal{Q}\,\mathbf{u}\,\delta^{(3)}(\mathbf{r}),
  \label{eq:vector_poisson_A}
\end{equation}
where $\mathbf{u}$ is the (slow) brane--parallel velocity of the throat and
$\kappa_A$ is a fixed calibration constant.  Identifying the vector potential
with the brane transverse wake,
\begin{equation}
  \mathbf{A} \equiv \kappa_A\,\mathbf{v}_T,
  \label{eq:A_from_vT}
\end{equation}
the solution of \eqref{eq:vector_poisson_A} is
\begin{equation}
  \mathbf{A}(\mathbf{r}) = \frac{\kappa_A\,\mathcal{Q}}{4\pi}\,\frac{\mathbf{u}}{r},
  \label{eq:A_u_over_r}
\end{equation}
and therefore
\begin{equation}
  \mathbf{B}(\mathbf{r})
  = \nabla\times\mathbf{A}
  = \frac{\kappa_A\,\mathcal{Q}}{4\pi}\,\frac{\mathbf{u}\times\mathbf{r}}{r^3},
  \label{eq:biot_savart_point}
\end{equation}
which is the Biot--Savart form for the field of a moving point charge (or,
equivalently, a localized current element) up to the overall normalization.

Two points are essential.  First, \eqref{eq:A_u_over_r} is \emph{not} attributed
to viscous drag (a Stokeslet); the $1/r$ kernel arises because $\mathbf{v}_T$
is a brane--localized transverse mode with its own Green's function.  Second,
this construction leaves the bulk flow used for gravity irrotational in the
far zone: magnetism lives in the brane's transverse sector, while gravity
lives in the bulk--induced longitudinal sector.  A time--dependent completion
(radiation and induction) will require promoting \eqref{eq:vector_poisson_A} to
a dynamical wave equation on the brane, which we defer to future work.

\subsection{Mass as drainage and volume flux}
\label{subsec:mass_flux}

In contrast to charge, which is associated with circulation and vorticity
flux, the mass of the defect is associated with drainage of fluid into the
throat and the resulting volume deficit in the surrounding superfluid.  From
the brane point of view, a defect appears as a localized sink: the normal
component of the bulk velocity at the mouth is nonzero and directed into
the throat,
\begin{equation}
  v_w(\mathbf{x},w=0,t) < 0
\end{equation}
within the mouth region $\abs{\mathbf{x}}\lesssim a$.  The net volumetric
flux into the throat is
\begin{equation}
  \dot{V}
  = \int_{\mathrm{mouth}}
      v_w(\mathbf{x},w=0,t)\,\mathrm{d}^2S,
\end{equation}
where the integral is over the effective mouth area on the brane.  In a
steady configuration this flux is balanced by either compression within the
throat, circulation of fluid in closed streamline patterns, or outflow at
the bottom of the throat, depending on the global topology.  The resulting
density deficit in the near field is what sources the effective Newtonian
potential on the brane.

At the level of scaling, the mass associated with the throat is
\begin{equation}
  M \sim \rho_0 \,(\text{effective throat volume})
    \sim \rho_0 \,\pi a^2 L,
\end{equation}
modulo order--unity geometric factors and corrections from the detailed
profile of $\rho(\mathbf{x},w)$.  This is consistent with the Gaussian toy
model of Sec.~\ref{subsec:gaussian_toy_model}, where $M\propto \rho_0 L a^3$
up to numerical factors, and with the interpretation of $M$ as the total
mass deficit localized within the throat and its immediate surroundings.

Putting these pieces together, the throat ontology suggests a simple
geometric picture:
\begin{itemize}
  \item Mass is associated with the \emph{volume} of the throat and the
        associated drainage of fluid into it.
  \item Charge is associated with the \emph{vorticity flux} threading the
        throat cross--section.
\end{itemize}
Both are properties of the same geometric object---a brane--bulk throat of
radius $a$ and depth $L$---viewed through different hydrodynamic
projections.  The gravitational sector is sensitive mainly to the integrated
mass deficit and the resulting monopolar potential on the brane, while the
electromagnetic sector is sensitive to the internal cavity modes and the
vorticity flux they support.

In this way, the sphere--cylinder tension that appeared when Papers~I--III
and IV were viewed in isolation is naturally resolved: the same throat can
appear as a spherical sink on the brane for gravity and as a cylindrical
cavity in the bulk for electromagnetism.  In the next section we move back
toward the brane and examine the transition region near the throat mouth,
where finite--size effects and 2PN corrections originate.

\section{Near-Field Transition and Finite-Size Effects}
\label{sec:near_field}

So far we have treated the throat in two complementary limits: the far field
on the brane, where only the monopole mass and velocity--dependent 1PN
interactions matter, and the deep interior of the throat, where cylindrical
cavity modes control the electromagnetic sector.  Between these lies a
transition region near the throat mouth where the flow is strongly curved
and neither limit applies cleanly.  In this section we give a qualitative
description of this near--field region, argue that its imprint on the
effective 3D theory is naturally multipolar and suppressed by $(a/r)^2$, and
connect this picture to the shape--sensitivity tests of the spin and
$N$--body paper (Paper~III).

\subsection{Flow regimes: far field, throat interior, and transition region}
\label{subsec:flow_regimes}

It is useful to distinguish three qualitatively different flow regimes
associated with a single throat:

\begin{enumerate}
  \item \emph{Far field on the brane} ($r \gg a,L$): Here the flow is almost
  purely radial in the brane directions and nearly independent of $w$.  The
  effective density $\rho_{\mathrm{3D}}(\mathbf{x})$ obtained by integrating
  over $w$ is well approximated by a spherically symmetric monopole plus
  small multipole corrections, as in Sec.~\ref{subsec:gaussian_toy_model}.
  The velocity field $\mathbf{v}_{\mathrm{3D}}(\mathbf{x})$ is essentially
  that of a point--like sink at the origin.

  \item \emph{Throat interior} ($r \lesssim a$, $0<w<L$): Inside the throat
  the flow is predominantly along the $w$ direction, with relatively weak
  dependence on the brane coordinates except through the radial cavity
  profile $J_0(x_{01}r/a)$ of the fundamental mode.  The relevant modes are
  the separated solutions of the 4D acoustic equation
  \eqref{eq:4d_acoustic_recall}, and the leading dynamics are controlled by
  the discrete set of $(k_r,k_w)$ pairs and the aspect ratio $L/a$.

  \item \emph{Transition region} ($r\sim a$, $w\sim 0$): Near the mouth of
  the throat the flow lines bend from radial inflow on the brane to axial
  flow into the bulk.  The geometry here is strongly curved; the brane
  deviates from a flat $w=0$ hypersurface, and the fluid experiences both
  significant shear and pressure gradients.  This is the region where
  departures from perfect spherical symmetry on the brane originate and
  where the throat geometry imprints itself most strongly on the effective
  3D source $\rho_{\mathrm{3D}}$.
\end{enumerate}

In the previous sections we treated the far field and the throat interior in
controlled approximations: the far field through multipole expansion and the
interior through mode separation.  The transition region does not admit such
a simple analytic treatment, but we can still characterize its effects on
the brane in terms of symmetry and scaling.  Its characteristic size is set
by $a$; its angular structure is determined by how the brane is deformed
into the bulk and by how streamlines reorient from brane--parallel to
bulk--directed.  These features naturally generate higher multipoles in
$\rho_{\mathrm{3D}}$ and $\mathbf{v}_{\mathrm{3D}}$ that are suppressed by
powers of $(a/r)$ at large distances.

\subsection{Streamline bending and induced stresses}
\label{subsec:streamline_bending}

To get a feel for the transition region, consider streamlines of the fluid
in a quasi--steady configuration.  Far from the defect, on the brane, the
streamlines are nearly straight and radial, converging toward the origin
with speed $v_r(r)\propto 1/r^2$ for an incompressible sink.  Deep inside
the throat, by contrast, streamlines are nearly parallel to the $w$ axis,
with a roughly uniform axial velocity $v_w$ that carries fluid along the
throat.

In order to connect these two regimes, streamlines must bend sharply in a
neighborhood of the mouth.  The curvature of a streamline is controlled by
the local acceleration,
\begin{equation}
  \frac{\mathrm{D}\mathbf{v}_4}{\mathrm{D}t}
  = \partial_t \mathbf{v}_4
    + (\mathbf{v}_4\cdot\nabla_4)\mathbf{v}_4
  = -\frac{1}{\rho}\,\nabla_4 P
    + \frac{\mathbf{f}_{\mathrm{ext}}}{\rho},
\end{equation}
so large curvature implies large pressure gradients in the transition
region.  Even in a quasi--steady state where $\partial_t\mathbf{v}_4\simeq 0$,
the advective term $(\mathbf{v}_4\cdot\nabla_4)\mathbf{v}_4$ is enhanced
where streamlines squeeze together and turn into the throat, and this must
be balanced by correspondingly strong gradients in $P$ and $h$.

From the brane point of view, these pressure gradients generate stresses in
the effective 3D fluid.  In particular, the stress tensor constructed from
the velocity field and the enthalpy,
\begin{equation}
  T_{ij}
  \sim \rho v_i v_j + \delta_{ij} P + \cdots,
\end{equation}
has nontrivial angular dependence in the transition region.  The components
tangent to the brane ($i,j\in\{x,y,z\}$) feel both the radial inflow and the
shear associated with bending into the $w$ direction.  When this structure
is projected onto the brane and coarse--grained over scales larger than $a$,
it manifests as effective multipole moments in the 3D source terms that
enter the gravitational and electromagnetic sectors.

Although a full solution of the transition--region flow would require
solving the nonlinear 4D Euler equations with a deformed brane geometry,
symmetry already constrains the leading corrections.  For an isolated
defect with no preferred direction on the brane, the transition region must
be axisymmetric about the throat axis and reflection--symmetric under
$w\to -w$ if we consider a mirrored continuation into the bulk.  These
symmetries forbid dipolar contributions in $\rho_{\mathrm{3D}}$ and
$\mathbf{v}_{\mathrm{3D}}$ and make the quadrupole ($\ell=2$) the leading
nontrivial multipole.

\subsection{Multipole expansion of the transition region}
\label{subsec:transition_multipoles}

We can formalize this intuition by writing the effective 3D density on the
brane as a sum of a purely radial piece and angular corrections sourced by
the transition region.  In spherical coordinates $(r,\theta,\phi)$ on the
brane, with the throat axis taken as the polar axis, we expand
\begin{equation}
  \rho_{\mathrm{3D}}(r,\theta)
  = \rho_0(r)
    + \sum_{\ell\ge 1} \rho_{\ell}(r) P_{\ell}(\cos\theta),
  \label{eq:rho3d_multipole}
\end{equation}
where $\rho_0(r)$ is the spherically symmetric part and
$P_{\ell}$ are Legendre polynomials.  Axisymmetry removes any dependence on
$\phi$, and reflection symmetry around the throat axis suppresses odd $\ell$
for the density; the leading correction is thus the quadrupole term
$\ell=2$.

The Gaussian toy model of Sec.~\ref{subsec:gaussian_toy_model} provides an
explicit realization of this structure, with
\begin{equation}
  \rho_{\mathrm{3D}}(r,\theta)
  \propto e^{-r^2/a^2}
          \Big[1 + \varepsilon P_2(\cos\theta)\Big],
\end{equation}
so that $\rho_0(r)$ and $\rho_2(r)$ are both localized near $r\lesssim a$.
Integrating $\rho_{\mathrm{3D}}$ against $r^2 P_{\ell}(\cos\theta)$ over a
volume enclosing the throat then yields the multipole moments, and we saw
explicitly that$\,$$Q/M\sim \varepsilon a^2$ for the quadrupole.  Higher
multipoles would involve higher powers of $a$ and smaller dimensionless
coefficients.

The same logic applies, at least at the level of scaling, to any reasonable
model of the transition region: as long as the region is localized within a
radius of order $a$ and preserves axisymmetry, all multipole moments higher
than the monopole are suppressed by powers of $a$ and decay with
appropriate powers of $1/r$ in the far field.  The resulting potential can
be written as
\begin{equation}
  \Phi(r,\theta)
  = -\frac{GM}{r}
    \Bigg[
      1
      + \sum_{\ell\ge 2}
          \alpha_{\ell}
          \left(\frac{a}{r}\right)^{\ell}
          P_{\ell}(\cos\theta)
    \Bigg],
  \label{eq:phi_multipole_general}
\end{equation}
with dimensionless coefficients $\alpha_{\ell}$ that encode the detailed
structure of the throat mouth and transition region.  For $\ell=2$ we
recover the $(a/r)^2$ suppression emphasized earlier.  In the absence of
fine tuning, the $\alpha_{\ell}$ are expected to be of order one times small
shape parameters characterizing how much the throat deviates from an ideal
spherical mouth.

\subsection{Quantitative verification: The rounded funnel model}
\label{subsec:rounded_funnel}

While the Gaussian model of Appendix~\ref{app:gaussian_multipoles} provides useful analytic scaling relations,
it is a simplified caricature of the throat.  To quantitatively assess finite--size corrections for a physically
bounded defect, we numerically integrate the effective density for a geometry that explicitly connects the
cylindrical bulk interior to the planar brane.

We model this transition using a hard--bounded ``rounded funnel'' geometry.  Deep in the bulk ($w\gg a$),
the throat is a cylinder of radius $a$.  Near the brane ($w\to 0$), the throat flares outward and smoothly
joins the bulk tube to the brane mouth.  We parameterize the throat radius $R(w)$ at depth $w$ as
\begin{equation}
  R(w) = a\left[1+\frac{1}{2}\exp\!\left(-\frac{5w}{a}\right)\right],
  \label{eq:funnel_radius}
\end{equation}
so that the mouth radius is $R(0)=1.5a$ and the profile relaxes exponentially to $R(w)\to a$ for $w\gtrsim a$.
The effective density inside this region is taken to be uniform, $\rho=\rho_0$, modulated by a small quadrupolar
deformation $\varepsilon P_2(\cos\theta)$ to represent the stress anisotropy discussed in
Sec.~\ref{subsec:streamline_bending}.

We performed a numerical integration of the resulting effective 3D multipole moments for a throat with aspect
ratio $L/a=2$ and deformation $\varepsilon=0.1$ (details in Appendix~\ref{app:physical_throat}).  The calculation
yields the dimensionless quadrupole coefficient
\begin{equation}
  \alpha_2 \simeq 1.4\times 10^{-2}.
\end{equation}
This result is significant for two reasons.  First, it confirms the sign of the correction in our convention:
$\alpha_2>0$ means the far--field potential is slightly deeper along the throat axis ($P_2>0$) and slightly
shallower near the equator ($P_2<0$), consistent with taking a positive quadrupolar anisotropy $\varepsilon$ in the
effective density ansatz.  Second, the magnitude is small: for the same $\varepsilon$ the Gaussian toy model gives
$\alpha_2=(3/10)\varepsilon=0.03$ (cf.\ Eq.~\eqref{eq:q_over_m_scaling}), whereas the hard--bounded rounded funnel
yields $\alpha_2\simeq 0.014$, about a factor of two smaller.  This suggests that physically bounded throat
geometries can ``hide'' finite--size multipoles more efficiently than soft Gaussian tails.

Consequently, finite--size corrections from the throat geometry enter the brane potential at the level
$\sim 10^{-2}(a/r)^2$, providing quantitative support for the spherical sink approximation used in the 1PN
orbital sector (Paper~I).

\subsection{Interpretation as 2PN finite-size corrections}
\label{subsec:2pn_interpretation}

From the point of view of post--Newtonian theory, the multipole corrections
generated by the transition region are naturally interpreted as finite--size
effects that first appear at 2PN order and beyond.  In standard GR language,
the small parameter controlling the PN expansion is
\begin{equation}
  \epsilon
  \sim \frac{GM}{rc^2}
  \sim \left(\frac{v}{c}\right)^2,
\end{equation}
with 1PN terms scaling as $\epsilon$ and 2PN terms as $\epsilon^2$.  For a
compact object of radius $R$ in GR, finite--size effects typically scale as
$(R/r)^2$ or higher powers, and in many situations they enter observables at
2PN order or later.

In the throat ontology, the role of the body's size is played by the throat
radius $a$, and the relevant dimensionless parameter is $a/r$.  For
well--separated bodies we have $a\ll r$, so $(a/r)^2$ is parametrically
small compared to unity.  The multipole expansion
\eqref{eq:phi_multipole_general} then suggests that the leading corrections
to the monopole potential scale as
\begin{equation}
  \delta\Phi
  \sim \Phi_{\mathrm{N}}\,
       \alpha_2 \left(\frac{a}{r}\right)^2
  \sim -\frac{GM}{r}
       \alpha_2 \left(\frac{a}{r}\right)^2.
\end{equation}
If $a$ is itself of order the gravitational radius $r_g=GM/c^2$ of the
object, then $(a/r)^2\sim (GM/rc^2)^2\sim\epsilon^2$, and these corrections
would indeed be 2PN in the usual counting.  Even if $a$ is somewhat larger
than $r_g$, as might be natural in a superfluid context, the $(a/r)^2$
suppression still pushes these effects beyond the leading 1PN terms for
sufficiently large separations.

This scaling argument underlies our claim in the Introduction that the
throat ontology provides a geometric roadmap for 2PN corrections: once $a$
is fixed for each object (e.g.\ by its mass and charge through the throat
geometry), the coefficients $\alpha_{\ell}$ in
Eq.~\eqref{eq:phi_multipole_general} are not arbitrary; they are determined
by the way the brane deforms into the bulk and by the structure of the
transition region.  A future 2PN calculation in this model would therefore
not introduce an unrestricted zoo of new parameters, but a constrained set
of finite--size couplings tied to the same throat geometry that already
controls the 1PN and electromagnetic sectors.

\subsection{Relation to shape sensitivity in the spin/$N$-body paper}
\label{subsec:shape_sensitivity}

The spin and $N$--body paper (Paper~III) already provided a first glimpse of
these geometric finite--size effects.  There, the authors replaced the
spherical defects of the orbital paper with mildly oblate spheroids and
computed how the 1PN perihelion precession coefficient changed as a function
of the oblateness.  The result was that even a relatively large shape
distortion at the level of $\sim 10\%$ in the equatorial radius produced
only a modest change of order a few percent in the precession coefficient.

In a purely three--dimensional description, this shape sensitivity might
seem puzzling or arbitrary: why does a substantial deformation of the body's
shape have only a small effect on the orbit?  In the throat ontology the
answer is straightforward.  Changing the apparent 3D shape of the defect on
the brane corresponds to modifying the geometry of the throat mouth and the
transition region, while holding fixed the deeper throat interior and its
overall mass content.  The resulting changes in the effective 3D density are
encoded primarily in the multipole moments of $\rho_{\mathrm{3D}}$, which
are suppressed by powers of $(a/r)$ and thus have only a small effect on the
far--field potential and on 1PN observables.

More concretely, a modest oblateness changes the quadrupole coefficient
$\alpha_2$ in Eq.~\eqref{eq:phi_multipole_general} by some amount of order
the oblateness parameter.  But because the quadrupole contribution is
already suppressed by $(a/r)^2$, the net fractional change in the potential
and in the 1PN precession coefficient is doubly suppressed: once by the
small shape parameter and once by $(a/r)^2$.  This is entirely consistent
with the $\sim 2\%$ shifts reported in Paper~III for $\sim 10\%$ geometric
distortions.

From this perspective, the shape--sensitivity experiment in Paper~III is not
an ad hoc curiosity but an indirect probe of the transition region and the
throat geometry.  It shows that the toy model already behaves like a
finite--size object in GR: small departures from spherical symmetry induce
small corrections to 1PN observables, in a way that is naturally understood
as a preview of the 2PN finite--size structure.  The throat ontology turns
this qualitative observation into a quantitative program: in a full 2PN
analysis, one would compute the coefficients $\alpha_{\ell}$ and the
associated finite--size couplings directly from the geometry and dynamics of
the transition region.

In the next section we will see how a different, but related, aspect of the
throat geometry---the mixing of brane--parallel and bulk--directed modes in
the vector sector---provides a natural interpretation of the longitudinal
mixing $\alpha^2=3/4$ required to match the Einstein--Infeld--Hoffmann
Lagrangian at 1PN order.

\section{Wake Mixing Constraint and \texorpdfstring{$\alpha^2 = 3/4$}{alpha^2 = 3/4}}
\label{sec:wake_mixing_constraint}

In Paper~III the 1PN \emph{vector} sector was obtained by decomposing the
translational wake of a moving defect into isotropic longitudinal and
transverse projector components and matching the resulting velocity--dependent
interaction terms to the Einstein--Infeld--Hoffmann (EIH) Lagrangian.
With the completed wake basis, the matching fixes a single real mixing
parameter,
\begin{equation}
  \alpha^2 = \frac{3}{4},
  \label{eq:alpha_squared_value}
\end{equation}
which may be viewed as the relative longitudinal/transverse weighting in the
effective vector kernel.  Importantly, $\alpha^2>0$ is compatible with an
everywhere positive--definite quadratic wake functional; the wake energy
remains bounded below.

The throat ontology provides a natural geometric interpretation of
Eq.~\eqref{eq:alpha_squared_value}.  A defect anchored to the brane is not a
purely three--dimensional object: its motion excites both (i) brane--parallel
transverse response in the surrounding superfluid and (ii) longitudinal
response associated with flow into (and out of) the bulk through the throat.
In the far zone, only the brane projection of this combined response enters
the 1PN observables, but the bulk degrees of freedom renormalize the
\emph{relative} strength of the longitudinal and transverse projector pieces.
The EIH match therefore constrains the throat's effective mode content rather
than requiring any indefinite ``Lorentzian'' signature in mode space.

Earlier exploratory versions of the wake decomposition (which truncated the
basis of allowed wake fields) can spuriously suggest a negative longitudinal
weight.  The corrected value \eqref{eq:alpha_squared_value} should be regarded
as the consistent outcome once the full isotropic projector basis is used and
the scalar, optical, and spin sectors are simultaneously respected.
\section{Toward 2PN: Predictions and Falsifiability}
\label{sec:2pn_predictions}

Up to this point we have used the throat ontology primarily to reinterpret
existing 1PN and electromagnetic results.  In this section we look forward:
we outline how a full 2PN analysis would be organized in this framework, and
we sketch how the resulting finite--size corrections could, in principle, be
confronted with observations.  The aim is not to perform a 2PN calculation,
but to make clear that such a calculation is both well--posed and
constrained, rather than an open invitation to add arbitrary new terms.

\subsection{What 2PN means in this toy model}
\label{subsec:2pn_meaning}

In standard post--Newtonian theory, the expansion parameter is
\begin{equation}
  \epsilon
  \sim \frac{GM}{rc^2}
  \sim \left(\frac{v}{c}\right)^2,
\end{equation}
where $M$ is a characteristic mass, $r$ is a characteristic separation, and
$v$ is a characteristic orbital speed.  Newtonian gravity is $\mathcal{O}(\epsilon^0)$,
1PN corrections are $\mathcal{O}(\epsilon)$, 2PN corrections are
$\mathcal{O}(\epsilon^2)$, and so on.  In GR this hierarchy is built into
the Einstein equations via the expansion of the metric components in powers
of $1/c$.

In the superfluid defect toy model, the same small parameter appears, but
there is an additional geometric scale: the throat radius $a$ (and, to a
lesser extent, the depth $L$).  We can therefore construct two independent
dimensionless quantities:
\begin{equation}
  \epsilon
  \sim \frac{GM}{rc^2},
  \qquad
  \delta
  \sim \frac{a}{r}.
\end{equation}
The 1PN analysis in Papers~I--III effectively assumes that $\delta$ is small
enough that finite--size effects can be neglected at leading order, and that
all relevant corrections scale with $\epsilon$ alone.  The throat ontology
makes this assumption explicit: the far--field theory on the brane is
controlled by the monopole mass $M$ and the velocity--dependent interactions
encoded in the vector kernel, while the structure of the throat and
transition region enters only through multipole corrections suppressed by
powers of $\delta$.

In a full 2PN analysis of the toy model, one would therefore organize terms
in a double expansion in $\epsilon$ and $\delta$, keeping track of how
powers of $a/r$ enter alongside powers of $GM/(rc^2)$.  Schematically, the
effective action or Lagrangian for a binary system would contain terms of
the form
\begin{equation}
  L_{\mathrm{eff}}
  = L_{\mathrm{N}}
    + \epsilon L_{\mathrm{1PN}}
    + \epsilon^2 L_{\mathrm{2PN}}
    + \delta^2 \tilde{L}_{\mathrm{2PN}}
    + \cdots,
\end{equation}
where $L_{\mathrm{2PN}}$ contains the usual point--particle 2PN corrections
and $\tilde{L}_{\mathrm{2PN}}$ contains finite--size corrections tied to the
throat geometry.  The key point is that the throat ontology predicts the
\emph{structure} of $\tilde{L}_{\mathrm{2PN}}$ in terms of a small set of
geometric parameters, rather than leaving it arbitrary.

\subsection{Organizing 2PN corrections via the throat geometry}
\label{subsec:2pn_organization}

Within the throat ontology, the most natural way to organize 2PN corrections
is to group them according to which part of the geometry they probe:

\begin{enumerate}
  \item \emph{Far--field multipoles on the brane}: These are corrections to
  the scalar potential and its velocity--dependent companions arising from
  the multipole structure of $\rho_{\mathrm{3D}}$ and
  $\mathbf{v}_{\mathrm{3D}}$ on the brane.  They are controlled by the
  coefficients $\alpha_\ell$ in the expansion
  \begin{equation}
    \Phi(r,\theta)
    = -\frac{GM}{r}
      \left[
        1
        + \sum_{\ell\ge 2}
            \alpha_\ell
            \left(\frac{a}{r}\right)^\ell
            P_\ell(\cos\theta)
      \right],
  \end{equation}
  with $\alpha_\ell$ determined by the geometry of the throat mouth and the
  transition region.  The leading quadrupole term ($\ell=2$) is naturally
  associated with 2PN finite--size effects in the orbital dynamics.

  \item \emph{Velocity--dependent couplings}: Just as the 1PN vector sector
  is controlled by the transverse and longitudinal parts of the vector
  kernel and the parameter $\alpha^2=3/4$, the 2PN vector and tensor
  sectors will receive corrections from the same kernel but with additional
  structure induced by the throat.  For example, the effective kernel could
  acquire mild dependence on $a$ and $L$ through higher--derivative terms,
  leading to corrections that scale as $(a/r)^2\epsilon$ or $\epsilon^2$.
  These terms would modify the coefficients of velocity--dependent
  structures in the many--body Lagrangian, but in a way that is tied to the
  same geometric data $(M,a,L,\ldots)$.

  \item \emph{Electromagnetic backreaction}: In the EM paper, charge is
  associated with vorticity flux threading the throat and the electromagnetic
  field is sourced by cavity modes inside the throat.  At 2PN order, the
  energy stored in these modes and in the surrounding electromagnetic field
  will backreact on the gravitational sector.  This will generate
  corrections coupling mass, charge, and throat geometry, schematically of
  the form
  \begin{equation}
    L_{\mathrm{eff}}
    \supset \epsilon^2
      \left(
        \beta_1 \frac{Q^2}{M^2}
        + \beta_2 \frac{Q^2}{M^2}\frac{a^2}{r^2}
        + \cdots
      \right),
  \end{equation}
  with coefficients $\beta_i$ determined by the cavity spectrum.  Thus 2PN
  corrections in this model are expected to carry signatures of both mass
  and charge in a way that reflects the common throat origin of the
  gravitational and electromagnetic sectors.
\end{enumerate}

A practical 2PN calculation in this framework would proceed roughly as
follows:

\begin{itemize}
  \item Choose a parametrization of the throat geometry (e.g.\ a family of
        smooth embeddings of the brane into the bulk near each defect,
        characterized by $a$, $L$, and a small number of shape parameters).

  \item Solve the 4D Euler and continuity equations perturbatively in the
        near field of each throat, matching onto the far--field multipole
        expansions on the brane and onto the cavity modes in the interior.

  \item Compute the induced interaction Lagrangian for a system of moving
        throats by integrating out the fluid degrees of freedom to the
        appropriate order in $\epsilon$ and $\delta$.

  \item Match the resulting effective Lagrangian onto a PN--style basis of
        scalars built from positions, velocities, and spins, and read off
        the finite--size coefficients as functions of $(M,a,L,Q,\ldots)$.
\end{itemize}

While technically demanding, this program is sharply constrained: once $M$,
$a$, $L$, and $Q$ are fixed for each object by low--order data (Newtonian
mass, EM charge, and perhaps one additional observable), there is little
room to adjust higher--order coefficients without spoiling the consistency
of the throat geometry.

\subsection{Falsifiability and observational channels}
\label{subsec:falsifiability}

The existence of a geometrically constrained 2PN sector raises a natural
question: how could the toy model be tested or ruled out?  While a full
phenomenological analysis lies beyond the scope of this paper, we can
identify several promising observational channels.

\paragraph{Binary pulsars and compact binaries.}
Timing measurements of binary pulsars and phase evolution of compact
binaries (including black hole and neutron star mergers) are sensitive to
2PN--order corrections in the orbital dynamics.  In GR, these corrections
are controlled by the masses, spins, and, in some cases, tidal deformability
parameters of the bodies.  In the throat model, there would be additional
finite--size contributions determined by the throat radii $a_a$ and $a_b$
and by the way the throats deform under mutual tidal fields.  If the model
predicts a specific relation between $a$ and $M$ (or between $a$, $M$, and
$Q$), then the pattern of 2PN corrections in different systems is highly
constrained.  Significant inconsistencies between the inferred 2PN
coefficients across different binaries would rule out the model.

\paragraph{Solar--system tests.}
Precision measurements of planetary ephemerides, light deflection, and
Shapiro delay in the solar system already probe some aspects of 1PN and
2PN physics.  The throat model is designed to match the 1PN sector by
construction; deviations are expected to appear, if at all, in subtle
finite--size effects associated with the Sun's throat geometry.  For
instance, small angular dependence in the effective potential or small
modifications to perihelion precession beyond the standard 1PN terms could
signal a non--GR finite--size structure.  Any 2PN calculation in the throat
model would need to be checked against the tight solar--system bounds.

\paragraph{Short--range gravity experiments.}
At laboratory scales, tests of Newton's law probe potential deviations from
the $1/r^2$ force at distances comparable to or smaller than the effective
size of the gravitating bodies.  In the throat ontology, if the throat
radius $a$ for ordinary matter is not vanishingly small compared to these
scales, one might expect measurable finite--size corrections to the force
law at short range.  Conversely, the absence of such deviations can be used
to place upper bounds on $a$ for laboratory masses, which must be consistent
with any $a$--values inferred from astrophysical systems.

\paragraph{Electromagnetic--gravitational cross--checks.}
Because mass and charge are different projections of the same throat
geometry, the model predicts relationships between gravitational and
electromagnetic observables that have no analogue in GR.  For example, if
$a$ is related to $Q$ via the enthalpy--selected aspect ratio and the cavity
structure, then objects with different charge--to--mass ratios should have
different patterns of finite--size gravitational corrections at 2PN order.
If observations show no such dependence (within experimental uncertainties),
or if the required $a(Q,M)$ relations are inconsistent across systems, the
model would be disfavored.

In all of these channels, the logical structure is the same: the throat
ontology fixes how 2PN coefficients depend on a small set of geometric
parameters; observations overdetermine those parameters; inconsistency
rules out the model.  In this sense, the model is, at least in principle,
falsifiable.

\subsection{Why the ontology paper must come before a full 2PN analysis}
\label{subsec:2pn_role}

It is natural to ask why we have not simply proceeded directly to a 2PN
calculation.  The answer is that, without the throat ontology, the space of
possible 2PN corrections in the superfluid defect model is too large and too
poorly organized to be meaningfully constrained.

Viewed purely as a 3D fluid model with defects, one could write down an
enormous number of higher--order terms in the effective action, including
arbitrary multipole couplings, higher--derivative corrections to the vector
kernel, and a variety of nonlocal interactions.  Without additional
structure, there is no principled way to decide which of these terms should
be present, how large their coefficients should be, or how they should be
related across different sectors (gravity vs electromagnetism) and different
systems.

The throat ontology changes this situation in three important ways:

\begin{itemize}
  \item It \emph{geometrizes} finite--size effects, tying them to the
        geometry of brane--bulk throats (radius $a$, depth $L$, shape of the
        mouth, mode spectrum in the interior), rather than treating them as
        arbitrary parameters in a 3D effective theory.

  \item It \emph{links} the gravitational and electromagnetic sectors:
        the same throat that determines the monopole mass and the 1PN vector
        couplings also determines the cavity structure and the enthalpy
        selection of $L/a$ in the EM sector.  Any 2PN corrections must be
        compatible with this shared origin.

  \item It \emph{explains} the effective projector signature in the 1PN
        vector sector as a consequence of brane--bulk mode mixing, rather
        than as an ad hoc choice in the sign of the longitudinal kernel.
        This strongly constrains how higher--order vector and tensor
        corrections can appear.
\end{itemize}

With these ingredients in place, a 2PN calculation is no longer a fishing
expedition in an enormous parameter space but a targeted computation in a
geometric setting with clear boundary conditions.  The present paper is
therefore a necessary precursor: it defines the ontology in which the 2PN
problem is posed, identifies the small set of geometric parameters that
control the finite--size structure, and clarifies the relationships among
the different sectors of the model.

In the next and final section we step back from technicalities and discuss
the broader implications of the throat ontology, including open questions
about the bottom of the throat, the stabilization of the brane, and the
behavior of multiple defects interacting through the bulk.

\section{Discussion and Outlook}
\label{sec:discussion_outlook}

We have proposed a brane--bulk throat ontology for the superfluid defect toy
universe developed in Papers~I--IV and shown how it resolves several
apparent tensions in the earlier work.  In this final section we summarize
the main elements of this picture, highlight the open questions it raises,
and outline directions for future work.

\subsection{Summary of the ontology and main results}
\label{subsec:summary}

The central idea of this paper is that the defects representing massive,
possibly charged bodies in the superfluid toy model should be understood as
\emph{throats} connecting a three--dimensional brane to a four--dimensional
superfluid bulk, rather than as purely three--dimensional objects.  Each
defect corresponds to a localized region where the brane at $w=0$ pinches
into the bulk, forming a tube--like region of radius $a$ and depth $L$.

From the perspective of a brane--bound observer, the intersection of a
throat with the brane is a nearly spherical mouth of radius $a$.  The
effective 3D density and velocity fields induced on the brane by drainage
into this mouth reproduce the spherical sink picture used in the orbital,
optical, and spin/$N$--body papers: at distances $r\gg a,L$, the defect
appears as a point mass with potential $\Phi\sim -GM/r$ and 1PN corrections
controlled by the same hydrodynamic energy functional that was matched to
the EIH Lagrangian.  Dimensional reduction from 4D to 3D shows that
finite--size effects from the throat interior and transition region enter
the far field only through multipole corrections suppressed by $(a/r)^2$ and
higher powers.

From the perspective of the bulk, the region $0<w<L$ with
$\abs{\mathbf{x}}\lesssim a$ is approximately a cylindrical cavity of
cross--sectional area $\pi a^2$ and length $L$.  The 4D acoustic wave
equation separated inside this cavity yields Bessel--type radial modes and
standing waves along $w$; imposing boundary conditions at $r=a$ and at
$w=0,L$ picks out a discrete mode spectrum.  Minimizing the enthalpy at
fixed ``charge'' selects a preferred aspect ratio
$L/a = \sqrt{2}\pi/x_{01} \approx 1.85$ for the fundamental mode, exactly as
found in the electromagnetic paper.  In the throat ontology, this result is
reinterpreted as a geometric property of the throat, not as an arbitrary
parameter choice for a 3D cylinder.

The throat picture also clarifies the physical meaning of mass and charge in
the model.  Mass is associated with drainage of fluid into the throat and
the resulting volume deficit in the surrounding superfluid, with
$M\sim\rho_0\pi a^2L$ up to order--unity factors.  Charge is associated with
circulation and vorticity flux threading the throat cross--section; in the
hydrodynamic--electromagnetic dictionary, this vorticity flux plays the role
of electric charge, and the cavity modes inside the throat generate the
effective electromagnetic fields.  Thus mass and charge are different
projections of the same geometric object---a brane--bulk throat characterized by $(a,L)$ and its internal mode structure.

In the near--field region around the throat mouth, where streamlines bend
from radial inflow on the brane to axial flow into the bulk, the geometry is
strongly curved and the flow experiences significant shear and pressure
gradients.  Coarse--graining this region onto the brane generates a tower of
multipole corrections to the effective 3D density and potential, with a
leading quadrupole term suppressed by $(a/r)^2$.  This provides a natural
geometric interpretation of finite--size effects and an organizing principle
for 2PN corrections: once the throat radius $a$ is fixed for a given
defect, the coefficients of these multipoles are determined by the geometry
of the throat mouth and its deformation under external fields.

Finally, the throat ontology offers a new perspective on the wake mixing
$\alpha^2=3/4$ fixed in the 1PN vector sector.  In the vector kernel that
mediates velocity--dependent interactions, transverse (brane--parallel) and
longitudinal (bulk--directed) response enter with different \emph{weights}.
The EIH matching selects the specific real weighting encoded by
$\alpha^2=3/4$, consistent with a positive--definite quadratic wake
functional.  In the throat picture this parameter quantifies how brane motion
couples to bulk--directed flow through the throat and thereby renormalizes
the effective longitudinal/transverse projector content seen on the brane.

\subsection{What happens at the bottom of the throat?}
\label{subsec:bottom_throat}

Throughout this paper we have treated the throat as a finite tube of depth
$L$ whose bottom at $w=L$ is left deliberately unspecified.  This is
sufficient for the 1PN and near--field considerations we have focused on,
but it leaves open an important question about the global structure of the
model: what happens at the bottom of the throat?

Several possibilities present themselves:

\begin{itemize}
  \item \emph{Closed cavities.}  The throat could close off smoothly, with
        the brane folding back into itself and the superfluid forming a
        finite cavity.  In this case the mode spectrum inside the throat is
        purely discrete, and there is no leakage of energy or vorticity into
        the deep bulk.  Charge and mass would then be strictly localized
        objects, with their values determined entirely by the geometry and
        mode content of the closed throat.

  \item \emph{Open throats.}  The throat could open into a larger bulk
        region at $w=L$, allowing acoustic energy and vorticity to escape
        into the bulk.  This would introduce damping and radiation channels
        not present in the closed--cavity picture.  For instance, highly
        dynamical processes such as mergers could excite bulk modes that
        carry energy away from the brane, potentially modifying the
        gravitational wave signal seen by brane--bound observers.

  \item \emph{Connected throats.}  Different defects could be connected
        through a common bulk region, with throats that meet or merge away
        from the brane.  This would allow for bulk--mediated interactions
        between defects, beyond those encoded in the effective 3D theory on
        the brane.  In extreme cases, one could imagine networks of throats
        whose connectivity changes dynamically, perhaps in analogy with
        reconnection events in vortex tangles.
\end{itemize}

Each of these scenarios would leave a characteristic imprint on the mode
spectrum in the throat, the time dependence of charge and mass, and the way
energy is exchanged between defects and the bulk.  For example, an open
throat might lead to slow leakage of vorticity and hence slow evolution of
charge, while a closed cavity would make charge strictly conserved.  The
present paper does not attempt to choose among these options, but the
framework we have developed is flexible enough to incorporate any of them
once additional microscopic input is specified.

\subsection{What stabilizes the brane?}
\label{subsec:brane_stability}

Another major open question is the origin and stability of the brane itself.
In our construction, the brane at $w=0$ is treated as a given: a
distinguished hypersurface on which observers live and on which the
effective 3D physics of Papers~I--IV is realized.  However, in a more
complete model the brane should arise as a dynamical object, stabilized by
some underlying mechanism.

Several possibilities are familiar from other contexts:

\begin{itemize}
  \item \emph{Domain walls or defects.}  The brane might be a domain wall in
        an underlying order parameter of the superfluid, separating regions
        with different phases or densities in the bulk.  The tension of the
        domain wall and the coupling of the superfluid to this order
        parameter would determine the brane's rigidity and its response to
        the presence of throats.

  \item \emph{Potential minima.}  The brane could be a locus where some
        effective potential $V(w)$ for the fluid has a minimum, pinning
        density and pressure at $w=0$ and making deviations in the $w$
        direction energetically costly.  Throats would then be localized
        regions where this pinning is overcome or modified.

  \item \emph{Defect condensation.}  The brane could represent a condensate
        of lower--dimensional defects in the bulk superfluid, with
        throats corresponding to higher--dimensional defects that connect
        different regions of the condensate.
\end{itemize}

Any such mechanism would feed back into the throat ontology by determining
the allowed shapes of the brane near defects, the cost of bending the brane
into the bulk, and the boundary conditions satisfied by the fluid at
$w=0$.  In particular, the spectrum of small fluctuations of the brane
could interact with the cavity modes in the throats, potentially modifying
the relation between $L/a$, $M$, and $Q$.  Exploring specific brane
stabilization mechanisms and their consequences is an important direction
for future work.

\subsection{Multiple defects and bulk interactions}
\label{subsec:multi_defects}

So far we have focused on a single isolated throat.  In reality, the toy
universe contains many defects, representing stars, planets, compact
objects, and possibly charged bodies.  The throat ontology provides a
conceptual framework for thinking about how these defects interact through
the bulk as well as through the effective 3D fields on the brane.

At the level of the effective 3D theory, multiple throats interact via the
superposition of their monopole and multipole fields on the brane, as
captured by the 1PN and (future) 2PN Lagrangians.  This is the regime
directly comparable to GR.  However, the throats also interact through the
bulk superfluid: their interior modes can scatter off each other, their
transition regions can overlap when defects come close, and their bottom
regions at $w=L$ can come into contact if throats are sufficiently deep.

In a merger of two compact objects, for example, one might imagine the
following sequence: the throats approach each other on the brane; their
transition regions deform and eventually overlap; the interior cavities
interact and exchange energy; finally, a single, larger throat emerges,
possibly with a different $L/a$ ratio and a different internal mode
spectrum.  From the brane perspective this would be seen as a change in the
effective mass, spin, and charge of the merged object, together with the
emission of gravitational and electromagnetic waves.  From the bulk
perspective there would also be emission of acoustic radiation and
rearrangements of vorticity in the deep bulk.

While these speculations are necessarily qualitative at this stage, they
point to a rich dynamical phenomenology that goes beyond what is captured in
a purely 3D effective theory.  A long--term goal of the program initiated by
this toy model would be to simulate such multi--throat dynamics directly in
the 4D superfluid and compare the emergent signals (both on the brane and in
the bulk) with those predicted by GR and standard electromagnetism.

\subsection{Relation to brane-world and emergent gravity scenarios}
\label{subsec:relation_other_models}

The brane--bulk throat ontology developed here sits conceptually between
several strands of work in the broader theoretical physics literature.

On the one hand, it shares with brane--world scenarios the idea that the
observable universe is a lower--dimensional hypersurface embedded in a
higher--dimensional space, and that localized objects may be associated with
tubes or funnels connecting the brane to the bulk.  On the other hand, it
shares with emergent gravity and superfluid vacuum models the idea that
gravitational and electromagnetic phenomena can arise as effective
excitations of an underlying medium, with metric structures and gauge fields
emerging from the dynamics of collective modes.

However, the present construction is more concrete and mechanical than many
of these frameworks.  The underlying degrees of freedom are those of a
compressible superfluid obeying continuity and Euler equations in a
specified number of spatial dimensions; the brane is a geometric hypersurface
in this fluid; and defects are explicit deformations of the brane into the
bulk.  The emergent metric and electromagnetic fields are not fundamental
fields in a gravitational action, but effective descriptors of the
collective motion of the fluid.  This keeps the model firmly in the realm
of a toy universe: it is not proposed as a literal description of our
universe, but as a playground in which questions about emergence, locality,
and finite--size structure can be posed sharply and answered with explicit
calculations.

Within this playground, the throat ontology unifies several seemingly
disparate features of the earlier papers: the need for spherical sinks in
the 1PN gravitational sector, the cylindrical cavity in the EM sector, the
wake mixing $\alpha^2=3/4$, and the small but nonzero
shape--sensitivity of orbital dynamics.  All of these phenomena emerge as
different aspects of the same geometric structure: brane--bulk throats of
radius $a$ and depth $L$ embedded in a 4D superfluid.

\medskip

To move beyond the toy universe, at least conceptually, one would need to
address the microscopic origin of the superfluid, the dynamical nature of
the brane, the behavior of the model in strong--field and high--curvature
regimes, and the relationship between the emergent effective metric and a
fundamental description of spacetime.  These are ambitious questions, and we
do not attempt to answer them here.  Our more modest claim is that, within
the limited scope of weak--field, slow--motion phenomena, the throat
ontology provides a coherent and falsifiable framework in which gravity and
electromagnetism emerge from a single mechanical medium, and in which the
next steps---a full 2PN analysis and detailed comparisons with observations
---are clearly defined.

If that program succeeds, the superfluid defect toy universe will have
served its purpose: not as a rival to GR or Maxwell theory, but as a concrete
example of how a higher--dimensional, mechanical substrate can give rise to
metric gravity, gauge fields, and finite--size structure in a controlled
effective description.  If it fails, it will fail in an instructive way,
highlighting which aspects of our current theories of gravity and
electromagnetism are hardest to reproduce in any emergent, medium--based
picture.

\appendix

\section{Gaussian 4D Density and 3D Multipole Structure}
\label{app:gaussian_multipoles}

In this appendix we work out the Gaussian toy model of
Sec.~\ref{subsec:gaussian_toy_model} in detail.  The goal is to make
explicit how a localized, mildly aspherical 4D density profile produces an
effective 3D source with monopole mass
$M \propto \rho_0 L a^3$ and quadrupole moment
$Q/M \sim \varepsilon a^2$.

\subsection{Setup: 4D density and projection to the brane}

We start from the 4D density profile introduced in
Eq.~\eqref{eq:rho4_gaussian}, written here for convenience as
\begin{equation}
  \rho_4(r,\theta,w)
  = \rho_0
    \exp\!\left(-\frac{r^2}{a^2}\right)
    \exp\!\left(-\frac{w^2}{L^2}\right)
    \left[1 + \varepsilon P_2(\cos\theta)\right],
  \label{eq:app_rho4}
\end{equation}
where $(r,\theta,\phi)$ are spherical coordinates on the brane,
$w$ is the bulk coordinate, $a$ is the characteristic brane--radius of the
throat, $L$ is its depth along $w$, $\rho_0$ is a reference density, and
$\varepsilon$ is a small dimensionless parameter controlling the angular
asymmetry.  The $\ell=2$ Legendre polynomial is
\begin{equation}
  P_2(\cos\theta)
  = \frac{1}{2}(3\cos^2\theta - 1).
\end{equation}
For $\varepsilon=0$ the density is spherically symmetric on the brane and
separable in $r$ and $w$; for $\varepsilon\neq 0$ the density has a mild
quadrupolar distortion localized within $r\lesssim a$.

To obtain the effective 3D density on the brane we integrate over $w$,
taking the projection kernel $K(w)=1$ for simplicity:
\begin{equation}
  \rho_{\mathrm{3D}}(r,\theta)
  \equiv \int_{-\infty}^{+\infty}
          \rho_4(r,\theta,w)\,\mathrm{d}w.
\end{equation}
Using the Gaussian integral
\begin{equation}
  \int_{-\infty}^{+\infty}
    \exp\!\left(-\frac{w^2}{L^2}\right)\mathrm{d}w
  = \sqrt{\pi}\,L,
\end{equation}
we obtain
\begin{equation}
  \rho_{\mathrm{3D}}(r,\theta)
  = \sqrt{\pi}\,L\rho_0\,
    \exp\!\left(-\frac{r^2}{a^2}\right)
    \left[1 + \varepsilon P_2(\cos\theta)\right],
  \label{eq:app_rho3d}
\end{equation}
which is Eq.~\eqref{eq:rho3d_gaussian} in the main text.

\subsection{Total mass of the effective 3D source}

The total effective mass $M$ associated with this density is
\begin{equation}
  M
  = \int \rho_{\mathrm{3D}}(r,\theta)\,\mathrm{d}^3x.
\end{equation}
In spherical coordinates $(r,\theta,\phi)$ this becomes
\begin{equation}
  M
  = \int_0^{2\pi}\!\mathrm{d}\phi
    \int_0^{\pi}\!\mathrm{d}\theta
    \int_0^{\infty}\!\mathrm{d}r\;
      \rho_{\mathrm{3D}}(r,\theta)\,
      r^2\sin\theta.
\end{equation}
Substituting Eq.~\eqref{eq:app_rho3d} yields
\begin{align}
  M
  &= \sqrt{\pi}\,L\rho_0
     \int_0^{\infty}\!\mathrm{d}r\;
       r^2 \exp\!\left(-\frac{r^2}{a^2}\right)
     \int_0^{\pi}\!\mathrm{d}\theta\;
       \sin\theta
       \left[1 + \varepsilon P_2(\cos\theta)\right]
     \int_0^{2\pi}\!\mathrm{d}\phi.
\end{align}
The $\phi$ integral simply yields a factor of $2\pi$.  The angular integrals
are
\begin{align}
  \int_0^{\pi}\sin\theta\,\mathrm{d}\theta
  &= 2,
  \\
  \int_0^{\pi}\sin\theta\,P_2(\cos\theta)\,\mathrm{d}\theta
  &= 0,
\end{align}
where the second equality follows from the orthogonality of Legendre
polynomials.  Thus the $\varepsilon P_2$ piece does not contribute to the
total mass, as expected for a pure quadrupole distortion.  We are left with
\begin{equation}
  M
  = \sqrt{\pi}\,L\rho_0\,(2\pi)
    \left(\int_0^{\infty}
            r^2 e^{-r^2/a^2}\mathrm{d}r
    \right)\,(2).
\end{equation}
Using the standard Gaussian integral
\begin{equation}
  \int_0^{\infty} r^2 e^{-r^2/a^2}\,\mathrm{d}r
  = \frac{\sqrt{\pi}}{4}a^3,
\end{equation}
we find
\begin{equation}
  M
  = \sqrt{\pi}\,L\rho_0\,(2\pi)
    \left(\frac{\sqrt{\pi}}{4}a^3\right)\,(2)
  = \pi^2 L a^3 \rho_0.
  \label{eq:app_mass_result}
\end{equation}
This is Eq.~\eqref{eq:gaussian_mass} in the main text.  Up to the overall
numerical factor $\pi^2$, the mass scales as
\begin{equation}
  M \sim \rho_0 L a^3,
\end{equation}
as expected for a throat--like object with size $a$ and depth $L$.

\subsection{Quadrupole moment and the scaling \texorpdfstring{$Q/M \sim \varepsilon a^2$}{Q/M ~ eps a^2}}

To compute the leading angular deviation from spherical symmetry, we
consider a quadrupole--like scalar moment defined by
\begin{equation}
  Q_{20}
  \equiv \int \rho_{\mathrm{3D}}(r,\theta)\,
           r^2 P_2(\cos\theta)\,\mathrm{d}^3x,
  \label{eq:app_q20_def}
\end{equation}
up to an overall normalization convention.  Substituting
Eq.~\eqref{eq:app_rho3d} and writing out the volume element gives
\begin{equation}
  Q_{20}
  = \sqrt{\pi}\,L\rho_0
    \int_0^{2\pi}\!\mathrm{d}\phi
    \int_0^{\pi}\!\mathrm{d}\theta
    \int_0^{\infty}\!\mathrm{d}r\;
      \exp\!\left(-\frac{r^2}{a^2}\right)
      \left[1 + \varepsilon P_2(\cos\theta)\right]
      r^4 P_2(\cos\theta)\sin\theta.
\end{equation}
Again the $\phi$ integral gives $2\pi$.  Because $P_2$ is orthogonal to the
constant function on the unit sphere, only the term proportional to
$\varepsilon$ survives:
\begin{align}
  Q_{20}
  &= \sqrt{\pi}\,L\rho_0
     (2\pi)\varepsilon
     \left(
       \int_0^{\infty}
         r^4 e^{-r^2/a^2}\,\mathrm{d}r
     \right)
     \left(
       \int_0^{\pi}
         \sin\theta\,[P_2(\cos\theta)]^2 \mathrm{d}\theta
     \right).
\end{align}
We now evaluate the radial and angular factors separately.

For the radial integral, we use the standard formula
\begin{equation}
  \int_0^{\infty}
    r^{2n} e^{-\lambda r^2}\,\mathrm{d}r
  = \frac{(2n-1)!!}{2^{n+1}}\sqrt{\frac{\pi}{\lambda^{2n+1}}},
\end{equation}
with $n=2$ and $\lambda = 1/a^2$, giving
\begin{equation}
  \int_0^{\infty}
    r^4 e^{-r^2/a^2}\,\mathrm{d}r
  = \frac{3\sqrt{\pi}}{8}a^5.
\end{equation}

For the angular integral, we exploit orthogonality of Legendre polynomials:
\begin{equation}
  \int_0^{\pi}
    \sin\theta\,[P_\ell(\cos\theta)]^2\mathrm{d}\theta
  = \frac{2}{2\ell+1}.
\end{equation}
Setting $\ell=2$ yields
\begin{equation}
  \int_0^{\pi}
    \sin\theta\,[P_2(\cos\theta)]^2\mathrm{d}\theta
  = \frac{2}{5}.
\end{equation}

Putting these pieces together, we find
\begin{align}
  Q_{20}
  &= \sqrt{\pi}\,L\rho_0
     (2\pi)\varepsilon
     \left(\frac{3\sqrt{\pi}}{8}a^5\right)
     \left(\frac{2}{5}\right)
  \\
  &= \frac{3}{10}\,\pi^2 L a^5 \varepsilon \rho_0.
  \label{eq:app_q20_result}
\end{align}
This is Eq.~\eqref{eq:q20_result} in the main text.

Dividing Eq.~\eqref{eq:app_q20_result} by the mass
Eq.~\eqref{eq:app_mass_result} we obtain
\begin{equation}
  \frac{Q_{20}}{M}
  = \frac{\frac{3}{10}\pi^2 L a^5 \varepsilon \rho_0}
         {\pi^2 L a^3 \rho_0}
  = \frac{3}{10}\,\varepsilon a^2.
  \label{eq:app_q_over_m}
\end{equation}
Up to the numerical factor $3/10$, this confirms the scaling
\begin{equation}
  \frac{Q}{M} \sim \varepsilon a^2,
\end{equation}
used in Sec.~\ref{subsec:gaussian_toy_model}.  The quadrupole moment per
unit mass is proportional to the square of the source size $a$ and to the
dimensionless asymmetry parameter $\varepsilon$.

\subsection{Remarks and generalizations}

A few comments are in order:

\begin{itemize}
  \item The precise numerical coefficients ($\pi^2$, $3/10$, etc.) depend on
        our choice of Gaussian profile and normalization conventions for the
        multipole moments.  Different smooth, localized profiles with the
        same characteristic size $a$ and small quadrupolar distortion
        $\varepsilon P_2$ would yield different order--unity factors but the
        same scaling $M\sim \rho_0 L a^3$ and $Q/M\sim \varepsilon a^2$.

  \item The fact that the $\varepsilon$--dependent term drops out of the
        total mass but dominates the quadrupole is a direct consequence of
        the orthogonality of Legendre polynomials.  This is generic: pure
        multipole distortions do not alter the monopole mass but do control
        the higher multipole moments.

  \item Higher multipoles can be generated by replacing $P_2(\cos\theta)$ in
        Eq.~\eqref{eq:app_rho4} with $P_\ell(\cos\theta)$ for $\ell>2$, or
        by adding a sum $\sum_\ell \varepsilon_\ell P_\ell(\cos\theta)$.  A
        similar calculation then yields
        \begin{equation}
          \frac{Q_\ell}{M}
          \sim \varepsilon_\ell a^\ell,
        \end{equation}
        up to order--unity coefficients, leading to potential corrections
        of order $(a/r)^\ell$ in the far field.

  \item The exponential factors in $r$ and $w$ are chosen for analytic
        convenience.  In the context of the throat ontology, one should view
        this Gaussian as a caricature of the true density profile near the
        throat mouth: it captures the localization at scales $r\sim a$,
        $w\sim L$ and the leading angular structure without pretending to be
        a faithful solution of the full 4D Euler equations.
\end{itemize}

These calculations justify the scaling arguments used in the main text:
dimensional reduction of a localized 4D throat--like density produces an
effective 3D source whose finite--size corrections to the potential are
naturally organized as a multipole expansion with coefficients suppressed by
powers of $(a/r)$ and controlled by a small set of geometric parameters
$(a,L,\varepsilon_\ell,\ldots)$.

\section{Numerical Evaluation of a Physical Throat Model}
\label{app:physical_throat}

In this appendix we detail the numerical evaluation of the finite--size coefficient $\alpha_2$ quoted in
Sec.~\ref{subsec:rounded_funnel}.  Unlike the analytic Gaussian model of Appendix~\ref{app:gaussian_multipoles},
this calculation uses a hard--bounded ``rounded funnel'' geometry that explicitly connects the cylindrical bulk
interior to the brane mouth.

\subsection{Geometry and Integration}

We use spherical coordinates $(r,\theta,\phi)$ on the brane and a depth coordinate $w$ into the bulk.  The throat
domain $\mathcal{T}$ is bounded by $0\le w\le L$ and $0\le r\le R(w)$, where the radius function flares near the
brane according to
\begin{equation}
  R(w) = a\left[1+\frac{1}{2}\exp\!\left(-\frac{5w}{a}\right)\right].
\end{equation}
This profile describes a cylinder of radius $a$ that widens to $1.5a$ at the mouth ($w=0$), providing a simple
proxy for the smooth bending of streamlines in the transition region.

The effective density within this volume is taken to be
\begin{equation}
  \rho(r,\theta,w) = \rho_0\big[1+\varepsilon P_2(\cos\theta)\big],
\end{equation}
where $\varepsilon$ parametrizes a small quadrupolar anisotropy.  The monopole mass $M$ and quadrupole moment $Q$
entering the far--field potential \eqref{eq:phi_multipole_general} are obtained by integrating over $\mathcal{T}$:
\begin{align}
  M &= \int_{\mathcal{T}} \rho\,\mathrm{d}^3x\,\mathrm{d}w
     = \int_0^L \mathrm{d}w \int \mathrm{d}\Omega \int_0^{R(w)} r^2\,\mathrm{d}r \,\rho(r,\theta,w), \\
  Q &= \int_{\mathcal{T}} r^2 P_2(\cos\theta)\,\rho\,\mathrm{d}^3x\,\mathrm{d}w
     = \int_0^L \mathrm{d}w \int \mathrm{d}\Omega \int_0^{R(w)} r^2\,\mathrm{d}r \, r^2 P_2(\cos\theta)\,\rho(r,\theta,w),
\end{align}
with $\mathrm{d}\Omega=\sin\theta\,\mathrm{d}\theta\,\mathrm{d}\phi$.  Using $\int \mathrm{d}\Omega\,P_2=0$ and
$\int \mathrm{d}\Omega\,P_2^2 = 4\pi/5$, the angular integrals reduce these expressions to one--dimensional
integrals over the funnel profile:
\begin{equation}
  M = \frac{4\pi\rho_0}{3}\int_0^L R(w)^3\,\mathrm{d}w,
  \qquad
  Q = \frac{4\pi\rho_0\,\varepsilon}{25}\int_0^L R(w)^5\,\mathrm{d}w.
\end{equation}
We evaluate these integrals numerically for the parameter choices quoted below.

\subsection{Results}

For the representative parameter set $a=1.0$, $L=2.0$, $\rho_0=1.0$, and anisotropy $\varepsilon=0.1$, numerical
quadrature yields
\begin{equation}
  M \approx 9.98,
  \qquad
  Q \approx 0.143.
\end{equation}
The dimensionless finite--size coefficient $\alpha_2$ is defined by
$\alpha_2 = Q/(M a^2)$, so the above values imply
\begin{equation}
  \alpha_2 = \frac{0.143}{9.98\times (1.0)^2} \approx 0.014.
\end{equation}
Thus, for a physically bounded throat geometry with modest anisotropy, the leading multipole coefficient is
small ($\alpha_2\ll 1$), ensuring that the defect behaves as an effectively spherical source in the far field.

\section{4D Mode Separation in the Cylindrical Throat}
\label{app:mode_separation}

In this appendix we record the basic mode structure of the 4D acoustic
equation inside the cylindrical throat.  The aim is to make explicit the
separation of variables leading to Bessel--type radial profiles and standing
waves along the bulk direction $w$, together with the corresponding
eigenfrequencies and orthogonality relations.  This is the mode structure
used implicitly in Sec.~\ref{sec:throat_em} and in the electromagnetic
paper when discussing cavity modes and the enthalpy--selected aspect ratio
$L/a$.

\subsection{Acoustic equation and throat geometry}

We begin from the linear 4D acoustic equation for the enthalpy perturbation
$h$,
\begin{equation}
  \partial_t^2 h
  - c_s^2 \nabla_4^2 h
  = 0,
  \qquad
  \nabla_4^2
  = \partial_x^2 + \partial_y^2 + \partial_z^2 + \partial_w^2,
  \label{eq:app_4d_wave}
\end{equation}
as derived in Sec.~\ref{subsec:bulk_fields}.  Inside the throat we
approximate the geometry as a straight cylinder of radius $a$ in the brane
directions and depth $L$ along the bulk direction $w$.  We choose
coordinates adapted to this geometry:
\begin{equation}
  (r,\phi,w),
\end{equation}
where $(r,\phi)$ are polar coordinates in a cross--section through the
throat mouth on the brane (with $r$ measured from the throat center), and
$w\in[0,L]$ measures distance along the throat into the bulk.  For the
lowest--lying modes of interest we assume axisymmetry in $\phi$ and no
dependence on the third brane direction orthogonal to the throat axis.

In this approximation the Laplacian restricted to axisymmetric configurations
becomes
\begin{equation}
  \nabla_4^2 h
  = \frac{1}{r}\partial_r\!\left(r\partial_r h\right)
    + \partial_w^2 h,
  \label{eq:app_laplacian}
\end{equation}
so the wave equation reads
\begin{equation}
  \partial_t^2 h
  - c_s^2
    \left[
      \frac{1}{r}\partial_r\!\left(r\partial_r h\right)
      + \partial_w^2 h
    \right]
  = 0.
  \label{eq:app_wave_rw}
\end{equation}

\subsection{Separation of variables and eigenvalue problem}

We look for separated, time--harmonic solutions of the form
\begin{equation}
  h(t,r,w)
  = \Re\!\left\{
      H(r)\,W(w)\,e^{-i\omega t}
    \right\},
  \label{eq:app_separated_ansatz}
\end{equation}
with real frequency $\omega$.  Substituting
Eq.~\eqref{eq:app_separated_ansatz} into Eq.~\eqref{eq:app_wave_rw} and
dividing by $H(r)W(w)e^{-i\omega t}$ yields
\begin{equation}
  -\omega^2
  - c_s^2
    \left[
      \frac{1}{H}
      \frac{1}{r}\partial_r\!\left(r\partial_r H\right)
      + \frac{1}{W}\partial_w^2 W
    \right]
  = 0.
\end{equation}
Rearranging gives
\begin{equation}
  \frac{1}{H}
  \frac{1}{r}\partial_r\!\left(r\partial_r H\right)
  + \frac{1}{W}\partial_w^2 W
  = -\,\frac{\omega^2}{c_s^2}.
  \label{eq:app_separation_eq}
\end{equation}
The left--hand side is a sum of a function of $r$ and a function of $w$,
while the right--hand side is a constant, so each term must separately be
equal to a constant.  Introducing separation constants $-k_r^2$ and
$-k_w^2$, we write
\begin{align}
  \frac{1}{H}
  \frac{1}{r}\partial_r\!\left(r\partial_r H\right)
  &= -k_r^2,
  \label{eq:app_radial_eq_pre}
  \\
  \frac{1}{W}\partial_w^2 W
  &= -k_w^2,
  \label{eq:app_axial_eq_pre}
\end{align}
and the dispersion relation
\begin{equation}
  \omega^2
  = c_s^2(k_r^2 + k_w^2).
  \label{eq:app_dispersion}
\end{equation}

Equations~\eqref{eq:app_radial_eq_pre}--\eqref{eq:app_axial_eq_pre}
constitute a pair of ordinary differential equations:
\begin{align}
  \frac{1}{r}\partial_r\!\left(r\partial_r H\right)
  + k_r^2 H &= 0,
  \label{eq:app_radial_eq}
  \\
  \partial_w^2 W
  + k_w^2 W &= 0.
  \label{eq:app_axial_eq}
\end{align}
The allowed values of $k_r$ and $k_w$ are determined by boundary conditions
on $H(r)$ and $W(w)$ inside the throat.

\subsection{Radial modes and Bessel functions}

The radial equation \eqref{eq:app_radial_eq} is the standard Bessel equation
of order zero.  Its general solution is
\begin{equation}
  H(r)
  = A J_0(k_r r) + B Y_0(k_r r),
\end{equation}
where $J_0$ and $Y_0$ are Bessel functions of the first and second kind,
respectively, and $A,B$ are constants.  Regularity at $r=0$ requires $B=0$,
since $Y_0(x)$ is singular at the origin.  We therefore take
\begin{equation}
  H(r) = A J_0(k_r r).
\end{equation}

At the throat wall $r=a$ we impose a boundary condition on $h$ corresponding
to a pinned phase boundary of the stiff superfluid vacuum.  Concretely, we
take a Dirichlet condition on the enthalpy perturbation,
\begin{equation}
  h(r=a,w,t) = 0
  \quad\Rightarrow\quad
  H(a) = 0,
\end{equation}
This pinning stabilizes the cylindrical throat against radial collapse:
expansions or contractions of the wall necessarily excite bulk modes that
raise the vacuum enthalpy.  This holds for all $w$ and $t$ and implies
\begin{equation}
  J_0(k_r a) = 0.
\end{equation}
Let $x_{0n}$ denote the $n$th zero of $J_0(x)$, with
\begin{equation}
  0 < x_{01} < x_{02} < \cdots.
\end{equation}
Then the allowed radial wavenumbers are
\begin{equation}
  k_r^{(n)}
  = \frac{x_{0n}}{a},
  \qquad
  n = 1,2,\dots,
\end{equation}
and the corresponding radial eigenfunctions are
\begin{equation}
  H_n(r)
  = A_n J_0\!\left(\frac{x_{0n} r}{a}\right).
\end{equation}

These eigenfunctions satisfy an orthogonality relation with respect to the
measure $r\,\mathrm{d}r$:
\begin{equation}
  \int_0^a r
    J_0\!\left(\frac{x_{0m} r}{a}\right)
    J_0\!\left(\frac{x_{0n} r}{a}\right)
  \mathrm{d}r
  = \frac{a^2}{2}\,[J_1(x_{0n})]^2\,\delta_{mn},
  \label{eq:app_bessel_orth}
\end{equation}
where $J_1$ is the Bessel function of order one and $\delta_{mn}$ is the
Kronecker delta.  One can therefore choose the normalization constants $A_n$
so that the $H_n(r)$ form an orthonormal basis on $[0,a]$.

\subsection{Axial modes and standing waves along \texorpdfstring{$w$}{w}}

The axial equation \eqref{eq:app_axial_eq} has the general solution
\begin{equation}
  W(w)
  = C\cos(k_w w) + D\sin(k_w w),
\end{equation}
with $C,D$ constants.  The appropriate boundary conditions depend on the
microscopic physics at the mouth ($w=0$) and bottom ($w=L$) of the throat.
For definiteness, and to match the cavity analysis in the main text, we
impose Dirichlet conditions at both ends:
\begin{equation}
  W(0) = W(L) = 0.
\end{equation}
The condition $W(0)=0$ implies $C=0$, so $W(w) = D\sin(k_w w)$.  The
condition $W(L)=0$ then requires
\begin{equation}
  \sin(k_w L) = 0
  \quad\Rightarrow\quad
  k_w L = n\pi,
  \qquad n=1,2,\dots,
\end{equation}
so the allowed axial wavenumbers are
\begin{equation}
  k_w^{(n)}
  = \frac{n\pi}{L},
  \qquad n=1,2,\dots,
\end{equation}
and the axial eigenfunctions are
\begin{equation}
  W_n(w)
  = D_n \sin\!\left(\frac{n\pi w}{L}\right).
\end{equation}

These functions are orthogonal on $[0,L]$:
\begin{equation}
  \int_0^L
    \sin\!\left(\frac{m\pi w}{L}\right)
    \sin\!\left(\frac{n\pi w}{L}\right)\mathrm{d}w
  = \frac{L}{2}\,\delta_{mn},
  \label{eq:app_sine_orth}
\end{equation}
so the constants $D_n$ can be chosen to yield an orthonormal set.

Other choices of boundary conditions (Neumann or mixed) are possible and
would lead to cosine or sine--cosine combinations with shifted eigenvalues,
but the qualitative structure---a discrete tower of axial modes with
$k_w\sim n\pi/L$---is the same.

\subsection{Mode spectrum and fundamental throat mode}

Combining the radial and axial modes, the general separated solution inside
the throat is a superposition of eigenmodes labeled by a pair of integers
$(m,n)$:
\begin{equation}
  h_{mn}(t,r,w)
  = A_{mn}
    J_0\!\left(\frac{x_{0m} r}{a}\right)
    \sin\!\left(\frac{n\pi w}{L}\right)
    \cos(\omega_{mn} t + \varphi_{mn}),
  \label{eq:app_full_mode}
\end{equation}
where $A_{mn}$ and $\varphi_{mn}$ are real amplitude and phase, and the
eigenfrequencies are given by
\begin{equation}
  \omega_{mn}^2
  = c_s^2
    \left[
      \left(\frac{x_{0m}}{a}\right)^2
      + \left(\frac{n\pi}{L}\right)^2
    \right],
  \qquad m,n=1,2,\dots.
  \label{eq:app_mode_spectrum}
\end{equation}
The lowest--lying mode (apart from any zero modes associated with trivial
symmetries) is the fundamental $(m,n)=(1,1)$ mode:
\begin{equation}
  h_{11}(t,r,w)
  \propto J_0\!\left(\frac{x_{01} r}{a}\right)
           \sin\!\left(\frac{\pi w}{L}\right)
           \cos(\omega_{11} t),
  \qquad
  \omega_{11}^2
  = c_s^2
    \left[
      \left(\frac{x_{01}}{a}\right)^2
      + \left(\frac{\pi}{L}\right)^2
    \right].
  \label{eq:app_fundamental_mode}
\end{equation}
This is the mode used in Sec.~\ref{subsec:throat_modes} and in the
enthalpy--minimization argument for the preferred aspect ratio $L/a$.

\subsection{Normalization and enthalpy functional (sketch)}

For completeness, we briefly record how the normalization of the modes
enters the enthalpy functional, leaving detailed variations to the main
text and to the electromagnetic paper.

The time--averaged perturbation energy (or enthalpy) associated with a mode
$h$ can be written schematically as
\begin{equation}
  \mathcal{E}[h]
  = \frac{1}{2}
    \int_{\mathcal{T}}\!\mathrm{d}V_4\;
      \rho_0
      \left[
        \frac{1}{c_s^2}\,\langle(\partial_t h)^2\rangle
        + \langle(\nabla_4 h)^2\rangle
      \right],
  \label{eq:app_energy_func}
\end{equation}
where $\mathrm{d}V_4 = r\,\mathrm{d}r\,\mathrm{d}\phi\,\mathrm{d}w$ is the
4D volume element restricted to the throat and $\langle\cdot\rangle$
denotes an average over one oscillation period.

For a single eigenmode of the form \eqref{eq:app_full_mode} with frequency
$\omega_{mn}$, the time average gives
\begin{align}
  \langle(\partial_t h_{mn})^2\rangle
  &= \frac{1}{2}\omega_{mn}^2 A_{mn}^2
     J_0^2\!\left(\frac{x_{0m} r}{a}\right)
     \sin^2\!\left(\frac{n\pi w}{L}\right),
  \\
  \langle(\nabla_4 h_{mn})^2\rangle
  &= \frac{1}{2} A_{mn}^2
     \left[
       \left(\frac{x_{0m}}{a}\right)^2
       + \left(\frac{n\pi}{L}\right)^2
     \right]
     J_0^2\!\left(\frac{x_{0m} r}{a}\right)
     \sin^2\!\left(\frac{n\pi w}{L}\right),
\end{align}
so that the integrand in Eq.~\eqref{eq:app_energy_func} is proportional to
$(k_r^2+k_w^2)A_{mn}^2 J_0^2 \sin^2$, with $k_r=x_{0m}/a$ and
$k_w=n\pi/L$.  Using the orthogonality relations
\eqref{eq:app_bessel_orth} and \eqref{eq:app_sine_orth}, one finds that the
total energy stored in the mode scales as
\begin{equation}
  \mathcal{E}_{mn}
  \propto A_{mn}^2 (k_r^2 + k_w^2)\,a^2 L,
\end{equation}
up to order--unity constants coming from the angular integrals and
normalization choices.

A ``charge''--like quantity associated with the mode amplitude, such as the
vorticity flux through the throat cross--section, typically scales as
\begin{equation}
  \mathcal{Q}_{mn}
  \propto A_{mn}^2 a^2 L,
\end{equation}
again up to order--unity factors.  At fixed $\mathcal{Q}_{mn}$, the energy
is therefore proportional to $k_r^2 + k_w^2$:
\begin{equation}
  \mathcal{E}_{mn}
  \propto (k_r^2 + k_w^2)\,\mathcal{Q}_{mn}.
\end{equation}
For the fundamental mode $(m,n)=(1,1)$, this reduces to
\begin{equation}
  \mathcal{E}_{11}
  \propto
    \left[
      \left(\frac{x_{01}}{a}\right)^2
      + \left(\frac{\pi}{L}\right)^2
    \right]\mathcal{Q}_{11}.
\end{equation}
The enthalpy--minimization problem at fixed ``charge'' discussed in
Sec.~\ref{subsec:aspect_ratio} and in the electromagnetic paper is
equivalent to minimizing this combination with respect to $a$ and $L$, subject
to appropriate constraints (e.g.\ fixed mass and/or other geometric
quantities).  The resulting optimum selects a particular aspect ratio
$L/a$, which in the toy model of Paper~IV and in the present ontology is
\begin{equation}
  \frac{L}{a}
  = \frac{\sqrt{2}\,\pi}{x_{01}},
\end{equation}
interpreted as a geometric property of the throat rather than an externally
imposed parameter.

We will not repeat the full variational calculation here; the purpose of
this appendix is to show how the basic mode structure and its dependence on
$a$ and $L$ arise from the 4D acoustic equation inside the cylindrical
throat.

\section{Two-Mode Toy Model for \texorpdfstring{$\alpha^2 = 3/4$}{alpha^2 = 3/4}}
\label{app:two_mode_details}

It is useful to keep a minimal algebraic picture in mind for the meaning of the
wake--mixing parameter $\alpha^2$ that appears in the isotropic longitudinal /
transverse projector decomposition of the vector kernel.

Consider two quadratic degrees of freedom, a transverse amplitude $u_T$ and a
longitudinal amplitude $u_L$, with a positive--definite energy
\begin{equation}
  E = \frac{1}{2}\left(A_T u_T^2 + A_L u_L^2\right),
  \label{eq:app_two_mode_energy}
\end{equation}
where $A_T>0$ and $A_L>0$.
Suppose that the physical (EIH--mediating) combination excited by a moving
throat involves a fixed mixing $u_L=\alpha\,u_T$.  Substituting into
Eq.~\eqref{eq:app_two_mode_energy} gives an effective single--mode energy
\begin{equation}
  E = \frac{1}{2}\left(A_T+\alpha^2 A_L\right)u_T^2.
\end{equation}
For the calibrated value $\alpha^2=3/4$,
\begin{equation}
  E = \frac{1}{8}\left(4A_T+3A_L\right)u_T^2,
\end{equation}
which is manifestly positive if $A_T,A_L>0$.

In matrix language the quadratic form in \eqref{eq:app_two_mode_energy} has
eigenvalues $\{A_L,A_T\}$, so the underlying wake functional is Euclidean and
bounded below.  The parameter $\alpha^2$ therefore functions as a \emph{real
weighting} of longitudinal relative to transverse content in the particular
combination that feeds the 1PN interaction, rather than as an indicator of an
indefinite (``Lorentzian'') signature.
\begin{thebibliography}{99}

\bibitem{Norris:2025Orbits}
Norris, T. (2025).
\newblock \emph{Newtonian and 1PN Orbital Dynamics from a Superfluid Defect Toy Model}.
\newblock Zenodo.
\newblock \href{https://doi.org/10.5281/zenodo.17875005}{doi:10.5281/zenodo.17875005}.

\bibitem{Norris:Paper2}
Norris, T. (2025).
\newblock \emph{Gravitational Optics and Soliton Geodesics in a Superfluid Defect Toy Model}.
\newblock Zenodo.
\newblock \href{https://doi.org/10.5281/zenodo.17918131}{doi:10.5281/zenodo.17918131}.

\bibitem{Norris:Paper3}
Norris, T. (2025).
\newblock \emph{Spin, Vorticity, and N-Body Dynamics in a Superfluid Defect Toy Model}.
\newblock Zenodo.
\newblock \href{https://doi.org/10.5281/zenodo.17798511}{doi:10.5281/zenodo.17798511}.

\bibitem{Norris:Paper4}
Norris, T. (2025).
\newblock \emph{Electromagnetic Fields and Charged Defects in a Superfluid Defect Toy Model}.
\newblock Zenodo.
\newblock \href{https://doi.org/10.5281/zenodo.17967962}{doi:10.5281/zenodo.17967962}.

\end{thebibliography}

\end{document}
